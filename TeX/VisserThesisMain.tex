%%%%%%%%%%%%%%%%%%%%%%%%%%%%%%%%%%%%%%%%%
% NOTICE: THIS VERSION HAS BEEN HEAVILTY MODIFIED
% BY MARCO D. VISSER from it's original (below)
% Tufte-Style Book (Documentation Template)
% LaTeX Template
% Version 1.0 (5/1/13)
%
%
% Original author:
% The Tufte-LaTeX Developers (tufte-latex.googlecode.com)
%
% License:
% Apache License (Version 2.0)
%
% IMPORTANT NOTE:
% In addition to running BibTeX to compile the reference list from the .bib
% file, you will need to run MakeIndex to compile the index at the end of the
% document.
%
%%%%%%%%%%%%%%%%%%%%%%%%%%%%%%%%%%%%%%%%%

%----------------------------------------------------------------------------------------
%	PACKAGES AND OTHER DOCUMENT CONFIGURATIONS
%----------------------------------------------------------------------------------------

\documentclass[b5paper,justified]{tufte-book} % Use the tufte-book class which in turn uses the tufte-common class


%\usepackage{tufteLargePrint} % I want to use my own custom large print style

%\usepackage[b5paper]{geometry} % I want to change the book size
%\geometry{height=21cm,width=14.8cm}

%\hypersetup{colorlinks} % Comment this line if you don't wish to have colored links

\usepackage{microtype} % Improves character and word spacing
\usepackage{fancybox} % I want text boxes
\usepackage{lipsum} % Inserts dummy text
\usepackage{pdflscape} % I want certain sections as a landscape
\usepackage{longtable}  % I have long tables that extend over pages
\usepackage{amsmath} % I want to add matrix algrebra and matrices
\usepackage{setspace} % I want to set line spacing
\usepackage{flafter}
\usepackage{fancyhdr}
\usepackage{enumitem} % I need to control spacing after itemize and enumerate lists
\usepackage{array,multirow} % I need multirow cells in table with rotated text
\fancypagestyle{lscape}{% 
\fancyhf{} % clear all header and footer fields 
\fancyfoot[LE]{}
\fancyfoot[LO] {}
\renewcommand{\headrulewidth}{0pt} 
\renewcommand{\footrulewidth}{0pt}}

%%%%%%%%%%%%%%%%%%%%%%%%%%%%%%%%%%%%%%%%%%%%%%%%%%%%%%%%%%%%%%%%%%%%%%%%%%%%%%%%%%%%%%%%%
%\usepackage[round]{natbib} % I want to cite from my bibtex library
\bibliographystyle{mee} % the style of referencing

\usepackage{booktabs} % Better horizontal rules in tables

\usepackage{graphicx} % Needed to insert images into the document
\graphicspath{{graphics/}} % Sets the default location of pictures
\setkeys{Gin}{width=\linewidth,totalheight=\textheight,keepaspectratio} % Improves figure scaling

\usepackage{fancyvrb} % Allows customization of verbatim environments
\fvset{fontsize=\normalsize} % The font size of all verbatim text can be changed here

\newcommand{\hangp}[1]{\makebox[0pt][r]{(}\#1\makebox[0pt][l]{)}} % New command to create parentheses around text in tables which take up no horizontal space - this improves column spacing
\newcommand{\hangstar}{\makebox[0pt][l]{*}} % New command to create asterisks in tables which take up no horizontal space - this improves column spacing

\usepackage{float} % I want to be able to stop repositioning of floats
\usepackage{xspace} % Used for printing a trailing space better than using a tilde (~) using the \xspace command

%%%%%%%%%%%%%%%%%%%%%%%%%%%%%%%%%%%%%%%%%%%%%%%%%%%%%%%%%%%%%%%%%%%%%%%%%%%%%%%%%%%%%%%%%%%%%%%%%%%%%%%%%%%%%%%%%%%%%
%% Custoom commands

\newcommand{\specialcell}[2][c]{%
  \begin{tabular}[\#1]{@{}c@{}}\#2\end{tabular}} % I want line breaks within a table cell

\newcommand{\monthyear}{\ifcase\month\or January\or February\or March\or April\or May\or June\or July\or August\or September\or October\or November\or December\fi\space\number\year} % A command to print the current month and year

\newcommand{\openepigraph}[2]{ % This block sets up a command for printing an epigraph with 2 arguments - the quote and the author
\begin{fullwidth}
\sffamily\large
\vspace{5mm}
\noindent\textit{\#1}\\ % The quote
\noindent\textbf{\#2} % The author
\vspace{5mm}
\end{fullwidth}
}

\newcommand{\blankpage}{\newpage\hbox{}\thispagestyle{empty}\newpage} % Command to insert a blank page

\usepackage{units} % Used for printing standard units

\newcommand{\hlred}[1]{\textcolor{Maroon}{\#1}} % Print text in maroon
\newcommand{\hangleft}[1]{\makebox[0pt][r]{\#1}} % Used for printing commands in the index, moves the slash left so the command name aligns with the rest of the text in the index 
\newcommand{\hairsp}{\hspace{1pt}} % Command to print a very short space
\newcommand{\ie}{\textit{i.\hairsp{}e.}\xspace} % Command to print i.e.
\newcommand{\eg}{\textit{e.\hairsp{}g.}\xspace} % Command to print e.g.
\newcommand{\na}{\quad--} % Used in tables for N/A cells
\newcommand{\measure}[3]{\#1/\#2$\times$\unit[\#3]{pc}} % Typesets the font size, leading, and measure in the form of: 10/12x26 pc.
\newcommand{\tuftebs}{\symbol{'134}} % Command to print a backslash in tt type in OT1/T1

\providecommand{\XeLaTeX}{X\lower.5ex\hbox{\kern-0.15em\reflectbox{E}}\kern-0.1em\LaTeX}
\newcommand{\tXeLaTeX}{\XeLaTeX\index{XeLaTeX@\protect\XeLaTeX}} % Command to print the XeLaTeX logo while simultaneously adding the position to the index

\newcommand{\doccmdnoindex}[2][]{\texttt{\tuftebs\#2}} % Command to print a command in texttt with a backslash of tt type without inserting the command into the index

\newcommand{\doccmddef}[2][]{\hlred{\texttt{\tuftebs\#2}}\label{cmd:\#2}\ifthenelse{\isempty{\#1}} % Command to define a command in red and add it to the index
{ % If no package is specified, add the command to the index
\index{\#2 command@\protect\hangleft{\texttt{\tuftebs}}\texttt{\#2}}% Command name
}
{ % If a package is also specified as a second argument, add the command and package to the index
\index{\#2 command@\protect\hangleft{\texttt{\tuftebs}}\texttt{\#2} (\texttt{\#1} package)}% Command name
\index{\#1 package@\texttt{\#1} package}\index{packages!\#1@\texttt{\#1}}% Package name
}}

\newcommand{\doccmd}[2][]{% Command to define a command and add it to the index
\texttt{\tuftebs\#2}%
\ifthenelse{\isempty{\#1}}% If no package is specified, add the command to the index
{%
\index{\#2 command@\protect\hangleft{\texttt{\tuftebs}}\texttt{\#2}}% Command name
}
{%
\index{\#2 command@\protect\hangleft{\texttt{\tuftebs}}\texttt{\#2} (\texttt{\#1} package)}% Command name
\index{\#1 package@\texttt{\#1} package}\index{packages!\#1@\texttt{\#1}}% Package name
}}

% A bunch of new commands to print commands, arguments, environments, classes, etc within the text using the correct formatting
\newcommand{\docopt}[1]{\ensuremath{\langle}\textrm{\textit{\#1}}\ensuremath{\rangle}}
\newcommand{\docarg}[1]{\textrm{\textit{\#1}}}
\newenvironment{docspec}{\begin{quotation}\ttfamily\parskip0pt\parindent0pt\ignorespaces}{\end{quotation}}
\newcommand{\docenv}[1]{\texttt{\#1}\index{\#1 environment@\texttt{\#1} environment}\index{environments!\#1@\texttt{\#1}}}
\newcommand{\docenvdef}[1]{\hlred{\texttt{\#1}}\label{env:\#1}\index{\#1 environment@\texttt{\#1} environment}\index{environments!\#1@\texttt{\#1}}}
\newcommand{\docpkg}[1]{\texttt{\#1}\index{\#1 package@\texttt{\#1} package}\index{packages!\#1@\texttt{\#1}}}
\newcommand{\doccls}[1]{\texttt{\#1}}
\newcommand{\docclsopt}[1]{\texttt{\#1}\index{\#1 class option@\texttt{\#1} class option}\index{class options!\#1@\texttt{\#1}}}
\newcommand{\docclsoptdef}[1]{\hlred{\texttt{\#1}}\label{clsopt:\#1}\index{\#1 class option@\texttt{\#1} class option}\index{class options!\#1@\texttt{\#1}}}
\newcommand{\docmsg}[2]{\bigskip\begin{fullwidth}\noindent\ttfamily\#1\end{fullwidth}\medskip\par\noindent\#2}
\newcommand{\docfilehook}[2]{\texttt{\#1}\index{file hooks!\#2}\index{\#1@\texttt{\#1}}}
\newcommand{\doccounter}[1]{\texttt{\#1}\index{\#1 counter@\texttt{\#1} counter}}

\usepackage{makeidx} % Used to generate the index
\makeindex % Generate the index which is printed at the end of the document

% This block contains a number of shortcuts used throughout the book
\newcommand{\vdqi}{\textit{VDQI}\xspace}
\newcommand{\ei}{\textit{EI}\xspace}
\newcommand{\ve}{\textit{VE}\xspace}
\newcommand{\be}{\textit{BE}\xspace}
\newcommand{\VDQI}{\textit{The Visual Display of Quantitative Information}\xspace}
\newcommand{\EI}{\textit{Envisioning Information}\xspace}
\newcommand{\VE}{\textit{Visual Explanations}\xspace}
\newcommand{\BE}{\textit{Beautiful Evidence}\xspace}
\newcommand{\TL}{Tufte-\LaTeX\xspace}

%----------------------------------------------------------------------------------------
%	SECTION NUMBERING to subsubsections {3} see below
%----------------------------------------------------------------------------------------

\setcounter{secnumdepth}{3}

%----------------------------------------------------------------------------------------
%	custom subtitle command
%----------------------------------------------------------------------------------------


\renewcommand{\maketitlepage}{%
  \cleardoublepage%
  {%
  \sffamily%
  \begin{fullwidth}%
  \begin{center}

  \fontsize{25}{30}\selectfont\par\noindent\textcolor{black}{\allcaps{\thanklesstitle}}%
  \vspace{.5cm}
  \fontsize{18}{20}\selectfont\par\noindent\textcolor{darkgray}{\allcaps{\@subtitle}}%
  \vspace{3cm}
  \fontsize{10}{12}\selectfont\par\noindent\textcolor{darkgray}{
	Proefschrift \\
	ter verkrijging van de graad van doctor \\
	aan de Radboud Universiteit Nijmegen \\
	op gezag van rector magnificus prof. dr. J.H.J.M. van Krieken \\
	volgens besluit van college van decanen \\}
	\vspace{3cm}
   \fontsize{10}{12}\selectfont\par\noindent\textcolor{darkgray}{
	in het openbaar te verdedigen op woensdag 16 november 2016 \\
	des namiddags om 14.30 uur precies \\}
	\vspace{3cm}
    \fontsize{10}{12}\selectfont\par\noindent\textcolor{darkgray}{ 
    door\\
	Marco Dirk Visser\\ 
	Geboren op 5 oktober 1982 \\
	te Springs, Zuid Afrika
	 }
	\clearpage	 
	\end{center}		 
 
   \fontsize{10}{12}\selectfont\par\noindent\textcolor{darkgray}{
	\textbf{Promotor:} \\
	Prof. dr. J.C.J.M. de Kroon \\ }
	\vspace{1.5cm}
  \fontsize{10}{12}\selectfont\par\noindent\textcolor{darkgray}{	
	\textbf{Copromotoren:} \\
	Dr. H.C. Muller-Landau (Smithsonian Tropical Research Institute, Verenigde Staten)\\
	Dr. ir. E. Jongejans  
	}
   \vspace{1.5cm}\
   \fontsize{10}{12}\selectfont\par\noindent\textcolor{darkgray}{
	\textbf{Manuscriptcommissie: } \\
	Prof. dr. M.A.J. Huijbregts	 \\
	Prof. dr.  S.W. Harpole (Helmholtz-Zentrum f\"ur Umweltforschung - UFZ, Duitsland) \\
	Prof. dr.  P. Zuidema (Wageningen University \& Research)\\
	 }	 
	\clearpage	
	 
	\begin{center}
	
  \fontsize{25}{30}\selectfont\par\noindent\textcolor{black}{\allcaps{\thanklesstitle}}%
  \vspace{.5cm}
  \fontsize{18}{20}\selectfont\par\noindent\textcolor{darkgray}{\allcaps{\@subtitle}}%
  \vspace{3cm}
  \fontsize{10}{12}\selectfont\par\noindent\textcolor{darkgray}{
	Doctoral Thesis \\
	to obtain the degree of doctor \\
	from Radboud University Nijmegen \\
	on the authority of the Rector Magnificus prof. dr. J.H.J.M. van Krieken \\
	according to the decision of the Council of Deans\\}
	\vspace{3cm}
   \fontsize{10}{12}\selectfont\par\noindent\textcolor{darkgray}{
	to be defended in public on Wednesday 16 November 2016 \\
	in the afternoon at 14.30 precisely \\}
	\vspace{3cm}
    \fontsize{10}{12}\selectfont\par\noindent\textcolor{darkgray}{ 
    by\\
	Marco Dirk Visser\\ 
	Born on October 5,  1982 \\
	in Springs, South Africa
	 }
   \end{center}

	\clearpage 
   \fontsize{10}{12}\selectfont\par\noindent\textcolor{darkgray}{
	\textbf{Promotor:} \\
	Prof. dr. J.C.J.M. de Kroon \\ }
	\vspace{1.5cm}
  \fontsize{10}{12}\selectfont\par\noindent\textcolor{darkgray}{	
	\textbf{Co-promotors:} \\
	Dr. H.C. Muller-Landau (Smithsonian Tropical Research Institute, United States) \\
	Dr. ir. E. Jongejans  
	}
   \vspace{1.5cm}\
   \fontsize{10}{12}\selectfont\par\noindent\textcolor{darkgray}{
	\textbf{Manuscript commission: } \\
	Prof. dr. M.A.J. Huijbregts	 \\
	Prof. dr.  S.W. Harpole (Helmholtz-Centre for Environmental Research - UFZ, Germany) \\
	Prof. dr.  P. Zuidema  (Wageningen University \& Research)\\
	 }	
    
  \end{fullwidth}
  }
  \thispagestyle{empty}%
  \clearpage%
  
}

\renewcommand{\thepage}{\arabic{chapter}.\arabic{page}}  
\renewcommand{\thesection}{\arabic{chapter}.\arabic{section}}   
\renewcommand{\thetable}{\arabic{chapter}.\arabic{table}}   
\renewcommand{\thefigure}{\arabic{chapter}.\arabic{figure}}

%----------------------------------------------------------------------------------------
%	BOOK META-INFORMATION
%----------------------------------------------------------------------------------------

 % Title of the book
\newcommand{\subtitle}[1]{
  \gdef\@subtitle{#1}
}


\title{Trade-offs, enemies \& dispersal } 
\subtitle{cross-scale comparisons on tropical tree populations}



\author[Marco D. Visser]{Marco D. Visser} % Author

\publisher{Radboud University Nijmegen} % Publisher

%----------------------------------------------------------------------------------------

\begin{document}

\frontmatter

%----------------------------------------------------------------------------------------
%	EPIGRAPH
%----------------------------------------------------------------------------------------
\thispagestyle{empty}
\clearpage
\vspace*{\fill}
\begin{center}
\begin{minipage}{.6\textwidth}
"There is no single, simplistic answer to the question, why are there so many species in the tropics? But there are answers." \\
\textbf{John Kricher (1997)}
\end{minipage}
\end{center}
\vfill % equivalent to \vspace{\fill}
\clearpage


%----------------------------------------------------------------------------------------

\maketitle % Print the title page

%----------------------------------------------------------------------------------------
%	COPYRIGHT PAGE
%----------------------------------------------------------------------------------------

\newpage
\begin{fullwidth}
~\vfill
\thispagestyle{empty}
\setlength{\parindent}{0pt}
\setlength{\parskip}{\baselineskip}
Copyright \copyright\ \the\year\ \thanklessauthor

\par\smallcaps{Published by \thanklesspublisher}

\par Supplemental material can be found at: \\ https://github.com/MarcoDVisser/thesis/tree/master/SupplementalMaterial

\par all Tex code is archived under https://github.com/MarcoDVisser/Thesis

\par Notice this document uses the Tufte-LaTeX style. Code was substantially modified from the original Tufte-LaTeX Developers. It is licensed under the GNU Public License, Version 2.0; It is open source and you may use and adapt the archived files in compliance with this license. \index{license}

\par\textit{Printed \monthyear} 
\par\textbf{ISBN}: 978-94-6233-453-3

\end{fullwidth}

%----------------------------------------------------------------------------------------

\tableofcontents % Print the table of contents

%----------------------------------------------------------------------------------------

%\listoffigures % Print a list of figures

%----------------------------------------------------------------------------------------

%\listoftables % Print a list of tables

%----------------------------------------------------------------------------------------
%	DEDICATION PAGE
%----------------------------------------------------------------------------------------

\cleardoublepage
~\vfill
\begin{doublespace}
\noindent\fontsize{18}{22}\selectfont\itshape
\nohyphenation
To those \textit{Rattus sp.} gastrointestinal microbes that tried to kill me, but failed.\\
\end{doublespace}
\vfill
\vfill


%----------------------------------------------------------------------------------------
%	INTRODUCTION
%----------------------------------------------------------------------------------------

\begin{landscape}
\begin{figure}
\vspace*{-.6cm}\hspace*{4.4cm}\fbox{\includegraphics[width=19.5cm,height=22cm]{../figures/illustrations/chapter1dark.png}}
\label{fig:chap1}
\hspace*{5cm}\begin{minipage}{18cm} 
\footnotesize \textit{"... especially the tropics, the plumb line is still being let out; we have no idea where it will end." - E.O. Wilson (1999) \nocite{Wilson1999}}
\small 
\end{minipage}
\end{figure}
\end{landscape}

\cleardoublepage

\mainmatter % begin section numbering

\chapter{An exercise in cross-scale integration} % The asterisk leaves out this chapter from the table of contents

\vline

\section{The question of persistence in diverse communities}
Tropical forests harbor astonishing numbers of tree species, so many that we do not know exactly how many exist. Recently, ecologists estimated a minimum number and we now believe tropical forests host at the very least 40,000-53,000 tree species \citep{Slik2015}. Direct field measurements allow for more certainty and show that even a single hectare\sidenote[][-2cm]{A single hectare is 10 000 m$^2$ or approximately half the size of the Dam square in Amsterdam, the full size of Trafalgar Square or roughly the area surrounding the bastions of the Statue of Liberty.} of tropical forest contains 473 tree species ($\geq$ 5 c.m. diameter, \citealt{Valencia1994})\sidenote{Measured in 1 ha of Amazonian lowland rain forest (Ecuador). The site also had a total diversity of 942 vascular plant species \citep{Balslev1998}, and this is the current record at this scale for any ecosystem.  Ecologists are confident, however, that this is not the maximum number in nature \citep[see e.g.][]{Wilson2012}.}.  With a total of only 280 tree species on the entire continent \citep{FAO2006}, tree diversity in Europe is much more modest in comparison. Yet, even a simple walk in the European woods yields the familiar sight of many tree and plant species growing in close proximity, all seemingly coexisting and relying on the same basic resources of light, water and nutrients \citep{Silvertown2004}. However, what if this assumed coexistence isn't quite as obvious as it seems? What if there were real reasons to believe the world could have been far less diverse? 

\begin{minipage}{14cm} 
\vspace*{1cm}
\large "Two species of approximately the same food habits are not likely to remain long evenly balanced in numbers in the same region. One will crowd out the other ... " 
\linebreak \textbf{\citealt{Grinnell1904}}
\end{minipage}
\vspace{1cm}

\begin{fullwidth} 
Observations such as these motivated Georgii F. Gause to design his doctoral dissertation on the examination of competition between unicellular yeast and protozoa species. In his now famous experiments, one species always dominated and excluded the other, with different species winning under different conditions \citep{Gause1934}. Gause's observations have since been repeated in many lab experiments (e.g. on flour beetles, \citealt{Park1948}; phytoplankton, \citealt{Titman1976, Tilman1977}). The observation that under a given set of conditions one species always excludes others is now referred to as the competitive exclusion principle or simply "Gause's law" \citep{Kingsland1991}. The observations by Grinnell and experiments by Gause show us that the world we are accustomed to is not the only way it could be. Yet, given that we may expect competition to limit diversity, why are tropical forests so diverse? 

\section{Coexistence theory}
Competition limits diversity when superior competitors increase in abundance to the exclusion of other species \citep{Gause1934}, unless there are forces internal to a community by which they are able to coexist \citep{Chesson2000, Holt2007}. Identifying these forces remains a central question in community ecology today \citep{Siepielski2010, Vellend2010, Comita2014}. Over the years, ecologists have proposed an abundance of hypotheses on mechanisms that drive community dynamics and influence plant diversity (\citealt{Palmer1994} lists over 100), but few enjoy strong empirical support \citep{Wright2002}. Current theory \citep{Chesson2000, Adler2007}, suggests that coexistence is the result of a mix of equalizing and stabilizing forces. Equalizing forces reduce average fitness differences between species, while stabilizing mechanisms increase negative intraspecific competition relative to negative interspecific competition in such a way that species tend to decline when abundant and recover when rare \citep{Chesson2000, Adler2007}.  In the absence of stabilizing mechanisms, average fitness differences among species predict the outcome of competitive exclusion \citep{Chesson2008}. Under this framework, competing species may coexist neutrally when they have equal fitness (e.g. equal birth and deaths  rates as in neutral theory; \citealt{Hubbell2001}), but neutral coexistence is transient without stabilizing mechanisms.  

Life-history trade-offs are expected to reduce average fitness differences among competing species \citep{Chesson2000, Bell2001, Hubbell2001}. Trade-offs are among the most important evolutionary constraints \citep{Stearns1992}, and arise due to finite resources or when adaptations towards one habitat limit survival in another \citep{Fabian2012}. Trade-offs therefore bound fitness to certain limits as advantageous selection towards one phenotypic trait will have compensating disadvantages in another \citep{Stearns1989}. Noteworthy examples among tropical trees include the number of offspring versus size of offspring \citep{Muller-Landau2008} and growth in light versus survival in dark conditions \citep{Wright2010}. Hence species may have strongly differing demographic rates (births and deaths) but due to trade-offs may vary little in mean fitness \citep[e.g.][]{Ostling2012}. The existence of trade-offs shows that species are functionally divergent, having evolved different strategies in response to the environment, and occupy different niches. 

Trade-offs need not act purely equalizing: whenever species evolve different strategies intraspecific relative to interspecific competition may intensify. Hence, trade-offs invoke negative density dependence and thus may also stabilize coexistence \citep{Chesson2000}. Negative density dependence (NDD) is the phenomenon that a species' ability to reproduce, disperse, grow (acquire resources) and/or survive (tolerate stress) diminishes as it becomes more common. Such a pattern may arise through different mechanism. Limited availability of a necessary commodity, for instance, will ensure that species with very similar requirements compete more strongly \citep{Tilman1982}. Also, as plant species become more specialized, their natural enemies are forced to adapt and specialize in response \citep[e.g.][]{Coley2014}. This causes a link between the populations of plants and enemies: as a plant species becomes more common, its natural enemies become more common and attack rates intensify \citep{Gillett1962, Johnson2012}. Critically, NDD must influence species' per capita population growth rates ($\lambda$) before it can act as a stabilizing mechanism. When NDD depresses $\lambda$ as species become common, and elevates $\lambda$ when species are rare - this is viewed as the signature of niche differentiation and potential stabilized coexistence \citep{Chesson2000, Adler2007}.


\section{Trade-offs, traits and the issue of the full life-cycle}
The existence of various trade-offs have been documented among tropical trees \citep[e.g.][]{Gilbert2006}. In addition, theoretical models show that phenotypic traits associated with these trade-offs (termed functional traits; box 1), can lead to equivalent carbon balances among trees \citep{Sterck2011}. The idea is that species traits are physical manifestations of ecological strategies, specializations that result from evolutionary trade-offs \citep{Marks2006, McGill2006}. These traits can then be used to predict differences in demographic rates among species \citep[e.g.][]{Adler2014}. Years of demographic monitoring  have shown that tree species differ dramatically in how they develop, the speed at which they grow and at what size they reach maturity, how many offspring of a particular size they produce, or how long they live \citep{Muller-Landau2008, Metcalf2009, Salguero-Gomez2016}. However, tests have shown that the power of functional traits to predict such demographic rates has been very low \citep[e.g.][]{Iida2014, Paine2015}, with variance explained typically being lower than 8\% (chapter 5). How important can trade-offs be in tropical forests when their associated traits are such weak predictors of species performance? \bigskip

\shadowbox{
\begin{minipage}{14cm}
\textbf{Box 1: Plant function and associated trade-offs.} Functional traits are defined as easily measurable, well-defined properties of organisms that directly influence or strongly correlate with ecological performance \citep{Wright2010}. Such measurable traits are hypothesized to be derived from patterns of tissue construction and energy allocation, and engender mechanical strength and rates of tissue turnover. Here we specifically look at seed mass ($g$), wood specific gravity (wood density; $gm.cm^{-3}$), Leaf mass per area (LMA: $g m^{-2}$) and maximum stature (species maximum height or diameter).  Each trait represents a key independent axis along which plants differentiate in strategies.  Seed mass reflects how species invest in the number vs. quality of future offspring: the more resources are packed within the seed the better the seedling will perform - but fewer large seeds can be produced \citep{Smith1974, Moles2004}. Wood density is associated with how much a species invests in structural strength and maintenance for a unit increase in size \citep{Chave2009}.  Low wood density infers greater strength at lower construction costs (i.e. trees reach larger diameters with lower investment), but low density wood has greater maintenance costs. High density wood is costly to construct for a given strength compared to low density wood, but is expected to have lower maintenance costs \citep{Larjavaara2010}. Hence, wood density predicts whether a species has an acquisitive or conservative resource gathering strategy, i.e. whether it is able to grow fast in the light (light wood) or survive well in the dark (dense wood; \citealt{Wright2010}; \citealt{Sterck2011}). Investments in leaf mass will increase leaf life span but also increase return times on leaf investments \citep{Wright2004}. Adult stature represents the trade-off between early reproduction and late reproduction. Any benefit of early reproduction is balanced by a shorter life-expectancy, while the copious seed production of late-mature, large-stature trees is limited as only few individuals reach mature sizes \citep{Kohyama1993}.
\end{minipage}
} \bigskip

Past tests have, however, focused on single life stages and ignored size and interrelatedness among traits and demography. Size is especially important to include in analyses as it often characterizes developmental stage. As an individual tree grows from a seedling onward, it increases leaf and root area and thus expands its ability to capture light and gain resources with size. With increasing size, maintenance and support costs of existing tissue also increase \citep{Givnish1988, Enquist2007} and individuals eventually start to reproduce \citep{Falster2003, Wright2005, Thomas2011}. Consequently growth or survival also change with size \citep{Muller-Landau2006}, as will the importance of different trade-offs and associated traits \citep[e.g.][]{Lasky2013}. Different traits will also influence organisms at different developmental stages. At the start of life, seed mass determines the resources available and initial recruit size. The proportion of resources devoted to construction or maintenance will vary with leaf mass and wood density as well as size. Finally, species stature (generally defined as maximum height) will in turn influence the size at which resources are allocated to reproduction. When viewed across the entire life-cycle, it is self-evident that multiple traits will influence any species ecological strategy, and hence the patterns of growth, survival and reproduction of trees across ontogeny \citep{Marks2006, Laughlin2015}. Failing to account for such developmental shifts with size, and the interactions between multiple traits, will limit the variation explained by traits. Moreover, fitness is the net integrated result of growth, survival and reproduction of each species across its entire life-cycle (\citealt{Caswell2001}; Fig. \ref{fig:chap7fig1}).  For these reasons, it does not suffice to show the effect of a trait on a single vital rate at a single life stage. Instead, the impact of a trait should be studied for all life cycle components, and in relation to other traits.  Efforts to identify species strategies (niches) and the hypothesized equalizing trade-offs, must take multiple traits and the full life-cycle into account. I have endeavored to do so in this thesis. 

 
As established above, natural selection operates on a suite of traits within populations of tree species. Thus, when one is interested in a specific trait, of relatively minor importance (e.g., operating on a minor life stage, or having a lower effect size), it becomes difficult to quantify the influence of this trait in the presence of variation in other potentially more important traits. Plant reproductive strategies, for example, take on a multitude of fascinating forms \citep[see e.g.][]{Visser2011}, but the effect of breeding system on seed production is relatively minor when compared to, for instance, the effect of seed size on seed production. The trade-off  between seed mass and seed number is well-known \citep{Moles2006, Muller-Landau2008}, and seed mass differs by orders of magnitude among species \citep{Muller-Landau2010} and is one of the main determinants of crop size. Plant breeding system also affects crop size \citep{Queenborough2009} but this effect is modest compared to the direct effect of seed mass.  Consequently, differences in seed production between species due to breeding system will be difficult to detect as species differ not only in breeding system but also in seed mass. In conclusion, only by controlling for the effects of important traits, like seed mass, can we focus on one particular aspect of the cloud of life history strategies such as breeding strategy. The potentially co-varying and opposing effects of traits thus need to be taken into account when comparative studies aim to focus on among-species variation caused by single traits like breeding system \citep[see also][]{Laughlin2015}. \end{fullwidth} 

\bigskip
\shadowbox{
\begin{minipage}{14cm}
\textbf{Box 2: The full life-cycle.} Change in population abundance is caused by performance of all individuals across all life stages, seeds must survive and germinate to become seedlings and seedlings must survive and grow long enough to become mature trees (Figure \ref{fig:fulllifecycle}).  Mathematically, life stages make highly unequal contributions to population growth (de Kroon \emph{et al.} 1986; Caswell 2001). This can be intuitively understood. The vast majority of seeds die before they establish as seedlings (Crawley 2000), and only very few seedlings survive the decades needed to reach reproductive sizes.  Once mature, however, a single tree can produce hundreds of millions of seeds during its life. The loss of any given seed or seedling hardly impacts the population of mature trees, as most die anyway, but the loss of a single mature tree impacts the number of seeds and seedlings instantly and substantially.
\end{minipage}
} 

\begin{marginfigure}[7cm]
\includegraphics[width=\linewidth]{../figures/lifecycle.pdf}
\caption[Simplified schematic representation of the life-cycle of a tree]{Simplified schematic representation of the life-cycle of a tree, depicting 4 distinct life stages. The thickness of arrows represent the relative number of individuals that transition towards each life-stage. Natural selection, enemies or competition may affect any stage, but the net consequences of these processes depend on how they influence transitions across all life-stages. }
\label{fig:fulllifecycle}
\end{marginfigure}

Dioecy is a plant breeding system where each individual has either male or female flowers. In the long term, this is thought to provide important advantages for plant species leading to less inbreeding depression and higher genetic diversity \citep{Bawa1980, Givnish1982, Barton2009}. Through loss of either male or female function, dioecious trees are also expected to have more resources available for growth, survival or reproduction. However, there is a trade-off, as it simultaneously confers a sizable demographic cost. Only female trees produce seeds, and therefore every additional male in the population will result in a drop in seed output compared to monoecious or hermaphroditic species where every individual reproduces \citep{Crawley1996}.  How can dioecious plants then coexist with non-dioecious plant, when hermaphroditism seemingly has such clear-cut demographic benefits? Are species with dioecious and hermaphroditic breeding strategies equalized in the sense that benefits from each strategy balance costs?  In this thesis, my collaborators and I control for species differences in wood density, seed mass and maximum size, and use a population model to integrate across the life-cycle and perform a cost-benefit analysis of dioecy at the population level. This means that the effects of dioecy on all parts of the life cycle are compared with the same fitness currency. 


\section{Density-dependence and the issue of scale}  
Negative density dependence appears to be pervasive among most life-forms \citep{Harms2000, Brook2006}.  However, NDD at some life stage may only contribute to stabilization and thus coexistence if it leads to NDD in population growth rates ($\lambda$), and to influence $\lambda$, negative density dependence must first affect per capita vital rates;  i.e growth, survival and/or reproduction.  Here the scale at which we measure vital rates is crucial.  For instance, seed predators may attack seeds where seed densities are high, usually close to an isolated fruiting tree \citep{Comita2014}, leading locally to apparent NDD.  Yet, on larger scales seed predators may become satiated with food, especially where fruiting trees are locally common, leading to positive density dependence of seed survival \citep[e.g.][]{Schupp1992}.  We need to study NDD at spatial scales at which local effects around individuals can be integrated to a population level: a scale allowing quantification of the relationship between conspecific density and per capita rates of growth, survival or reproduction. Further, any density-dependent effects must be integrated across the life-cycle. Any documented density-dependent mechanism operating at one life-stage constitutes weak evidence for population regulation as it can be counteracted by its inverse at another life stage\sidenote[][-10cm]{In long-lived organisms like trees, different life-stages and vital rates make highly unequal contributions to population growth rates ($\lambda$; \citealt{DeKroon1986}; \citealt{DeKroon2000}). Hence, the net impact of any documented mechanism on $\lambda$ not only depends on its absolute strength (i.e. effect size), but also on the importance of the life stage and vital rate upon which it acts (see also box 1).  A mechanism strongly affecting a life stage of minor importance may even be offset by weaker opposing effects at another more critical life stage \citep{Caswell1978}. For example, \citet{Zhu2015} recently quantified the strength of negative density-dependence across four tree life-stages. Seedlings and saplings showed strong negative responses to conspecific neighbors while adults showed modestly positive responses. Interestingly, empirical evidence  suggests that life-stage contributions to $\lambda$ should steeply increase with tree size \citep{DeKroon2000}.  A fundamental question arising from the work of \citet{Zhu2015} - given the pivotal importance of adult stages in long lived organisms \citep{Franco2004, Visser2011} - is whether the modest positive density-dependent effects on adults outweigh the sizable negative effects on seedlings and saplings. }  (see e.g. \citealt{Turchin1995}, Box 1).  NDD must influence the performance of species on a scale relevant for coexistence, the population scale.  Yet exactly here our knowledge is lacking.  Virtually all previous studies in tropical forests either focus on a local scale such as natural enemy attack close to a single adult tree \citep{Bagchi2010}, or on a single life stage and vital rate \citep{Harms2000, Comita2010, Mangan2010, Johnson2012, Bagchi2014}. Despite the wealth of insights these studies have provided, this is insufficient to establish whether such mechanisms influence per capita population growth rates ($\lambda$) 

\citet{DeKroon2016} issue another warning: using examples from aquatic systems they argue that deterministic factors at one scale of observation may translate in stochastic factors at another scale. \begin{fullwidth} Classic work from population ecology shows that density-dependence at one stage, depending on its magnitude, may even cause chaotic and destabilizing effects at the population scale \citep{May1976, Hassell1986}.  All these examples paint a clear picture, accurate estimation of net impact of any mechanism hypothesized to drive population dynamics - requires 1) a full life-cycle approach on 2) a scale relevant to the population. Both steps are taken in this thesis. 
\end{fullwidth}.  

\section{Ecological strategy \& the full life cycle in an era of global change}
Collectively tropical forests make up only 7\% of the earth's surface area but they contain approximately half of earth's terrestrial biodiversity \citep{May1992, MEA2005}, account for one third of the world's carbon stores (Lewis \emph{et al.} 2009; Wright 2013) and at least a third of the planet's net productivity \citep{Field1998, Roy2001}. The world is however changing, and tropical forests are no exception \citep{Malhi2014}. Recent work forewarns that biodiversity, carbon stores and productivity may all be altered by factors such as increased frequency of extreme droughts \citep{Bennett2015} or drastic increases in liana abundance \citep{Phillips2002, Schnitzer2011}. A challenge is to predict how the emergent properties of tropical forests such as biodiversity or carbon storage, will be altered by global anthropogenic change. In theory, the responses of any ecosystem to change will reflect the different strategies and properties of the organisms within them.  Large or small shifts may be expected, dependent on how functionally divergent the species assembly is, i.e. if they vary widely in their ecological strategies or if they are fairly similar \citep{Fonseca2001}\sidenote[][-11cm]{An emergent property of an ecosystem is for instance the carbon stored within its total biomass. How change will disrupt this property will depend on how individual species respond to this change. When fast-growing species with generally low wood density respond differently than slow-growing trees with high wood density, this will impact carbon stock. Depending on how future species composition changes, simulation studies show that carbon stocks may vary more than 600\% \citep{Bunker2005}.}.  However, without an understanding of the different ecological strategies within forests how can we hope to predict change in the emergent properties of tropical forests? 

Over the past decades, lianas\sidenote[][-6cm]{Lianas are long and slender stemmed, woody vines that root at ground level but use trees as a means of support, to climb vertically up to the canopy to get access to well-lit areas of the forest without investing in their own structural support \citep{Putz1984, Schnitzer2002}. Once in the canopy they deploy a layer of leaves above their host's canopy \citep{Avalos1999} intercepting light resources while also competing intensely with trees for water and nutrients \citep[reviewed in][]{Schnitzer2011}.  Lianas are considered to be structural parasites,  or macro parasites \citep{Stevens1987}, that have strong negative effects on tree growth and survival \citep{Ingwell2010, Wright2015}. } have shown dramatic increases in abundance across the Neotropics. On BCI, reported tree crown infestation rates have risen from 43-47\% in 1980  \citep{Putz1984} to 73.6\% in 2007 \citep{Ingwell2010}.  These increases are hypothesized to \begin{fullwidth} be the result of increasing $CO_2$ levels ($CO_2$ fertilization), decreasing precipitation \citep{Schnitzer2011} and/or increased disturbance and hunting \citep{Marvin2015, Wright2015}. Regardless of the direct cause of liana proliferation, liana infestation invariably has a negative impact on tree performance. This begs the question of how such documented increases in liana abundance will affect tropical tree communities. Any process that operates non-randomly with respect to species identity can be expected to contribute to the structure of natural communities \citep{Chase2003, Vellend2010}. Hence, a fundamental question is whether changes in liana abundance affects some tree species more severely than others. If so, it follows that any shifts in liana abundance should alter tropical forest community composition.  

An increased frequency of liana infestation is expected to affect multiple life stages of trees, influencing not only early life stages \citep{Schnitzer2000} but critically also the growth and reproduction of adult trees. The net impacts on the abundance of any given tree species will depend on the interplay between the size and direction of the impact at each life stage and the demographic importance of that life stage (box 2) \citep{Zuidema2001}. As mentioned earlier$^3$,  late life stages are considered overwhelmingly more important to fitness in long-lived organisms such as trees \citep{DeKroon2000, Franco2004, Visser2011}. Alterations in vital rates at these life stages, such as an increased mortality, should in theory result in the largest shifts in community composition within the shortest timeframe \citep[e.g.][]{Caswell2001}. Accordingly, quantification of the net effects of such changes across all life stages is urgently needed. 

This dissertation, takes a step in that direction by quantifying the full-life cycle effects of liana infestation on different tree species.  We evaluate whether the population-level impact of liana infestation differs among species, and whether these are systematically related to species functional traits or trade-offs. Finally, we estimate the most important mechanisms that regulate liana prevalence, or the proportion of infested individuals among different tree species.

\section{This thesis}
This thesis has one overarching theme, scaling up processes across spatial and organizational scales (from the individual towards the population, or across trophic levels).  In particular it takes a look at processes hypothesized to structure tropical forest communities, such as negative density dependence and trade-offs, at the scale of the population.  Across the 8 chapters presented, I will look at patterns over many hectares of forests and integrate several large-scale datasets across multiple life stages and vital rates. The central philosophy is that by moving up in scale, we are forced to abstract detail, and identify which details at the fine scale truly matter for the phenomenon at higher scale of organization \citep{Levin1992}. In the final chapter, I summarize what I believe each exercise in cross-scale examination reveals about which details are key to understanding the pattern at hand, and which can be swept under the rug of generality. 

In general, I use two broad approaches.  The first is a model system approach, looking at the palm \textit{Attalea butyracea}, and the second is a broad comparative approach across multiple species. Both approaches have their advantages and disadvantages, and offer unique insights into the processes that regulate populations and eventually structure communities within tropical forests.  

\vspace{1cm} \shadowbox{
\begin{minipage}{12.5cm}
\textbf{Box 3: The \textit{Attalea} model system.} \textit{Attalea butyracea} (Fig \ref{fig:attaleaplate}) is a common species in central and south America, even christened a "hyper-dominant" by virtue of being the 17th most common species in the Amazon \citep{Steege2013}. As such, it is a prime candidate to study the mechanism of population control by negative density dependence.  Two other traits of \textit{Attalea} make it even more uniquely suited for the study of density dependence and a highly suitable model system. First, it is easily identifiable from aerial pictures \citep{Jansen2008} , which allows us to control for adult palm density as a dependent variable and to study how its dynamics change across this gradient. Second, \textit{Attalea} seeds are protected by a stony protective layer (an endocarp), which is so tough that only a few specialized species are able to destroy its seeds \citep{Wright2001a}. In contrast to other tropical species, this greatly limits the species of possible seed predators, and allows us to study how seed predation changes in relation to increasing conspecific density. Moreover, seeds decay slowly, and each predator leaves a unique scarring pattern which enables crime-scene like investigation of the identity of killer of each seed, years after the event occurred (see chapter 2). 
\end{minipage}
}
\subsection{Model system}
We will never know everything about every species, but we may learn invaluably by studying the intricate details of a model organism.  Model systems have allowed the discovery of some of the deep secrets of life. The fruit fly \textit{Drosophila melanogaster} and the bacterium \textit{Escherichia coli} are famous for the abundant insights into the subject of genetics. Here, I will take a detailed look at the less renowned neotropical palm \textit{Attalea butyracea} (hereafter \textit{Attalea}).  \textit{Attalea} makes an ideal model organism because (amongst other reasons) the species is fairly common, easily recognized and has a persistent seed with limited set of seed predators (see box 3). My main objective is to study negative density dependence at the population scale, quantifying seed predation and dispersal rates for \textit{Attalea} populations at contrasting conspecific densities, and finally scaling up NDD across the life cycle.
  

\begin{figure*}
\includegraphics[width=12cm,height=15cm,keepaspectratio]{../figures/Attalea_plantillustrations_org.jpg}
\caption[Illustrations from Karsten's "Florae Columbiae"][-10cm]{Illustrations from Karsten's "Florae Columbiae" (\citealt{Karsten1858};  available under the  Creative Commons License). The plate shows a reproductive adult \textit{Attalea butyracea} palm in the background and in front of it, an immature rosette with leaves transitioning from the simple to the compound leaf phase.  Details of the inflorescences (left) and infructescences (right) are shown in the foreground. 
 }
\label{fig:attaleaplate}
\end{figure*}

 
\subsection{Multi-species comparison} 
At a community scale, a single species yields only a single sample and any insights gleaned may be unique to that species. My second approach, therefore, is to search for general organizing principles among species. Such an "among-species comparison" approach can be hugely insightful. Given that it was used by Darwin to compare finches in the Galapagos, one could even argue that this approach forms the basis of modern biology. Here, I compare over 100 tropical tree species, looking to identify and explain broader patterns among species. My goals include identifying and evaluating functional traits as a framework for predicting the full life-cycle dynamics of species with diverse ecological strategies;  using traits as a framework to make more robust inferences regarding coexistence of species with different reproductive strategies; estimating population level effects of liana infestation and whether traits work in predicting population level responses to stress induced by liana infestation; and disentangling the processes that control liana prevalence among tree species.

\subsection{Speeding up: the need for biology to learn from computer science} 
Computing has become fundamental to the practice of science \citep[e.g.][]{Michener2012, Wilson2012}, and the size and scope of ecological datasets and computations have exploded \citep{Bolker2013}.  In this data-rich age, biologists are collecting and integrating datasets at an unprecedented scale \citep{Petrovskii2012} which simultaneously enables more sophisticated analyses but also presents new computational challenges. This was no different for the majority of work presented in this thesis.  My research questions were invariably both aided and limited by computation. I was often forced to balance potential research questions within the limits of computational speed. This motivated my unplanned foray into the world of high performance programming. The result of this work is summarized in chapter 9, where we argue that programming can have a substantial impact on modern biological science. Chapter 9 reviews various techniques from computer science and shows that by learning how to program more efficiently, biologists will feasibly solve more complex tasks, ask more ambitious questions, and include more sophisticated analyses in their work.  In a time where the programming abilities of scientists are being challenged \citep[e.g.][]{Merali2010}, chapter 9 provides an accessible toolbox for any biologist. 

\subsection{Scaling up through the chapters}
Chapters 2-4 scale up in spatial scale, trophic level and towards the population level for our model species \textit{Attalea butyracea}. Chapter 2 documents the intricate interactions between a host palm and its vertebrate and insect seed predators in determining seed survival rates.  Chapter 3 discusses how palms compete for seed dispersers and how this can lead to complete failure of dispersal. Finally, chapter 4 integrates NDD in all vital rates across the life-cycle, asking whether there is NDD in per capita population growth rate at the population scale, an exercise that has rarely if ever been done before. Chapters 5-8 scale up across the life-cycle and trophic levels. In chapter 5, I study how well a combination of several important traits (specifically wood density, seed mass, leaf mass and adult size) predict the performance of tropical trees across their life cycle, from seeds to forest giants. Chapter 6 explores whether the loss of either male or female function confers benefits to dioecious trees, and whether any such benefits compensate for the loss of seed-producing individuals so as to equalize fitness with hermaphroditic species. In chapter 7, I seek to answer the crucial question if species respond differentially to change, by quantifying the species-specific impacts of liana infestation on tree growth, survival, reproduction and total tree population growth rates.  I also evaluate whether differences in tree species responses are associated with two key life-history axes: adult stature and shade-tolerance. Insofar as population level responses to change are strongly related to species traits, this provides evidence that species differ in their ecological strategies and niches.  In chapter 8, I use models from disease ecology in combination with long term data on liana infestation to explain variation in liana prevalence among tree species. Chapter 8 shows that understanding the potential of lianas to exert selective pressure on and among tree species is crucial to understanding both tree and liana species future and current abundance.  Chapter 9 crosses disciplinary boundaries, using insights from computer science to showcase the impact of efficient computing on modern biological science - without which many of the preceding chapters in this thesis would not have been feasible. Finally, I end by discussing how scaling up over the life cycle, community, trophic and spatial scales yielded insight into which details are essential to patterns at the scale of interest. 
    
\section{Data and study site}
All data were collected on Barro Colorado Island (BCI), Republic of Panama.  The island was formed when the Chagres river was dammed to form Gatun Lake during 1907-1913 \citep{Enders1935}. BCI is now a 15 km$^2$ km island lying in the central Gatun Lake section of the Panama Canal. Administered for research since 1923,  BCI is now arguably the best-studied tropical forest site in the world \citep{Leigh1999}.  Seven datasets enable this thesis. 

\begin{enumerate}
\subsection{Model system} 
\item Full life-cycle data on \textit{Attalea butyracea}.  Using aerial pictures of BCI to find the most dense and least dense \textit{Attalea} stands on BCI, we established 10 four ha plots spanning twenty-fold variation in adult density. Plots were initially established in 2008, and revisited in 2010 and 2012. All palms larger than 1.3 meters in height are were measured, and plots contain a total of 330 seed quadrats and 2.5 ha of mapped seedlings plots.

\subsection{Multi-species comparison}
My research in this thesis leans heavily on the foresight of others in the establishment of the BCI 50 ha forest dynamics plot (FDP) in 1980 (Condit 1995). Within the plot, six datasets enable the work done in chapters 5-8 of this thesis. 
\item The core FDP census identifies every free-standing woody plant greater than 1 cm in diameter to species, maps it to the nearest 0.5 m, and measures its diameter at breast height (DBH). Censuses took place in 1980-82, 1985, 1990, 1995, 2000, 2005 and 2010. These censuses tallied 468,000 individuals of 314 species. 
\item Seed production is quantified weekly since January 1987 using 250 seed traps located in a stratified random design along 2.7 km of pre-existing trails within the FDP \citep{Wright2005a}. Each trap has an effective surface area of 0.5 m$^2$ (125 m$^2$ total). Through May 2010, these censuses tallied 1,362,600 seeds and fruits of 509 species. Species number exceeds the FDP under 2 because these include all seed species including lianas, epiphytes, and herbaceous plants. 
\item  Establishment, growth and mortality of all woody seedlings and saplings under 1 cm DBH has been monitored annually since 1994 using 800 one-square-meter plots located near the seed traps. The census provides estimates of seed-to-seedling transition probabilities \citep{Harms2000a, Visser2016}. Through 2009, these censuses tallied 60,428 seedlings of 445 woody species.  
\item  Growth, mortality, and recruitment of seedlings and saplings greater than 20 cm in height has been monitored annually since 2001 in 20,000 one-square-meter plots located in a uniform design throughout the FDP. Through 2009, these censuses tallied 107,353 small saplings of 414 species \citep{Comita2010}. 
\item  Plant functional trait data, including seed mass (g), wood density (g/cm3), leaf mass per area (g/m2), and maximum adult height (m) have been measured for a sub sample of individuals belonging to virtually all tree species  found on the BCI plot (dataset 2). 
\item  Reproductive data. My collaborators and I added data on assessment of reproductive status and proportion of the canopy covered by lianas (liana load) for 21,157 trees of 131 species between March 2011 and October 2015. Supplemented with earlier such data for wind dispersed species \citep{Wright2005}, this census provides a crucial link between the individual-level responses required to model population dynamics and the population-level estimates of seed production provided by the ongoing seed trap census.
\end{enumerate}
\end{fullwidth} 


%----------------------------------------------------------------------------------------
%	CHAPTER 1
%----------------------------------------------------------------------------------------
\vspace*{20cm}

\begin{landscape}
\begin{figure}
\vspace*{-.6cm}\hspace*{4.4cm}\fbox{\includegraphics[width=19cm,height=22cm]{../figures/illustrations/chapter2.png}}
\hspace*{5cm}\begin{minipage}{18cm}
 \textit{ \footnotesize "Each consumer species opens \textit{Attalea} endocarps in a characteristic fashion, leaving species-specific openings and tooth marks" - Kirsten M. Silvius (2002) }
\end{minipage}
\label{fig:chap2}
\end{figure}
\end{landscape}

\chapter{Tri-trophic interactions affect density dependence of seed fate in a tropical forest palm}
\label{ch2} 

\marginnote[-1.7cm]{Marco D. Visser, Helene C. Muller-Landau, S. Joseph Wright, Gemma Rutten and Patrick A. Jansen. \textbf{Ecology Letters, (2011) 14: 1093-1100}. Supplementary material can be found online: http://tinyurl.com/zghs7t8}

\section{Abstract} 
\begin{fullwidth} 
Natural enemies, especially host-specific enemies, are hypothesised to facilitate the coexistence of plant species by disproportionately inflicting more damage at increasing host abundance. However, few studies have assessed such Janzen-Connell mechanisms on a scale relevant for coexistence and no study has evaluated potential top-down influences on the specialized pests. We quantified seed predation by specialist invertebrates and generalist vertebrates, as well as larval predation on these invertebrates, for the Neotropical palm \textit{Attalea butyracea} across ten 4-ha plots spanning 20-fold variation in palm density. As palm density increased, seed attack by bruchid beetles increased, whereas seed predation by rodents held constant. But because rodent predation on bruchid
larvae increased disproportionately with increasing palm density, bruchid emergence rates and total seed
predation by rodents and bruchids combined were both density-independent. Our results demonstrate that top down effects can limit the potential of host-specific insects to induce negative density-dependence in plant
populations.

\section{Introduction} 
Tropical forests are astonishingly rich in tree species, which in turn support an even more impressive diversity of animals and microbes. Ecologists have long sought to identify the mechanisms underlying
the coexistence of so many competing plant species \citep[reviewed in][]{Leigh2004, Pennisi2005}. One of the most widely supported explanations holds that specialised natural enemies play a key role in making each species' per capita population growth negatively density dependent \citep{Gillett1962, Janzen1970}. The idea is that individuals in a population of a given tree species will become more susceptible to attack by natural enemies, as they become more common, reducing individual growth, survival and / or reproduction \citep{Gillett1962, Carson2008}. Such effects are expected to be especially strong at early life stages, as seeds and seedlings are particularly vulnerable to attack \citep{Wright2002}.

\citet{Janzen1970} and \citet{Connell1971} independently postulated that higher densities of host-specific seed predators, pathogens and herbivores near adult plants would suppress recruitment near conspecific adults, thereby providing an advantage to other species. \citet{Janzen1970} recognised that this mechanism required population-level density dependence to facilitate coexistence: "The percentage of seed mortality on a parent tree should be inversely correlated with its distance to other fertile adults of the same species ... [and] ... the average seed mortality on these parents should be an inverse function of the density of reproducing adults". This reduction in per capita
success with increased density is crucial to most mechanisms of species coexistence \citep{Chesson2000}. Nonetheless, empirical tests of the Janzen-Connell hypothesis have focused almost exclusively on the actions of natural enemies close to conspecific adults, hereafter the local spatial scale (see reviews by \citealt{Clark1984, Hammond1998, Carson2008}. Very few studies have examined enemy influences at the larger spatial scales, necessary to evaluate relationships between per capita recruitment and the density of reproductive plants, hereafter the population-level spatial scale \citep{Schupp1992}.

Natural enemies differ in their potential to suppress plant recruitment at the local spatial scale near conspecific adults. Many insect herbivores are short-lived host specialists with high potential reproductive rates \citep{Janzen1970}. In contrast, herbivorous vertebrates are long-lived, feed on many plant species and have low reproductive rates relative to insects. In addition, the movements of many herbivorous vertebrates are restricted to circumscribed home ranges or territories. For these reasons, insects might respond more strongly than vertebrates to the proximity of adult trees and aggregations of seeds and seedlings. \citet{Hammond1998} review 46 studies that compare seed or seedling performance near and far from conspecific adults. Performance was lower near conspecific adults for 15 of 19 populations whose principal herbivore was an insect, but for just two of 27 populations whose principal herbivore was a vertebrate (P $< 10^{-6}$ , Fisher Exact Test). Thus, insect and vertebrate enemies apparently respond differently to the dense seed and seedling shadows near seed-bearing trees in tropical forests. Much less is known about natural enemy responses to population-level variation in the density of seed-bearing trees (\citealt{Schupp1992}, \citealt{Lewis2008}, see Discussion).

Higher trophic levels also influence interactions between herbivores and plants \citep{Terborgh2010}. It is well known that predators can indirectly alter plant species composition through their influences on herbivores \citep{Schmitz2006}. Such trophic cascades are particularly strong for endothermic predators of insect herbivores
\citep{Borer2005}. In tropical forests, in particular, insectivorous birds and bats have significant impacts on insect abundances and plant damage \citep{VanBael2005, Kalka2008}. Predators of insect pests that focus their efforts where prey are more abundant \citep[e.g.][]{Oksanen1981} may especially limit the ability of insect
pests to control host-plant populations. Furthermore, vertebrate seed predators that consume the larvae of insect seed predators serve as intraguild predators \citep{Silvius2002}, and theoretical work has shown that intraguild predation can be particularly important in structuring communities \citep{Holt1997}. Insectivores might thus influence the role that these insect pests play in promoting plant coexistence. Herein, we quantified the responses of generalist vertebrate and specialised insect seed predators to population-level variation in density of the palm \textit{Attalea butyracea} in a diverse moist tropical forest. Previous studies at the local scale show that the proportion of seeds that escaped bruchids and rodents increased with distance from the nearest seed-bearing \textit{Attalea} \citep{Wright1983, Wright2001a}.

Herein, we evaluated levels of (1) seed predation by a host-specific insect and (2) seed predation by generalist vertebrates, (3) predation of insect larvae by vertebrates, and (4) the net effect on seed survival over a 20-fold range in population density of the host palm. We found important top-down influences that were stronger at higher
host plant abundances, and that prevented the insect seed predator from increasing disproportionately with host-plant density, strongly limiting the potential to induce negative density-dependent seed survival at the population scale. Our study highlights the importance of considering feedbacks from higher trophic levels and population-level variation when evaluating natural enemy influences on plant communities.

\section{Methods} 

\subsection{Study site and species}
Barro Colorado Island (BCI) is a 1560-ha island in Lake Gatun, Panama (9$^{\circ}$9'N, 79$^{\circ}$51'W). Annual rainfall averages 2600 mm, there is a distinct 4-month dry season, and the vegetation is tall semi-deciduous forest \citep{Leigh1999}. The arborescent palm \textit{Attalea butyracea }(Mutis ex L.f.) Wess.Boer is monoecious, grows to heights of 30 m and is abundant in central Panama \citep{Foster1982, DeSteven1987}. Canopy trees produce one to three infructescences with 100-600 large fruits (3-5 cm) each year \citep{Wright1983}. Fruits have a tough exocarp, soft mesocarp and a single hard "stony" endocarp. Most endocarps enclose one seed, but 2\% contain two or three seeds \citep{Wright1983}.The endocarp remains on the forest floor for more than 3 years as it slowly decomposes \citep{Wright1983}.

Two bruchid beetles, \textit{Speciomerus giganteus }(Chevrolat) and \textit{Pachymerus cardo} (Fahraeus) prey on \textit{Attalea} seeds. Both species require \textit{Attalea }seeds to complete their life cycles and have no other local host species (S. Pinzon, S. Gripenberg \& O.T. Lewis, unpublished data). Both species lay eggs on \textit{Attalea} fruit; larvae then drill through the endocarp and develop inside, consuming the endosperm (Wright
1983). Adults emerge at the onset of the following wet season (S.J. Wright, personal observation). An analysis of exit holes, which are substantially larger for the bigger \textit{Speciomerus}, indicates that \textit{Speciomerus} comprises more than 95\% of the bruchids emerging from \textit{Attalea} endocarps on BCI (S.J. Wright and K. Silvius, unpublished data). We have reared several thousand bruchids from \textit{Attalea }endocarps in central Panama and have yet to encounter a parasitoid (S.J. Wright, unpublished data).

Many mammal species consume the soft mesocarp of \textit{Attalea} fruits; however, just three rodents can open the endocarps to access the seeds \citep{Wright2001a}. The three rodents are red-tailed squirrel (\textit{Sciurus granatensis}), Central American agouti (\textit{Dasyprocta punctata}) and Central American spiny rat (\textit{Proechimys semispinosus}). Endocarps with the characteristic scars left by the small teeth of spiny rats were not encountered in this study, and the spiny rat is not considered further. Agoutis and squirrels are generalist seed predators, known to feed on a wide variety of seeds and fruits, and both are important seed predators of \textit{Attalea} on BCI \citep{Heaney1978, Forget1994}. The white-lipped peccary, which once inhabited BCI, cannot open \textit{Attalea} endocarps \citep{Kiltie1982}. 

Mammal abundances on BCI are comparable to other Neotropical sites \citep{Wright1994}. Squirrel abundance is somewhat lower than at most Neotropical sites, and agouti abundances are at the 75th percentile of other Neotropical sites \citep{Wright1994}. Large mammalian predators are also present. BCI supports 25-30 Ocelots (\textit{Leopardus pardalis}), and Jaguar (\textit{Panthera onca}) and Puma (\textit{Puma concolor}) are regular visitors (J. Giacalone and P.A. Jansen, unpublished camera-trap data). Agoutis comprise almost 20 and 15\% of the diets of ocelots
and puma on BCI, respectively, and squirrels are a minor diet item for both cats \citep{Moreno2006}.

\subsection{Plot selection}
We selected locations for 10 square 4-ha plots using BCI-wide maps of canopy \textit{Attalea} (C.X. Garzon-Lopez \emph{et al.}, unpublished data) developed from high-resolution aerial photographs \citep{Jansen2008}. These maps allow identification of a wide range of adult \textit{Attalea} densities including the most dense stands on BCI. We chose
4-ha plots because the seed predators function at this scale. Agouti and squirrel home ranges on BCI are typically 1.34-2.45 ha \citep{Aliaga-Rossel2008} and 0.83-2.15 ha \citep{Heaney1978}, respectively. Bruchids oviposit on \textit{Attalea} endocarps at similar high levels within 16 m of fruiting palms and at just 16\% of this level at
100 m from fruiting palms \citep{Wright1983}. Each plot was laid out to have a reproductive \textit{Attalea} at its centre, and the ten plots included a wide range of adult \textit{Attalea} densities. To minimise confounding
factors, all 10 plots were located in a secondary forest, avoiding streams, steep slopes ($>$ 30\%) and lake edge ($>$ 200 m from the lake). \textit{Attalea} reaches its peak abundance on BCI in secondary forests that are about 130 years old and cover roughly half of BCI \citep{Svenning2004}.

\subsection{Adult census}
We mapped the position of every \textit{Attalea} palm with a bole height (ground to lower crown) greater than 1.3 m for the ten 4-ha plots between October and December 2007, using a precision compass (Suunto KB-14 precision, Vantaa, Finland) and an ultra-sonic rangefinder (Hagloff DME-201 cruiser, Langsele, Sweden). The presence of infructescenses and inflorescences (these remain on palms for a year) identified reproductive individuals \citep{Wright1983}. Reproductive \textit{Attalea} density ranged from 1.25 to 23.25 individuals per hectare (Table S1 Supporting information).

\subsection{Endocarp census}
We investigated seed fate for endocarps collected from the forest floor and topsoil between January and August 2008. We collected endocarps from thirty-two 1-m$^2$ quadrats for each 4-ha plot (320 quadrats total). We located quadrats in a stratified random manner, with two quadrats in each of the sixteen 25 $\times$ 25 m subplots in the central hectare of each 4-ha plot. If a rock, tree or debris covered a randomly selected point, the quadrat was placed as close as possible
to the randomly selected point in a randomly generated direction. We collected endocarps from the surface to a depth of c. 5 cm using a small rake. This depth ensures that scatter-hoarded endocarps were recovered; rodents typically hoard \textit{Attalea} endocarps at depths of 2-4 cm on BCI (Smythe 1989; P.A. Jansen, personal observation). Endocarps that could be crushed by hand were excluded because decomposition could obscure bruchid and rodent scars \citep[\textit{sensu}][]{Wright2001a}. The remaining endocarps were up to 3 years old. 

\subsection{Seed fates}
Bruchids and rodents leave species-specific scars on \textit{Attalea} endocarps that can be distinguished by their size, location and shape \citep{Silvius2002}. Bruchid larvae leave pin-sized entrance holes when they initially drill into the endocarp. Adult bruchids leave large circular emergence holes [diameter 6.6 $\pm$ 0.86 mm (mean $\pm$ SD) for \textit{Speciomerus}]. Agoutis hold endocarps horizontally and use their lower jaws \citep{Silvius2002} to
gnaw holes in the middle of the endocarp, and leave broad tooth marks. Red-tailed squirrels leave long gashed tooth marks and triangular openings at one end of the endocarp. We opened unscarred endocarps with a vice to determine whether seeds were viable. 

Rodents make large openings in \textit{Attalea} endocarps to extract seeds and significantly smaller openings to extract bruchid larvae \citep{Galvez2007}. We used this difference to identify rodent predation of larvae, based on the size of the openings made by rodents. The size of the openings assigned to larval predation (mean $\pm$ SD; 39.4 $\pm$
20.5 mm$^2$ , n = 26) and seed extraction (115 $\pm$ 51.6 mm$^2$ , n = 69) closely matched the values for larval predation (38.1 $\pm$ 17.5 mm$^2$, n = 12) and seed extraction (116 $\pm$ 31.8 mm$^2$, n = 10) reported by \citep{Galvez2007}.

We distinguished eight seed fates. When a bruchid larva enters an \textit{Attalea} endocarp, this can lead to the successful development of a bruchid (BB), the consumption of the larva by an agouti (BA) or a
squirrel (BS) or the death of the larva inside the endocarp for unknown reasons (BX). Fate BB included both endocarps with live larvae and endocarps with emergence holes left by an adult beetle. For endocarps that escaped bruchids, the seed may be eaten by agoutis (A) or squirrels (S), may escape predation and die for unknown reasons (X) or remain viable (V). We pooled endocarps collected from the 32 quadrats for each 4-ha plot, and calculated the proportion of
endocarps for each of the eight observed seed fates, so that BB + BA + BS + BX + A + S + X + V = 1 for each 4-ha plot.

\subsection{Analyses}
We used logistic regression with binomial errors to evaluate relationships between palm density and 14 indices of seed fate. Model slope values ($\hat{a_1}$) were tested against zero with the z statistic (normal approximation of the binomial). Model fits were compared with null models containing only the intercept ($\hat{a_0}$, indicating no
relationship with palm density) using likelihood ratio tests (Table S2). We used a Bonferroni correction across our 14 analyses. To maintain a family wise error rate of $\alpha$ = 0.05, we considered an effect significant
if P $\leq$ 0.0036.

Eight of the indices of seed fate were simple sums of the proportion of endocarps over subsets of the eight observed seed fates. This included total seed predation (A + S + BA + BS + BB + BX), total rodent attack (A + S + BA + BS), total bruchid attack (BA + BS + BB + BX), agouti attack (A + BA), squirrel attack (S + BS), seed predation by rodents (A + S), larval predation by rodents (BA + BS) and seed predation by bruchids (BB). As these eight indices of seed
fate are proportions, significant slopes indicate disproportionate changes in seed fate with palm density.

The six remaining indices of seed fate provide additional information concerning the rodent's preference for larval infested and uninfested endocarps. Three indices concern the proportion of endocarps attacked by bruchids that were also attacked by agoutis, squirrels or rodents [BA / (BA + BS + BB + BX), BS / (BA + BS + BB + BX) or (BA + BS) / (BA + BS + BB + BX) respectively]. The final three indices concern the proportion of endocarps that escaped bruchids, but were attacked by agoutis, squirrels and rodents [A / (A + S + X + V), S / (A + S + X + V) and (A + S) / (A + S +
X + V), respectively]. Significant slopes for these six indices indicate that rodents' preference for larval infested and uninfested endocarps changed with palm density.

\section{Results}

We determined the fates of 2154 \textit{Attalea} endocarps across 320 m$^2$ of samples. Palm density was a strong predictor of endocarp density among the 4-ha plots (Table S1; Spearman $\rho$ = 0.9, P = 0.00039). Of the recovered endocarps, 72.5\% had scars from rodent feeding, with many more scarred by squirrels (83.2\%) than by agoutis (16.9\%); 40.3\% had been attacked by bruchids; and 23.2\% were unscarred of which just 0.8\% were viable. Viable seeds all appeared to be from the most recent fruiting season. These percentages sum to more than 100\% because 89.3\% of the endocarps attacked by bruchids were also attacked by rodents providing evidence of larval predation. Just 4.3\% of endocarps had living bruchid larvae or exit holes indicative of successful bruchid emergence. The proportion of endocarps attacked by rodents (A + S + BA + BS) was independent of palm density (Fig. \ref{fig:chap2fig1}a); however, agoutis and squirrels showed distinct responses to palm density. The proportion of endocarps attacked by agoutis (A + BA) decreased with palm density (Fig. \ref{fig:chap2fig1}b; $\hat{a_1} $  = 0.055, P < 0.001), whereas the proportion attacked by squirrels (S + BS) increased (Fig. \ref{fig:chap2fig1}c; $\hat{a_1} $  = 0.025, P < 0.001).

\begin{landscape}
 \begin{figure*}
\hspace*{4.5cm}\includegraphics[width=15cm,height=18cm]{../figures/Chap2Fig1.png}
\caption[Observed seed fates][-15cm]{Observed seed fates (open circles with area proportional to sample size) and fitted logistic regression models of seed fate (black solid lines) as a function of adult palm density in \textit{Attalea butyracea}, with 99\% model confidence intervals (grey dashed lines). Seed fates are grouped as (a) total rodent attack (A + S + BA + BS, see methods for definitions), (b) agouti attack (A + BA), (c) squirrel attack (S + BS), (d) bruchid attack (BA + BB + BS + BX), (e) bruchid emergence (BB; note the different y-axis scale in this panel), (f) total seed predation (A + S + BA + BS + BB + BX), (g) bruchid larval predation by rodents (BA + BS) and (h) seed predation by rodents (A + S). The letters 'n.s.' in the bottom right of panels (a, e and f) indicate that the fitted model slopes are not significantly different from zero in these cases. The final panel (i) shows overall seed fate as a function of adult palm density. Attack of the same seeds by both bruchids and rodents results in larval predation by rodents and seed death.
 }
\label{fig:chap2fig1}
\end{figure*}
\end{landscape}

The proportion of endocarps attacked by bruchids (BA + BS + BB + BX) also increased with palm density (Fig. \ref{fig:chap2fig1}d; $\hat{a_1} $  = 0.039, P < 0.001). However, the proportion of endocarps with viable bruchids (BB) was unrelated to palm density (Fig. \ref{fig:chap2fig1}e, note the unique vertical scale; $\hat{a_1} $  =  0.018, P = 0.196). Total seed predation by rodents and bruchids combined (A + S + BA + BS + BB + BX) was also unrelated to palm density (Fig \ref{fig:chap2fig1}f, $\hat{a_1} $  = 0.0057, P = 0.411). Given that the proportion of endocarps attacked by rodents was similar for all palm densities (Fig. \ref{fig:chap2fig1}a) and the proportion of endocarps attacked by bruchids increased with palm density (Fig.  \ref{fig:chap2fig1}d), it is surprising that the total seed predation by rodents and bruchids was similar for all palm densities (Fig. \ref{fig:chap2fig1}f). This discrepancy reflects an increase in larval predation by rodents (BA + BS) with palm density(Fig. \ref{fig:chap2fig1}g; $\hat{a_1} $ = 0.045, P < 0.001) and a decrease in seed predation by rodents (A + S) with palm density (Fig.  \ref{fig:chap2fig1}h; $\hat{a_1} $ = 0.044, P < 0.001).

To determine whether the increase in larval predation by rodents with palm density (Fig. \ref{fig:chap2fig1}g) is a passive consequence of the increase in bruchid attack with palm density (Fig. \ref{fig:chap2fig1}d), we examined the proportion of endocarps attacked by bruchids that were also attacked by rodents [(BA + BS) / (BA + BS + BB + BX)]. This last proportion increased with palm density (Fig. \ref{fig:chap2fig2}a, $\hat{a_1} $ = 0.048, P = 0.0012). Squirrels attacked endocarps containing larvae [BS / (BA + BS + BB + BX)] at more than twice the level they attacked uninfested endocarps [S / (A + S + X + V)] for all palm densities (Fig. \ref{fig:chap2fig2}e,f), and were entirely responsible for the increase in the proportion of bruchid attacked endocarps that were also attacked by rodents (Fig. \ref{fig:chap2fig2}e, $\hat{a_1} $ = 0.042, P = 0.003). In contrast, agoutis attacked a similar and very low proportion of bruchid infested endocarps [BA / (BA + BS + BB + BX)] at all palm densities (Fig. \ref{fig:chap2fig2}c, $\hat{a_1} $ = 0.017, P = 0.71) and attacked uninfested endocarps [A / (A + S + X + V)] at more than twice this level for all palm densities (Fig. \ref{fig:chap2fig2}c,d).


\begin{landscape}
 \begin{figure*}
\hspace*{5cm} \includegraphics[width=15cm,height=19cm,keepaspectratio]{../figures/Chap2Fig2.png}
\caption[Rodent attack rates][-13cm]{Rodent attack rates on seeds with (a, c and e) and without (b, d and f) bruchid larvae; the former leads to larval predation by rodents, whereas the latter leads to seed predation by rodents. Results are shown for all rodents combined (a and b), just agoutis (c and d) and just squirrels (e and f). Open circles represent plots, with circle area proportional to sample size. Fitted logistic regression models are shown as black solid lines, with 99\% confidence intervals shown as dashed lines. The letters 'n.s.' in the top right of panels b and c indicate that the fitted model slopes are not significantly different from zero.
 }
\label{fig:chap2fig2}
\end{figure*}
\end{landscape}

To determine whether the decline in seed predation by rodents with palm density (Fig. \ref{fig:chap2fig1}h) is a passive consequence of the increase in bruchid attack with palm density (Fig. \ref{fig:chap2fig1}d), which leaves a smaller proportion of uninfested endocarps to be attacked, we examined the proportion of the endocarps that escaped bruchids and were attacked by agoutis [A /(A + S + X + V)] (Fig. \ref{fig:chap2fig2}d) or by squirrels [S / (A + S + X + V)] (Fig. \ref{fig:chap2fig2}f). Both proportions decreased significantly with palm density ($\hat{a_1} $ = 0.059, P < 0.001 and $\hat{a_1} $ = 0.019, P = 0.003 for agoutis and squirrels respectively).This decline was only marginally significant with both rodents combined [(A + S) / (A + S + X + V)] (Fig. \ref{fig:chap2fig2}b, $\hat{a_1} $ = 0.012, P = 0.047).

\section{Discussion}
We found that the density dependence of seed predation at the population scale differed from what was expected based on distance-dependence at the local scale, and that trophic interactions were critical to explaining the population-level patterns. The proportion of seeds attacked by bruchid beetles increased with palm density, whereas the proportion attacked by rodents held steady, as expected based on their degree of specialisation and responses to local variation in seed density around individual fruiting trees \citep{Hammond1998}. However unexpectedly, rodent predation of bruchid larvae increased disproportionately with increasing palm density, completely negating the increase in bruchid infestation, and causing bruchid emergence rates to be very low and density-independent (with a trend towards a decrease with palm density). These results demonstrate that top-down influences can limit the potential of host-specific insects to regulate the populations of their host plants.

\subsection{Variation among natural enemies}
The two vertebrate seed predator species exhibited contrasting responses to population-level variation in seed densities: agouties were satiated at higher seed densities, whereas squirrels exhibited increasing attack rates. These different responses are consistent with the different roles of \textit{Attalea} seeds in their diet. \textit{Attalea} seeds (or the larvae inside) comprise more than half of the diet of squirrels \citep{Heaney1978}, whereas agoutis clearly prefer seeds of other species, particularly the palm \textit{Astrocaryum standleyanum} and the legume
\textit{Dipteryx panamensis} (\citealt{Smythe1978}; P.A. Jansen, unpublished data). The focus of squirrels on \textit{Attalea} essentially makes them behave more like specialist than generalist predators. More generally, the role of a particular plant species in the diet of a generalist seed predator may be expected to affect how that generalist responds to spatial variation in the plant's abundance, and thus whether its seed predation is positively or negatively density dependent. Social systems and specifically territoriality may also influence vertebrate responses. Both squirrels and agoutis have similar small home range sizes on BCI (see Methods: Plot selection), and would be expected to experience similar associated limits on their ability to respond to population-level variation in tree density.

The observed increase in the proportion of bruchid attack with adult palm density is consistent with the expectation that specialist insects have the potential to respond disproportionately to host density \citep{Hammond1998, Lewis2008}. However, the increased attack cannot be explained by a disproportionate numerical response as adult beetle emergence was proportional with palm density. The increased attack rates may reflect increased oviposition efficiency. Increased seed densities in plots with high adult palm density should improve the ability of female beetles to locate seeds \citep{Wilson1972}. Furthermore, where a tree species is more abundant, seed dispersers may become satiated \citep{Hampe2008}, which might cause seed removal and dispersal to decline \citep{Forget2007,Klinger2009} and expose more seeds for longer periods to bruchid attack. Therefore, increased per-beetle oviposition efficiency, a functional response caused by greater seed availability and exposure time, likely combined with a proportional numerical response in adult beetle densities lead to increased infestation rates.

\subsection{Tri-trophic interactions}
Though bruchids attacked seeds at an increasing rate at higher densities of their host palm, they also suffered a compensatory increase in larval predation by rodents, and thus adult emergence rates did not change with palm density. This increase in intraguild predation of bruchid larvae has a passive and an active component. The passive
component arises because the proportion of endocarps attacked by bruchids increased with palm density, thus increasing the opportunity to attack larvae (compare Fig. \ref{fig:chap2fig1}d, g). The active component arises because squirrels, which prefer infested seeds (compare Fig. \ref{fig:chap2fig2}e, f), exhibited an increase in preference for infested seeds as palm density increased (Fig. \ref{fig:chap2fig2}e). Had rodents attacked seeds with larvae in the same proportion they attacked seeds without larvae, bruchid emergence, seed predation by bruchids and total seed predation would all have increased with increasing palm density. Together, satiation of agoutis (Fig. \ref{fig:chap2fig1}b) and increasing intraguild predation of bruchid larvae by squirrels at high palm densities (Fig. \ref{fig:chap2fig2}e) combine to neutralise increases in the proportion of endocarps attacked by bruchids and squirrels at high palm densities, so that overall levels of seed predation are independent of density (Fig. \ref{fig:chap2fig2}f).

This unexpectedly strong intraguild predation of bruchid larvae raises the question of how common such top-down control of pests \citep[\textit{sensu}][]{Gillett1962} might be in tropical forests. Insect larval predation
on seeds is high in many tropical plant species \citep[e.g.][]{Herrera1989}. Bruchid predation on palm seeds is particularly well-documented \citep[e.g.][]{Delobel1995}. Vertebrates often consume seeds of species that have insect larval predators, and commonly feed on infested fruit \citep{Silvius2002, Galvez2007}. Silvius (2002) argued that
vertebrate granivores feeding on palm nuts should prefer larvae to seeds, as they require less effort to extract and are likely to be more nutritious than seeds. We found that squirrels indeed consumed a higher proportion of infested than uninfested fruit, consistent with such preference. This finding suggests abundant potential for intraguild predation and associated top-down control of insect seed predators, whenever vertebrates and insects prey on the same seeds.


\subsection{Comparing density dependence at local and population scales}
The interaction among \textit{Attalea} seeds and their predators illustrates the problem posed by the spatial scale used to evaluate the Janzen-Connell hypothesis \citep{Schupp1992}. In earlier studies of the \textit{Attalea}-bruchid-rodent interaction, the proportion of seeds that escaped bruchids and rodents and the number of seedlings recruited per seed all increased with distance from the nearest seed-bearing \textit{Attalea} \citep{Wright1983, Wright2001a}. The conditions emphasised by \citet{Janzen1970} and \citet{Connell1971} are present at the local spatial scale. Nonetheless, seed predation is independent of variation in conspecific density at the larger spatial scale used in this study (Fig. \ref{fig:chap2fig1}f). The two mechanisms that neutralise population-level density dependence of
seed survival in our system are unlikely to operate at the small spatial scales around single fruiting trees. Agoutis are unlikely to be satiated by variation in endocarp density around single fruit trees because extra seeds are avidly scatter hoarded for future use \citep{Forget1994}. The preference of squirrels for bruchid infested seeds is also unlikely to vary spatially around individual \textit{Attalea} trees because most \textit{Attalea} fruit are dispersed away from fruiting trees on BCI \citep{Wright2000, Wright2001a}. Thus, the mechanisms that neutralise population-level density dependence of seed survival are unlikely to operate at the small spatial scales around single fruiting trees, emphasised by the Janzen-Connell hypothesis. The result is a disconnect between the local spatial signature of the Janzen-Connell mechanism and population-level density dependence.

A few other studies have found contrasting patterns of density- and distance-dependence at local and population scales. \citet{Schupp1992} reports increasing seed survival with distance to the nearest adult conspecific at the local spatial scale and positive density dependence at the population-level scale for a second BCI plant species, \textit{Faramea occidentalis}. In both \textit{Faramea} and \textit{Attalea}, the local spatial signature of the Janzen-Connell mechanism is present, but the negative density dependence necessary to regulate populations is absent. Studies of the dipterocarp \textit{Shorea laxa} find a disconnect in the opposite direction: in this case, seed survival decreases with distance from the nearest seed-bearing tree \citep{Takeuchi2007} and, at larger spatial
scales, with the density of conspecific adults \citep{Takeuchi2010}. The local spatial signature of the Janzen-Connell mechanism is reversed, yet the negative density dependence necessary to regulate populations and promote coexistence is present at population-level scales. \citet{Norghauer2008} present a final example from Brazil in
which the local spatial signature of the Janzen-Connell mechanism and negative density dependence at the population-level are both present in \textit{Swietenia macrophylla}. A reasonable tentative conclusion is that the presence or absence of the spatial signature of the Janzen-Connell mechanism at local spatial scales provides no insight into the
presence or absence of negative density dependence and the facilitation of species coexistence at population-level scales.

New studies will be needed to explore the responses of natural enemies to population-level variation in the density of tropical trees, not only for seed predators but also for seedling herbivores, seedling pathogens, etc. Studies now document negative-density dependence in plant recruitment, growth and / or survival at appropriate spatial scales, but do not evaluate the impacts of natural enemies (or any other mechanism; reviewed by Wright 2002). \citet{Schupp1992} speculated that territoriality limited rodent responses to plant density, contributing to positive density-dependent seed escape. \citet{Takeuchi2010} and \citet{Chauvet2004} speculated that variation in
community-level seed production overwhelmed the responses of generalist seed predators to spatial variation in the density of individual tree species. The record of mortality agents recorded in scars on the stony endocarps of \textit{Attalea butyraceae} provides the first opportunity to evaluate responses of natural enemies to population-
level variation for a tropical tree. 


\subsection{Conclusions and implications}
Specialised natural enemies are widely believed to be one of the most important agents contributing to the maintenance of plant species diversity in tropical forests \citep{Wright2002, Leigh2004}. Attack by specialised pests is expected to increase disproportionately with increasing host plant abundance \citep{Gillett1962, Janzen1970, Connell1971, Hammond1998, Leigh2004, Lewis2008}, leading to negative density dependence of plant species vital rates and giving each species an advantage when rare, a crucial feature of stable species coexistence \citep{Chesson2000}. Classic papers by \citet{Janzen1970} and \citet{Connell1971} stimulated a huge volume of tropical research testing for negative density dependence in general and density dependence of enemy attack in particular on local spatial
scales (Wright 2002). However, in all this literature about the importance of natural enemies to plants, there is virtually no mention of the potential impact of the enemies of those enemies. That predators affect herbivores and thereby herbivory on plants, and that predators respond to herbivore abundance, has been abundantly demonstrated, in tropical forests as well as in many other systems \citep{Terborgh2010}. However, the potential for top-down
influences to interact with Janzen-Connell effects has been neglected. This study provides evidence that top-down influences can critically alter population-level patterns of density dependence, and must be considered. We also show that local scale studies are not a reliable basis for inferring population-level patterns.

A key question for future research concerns the generality and importance of this phenomenon: Do top-down influences often prevent natural enemies from exerting increasingly negative effects at increasing abundances of the focal plant species? A better understanding of which enemies are likely to exert negative density-dependent effects on plant vital rates could illuminate which interactions are most crucial for the maintenance of plant diversity. In general, we expect that top-down influences of predators and parasitoids may be important for many invertebrate enemies \citep{VanBael2005, Kalka2008}. Because pathogens, in contrast to insects, are rarely subject to top-down control from predators \citep{Agrios2005}, microbes may be a fundamentally more powerful force for inducing negative density dependence at the population scale than other natural enemies \citep{Mangan2010}. 

\end{fullwidth}

%------------------------------------------------
% Chapter 3
%--------------------------------------------------

\begin{landscape}
\begin{figure}
\vspace*{-.6cm}\hspace*{4.4cm}\fbox{\includegraphics[width=19cm,height=22cm]{../figures/illustrations/chapter3.png}}
\hspace*{5cm}\begin{minipage}{18cm} 
\textit{ \footnotesize "The evolutionary function of fruit is to advertise itself to some sort of animal so it will be eaten" - John Kricher (2011) \nocite{Kricher2011}}
\end{minipage}
\end{figure}
\end{landscape}

\chapter{Negative density dependence of seed dispersal and seedling recruitment in a Neotropical palm}
\label{ch3} 

\marginnote[-1cm]{Patrick A. Jansen, Marco D. Visser, S. Joseph Wright, Gemma Rutten and Helene C. Muller-Landau \textbf{Ecology Letters, (2014) 17: 1111-1120}. Supplementary material can be found online: http://tinyurl.com/zghs7t8}
\vspace*{.8cm}

\section{Abstract} 
\begin{fullwidth}
Negative density dependence (NDD) of recruitment is pervasive in tropical tree species. We tested the hypotheses that seed dispersal is NDD, due to intraspecific competition for dispersers, and that this contributes to NDD of recruitment. We compared dispersal in the palm \textit{Attalea butyracea} across a wide range of population density on Barro Colorado Island in Panama and assessed its consequences for seed distributions. We found that frugivore visitation, seed removal and dispersal distance all declined with population density of \textit{A. butyracea}, demonstrating NDD of seed dispersal due to competition for dispersers. Furthermore, as population density increased, the distances of seeds from the nearest adult decreased, conspecific seed crowding increased and seedling recruitment success decreased, all patterns expected under poorer dispersal. Unexpectedly, however, our analyses showed that NDD of dispersal did not contribute substantially to these changes in the quality of the seed distribution; patterns with population density were dominated by effects due solely to increasing adult and seed density.


\section{Introduction}
Conspecific neighbours often have more negative effects on individual plant performance (e.g. survival, growth and
recruitment) than do heterospecific neighbours, a phenomenon known as negative density dependence (NDD). NDD constitutes a strong stabilising force for population regulation and species coexistence (Chesson 2000). Many studies have found NDD of plant performance in diverse tropical forests \citep{Wright2002}. NDD can arise because attacks by natural enemies increase disproportionately with local conspecific density \citep{Janzen1970, Connell1971}, and/or because competition for resources is more intense between conspecifics than heterospecifics \citep[e.g.][]{Tilman1996}. In tropical forests, NDD is generally attributed to natural enemies \citep{Hammond1998, Terborgh2012}. However, it is also possible that NDD results, at least in part, from intraspecific competition for seed dispersers. 

\subsection{Seed dispersers as a limiting resource}
Seed dispersal is critical for colonising vacant sites, escaping natural enemies concentrated around parents and reducing kin competition \citep{Nathan2000}. Vertebrates are agents of seed dispersal for 70-100\% of tropical tree species \citep{Willson1989}, and may thus be a key limiting resource for the majority of tree species in most tropical forests \citep{Howe1977, Manasse1983}. Because disperser populations are limited by food availability
during the season of greatest food scarcity \citep{Leigh1999}, they are easily satiated in times of greater food availability, setting the stage for competition for dispersers \citep[e.g.][]{Wheelwright1985, Hampe2008}.

Plant species vary widely in the timing of fruit production and the composition of their disperser coteries. Thus, they should compete for dispersers more strongly with conspecifics, which have exactly the same disperser coteries and fruiting phenologies, than with heterospecifics \citep{Howe1977}. Therefore, intraspecific competition for dispersers should increase with a tree species' abundance \citep{Hampe2008}, and the rate at which fruits are consumed, and seeds removed and dispersed should decline, as should the dispersal distance. 

Empirical studies have shown that seed removal indeed depends positively on disperser abundance \citep[e.g.][]{Alcantara1997}, and negatively on seed abundance \citep[e.g.][]{Jansen2004} and tree density \citep[e.g.][]{Manasse1983,Beck2002}. This indicates that intraspecific competition for dispersers - sometimes referred to as disperser satiation - exists and can indeed lead to NDD of seed removal. Intraspecific competition can also reduce dispersal distance in years of higher seed abundance \citep{VanderWall2002, Jansen2004} and in areas with higher fruit availability \citep{Galvez2009, Klinger2009, Morales2012}. To date, however, no study has evaluated whether competition for an entire community of seed dispersers can cause NDD of dispersal, and whether and how this contributes to NDD of recruitment.

\subsection{Consequences for recruitment}
NDD of seed dispersal may translate to population-level NDD of recruitment because dispersal promotes recruitment
in at least three fundamentally different ways. First, dispersal reduces seed limitation, i.e. the failure of trees to establish recruits at potentially favourable sites because no seeds or insufficient seeds arrive there \citep{Nathan2000, Schupp2002}. Shorter dispersal implies that, all else equal, the available seeds end up in fewer sites. Second, dispersal allows offspring to escape the vicinity of the parent, a conspecific adult. Many studies have shown that offspring mortality rates are higher near conspecific adults (reviewed in \citealt{Hammond1998, Wright2002}), a pattern most often attributed to so-called distance-responsive natural enemies associated with adults \citep{Janzen1970, Connell1971}, such as host-specialised pathogens and arthropods \citep[e.g.][]{Augspurger1984, Alvarez-Loayza2011}. Shorter dispersal implies that more seeds end up near their parent. Finally, dispersal can decrease the spatial aggregation (clumping or crowding) of offspring by spreading seeds over a larger area. Establishment success and subsequent survival decreases with local conspecific seed density \citep[e.g.][]{Harms2000,
Comita2010, Bagchi2014}, a pattern generally attributed to density-responsive natural enemies \citep[\textit{sensu}][]{Janzen1970} such as host-specialised pathogens and arthropods \citep[e.g.][]{Augspurger1984, Bell2006, Bagchi2014}. Shorter dispersal implies that seeds, on average, experience higher local seed density.

Despite these important functions of dispersal, it is not evident that NDD of dispersal will translate to NDD of recruitment. Dispersal is inherently less likely to result in escape from conspecifics as density of conspecific adults and offspring increases. For example, increased dispersal limitation in dense populations will to some degree be offset by reduced source limitation because so many more seeds will be available to colonise sites. Also, even if seed dispersal is identical, the mean distance of dispersed seeds to the nearest adult will still decline with density because, on average, there is an adult closer to every point. The essential question is whether NDD
dispersal causes seed limitation, distance-to-adult and crowding to change faster or slower (in the case of seed limitation) than they would as passive consequences of changes in adult density.

\subsection{This study}
We tested the hypotheses that (1) competition for seed dispersers increases with population density, depressing seed
removal rates and dispersal distances, and that (2) reduced seed dispersal at higher population density significantly
increases seed limitation, proximity to adults and crowding beyond the changes due solely to increasing adult and seed density. Our approach was to compare seed dispersal of the vertebrate-dispersed palm species \textit{Attalea butyracea} among areas on Barro Colorado Island (BCI), Panama that ranged widely in adult density. We found that intraspecific competition for dispersers indeed caused NDD of seed dispersal. Unexpectedly, however, this did not affect NDD of seedling recruitment beyond effects solely due to increasing adult density.


\section{Methods}

\subsection{Study site and species}
Fieldwork was conducted in old secondary moist tropical forest (90-130 years old) on the south-eastern half of BCI
(9$^{\circ}$9'N, 79$^{\circ}$51'W), a 1560-ha island located in the Gatun Lake section of the Panama Canal \citep{Leigh1999}. 

\textit{Attalea butyracea} (Mutis ex L.f.) Wess. Boer (henceforth \textit{Attalea}) is a monoecious palm abundant in Central Panama (Wright 1990). Adults can reach heights of $\approx$30 m \citep{DeSteven1987}. They produce 1-3 pendulous infructescences annually, each containing 100-600 fruits of 3-5 cm. Fruits consist of a hard exocarp enclosing a sweet fleshy mesocarp and a stone (henceforth 'seed'), a hard endocarp that usually contains one seed. Endocarps persist on or in the soil for several years before decomposing \citep{Wright1983, Wright1990}. Fruits are consumed by many mammal species \citep[][and references therein]{Wright2001a}, but only Baird's tapir \textit{Tapirus bairdii} ingests the seeds.

On BCI, seed dispersal of \textit{Attalea} is primarily by two rodent species - the Central American agouti (\textit{Dasyprocta punctata}) and the Red-tailed squirrel (\textit{Sciurus granatensis}) - that carry seeds away in their mouth, one at a time, to scatter hoard \citep{Forget1994, Wright2001a}. Reciprocal theft and recaching of seeds can produce stepwise dispersal over distances >100 m \citep{Hirsch2012a, Jansen2012}. Agoutis and squirrels, along with the Central American spiny rat (\textit{Proechimys semispinosus}), are the only vertebrates reported to eat \textit{Attalea} seeds on BCI \citep{Forget1994a}. Two insect seed predators, the bruchid beetles \textit{Speciomerus giganteus} (Chevrolat) and \textit{Pachymerus cardo} (Fahraeus), also attack \textit{Attalea} seeds \citep{Wright1983}. 

\subsection{Forest plots}
We used two sets of plots varying in adult \textit{Attalea} density, all located in the same area of old secondary forest (Fig. S1). Seed removal rates and initial seed dispersal distances were quantified in six square 1-ha experimental plots established in 2005, with adult density ranging over 5-29 individuals ha$^{-1}$. Frugivore visitation, ultimate seed dispersal distance and seedling recruitment were quantified in ten square 4-ha plots established in 2008, with adult densities ranging over 1-25 individuals ha$^{-1}$ (see chapter 2). Plot centres were separated by at least 300 m (Fig. S1). In each plot, we mapped every palm with a bole height >1.3 m, and determined its reproductive status by the presence of infructescences and inflorescences. In the 4-ha plots, we also recorded bole height, a potential predictor of fecundity. 

\subsection{Tree visitation by dispersers}
Tree visitation by seed dispersers was estimated by deploying camera traps with passive infrared motion sensors (RC55, Reconyx, Inc. Holmen, WI, USA) below 1-3 fruiting \textit{Attalea} in each of nine 4-ha plots during July-August 2009 (21 palms in total; the tenth plot had no suitable fruiting \textit{Attalea} during the study period). Cameras were placed ca. 20 cm above the ground 3-4 m away from the focal palm facing the fallen fruits. Deployment duration averaged 8.4 days per tree, and 19.6 days per plot. We analysed the photographs and identified all mammal to species \citep{Kays2011}. For each species, visitation rate was calculated as the number of visits (separated by >3 min) divided by deployment duration.

\subsection{Seed removal and initial dispersal distance}
Seed removal rates and initial dispersal distances were measured in the six 1-ha plots during August-September 2005. To quantify seed removal, we placed three sets of 15 seeds - a mix of seeds with and without fruit flesh - below the crowns of three fruiting \textit{Attalea} in each plot (135 seeds per plot), and recorded how many remained after 1, 2, 4, 8, 16, 32, 64 and 128 days. We calculated time to removal, the complement of removal rate, as the geometric mean of the days until the census at which the seed was no longer present. Initial seed dispersal distance was quantified by tracking thread-tagged seeds \citep{Forget2005}. A 70-cm thread with 10 cm of pink flagging tape with a unique number written on it was attached to the seed via a tiny hole drilled though the woody tip of the endocarp. Thread tags enable retrieval of seeds after removal even if the seed is buried, as the pink flagging remains above
ground. We placed five sets of five tagged, de-fleshed seeds at the base of three fruiting individuals in each plot (75 seeds per plot), relocated all seeds 7 days after placement by searching the surrounding area for thread marks, measured the distance from the original location, fitted a log-normal distribution to the Kaplan-Meier function of distance \citep[after]{Jansen2004, Hirsch2012} and extracted the median distance.

\subsection{Ultimate seed dispersal distance}
Ultimate dispersal distances can be much larger than initial distances, due to reciprocal theft and recaching of seeds by rodents \citep{Jansen2012}. We used inverse modelling (IM) to estimate ultimate dispersal distance from the post-dispersal spatial distribution of empty and filled endocarps in the ten 4-ha plots. We sampled 33 1 $\times$ 1 m quadrats within the central 100 $\times$ 100 m of each plot during January-August 2008 \citep[see][chapter 2]{Visser2011a}. Two quadrats were placed at computer-generated random locations in each of the 16 subplots of 25 $\times$ 25 m, a 33rd quadrat was placed randomly beneath the central palm. The surface and top 5 cm of soil of each quadrat were thoroughly searched for endocarps using a small rake. We excluded fresh endocarps from the early fruiting season of 2008, as well as endocarps so old they could be crushed by hand, hence the endocarps we counted represent those produced during the fruiting seasons of 2007, 2006 and probably 2005.

We used standard IM methods \citep{Ribbens1994, Muller-Landau2008} to model the seed shadow of an individual tree as the product of its estimated seed production and its dispersal kernel, the two-dimensional probability distribution of seed displacement. The expected number of endocarps at each quadrat was the sum of expected contributions of all adults on the plot, and from adults outside the plot under the assumption that endocarp production per unit area off-plot equalled that on the plot \citep{Muller-Landau2008}. Observed endocarp numbers were assumed to follow a negative binomial distribution around expected values \citep{Clark1999}.

To test for the influence of palm density on seed dispersal distances, we first separately fitted endocarp dispersal kernels for each plot, and second fitted an overall model in which dispersal parameters varied with adult density among plots. We combined three fecundity models (constant, linear increase with height, asymptotic increase with height) with seven dispersal kernels (Erlang, 1- and 2-parameter2Dt, 1- and 2-parameter Weibull and 1- and 2-parameter lognormal; see Table S1)\citep{Ribbens1994, Clark1999, Klein2006, Jongejans2008}. In the overall model, the scale parameter of the dispersal kernels was allowed to vary among plots as a function of adult density (Table S1). We tested linear and exponential dependence on adult density against a constant model (no dependence). For each model, we searched for the maximum likelihood parameter estimates using the Nelder-Mead downhill simplex method \citep{Nelder1965}. We used Akaike's information criterion (AIC) for model selection, and calculated median dispersal distances for the best-fitting model.

\subsection{Seedling recruitment}
To assess recruitment success, we mapped all \textit{Attalea} seedlings across 2500 m$^2$ in each of the ten 4-ha plots during January-March 2008. We subdivided the central ha of each plot into 16 subplots of 25 $\times$ 25 m, and randomly selected one of the four inner subplots and three of the 12 outer subplots. We mapped all offspring in those subplots and tagged them with numbered vinyl loop tags. Seedlings were defined as individuals that had simple leaves only (as opposed to compound leaves). We calculated the ratio of seeds to seedlings and seedlings to adults based on their densities.

\subsection{Consequences for recruitment}
To assess the consequences of dispersal, we calculated three measures of seed distribution relevant for recruitment success from simulations parameterised with fitted seed dispersal models. (1) Seed limitation was quantified as its complement; the proportion of 1-m$^2$ quadrats reached by at least one seed. (2) Seed escape from adults was quantified as the mean distance of dispersed seeds to the nearest conspecific adult trunk in the mapped plot. (3) Seed crowding was quantified as the mean over all seeds of the number of conspecific seeds in the same 1-m$^2$ sample. To determine whether trends in these three measures among plots reflected declines in seed dispersal quality beyond effects solely due to adult density itself, we compared values of these metrics obtained in simulations using each plot's empirically fitted dispersal kernel ('observed dispersal') with two sets of simulations that applied the same dispersal kernel in all plots: 'Good dispersal' simulations used the kernel of the plot with the lowest population density (and most extensive dispersal), whereas 'poor dispersal' used the kernel of the plot with the highest population density (and least extensive dispersal). For all simulations, seed production was set to 4247 seeds per adult (the fitted value in the hierarchical model).

\subsection{Statistical analyses}
We used linear models to quantify among plot relationships of adult density with frugivore visitation, time to seed
removal, initial seed dispersal distance and ultimate dispersal distance. NDD of seed dispersal would be evident if the slopes of the log-log relationships with adult density were significantly greater than zero for time to removal, and were significantly less than zero for the other three metrics. Seed removal rates were additionally evaluated using a Cox proportional hazards model of time-to-removal. The relationship of ultimate dispersal distance to adult density was evaluated by comparing the fits of hierarchical models that assumed constant dispersal distances across all 10 plots with a model that had dispersal distance vary linearly or exponentially with adult density among plots.
We also used linear models to quantify the relationship between adult density and the three measures of seed distribution for each of the three types of simulation. To meet model assumptions, adult density, distance-to-adult and crowding were log$_{10}$ transformed, and the proportion of sites with seeds was logit transformed. To distinguish effects of NDD of seed dispersal from effects of adult density alone, we compared the slopes between the relationship for observed dispersal and the relationship for good dispersal (i.e. density-independent dispersal), using analysis of covariance (ANCOVA). A significantly different slope was interpreted as evidence of NDD of seed dispersal. We assessed how dispersal quality affected seed distributions across the gradient of adult density by comparing the intercepts and slopes of the fitted relationships with adult density between good and poor dispersal. All analyses
were conducted in R 3.02 \citep{RDCT2016}.

\begin{figure*}
\hspace*{2cm} \includegraphics[width=16cm,height=20cm]{../figures/Chap3Fig1.png}
\caption[Seed dispersal in the palm \textit{Attalea butyracea}][-15cm]{Seed dispersal in the palm \textit{Attalea butyracea} on Barro Colorado Island for populations with widely ranging adult density. (a) Rates of fruiting-tree visitation by the two principal seed dispersers. (b) Geometric mean time elapsed until seed removal by dispersers (the complement of removal rate). (c) Median seed dispersal distance from the source after 32 days. (d) Median ultimate dispersal distance estimated by hierarchical inverse modelling over all plots simultaneously (solid line), and - for illustration - by inverse modelling for plots individually (dots and dashed line; symbols vary by the function that gave the best model fit). Dots correspond to ten 4-ha (a and d) or six 1-ha (b and c) forest plots. All relationships (lines) imply negative density dependence of seed dispersal.
 }
\label{fig:chap3fig1}
\end{figure*}


\section{Results}

\subsection{Density dependence of seed dispersal}

The camera traps recorded a total of 2076 visits by 15 mammal species, with an average of 11.7 visits day$^{-1}$ (Table \ref{tab:chap6tab1}). Ten frugivorous species were photographed handling (i.e. holding in their paws and/or mouth) a total of 2201 \textit{Attalea} fruits (Table \ref{tab:chap3tab1}). Seeds were removed by agoutis and squirrels (all 21 trees) and tapirs (2 trees). Rates of tree visitation by frugivores ranged 40-fold among the nine plots for both agoutis (1.1-41.6 day$^{-1}$) and squirrels (0.08-3.4 day$^{-1}$). Visitation rates by agoutis and squirrels combined decreased significantly with tree density across the ten 4-ha plots (Fig. \ref{fig:chap3fig1}a; log-log regression: $F_{1,7}$ = 7.9, $R^2$ = 0.53, P = 0.026). The slope of this relationship was significantly smaller than zero ($\beta$ = -0.82, $t_7$ = 2.8, P = 0.013), indicating that visitation was NDD.

The seed removal rate differed strongly among the six 1-ha plots. The time elapsed until all seeds were removed ranged from 8 to 128 days. Time-to-removal increased with tree density (Fig. \ref{fig:chap3fig1}b; $R^2$ = 0.90, $F_{1,4}$ = 37.0, P = 0.004). The slope of this relationship was significantly larger than zero ($\beta$ = 1.34, $t_4$ = 6.08, P = 0.002), indicating that seed removal was NDD. Survival analysis of time-to-removal yielded a similar relationship (Cox regression: $e^\beta$ = 0.09, $Wald_1$ = 18.19, P < 0.001). 

Initial dispersal distance - the distances at which we found tagged seeds 7 days after placement - ranged from 0 to 38.4 m among seeds. Median distances ranged from 0.04 to 2.21 m among the six 1-ha plots, and decreased with adult density (Fig. \ref{fig:chap3fig1}c; log-log regression: $R^2$ = 0.60, $F_{1,4}$ = 6.0, P = 0.07). The slope of this relationship was significantly smaller than zero ($\beta$ = -2.25, $t_4$ = 2.46, P = 0.035), indicating that initial dispersal distance was NDD. 

We excavated a total of 2272 endocarps across 330 1-m$^2$ quadrats. Ultimate seed dispersal distance, estimated with IM from the distributions of these endocarps with respect to adult trees, declined with adult density. The dispersal kernels that best fitted the observed distribution patterns for individual plots were the Erlang and the Weibull, each in five of the ten 4-ha plots. Median dispersal distances calculated from these best-fitting dispersal models were negatively related to the density of \textit{Attalea} adults (straight dashed line in Fig. \ref{fig:chap3fig1}d; $R^2$ = 0.50, F$_{1,8}$ = 7.9, P = 0.023). The slope of this log-log relationship was again significantly smaller than zero ($\beta$= -0.44, $t_8$ = -2.81, P = 0.011). The best-fitting dispersal kernel fitted over all plots simultaneously, with hierarchical inverse modelling, was a 1-parameter 2Dt function with the shape parameter fixed at 2 and the scale parameter \textit{a} fitted as a linear function of palm density D: \textit{a = 7878 - 326 D} (Fig. S2). The median dispersal distance extracted from this model showed a sharp decline with adult density (curve in Fig.\ref{fig:chap3fig1}d). These results indicate that ultimate dispersal distance was NDD.

\begin{landscape}
\advance\vsize6cm
\csname @colroom\endcsname=\vsize
\textheight=\vsize
\csname @colht\endcsname=\vsize
\small
     \begin{longtable}{@{}p{7cm}p{1cm}p{1cm}p{1cm}p{1cm}p{1cm}p{1cm}p{1cm}}
\caption{Visitation of \textit{Attalea butyracea} palms by mammals on Barro Colorado Island, recorded with camera traps deployed below fruiting individuals.}
\label{tab:chap3tab1}\\
\hline
{} & {}& \multicolumn{2}{c}{\emph{No. of visits}} & \multicolumn{2}{c}{\emph{Visit duration (min)} }& \multicolumn{2}{c}{\emph{Fruits handled}}  \\
Species &  No. of plots & Total & Rate ($day^{-1}$) & Total & Rate ($day^{-1}$) & Total & Dispersal \\

\hline

\endfirsthead
\hline

{} & {}& \multicolumn{2}{c}{No. of visits} & \multicolumn{2}{c}{Visit duration (min)} & \multicolumn{2}{c}{Fruits handled}  \\
Species &  No. of plots & Total & Rate ($day^{-1}$) & Total & Rate ($day^{-1}$) & Total & Dispersal \\

\hline
\endhead
\hline \multicolumn{8}{r}{\emph{Continued on next page}}
\endfoot
\endlastfoot

\multicolumn{8}{l}{\textbf{Frugivores}} \\
Central American agouti (\textit{Dasyprocta punctata}) & 9 & 1097 & 6.21 & 824 & 4.66 & 561 & Y \\
Red-tailed squirrel (\textit{Sciurus granatensis}) & 9 & 293 & 1.66 & 112 & 0.63 & 150 & Y \\
Collared peccary (\textit{Tayassu tajacu}) & 8 & 272 & 1.54 & 1189 & 6.73 & 634 & N \\
White-nosed coati (\textit{Nasua narica}) & 8 & 232 & 1.31 & 1692 & 9.57 & 700 & N \\
Paca (\textit{Agouti paca}) & 8 & 232 & 1.31 & 1692 & 9.57 & 700 & N \\
Common opossum (\textit{Didelphis marsupialis}) & 5 & 24 & 0.14 & 44 & 0.25 & 22 & N \\
Tome's spiny rat (\textit{Proechimys semispinosus}) & 5 & 22 & 0.12 & 3 & 0.02 & 3 & Y \\
White-faced monkey (\textit{Cebus capucinus}) & 5 & 10 & 0.06 & 2 & 0.01 & 8 & N \\
Baird's tapir (\textit{Tapirus bairdii}) & 3 & 9 & 0.05 & 132 & 0.75 & 103 & Y \\
Mouse spec.& 4 & 9 & 0.05 & 0.5 & 0.00 & 0 & N \\
Robinson's mouse opossum (\textit{Marmosa robinsoni}) & 3 & 7 & 0.04 & 5 & 0.03 & 1 & N \\

\multicolumn{8}{l}{\textbf{Non-frugivores}} \\
Northern tamandua (\textit{Tamandua mexicana}) & 5 & 42 & 0.24 & 79 & 0.45 & 0 & N \\
Red-brocket deer (\textit{Mazama americana}) & 8 & 13 & 0.07 & 2 & 0.01 & 0 & N \\
Nine-banded armadillo (\textit{Dasypus novemcinctus}) & 3 & 7 & 0.04 & 8 & 0.05 & 0 & N\\
Tayra (\textit{Eira barbara}) & 1 & 1 & 0.01 & 0 & 0.00 & 0 & N \\
\multicolumn{2}{l}{\textbf{Total}} & 2076 & 11.75 & 4109 & 23.25 & 2201 & {} \\
\bottomrule

\end{longtable}
\end{landscape}


\subsection{Consequences for recruitment}
Average seed density ranged over 1.6 orders of magnitude across the ten 4-ha plots, from 0.8 to 31 m$^{-2}$ , and increased significantly with adult density (Fig. \ref{fig:chap3fig2}a; $R^2$ = 0.88, P < 0.001). The slope of this relationship did not differ significantly from one ($\beta$ = 1.23, $t_8$ = 1.44, P = 0.09), implying that seed density was proportional to adult abundance. Seedling density ranged over 1.1 orders of magnitude across the 10 plots, from 0.0124 to 0.142 m$^2$ , and also increased with adult density (Fig. \ref{fig:chap3fig2}a; $R^2$ = 0.70, P = 0.003), but much less than proportionally ($\beta$ = 0.56, $t_8$ = 3.45, P = 0.004). Thus, the ratio of seedlings to seeds declined with population density (Fig. \ref{fig:chap3fig2}b; $R^2$ = 0.58, F 1,8 = 11.0, P = 0.011), indicating NDD of seedling recruitment.  


\begin{figure*}
\hspace{1.5cm}
\includegraphics[width=8cm,height=15cm]{../figures/Chap3Fig2.png}
\caption[Seedling recruitment success in the palm \textit{Attalea butyracea}][-10cm]{Seedling recruitment success in the palm \textit{Attalea butyracea} on Barro Colorado Island, Panama, for 10 populations with widely ranging adult density. Log-log relationships with adult density of (a) the density of seeds and young (simple-leaved) seedlings and (b) the ratio of seedlings to seeds. Isometric scaling (i.e. proportional recruitment; dashed lines) is shown for comparison. The relationships are significant and imply negative density dependence of seedling recruitment. }
\label{fig:chap3fig2}
\end{figure*}

\begin{landscape}
\thispagestyle{lscape}
\pagestyle{lscape}
\begin{figure*}
\hspace{5.5cm} \includegraphics[width=18.5cm,height=19cm]{../figures/Chap3Fig3.png}
\caption[Predicted post-dispersal seed distributions][30cm]{.}
\label{fig:chap3fig3}
\hspace*{5cm}\begin{minipage}{20cm}
\small Figure \ref{fig:chap3fig3}: Predicted post-dispersal seed distributions across the lowest and highest density populations of \textit{Attalea butyracea} on Barro Colorado Island, Panama, modelled with the longest and shortest dispersal kernels fitted in this study. (a-d) Actual distributions of adult trees (black dots) with modelled distributions of seeds (density isoclines) over 1 ha. Each plot has one palm in its center. (e-h) Model-estimated seed density along a 100-m cross section of the plots (dotted lines in a-d) for individual trees (grey lines) and the entire population (bold lines). The patterns in a and e resemble the observed seed distributions for low-density populations, and those in d and h resemble the observed seed distributions for high-density populations. The other four panels show hypothetical distributions. The graphs show how reducing dispersal quality leads to more heterogeneous, clumped seed distribution.
\end{minipage} 
\end{figure*}
\end{landscape}

To determine to what degree NDD of seedling recruitment was a consequence of NDD of seed dispersal, beyond effects of adult density alone, we explored the effects of seed dispersal and adult density on seed distributions (Fig. \ref{fig:chap3fig3}). Plots with high adult population density had much greater spatial variation in seed density across soil samples than did plots with low adult population density (compare Fig. \ref{fig:chap3fig3}d and h with a and e). Comparison between simulations with fitted kernels for the worst and the best dispersal suggested that dispersal quality had a strong effect on seed distribution (compare Fig. \ref{fig:chap3fig3}c and g with d and h for illustration).

Seed limitation, however, was not significantly affected by NDD of dispersal. The proportion of sites reached by at least one seed - the complement of seed limitation - increased sharply with adult density (Fig. \ref{fig:chap3fig4}a; logit-log regression: $\beta$ = 2.77, $F_{1,8}$ = 315, P < 0.001, $R^2$ = 0.98). In fact, most of the sampling quadrats in dense plots received multiple seeds despite simulated observed and simulated density-independent dispersal (ANCOVA: $F_{1,17}$ = 0.35, P = 0.56). Where adult abundance decreased the distance 4.7-fold, NDD of dispersal decreased it only a further 1.09-fold (Fig \ref{fig:chap3fig4}b). The extent of seed dispersal affected distance-to-adult strongly in plots with low densities but again not at high adult densities. In dense populations, adult trees were so common that dispersed seeds tended to land near one regardless of dispersal quality. Thus, negative consequences of NDD seed dispersal for seed escape from adults were overwhelmed by direct consequences of high adult densities. 

Seed crowding, finally, was also not significantly affected by NDD of dispersal. The average neighbourhood seed density experienced by seeds - a measure of seed crowding - increased as a power function of adult density (Fig. \ref{fig:chap3fig4}c; log-log regression: $\beta$ = 1.22, $F_{1,8}$ = 338, P < 0.001, $R^2$ = 0.98). Again, this increase resulted largely from adult density itself, as the slope of the relationship did not differ significantly between simulated observed and density-independent dispersal (ANCOVA: $F_{1,17}$ = 3.17, P = 0.09). Where adult density increased perceived seed density 32-fold, NDD of dispersal increased it only a further 1.3-fold (Fig \ref{fig:chap3fig4}c). Crowding increased only marginally faster under poor dispersal than under good dispersal. Thus, plots with high adult density already had high seed densities because of the large number of seeds in these systems, and NDD of seed dispersal did not increase crowding further.

\section{Discussion}


\begin{figure*}
\hspace*{1cm}\includegraphics[width=10.5cm,height=19cm]{../figures/Chap3Fig4.png}
\caption[Effects of NDD seed dispersal][-15cm]{Effects of NDD seed dispersal on (a) seed limitation, (b) seed escape from adults and (c) seed crowding in the palm \textit{Attalea butyracea} on Barro Colorado Island, Panama, for 10 populations that varied widely in adult density. Plotted values are predictions modelled with the observed dispersal kernels for each plot (filled points and solid line), and - for comparison - with the longest dispersal kernels (crosses and dotted lines) and shortest dispersal kernels (open points and dotted lines) fitted in this study. Panels show (a) the proportion of sites reached by at least one seed (the complement of seed limitation), (b) the average distance of dispersed seeds to the nearest adult, and (c) the density of conspecific seeds perceived by dispersed seeds. The brackets to the right indicate the variation due to adult density (larger numbers) and to NDD of seed dispersal (smaller numbers) for each dependent variable. See text for further explanation.
 }
\label{fig:chap3fig4}
\end{figure*}

We tested the hypotheses that seed dispersal, a key component of reproductive success in seed plants, is NDD due to intra-specific competition for seed dispersers, and that NDD dispersal contributes to NDD of recruitment, a widely observed phenomenon hypothesised to be important to species coexistence in diverse tropical forests (Wright 2002). We found that competition for dispersers indeed caused NDD of seed dispersal. Tree visitation by dispersers, seed removal rate, initial seed dispersal distance and ultimate seed dispersal distance all decreased with increasing population density. We also found evidence for NDD of seedling recruitment. However, NDD of recruitment was not driven by NDD of seed dispersal, as the negative effects on declining dispersal quality were entirely overwhelmed by effects of increased adult and seed abundance alone. 

\subsection{Density dependence of seed dispersal }
Our field observations suggest that the NDD of dispersal observed in this study was a consequence of intraspecific competition for seed dispersers. The declines of removal and dispersal with tree population density all indicate that dispersers became increasingly satiated by the combined fruit crops. The concordance of the patterns for different measures of seed removal and dispersal, which reflect data taken over multiple years, strongly suggests that this is a general pattern for our focal species. Our study is the first to show that intraspecific competition for an entire community of seed dispersers (Table \ref{tab:chap3tab1}) can cause NDD of seed dispersal. 

Our results are consistent with previous studies that have found negatively density-dependent seed removal and/or dispersal by scatter-hoarding rodents, marsupials and birds. Scatter-hoarding rodents invest less effort in caching seeds when and where seeds are more abundant, resulting in reduced dispersal distances and reduced rates of recaching
\citep[e.g.][]{VanderWall2002, Jansen2012}. In tropical forests, seed removal and dispersal by scatter-hoarding rodents were reduced in areas with aggregations of adults for \textit{Astrocaryum murumuru} in Peru \citep{Beck2002} and for \textit{A. standleyanum} in Panama \citep{Galvez2009}. Intraspecific competition for seed dispersers has also been documented for one marsupial-dispersed and several bird-dispersed tropical plant species \citep{Howe1977, Howe1981, Manasse1983, Sargent1990, Poulin1999, Saracco2005, Morales2012}.

NDD of dispersal is a particular manifestation of the broader phenomenon of context-dependent dispersal. In general, the per-fruit odds of seed removal and dispersal decline in areas and times where food is so common relative
to disperser densities as to result in satiation \citep[e.g.][]{VanderWall2002, Jansen2004, Klinger2009}. One might then expect that dispersal NDD will be ubiquitous, as higher local focal species abundances lead to higher fruit and seed abundances, and thus increased potential for satiation, all else equal. However, multiple factors influence whether dispersal NDD actually emerges in a particular system, year and site, including especially spatial covariation in abundances of dispersers and of their alternative foods with the abundances of the focal plant species \citep{Klinger2009}. For example, dispersal NDD need not emerge if focal species fruits or seeds are sufficiently valued and scarce relative to disperser abundance that they are all removed everywhere, or if disperser abundances are able to track spatial variation in focal species fruit or seed availability. We hypothesise that such conditions are the exception rather than the rule, and that dispersal NDD is a common phenomenon among animal-dispersed plants in general.

\subsection{Consequences for recruitment}

We found NDD of seedling recruitment at the population level, consistent with earlier studies \citep[reviewed in][]{Wright2002}. No previous studies identified the causes of such NDD. We had hypothesised that NDD of seed dispersal translates to NDD of recruitment, but found that NDD of seed dispersal in fact had little effect on seed limitation, seed proximity to adults, or seed crowding, the three aspects of seed distributions that we examined. 

First, NDD of seed dispersal evidently increased dispersal limitation, a key component of recruitment limitation \citep{Nathan2000}, yet this was more than offset by a concomitant reduction of source limitation, resulting from elevated seed availability and wider scattering of seed sources over space. Thus, essentially all sites in dense populations were reached by seeds regardless of dispersal quality. Second, seed proximity to adults increased with adult density, but this was more a consequence of adults being omnipresent in these plots than of seeds dispersing less far from their parent. Thus, seeds were likely to land near a conspecific adult regardless of the distance over which they were dispersed. Third, seed crowding increased drastically with adult density, but as a direct consequence of greater seed abundance rather than due to NDD of dispersal. This implies that there is no way for seeds and seedlings to escape from juvenile conspecifics in dense populations, which makes them prone to attack by density-responsive natural enemies.

All tropical trees show aggregation \citep[e.g.][]{Condit2000}, and do so at a level that was sufficient for NDD of seed removal in \textit{Virola surinamensis} \citep{Manasse1983}, suggesting that NDD dispersal could be pervasive if linked with abundance. We expect the consequences of NDD of dispersal for recruitment will depend on the abundances and characteristics of the focal plant species and its dispersers, with NDD of dispersal having important negative consequences in some cases. Unlike our focal species, the vast majority of tropical plant species never reach high local densities, and thus may not reach the point at which dispersal effects on seed distributions become irrelevant. Furthermore, seed handling is important for seed germination and/or escape from seed pathogens and/or predators in many species \citep[e.g.][]{Traveset1998, Jansen2010}. In such species, any decline in seed removal would lead
directly to a decline in the proportion of surviving seeds, thus contributing to NDD of recruitment. Finally, we note that our analyses of recruitment success were limited to our plots, ignoring the potential for long-distance dispersal (LDD). Insofar as LDD is also NDD, associated reproductive output will be reduced disproportionately in more dense stands, especially considering that LDD is the only way to escape high local conspecific densities and associated threats in these stands.

Clearly, a key question concerns the generality of these results to other plant populations. Despite the voluminous literature on seed dispersal, few studies have reported tests of NDD dispersal, and none have previously quantified its consequences. More studies are needed to establish the frequency of NDD dispersal, to quantify its impacts when present, and to investigate what characteristics of plant populations and their dispersers explain variation in the occurrence of NDD dispersal and in its consequences when present.
\end{fullwidth}


\vspace*{20cm}


\begin{landscape}
\begin{figure}
\vspace*{-.6cm}\hspace*{4.4cm}\fbox{\includegraphics[width=19cm,height=22cm]{../figures/illustrations/chapter4.png}}
\hspace*{5cm}\begin{minipage}{18cm} 
\textit{ \footnotesize "Rainforest trees are not immortal, and each and every one of them will eventually die" - John Kricher (1997) \nocite{Kricher1997}}
\end{minipage}
\label{fig:chap4}
\end{figure}
\end{landscape}

%------------------------------------------------
% Chapter 4
%--------------------------------------------------
\chapter{Population-level density dependence in a tropical forest:  Regulation and limitation of a common palm}
\label{ch4} 

\marginnote[-2cm]{Marco D. Visser, Helene C. Muller-Landau, S. Joseph Wright, Eelke Jongejans, Hans de Kroon, Annieke C. W. Borst, Pieter A. Zuidema,  Gemma Rutten and Patrick A. Jansen. Supplementary material can be found online: http://tinyurl.com/zghs7t8}

\section{Abstract} 
\begin{fullwidth} 
Stable coexistence in any community is possible only when each species has positive population growth rates at invasion (low density), and has negative density dependence (NDD) that prevents dominance. Yet, few studies have quantified NDD of per-capita population growth rates, and none have evaluated the relative importance of the processes underlying population-level NDD, invasion growth rates, and equilibrium abundance. We measured seed production, seedling establishment, and the growth and survival of three life stages of the palm \textit{Attalea butyracea}, on Barro Colorado Island, Panama, for ten populations that spanned a 10-fold range in adult density. We quantified density dependence for each life stage, and then integrated these estimates across the entire life cycle, using a density-dependent integral projection model (IPM). We quantified invasion growth rates, equilibrium density and NDD in per capita population growth, and used sensitivity analysis to determine which demographic processes at which life stages were most influential. \textit{Attalea} exhibited stronger NDD in seedling establishment than in any other life phase, and this was strong enough to result in NDD population growth. The estimated invasion growth rate was 1.048, and the equilibrium density was just 0.5 adults per hectare. The most important bottleneck determining equilibrium density was the transition from seedling to juvenile.  We conclude that \textit{Attalea} is regulated mainly by strong density dependence of seedling establishment.  However, equilibrium population densities are more strongly limited by additional, density-independent bottlenecks, especially the light-limited transition from seedling to juvenile rosette. In essence, we show that while NDD seedling establishment is required for an equilibrium density to exist at all; other density-independent parameters determine the exact value of this equilibrium density.  Thus \textit{Attalea} can attain very high densities in disturbed forests where the density-independent bottleneck is relaxed, despite strong density-dependent regulation.

\section{Introduction}

Tropical forests are the most species-rich ecosystem in the world, hosting an estimated total of 40,000-53,000 tree species \citep{Slik2015} that in turn support an astonishing diversity of other life forms. How hundreds of tropical tree species with similar resource requirements are able to coexist at small spatial scales is a long-standing question in ecology \citep{Wright2002, Leigh2004}. Multiple mechanisms preventing competitive exclusion of tree species by superior competitors have been proposed, in many of which the performance of tree species is negatively dependent on their frequency \citep{Bagchi2010, Comita2014}. Examples include density dependence of seed predation \citep{Lewis2008} and seedling establishment \citep{Harms2000a}. Evidence has accumulated that negative density-dependent mechanisms indeed operate, and may also determine the relative abundance of species \citep{Comita2010,  Mangan2010, Johnson2012}. 

Negative density-dependent performance has been abundantly demonstrated for early life phases of trees, where the greatest losses occur \citep[e.g.][]{Crawley2000, Comita2014}.  Density dependence has also been proposed to affect performance of trees at later life stages \citep{Jones2008}, but studies considering this possibility are scarce \citep{Zhu2015}. The vast majority of studies focus on a single life stage, and even the best  studies consider early life stages only \citep{Comita2010, Mangan2010, Swamy2010, Johnson2012, Bagchi2014}.  Demographic theory  predicts that life-stage contributions to population growth are highly unequal and that contributions from early life-stages are likely to be very small \citep[e.g.][]{DeKroon2000}. Moreover, effects at one life stage can be offset by opposite effects at another \citep{Visser2016}. How these various effects add up, interact with other processes across species' life cycles, and affect the fitness of trees at the population level, has never been explored \citep{Caughlin2014}.  More broadly speaking, key population-level criteria for coexistence, such as invasibility, have rarely been quantified for natural populations \citep{Siepielski2010}. Whether negative density dependence at early life stages is strong enough to induce negative density dependence at the population scale is an open question. Moreover, whether that hypothetical population scale negative density-dependence regulates populations at their observed densities is also unknown.

Species abundances are not only regulated by density-dependence, but are simultaneously limited by the capacity of the environment to support them \citep{Turchin1995}.  We define population regulation as any process that causes NDD at the populations scale and population limitation as any and all factors affecting the abundance around which a population is regulated.  For instance, populations of the tropical pioneer species \textit{Cecropia}, are regulated by negative density dependence but are simultaneously limited by the frequency of canopy gaps \citep{Alvarez-Buylla1994}. Theoretically, environmental factors - such as the lack of recruitments sites - can cause severe population bottlenecks that strongly limit population growth rates. Yet, how important are both processes in determining the relative abundance of any given species? 

Population-dynamic models parameterized with field-measured vital rates from communities with contrasting densities of a focal species provide a platform for integrating density dependent effects, and quantifying the role of density dependence in relation to other limiting factors, across the life cycle.  Integral projection models (IPMs) are particularly suited for such evaluations \citep{Ellner2006}. IPMs can be used to model the life cycle of species with complex demographic attributes and alternations between discrete and continuous structures, such as abrupt transitions from seeds to seedlings as well as gradual transitions from seedling to reproductive adults. Model sensitivity analysis \citep{Caswell2000, Rees2009} allows quantification of the relative importance of different processes operating throughout the life-cycle to population-level statistics such as population growth rates or population density.  Studies evaluating density-dependence across the life cycle from seed to adult are extremely rare, while we know of no study which assesses the relative contributions of density dependence and other population bottlenecks to population regulation and thus the potential for coexistence.  

Here, we use IPMs to quantify the contributions of different processes across the entire life cycle to population regulation and limitation in the palm \textit{Attalea butyracea} (Arecaceae).  This is the 17th most abundant species among the 227  'hyper-abundant species' that together comprise >50\% of all stems in Amazonia \citep{Steege2013}.  We test the hypothesis that negative density dependence at the seedling stage, integrated across the life cycle, is sufficiently strong to cause negatively density-dependent population growth. Specifically, we quantify 1) the invasion growth rate, 2) equilibrium density and 3) local stabilization, i.e., the rate of change in the population growth rate around the equilibrium.  We then explore which vital rates are subject to density dependence and which constitute the most important constraints on performance. Our results show that that \textit{Attalea} is regulated by strong density dependence in seedling establishment, and that equilibrium population densities are more strongly limited by additional, density-independent factors.

\section{Methods}

\subsection{Study system}
Fieldwork was conducted in old secondary forest (90-130 years old) on the eastern half of Barro Colorado Island (BCI; 9$^{\circ}$9'N, 79$^{\circ}$51'W), a 1560-ha island in Gatun Lake, Panama.  Annual rainfall averages 2650 mm, with a dry season from December to April. BCI is well protected by game wardens, and mammal abundances on BCI are not fundamentally different from those at other Neotropical sites with intact fauna (\citealt{Wright1994}). 

\textit{Attalea butyracea} (Mutis ex L.f.) Wess. Boer (henceforth \textit{Attalea}) is a monoecious palm that can reach heights of $\approx$30 m \citep{DeSteven1987}. Adults produce 1-3 pendulous infructescences annually, each containing 100-600 fruits. Fruits are 3-5 cm long and consist of a hard exocarp enclosing a sweet, fleshy mesocarp and a hard endocarp or stone (henceforth 'seed') that usually contains one seed. The life-cycle of \textit{Attalea} palms is illustrated schematically in Figure \ref{fig:chap4fig1}.

\begin{landscape}
\begin{figure*}
\hspace*{5cm} \includegraphics[width=16cm,height=20cm]{../figures/Chap4Fig1.pdf}
\caption[Schematic representation of the life cycle of \textit{Attalea butyracea}][-4cm]{Schematic representation of the life cycle of \textit{Attalea butyracea} used for the Integral Projection Model in this study. Arrows indicate transitions between life stages (seedlings, rosettes and stemmed palms) and seed production. $\alpha$ and $\omega$ denote minimum and maximum sizes within each life stage. Transitions within the grey box take place within a single time step (2 years).}
\label{fig:chap4fig1}
\end{figure*}
\end{landscape}

\subsection{Field measurements}

We used BCI-wide maps of canopy \textit{Attalea} developed from high-resolution aerial photographs \citep{Jansen2008} to locate 10 square 200$\times$200 m (4-ha) plots that ranged twenty-fold in adult \textit{Attalea} densities, from 1.25 to 24.4 palms per hectare. We included the densest \textit{Attalea} stands on BCI (see ch. \ref{ch2} \& \ref{ch3}). Each plot had one reproductive \textit{Attalea} in its center.

\textit{Adults} - Between October and December 2007, every \textit{Attalea} with a bole height >1.3m was tagged and mapped in the 10 plots using a precision compass (Suunto KB-14 precision, Vantaa, Finland) and ultra-sonic rangefinder (Hagloff DME-201 cruiser, Langsele, Sweden). We recorded bole heights (height to lowest living palm frond) using a laser range finder (Nikon Forestry PRO). Reproductive individuals were identified by the presence of infructescences and inflorescences, which remain on the palm for a year. In June 2012 we conducted a full recensus of all previously mapped individuals, and mapped all recruits (palms >1.3 m tall).

\textit{Seeds} - We assessed seed fate between January and August 2008, by studying scars on endocarps that we collected from the forest floor and topsoil in 32 1-m$^2$ quadrats per plot (320 quadrats total). Quadrats were placed in a stratified random manner, with two quadrats in each of the sixteen 25$\times$25 m subplots in the central hectare of each 4-ha plot. If a randomly selected point was covered with a rock, tree or debris, the quadrat was placed as close as possible to the randomly selected point in a randomly generated direction. This protocol produces endocarp densities that reflect the sum of the previous three fruiting seasons (more details in ch. \ref{ch2} \& \ref{ch3}). 

\textit{Seedlings and saplings} - In January 2008, we mapped all juvenile palms (< 1.3 meters in height) within a 2500-m$^2$ subplot in each of the ten 4 ha plots. We divided the central 100$\times$100 m of each 4-ha plot into 16 25$\times$25 m subplots and randomly selected one of the four inner subplots and three of the 12 outer subplots. In these four 625-m$^2$ subplots, we tagged all \textit{Attalea} individuals with vinyl loop tags, and recorded the length of the longest leaf, as well as the number of simple or complex leaves. In June 2010 and June 2012, we revisited all juvenile plots, recorded survival, measured surviving individuals, and mapped and tagged new recruits.   


\subsection{Models} 
We used the field data to fit vital rate functions that included density dependence, and then used these functions to parameterize a density-dependent Integral Projection Model (IPM) \citep{Easterling2000, Zuidema2010}. From the IPM, we calculated invasion growth rates ($\lambda_i$), equilibrium density ($a_{\lambda=1}$) and the change in $\lambda_i$ at the equilibrium density (a measure of stabilization; $\theta$). We also calculated the sensitivity of all underlying model parameters to each of the three statistics.  These sensitivities quantify the relative importance of the biological processes represented by each model parameter for each statistic. 

All analyses and metrics were based on projections of asymptotic dynamics. This means that we asked the question how the populations would change if conditions continued as they were when our data were collected. Such analyses make it possible to evaluate the relative importance of different processes to the population in its current situation \citep[see e.g.][]{Caswell2000, Caswell2001}. We use capital Roman letters to denote functions, bold letters for matrices, Greek letters for parameters and lowercase Roman letters for variables (life stage, size, time, density etc.). All models were built in R 3.2.3 (R Development Core Team 2016), with computationally intense segments in C++, as recommended by \citet{Visser2015} (Ch. \ref{ch9}).

\subsection{Quantifying negative density-dependence across the life-cycle}
The \textit{Attalea} life cycle was structured along continuous size axes in three distinct stages (Fig. \ref{fig:chap4fig1}): 1) seedlings with simple leaves, 2) basal rosettes of compound leaves, and 3) individuals with woody trunks (which display height growth). This classification is based on morphological traits combined with significant changes in either growth or survival.  Seedlings have simple leaves whose lamina can be up to 90 cm long. Basal rosettes have compound leaves that can be up to 5 meters long. Individuals tend to persist as basal rosettes for extended periods before woody stems first appear enabling height growth. The time step was set to 2 years, i.e., the seedling-census interval.  We did not include a seed phase, as the vast majority of seeds either die or germinate within the two-year time step - the proportion of seeds that both escape predation and would be expected to germinate after 3 years is 0.0014\% \citep{Harms1995,Visser2011}.  For each life stage, we calculated growth, survival and the probability of transition to the next life stage. We also calculated reproduction as a function of size (\emph{x}, units vary depending on stage) and adult density (\emph{a}, in $ha^{-1}$).  Size was defined by the length of the longest leaf (cm) for simple leaf seedlings ($x_s$; stage \emph{s}) and basal rosettes ($x_r$; stage \emph{r}), and as bole height (m) for stemmed individuals ($x_h$ ; stage \emph{h}).
 
We fitted functions and estimated parameter uncertainty within a Bayesian framework, using flat or uninformative priors, unless stated otherwise. All models were fit using Monte-Carlo Markov Chain (MCMC) methods in JAGS and the R-package \emph{rjags}, which uses an adaptive Gibbs sampler to approximate posterior distributions \citep{Plummer2016}.  Model parameters were sampled independently from three chains, and convergence was checked using Gelman and Rubin's convergence diagnostic \citep{Brooks1998}, after a burn in period of 10 000 iterations for each chain. Each chain was checked for autocorrelation, and thinning was applied automatically based on within-chain autocorrelation. 

To evaluate density-dependence, we conducted Bayesian model averaging to obtain final models \citep{Hoeting1999,Bishop2006}, thus ensuring that models with similar fits were not ignored. Models were weighted by their posterior probabilities, and we averaged parameters over all density-dependent and density-independent models using the 'zero method' in which parameters are set to zero when absent from models \citep{Grueber2011}. Credible intervals (CI) were calculated as the 2.5 and 97.5 percentile of the model-averaged marginal distributions of each parameter (estimated from the joint posterior by sampling from posterior distributions using MCMC). Each selected model was finally checked visually, comparing model fits to moving-averages over size obtained from generalized additive models (GAMs). The procedure to fit each vital rate function is described below, and sample sizes at each stage and vital rate are given in Table S1.   

\textit{Probability of annual reproduction} - We fitted two logistic functions for the probability that individuals were reproductive in a given year. One, R($x_h$), included only size. The other, R($x_h$, $d$), included both size and density (see table S2). Both models included random intercepts for census period ($\gamma_{year}$).
 
\textit{Seed production} - We used the seed production function from Jansen \emph{et al.} (2014), which was estimated for the same populations (i.e., the same ten 4-ha plots). \citealt{Jansen2014} (Ch. \ref{ch3}) found that seed production is independent of palm height and adult density, and the average number of seeds produced is 498 seeds per year per adult palm (95\% CI; 148.05 - 774.5). 

\textit{Seedling establishment} - Seedling establishment probabilities E(\emph{a}) were calculated by dividing the density of newly recruiting seedlings at each 2-year census by the expected 2-year seed production density (seeds m$^2$) for each plot. The 2-year seed production density was calculated as 2/3 of the observed seed densities from 32 seed quadrats per plot because quadrat censuses capture seed deposition over three years. Seedling establishment probabilities were then related to log adult density (\emph{a}) through linear regression. Seed predation was found to be density-independent due to top-down control on \textit{Attalea}'s host-specific insect seed-predator by rodents (Ch. \ref{ch2}). 

\textit{Initial leaf length} - Initial leaf length, L(\textit{x}), was modeled using six alternative probability density functions (Table S3). We first tested density-independent versions of the exponential, Weibull and lognormal distributions. Then, using the best-fit density-independent model as determined by DIC scores \citep{Spiegelhalter2002}, we proceeded to test for density-dependence by linearly relating the parameters that governed the mean leaf length to adult density. These parameters were, for the exponential, Weibull and lognormal respectively, the rate, scale and mean parameter. Models included random effects for census ($\gamma_{year}$) that acted on the rate, scale or mean parameter. 

\textit{Growth} - For simple leaf seedlings and basal rosettes, growth, G(\emph{x}, \emph{a}), was defined as absolute leaf length growth (cm/year) and modelled as a function of initial leaf length (cm).  For stemmed individuals, growth was defined as absolute height growth (m/year) and modelled as a function of initial bole height (m). For each life stage, we fit seven growth models (Table S4), including 4 density-independent and three density-dependent models.  The density-independent models were (1) a constant growth null model without size or density dependence, (2) a size-dependent linear model with constant variance, (3) a size-dependent linear model with variance linearly related to size, and (4) a size-dependent linear model with variance log-linearly related to size.  The density-dependent models were (5) a size and density-dependent linear model with constant variance, (6) a size and density-dependent linear model with variance linearly related to size, and (7) a size and density-dependent linear model with variance log-linearly related to size.  All models included random intercepts for census ($\gamma_{year}$).

\textit{Survival} - Annual survival (S(\emph{x}, \emph{a})) was modeled as a function of size and adult density using the same procedure as for the fraction of reproductive individuals (see above; Table S2).

\textit{Transition probabilities} - The probabilities of transitioning from one stage to the next ($T_{s\rightarrow r}$, and $T_{r\rightarrow h}$) per two-year time step were fitted as logistic functions of size only (first model in Table S2), as there were insufficient data to test for density-dependence (just 9 and 7 transitions, respectively). For the transition from simple to complex stages, initial size in the complex stage was predicted from growth models for current leaf size. For the transition from basal rosette to stemmed palm, initial stem size B($x_h$) was chosen from a probability density function fitted to initial sizes following transition, chosen by evaluating the same candidate models used for initial leaf length L($x_s$) (Table S3). 


\subsection{Integrating negative density-dependence across the life cycle}
All fitted vital rate functions were combined in a three-stage IPM (integral projection model). The construction of IPMs is described in detail elsewhere \citep{Easterling2000, Ellner2006, Zuidema2010, Metcalf2013, Merow2014}. Briefly, IPMs combine size-dependent vital rates describing survival, growth and reproduction of a population into a projection kernel, P($y$,$x$). The projection kernel describes all possible transitions between individuals of size \emph{x} at time \emph{t} and individuals of size \emph{y} at time \emph{t+1}, as well as the production of new individuals. When the distribution (number of individuals) in a population at time \emph{t} is described by W(\emph{x},\emph{t}) as a function of size \emph{x}, then the distribution of individuals sized \emph{y} at time \emph{t+1} is given by: 

\begin{equation}
W(y,t+1) = \int_{\alpha}^{\omega} P(y,x,a)W(x,t)dx 
\end{equation}
	
where the limits [$\alpha$,$\omega$] represent the minimum and maximum sizes of individuals . Here, the projection kernel depends on the previous population structure, summarized by a or the total reproductive population size at time t. The total reproductive population, a,  is calculated as $a=\int R(x)W(x,t)dx$, where R(\emph{x}) denotes the fraction of individuals that are reproductive as a function of size. 	

We built an IPM consisting of six kernel functions that together described all transitions between and within the three life stages (Table \ref{tab:chap4tab1}). These included a (1) reproduction kernel $P_{h \rightarrow s}(y_h, x_s, a)$, (2) survival and growth kernel for seedlings $P_s(y_s,x_s, a)$ (3) transition kernel between seedlings and basal rosettes $P_{s\rightarrow r}(y_s,x_r, a)$, (4) a survival and growth kernel for basal rosettes $P_c(y_c,x_c, a)$,  (5) a transition kernel between basal rosettes and stemmed palms $P_{r\rightarrow h} (y_h, x_r, a)$, and (6) a survival and growth kernel of stemmed palms $P_h (y_h,x_h, a)$. The six functions combined into a mega-matrix (\textbf{M}) which described the transition within and between all stages at a given adult density (\emph{a}): 

\begin{equation}
\textbf{M}=
  \begin{bmatrix}
    P_s(y_s,x_s, a) & 0 & P_{h \rightarrow s}(y_h, x_s, a) \\
    P_{s\rightarrow r}(y_s,x_r, a) & P_r(y_r,x_r, a) & 0 \\
    0 & P_{r\rightarrow h} (y_h, x_r, a) & P_h (y_h,x_h, a)
  \end{bmatrix}
  \label{eq:chap4eq2}
\end{equation}
The dimensions of matrix \textbf{M} were set at 450$\times$450 (100 s-stage, 150 r-stage and 200 h-stage classes).  

We then used \textbf{M} to numerically simulate population growth and thereby determine population-level negative density-dependence (NDD), by which we mean the change in per capita population growth rates with abundance.  With this we estimated equilibrium population densities, invasion growth rates, and stabilization (rate of change in population growth rate with abundance at the equilibrium abundance). These statistics were calculated as follows.


\begin{landscape}
\begin{table}
\begin{center}
\footnotesize
\hspace*{4.5cm}\begin{tabular}{p{1.4cm}p{9.1cm}p{8.2cm}}
\multicolumn{3}{c}{Construction of the mega-matrix \textit{M}} \\
\hline 
Kernel & Formulation & Description  \\
\\
\hline 
\\
$P_{s}(x_s,y_s,a) $ & $ = \begin{cases} 
S_s(x_s,a)N(y_s,\mu=G_s(x_s,a),\sigma=V(x_s)) (1-T_{s \rightarrow r} (x_s)) & 0 \leq y_s \leq \omega_s \\
0 & y_s < 0  \end{cases}$ & 
$S_s()$ and $G_s()$ are size \& density dependent annual survival and simple leaf growth (mm/year) functions for seedlings fit to data (Fig. \ref{fig:chap4fig2} C-D), $N()$ denotes the normal distribution, $V(x_s)$ predicts $\sigma_s$ or the standard deviation of growth rates as a function of size and $T_{s \rightarrow r} (x_s)$ a function describing the transition from simple to complex leaves (Fig. \ref{fig:chap4fig2} E). The upper limit of the simple leaf seedling kernel is given by $\omega_s$. The total number of reproductive palms per hectare is given by $a$. 
\\ 
$P_{s \rightarrow c}(x_s,y_r,a)$ & $= \begin{cases}  
0 &  y_h < \alpha_r  \\
S_s(x_s,a)N(y_r,\mu=G_s(x_s,a),\sigma=V(x_s))T_{s \rightarrow r} (x_s) & \alpha_r \leq y_c \leq \omega_r
 \end{cases}
$ & Transition toward complex leaf rosettes with leaf length $y_r$, where $\omega_r$ is the upper limit of the complex leaf rosette kernel.
\\ 
$P_{r}(x_r,y_r,a) $ & $ = \begin{cases} 
S_r(x_s,a)N(y_r,\mu=G_s(x_r,a),\sigma=V(x_r)) (1-T_{r \rightarrow h} (x_r)) & \alpha_r \leq y_r \leq \omega_r     \\
0 & \alpha_r > y_r > \omega_r \end{cases}$ & 
All function definitions are identical to those for $P_{s}(x_s,y_s,a)$, but are specific to complex leaf rosettes. 
\\ 
$P_{c \rightarrow h}(x_r,y_h,a)$ & $= \begin{cases}  
0 & y_h \leq 0 \\
S_r(x_h,a)T_{c \rightarrow h}(x_r)B(y_h) & 0 \geq y_h \leq \omega_h
 \end{cases}
$ & Transition towards stemmed palms and height growth. Here $\omega_h$ denotes the upper limit of the $P_{h}(x_h,y_h)$ kernel. $B(y_h)$ is a the probability density distribution of initial stem heights (Fig. \ref{fig:chap4fig2} I).
\\ 
$P_{h}(x_h,y_h,a)$ & $= S_h(x_d,a)N(y_h,\mu=G_h(x_d,a),\sigma=\sigma_h) $ & 
All function definitions are identical to those for $P_{s}(x_s,y_s)$, but are specific to stemmed palms. 
\\ 
$P_{h \rightarrow s}(x_h,y_s,a)$ & $ = F(x_h)E(a)R(x_h,a)S_h(x_h,a)L(y_s,a)$ &  
$F(x_h)$ is the total number of seeds produced by a palm of height $x_h$. The density-dependent seedling establishment rate is given by $E(a)$, the fraction of individuals that are reproductive at size $x_h$ by $R(x_h)$, while $S_h()$ denotes the survival of stemmed palms. $L()$ denotes the distribution of initial leaf lengths for seedlings (Fig. \ref{fig:chap4fig2}B,J-K). .    
\\ 
\hline 
\end{tabular}
\hspace*{4.5cm}\begin{minipage}{20cm} Table \ref{tab:chap4tab1}: Quantification of all transition kernels in the mega-matrix M (equation \ref{eq:chap4eq2}), including the definition of the various functions and parameters, for the palm \textit{Attalea butyracea}. The functions P indicate each separate stage and stage-transition kernel (eq. \ref{eq:chap4eq2}), with x, y and a denoting the size at time t, size at time t+1, and the total reproductive population. \end{minipage} 
\label{tab:chap4tab1}
\end{center}
\end{table}
\end{landscape}

\end{fullwidth}\textit{Equilibrium densities} ($a_{\lambda=1}$) were calculated by iteratively varying a (adult palm density) until the asymptotic $\lambda$ was 1. Numerically projecting population growth from both high or low densities until a stable density was reached yielded identical estimates of equilibrium densities (Fig S2). We compared $a_{\lambda=1}$ to the 2010 density of stemmed \textit{Attalea} in the 50ha plot, which was 0.64 palms/ha\sidenote[][-2cm]{density given on \\ \small http://ctfs.si.edu/} 
\vspace{0.5cm}

\begin{fullwidth}
\emph{Invasion growth rates}, $\lambda_i$, are the rate of increase when a species is extremely rare (i.e. when its density is so low that it has no impact on the resident community), often visualized as the intercept of population growth over focal species density (Adler \emph{et al.} 2007). We therefore estimated $\lambda_i$ as the rate of $\lambda$ when \textit{a}=0.  
\vspace{0.5cm}

\emph{Stabilization} ($\theta$) is a measure of how strongly \textit{Attalea} palms are regulated at their estimated equilibrium abundance. To quantify stabilization, we numerically estimated the slope of $\lambda$ over \textit{a} at $a_{\lambda=1}$.  
\vspace{0.5cm}

All density-dependent vital rates were incorporated into the IPM. We then estimated the relative importance of each density-dependent vital rate to each of the above statistics by iteratively switching density-dependence "off" - by setting parameters to zero - and recalculating statistics. 
Finally we quantified uncertainty in all statistics to underlying uncertainty in each model parameter, by sampling from the (weighted) posterior distributions of all model parameters and recalculating each statistic. This allowed us to calculate confidence intervals for the population statistics that incorporated uncertainty in each lower-level parameter.

\subsection{Sensitivity analysis} 
We used the density-dependent IPM to evaluate the relative importance of different parameters and associated stages and processes, using "perturbation analysis" (for a more detailed explanation see \citealt{Caswell2001}; \citealt{Zuidema2001}).  We quantified the proportional sensitivity as the effect of a 1\% change in each lower-level parameter. Furthermore, we calculated classic elasticity values, the proportional change in the population growth rate for a proportional change in any matrix element (\textit{sensu} \citealt{Caswell2001}), and evaluated how these changed with adult density. 


\begin{landscape}
\begin{table}
\begin{center}
\vspace*{2cm}
% latex table generated in R 3.3.1 by xtable 1.8-2 package
% Thu Sep  1 15:31:58 2016
\scalebox{0.75}{\hspace*{5cm}\begin{tabular}{p{.1cm}p{2.5cm}p{3.5cm}p{3.5cm}p{2.5cm}p{2.5cm}p{4cm}p{5.5cm}}
  \hline
 & Vital rate & $\beta_0$ (intercept) & $\beta_1$ (size/sd) & $\beta_2$ (intercept) & $\beta_3 (size)$ & $\beta_4$ (density) & Formula \\ 
  \hline
1 & Seedling establishment & \textbf{ -3.3 [-4.3,-2.2] } &  &  &  & \textbf{ -0.71 [-1.2,-0.23] } & $E(a) = logit^{-1} (\beta_0 + \beta_4 a)$ \\ 
  2 & Initial size & \textbf{ 3.7 [3.6,3.8] } & \textbf{ 5.6 [4.8,6.5] } &  &  & -0.0026 [-0.0059,0.00073] & $L(x_s) \sim \frac{1}{x_{s}/\sqrt{2\pi\beta_1^2}} e^{-\frac{(ln x_{s}-(\beta_0 + \beta_4 a))^2}{2\beta_{1}^2 }}$ \\ 
  3 & Seedling growth & \textbf{ 8.7 [6.7,11] } & \textbf{ -0.13 [-0.17,-0.085] } & \textbf{ -9.6 [-15,-3.9] } & \textbf{ 4.4 [2.9,5.9] } & -0.00088 [-0.0087,0.007] & $G_s(x_s,a) = \beta_0 + \beta_1 x_s + \beta_4 a$ \& \newline $V_s(x_s) = \beta_2 + \beta_3 x_s$ \\ 
  4 & Seedling survival & \textbf{ -0.82 [-1.5,-0.17] } & \textbf{ 0.045 [0.031,0.058] } &  &  & \textbf{ 0.023 [0.0056,0.041] } & $S_s(x_s,a) = logit^{-1} (\beta_0 + \beta_1 x_s + \beta_4 a)$ \\ 
  5 & Rosette growth & \textbf{ 9.4 [2.6,16] } & -0.017 [-0.055,0.021] & 0.082 [-3.3,3.4] & \textbf{ 0.19 [0.17,0.21] } & -0.041 [-0.15,0.066] & $G_r(x_r,a) = \beta_0 + \beta_1 x_r + \beta_4 a$ \& \newline $V_r(x_r) = \beta_2 + \beta_3 ln(x_r)$ \\ 
  6 & Rosette survival & \textbf{ 1.2 [0.31,2.2] } & \textbf{ 0.0054 [0.0016,0.0091] } &  &  & \textbf{ 0.033 [0.0018,0.065] } & $S_r(x_r,a) = logit^{-1} (\beta_0 + \beta_1 x_r + \beta_4 a)$ \\ 
  7 & Adult growth & \textbf{ -0.021 [-0.034,-0.0087] } & \textbf{ 2.6 [2.2,2.9] } & \textbf{ 0.59 [0.42,0.76] } &  & 0.0038 [-0.0013,0.0088] & $G_h(x_h,a) = \beta_0 + \beta_1 x_h + \beta_4 a$ \& \newline $\sigma = \beta_2$ \\ 
  8 & Adult survival & \textbf{ 7.2 [5.3,9] } & \textbf{ -0.25 [-0.36,-0.14] } &  &  & -0.0023 [-0.017,0.013] & $S_h(x_h,a) = logit^{-1} (\beta_0 + \beta_1 x_h + \beta_4 a)$ \\ 
  9 & Reproduction & \textbf{ -5.6 [-6.6,-4.6] } & \textbf{ 0.8 [0.67,0.93] } &  &  & 0.00022 [-0.011,0.011] & $S_h(x_h,a) = logit^{-1} (\beta_0 + \beta_1 x_h + \beta_4 a)$ \\ 
   \hline
\end{tabular}
}

\hspace*{4cm}\begin{minipage}{20.5cm}
\vspace{0.3cm}
Table \ref{tab:chap4tab2}: Density-dependent functions of vital rates of the palm \textit{Attalea butyrecea}. Shown are averaged model coefficients under size- and/or density-dependence, with 95\% credible intervals (CI). Bold face indicates significance (no overlap of CI with zero). In each function, parameter $\beta_0$ is the intercept, $\beta_1$ is the size effect, $\beta_2$ \& $\beta_3$ define heteroscedasticity in growth, and $\beta_4$ is the density-dependent effect.  See Table S5 for fitted parameters and credible intervals for the density independent vital rates, and Tables S2-4 for model formulas and priors. \end{minipage} 
\label{tab:chap4tab2}
\end{center}
\end{table}
\end{landscape}

\section{Results}

\subsection{Density dependence of vital rates}
We fit a total of 41 models over 12 vital rates (Fig. \ref{fig:chap4fig2};  Table \ref{tab:chap4tab1} \& S5). We found evidence for both positive and negative density dependence. Effect sizes were small except for seedling establishment (Table \ref{tab:chap4tab2} \& Fig S1). Of the nine vital rates with sufficiently large sample size to test for negative density dependence, three differed significantly from zero (Table \ref{tab:chap4tab2}).  These included seedling establishment (Fig. \ref{fig:chap4fig2}a), seedling survival (Fig. S1d) and rosette survival (Fig. S1g).  Seedling establishment declined exponentially with adult density (Fig. \ref{fig:chap4fig2}a, Table \ref{tab:chap4tab2}, $R^2$ = 0.51), while seedling survival and rosette survival were weakly linearly positively density-dependent (Fig. S1d, S1g).  Survival of seedlings and rosettes increased with size more strongly than with adult density (Fig. \ref{fig:chap7fig2}d \& g, Fig S1, table \ref{tab:chap4tab2}). 

The six remaining vital rates with sufficiently large sample size to test for negative density dependence were not significantly density-dependent (Table \ref{tab:chap4tab2}), although posterior model probabilities in favor of density-dependent effects were often substantial (table S5).  Four of these vital rates tended towards negative (Fig. S1 panels b. c. f and k) and two tended towards positive relationships with adult density (Fig. S1 panels j and l). The log-normal model gave the best fit for initial leaf length of recruiting seedlings (Fig \ref{fig:chap4fig2}b), with mean size of establishing seedlings decreasing slightly with adult density (Table \ref{tab:chap4tab1}, Fig. S1b). Leaf length growth for seedlings and rosettes tended to be negatively density-dependent, but adult height growth was weakly positively density-dependent (Table \ref{tab:chap4tab2}, Figs. S1c, f and j, respectively). Estimated variances for growth rates were found to be constant for bole height (Figs. \ref{fig:chap4fig2}j) and to increase with size (heteroscedasticity) for seedlings and rosette leaf length (Fig. \ref{fig:chap7fig2} C \& F). Survival tended to decrease sharply for bole heights > 15 meter (Fig \ref{fig:chap4fig2}k) and was weakly negatively density-dependent (Table \ref{tab:chap4tab2}, Fig. S1k). The fraction of reproductive individuals increased with size (Fig \ref{fig:chap4fig2}l) and very weakly with adult density (Table \ref{tab:chap4tab2}, Fig. S1l). 

Three vital rates had too few samples to even attempt estimates of density-dependence: the transition between seedlings and basal rosettes (Fig. \ref{fig:chap4fig2}e), the transition between rosettes and stemmed individuals (Fig \ref{fig:chap5fig2}H), and the initial stem height distribution (Fig. \ref{fig:chap4fig2}i), for which an exponential distribution gave the best fit. Uncertainty due to these rare transitions was quantified for population-level statistics (described below) and is presented in Table S6.  


\begin{figure*}
\hspace*{.75cm}\includegraphics[width=20cm,height=19.5cm]{../figures/Chap4Fig2.pdf}
\caption[Vital rates of the palm \textit{Attalea butyracea}][40cm]{.}  
\label{fig:chap4fig2}
\begin{minipage}{14cm}
\small
Figure \ref{fig:chap4fig2}: Vital rates of the palm \textit{Attalea butyracea} and their dependence on adult density (panel a) or size (b through l) across the entire life cycle on Barro Colorado Island, Panama. \footnotesize [Continued below].
\end{minipage}
\end{figure*}


\begin{minipage}{14cm}
\small
Figure \ref{fig:chap4fig2} (continued): Individual data points are depicted in histograms (B,I), or with green dots (all other panels); in (A) each data point represents one 4-ha plot; in all other panels each data point represents one individual in one census or census interval.  Solid blue lines show the best-fitting models; in cases in which the best-fit models included density-dependence, the plotted lines are for an average adult density of 8.5 stems ha$^{-1}$. Vital rates of the palm \textit{Attalea butyracea} and their dependence on adult density (panel a) or size (b through l) across the entire life cycle on Barro Colorado Island, Panama. Individual data points are depicted in histograms (B,I), or with green dots (all other panels); in (A) each data point represents one 4-ha plot; in all other panels each data point represents one individual in one census or census interval.  Solid blue lines show the best-fitting models; in cases in which the best-fit models included density-dependence, the plotted lines are for an average adult density of 8.5 stems ha$^{-1}$. \end{minipage}

\subsection{Population-level density dependence} 
The full IPM, including all nine density-dependent vital rates, showed strong negative density dependence of population growth (Fig. \ref{fig:chap4fig3}). The invasion growth rate ($\lambda_i$) - the per capita population growth at low density - was estimated to be 1.0482 per 2 years (CI 1.043-1.064; Fig. \ref{fig:chap4fig3}, left inset). The estimated equilibrium density ($a_{\lambda_=1}$) was 0.51 reproductive palms per hectare (CI 0.285-1.298; Fig. \ref{fig:chap4fig3}, right inset). This is approximately the density of \textit{Attalea} in the old-growth forest of BCI, which was 0.64 stemmed palms per hectare (Fig. \ref{fig:chap4fig3}, grey star in right inset). Estimated Stabilization ($\theta$) - the rate of change in population growth rate with population density at the point of equilibrium - was -0.0199 (CI -0.036, -0.0078). The negative sign indicates local stabilization around the equilibrium.  The model reproduced the population's stage structure well (Fig. S2). 

Sensitivity analysis indicated that negatively density-dependent seedling establishment was the main driver of the observed population-level density dependence (Fig \ref{fig:chap4fig4}). Removal of negatively density-dependent seedling establishment reduced invasion growth rates below 1 and removed stabilization (switching $\theta$ from a negative to a positive trend with increasing adult density), meaning that no equilibrium is possible (Fig. \ref{fig:chap4fig4}, final row).   The second most important parameter was stem height growth, which was weakly positively density-dependent (Fig. \ref{fig:chap4fig4}). Removal of the weak positive density-dependence of stem height growth decreased invasion growth rates and equilibrium densities, and strengthened stabilization (Fig. \ref{fig:chap4fig4}, third row). 


\begin{landscape}
\begin{figure*}
\hspace*{2.9cm}\includegraphics[width=18cm,height=18cm]{../figures/Chap4Fig3.pdf}
\caption[Per-capita population growth rate as a function of adult density][-13cm]{Per-capita population growth rate ($\lambda$; solid black lines) as a function of adult density of the palm Attalea butyracea on Barro Colorado Island, Panama, obtained by integrating all vital rates in an Integral Projection Model. The horizontal and vertical dashed lines represent the stable population growth rate ($\lambda$=1) and the estimated equilibrium density, respectively. The equilibrium density and invasion growth rate ($\lambda$ when adult density equals 0) are indicated with crosses, and shown in more detail in two insets detailing the equilibrium density and invasion growth rate (black dots), with confidence intervals around the estimates (solid lines). The star in the equilibrium-density inset is the density of stemmed individuals of \textit{Attalea} in the old-growth forest of BCI in 2010.  The black and grey areas on the x-axis indicate whether the projections are within or outside the range of the data, respectively.}
\label{fig:chap4fig3}
\end{figure*}
\end{landscape}


\begin{landscape}
\begin{figure*}
\hspace*{2.75cm} \includegraphics[width=18cm,height=18cm]{../figures/Chap4Fig4.pdf}
\caption[ Effect of density dependence of vital rates on the invasion growth rate ($\lambda_i$)][-11.5cm]{ Effect of density dependence of vital rates on the invasion growth rate ($\lambda_i$), equilibrium density ($a_{\lambda=1}$) and stabilization ($\theta$) of the palm \textit{Attalea butyracea} on Barro Colorado Island, Panama. The black vertical line indicates the unaltered values from the full model. Black circles show the change when density dependence is removed from the model for the corresponding vital rate by replacing the density-dependent parameter with zero. Black and grey horizontal bars indicate uncertainty (95\% CI) due to variation in each density-dependence parameter alone (black) and due to all parameters (grey). Vital rate functions are given in table \ref{tab:chap4tab1}. Density-dependence parameters always refer to $\beta_4$ in table \ref{eq:chap4eq2} and tables S2-4.}
\label{fig:chap4fig4}
\end{figure*}
\end{landscape}

\subsection{Relative importance}

There were large differences in the sensitivity of the invasion growth rate ($\lambda_i$), the equilibrium density ($a_{\lambda_=1}$), and the stabilization ($\theta$) to model parameters (Fig. S3). The fifteen parameters associated with the largest changes in $\lambda_i$, $\theta$ and $a_{\lambda_=1}$ for a given small proportional change (1\%) in their respective parameter values are shown in figure \ref{fig:chap4fig5} (see Fig S3 for all parameters).  Note, that Figure \ref{fig:chap4fig4} compares the effect of changing estimated parameter values to zero, while Figure \ref{fig:chap4fig5} compares the effect of changing estimated parameters values by plus and minus 1\%.  The parameters that govern the inflection points in transition rates (Fig \ref{fig:chap4fig2}e \& h) had the greatest influence on $\lambda_i$, $\theta$ and $a_{\lambda_=1}$ (Fig \ref{fig:chap4fig5}).  Biologically, these parameters represent the size at which 50\% of seedlings transition towards basal rosettes or 50\% of basal rosettes transition to stemmed palms (Fig \ref{fig:chap4fig2}e \& h). This means that these are crucial population bottlenecks that limit population growth. Indeed, we rarely observed these transitions in the field. Other important parameters influenced the rate at which seedlings transitioned to basal rosettes, the rate of increase in transition probability with size (parameter $\beta_1$ in table S2, which is the slope over size in the seedling to rosette transition on a logit scale), and the change in seedling growth with size (parameter $\beta_1$ in table S4, seedling size-dependent growth). The relative impact of all other parameters was much smaller (Fig S3). 

Classic elasticity values for matrix elements of the IPM, which were summed over sizes at each life stage, showed a clear ranking of stages, with simple seedlings, basal rosettes and stemmed palms in order of increasing influence on the population growth rate (Fig S4A). This ranking hardly changed with adult density (Fig S4B). Elasticity values varied with plant size within each life stage (Fig S4C-E).

\begin{landscape}
\begin{figure*}
\hspace*{2.75cm} \includegraphics[width=18cm,height=18cm]{../figures/Chap4Fig5.pdf}
\caption[Sensitivity of the invasion growth rate, equilibrium density and stabilization][-13cm]{Sensitivity of the invasion growth rate ($\lambda_i$), equilibrium density ($a_{\lambda_=1}$) and stabilization ($\theta$) of the palm \textit{Attalea butyracea} to small alterations in model parameters. The black vertical line and the solid circles in the uppermost row indicate the unaltered values from the full model.  Circles below the top row show the response of each statistic to 1\% change, either positive or negative (plus and minus signs).   The grey bars show the difference in $\lambda_i$, $\theta$ or $a_{\lambda_=1}$ between a 1\% increase and a 1\% decrease for each parameter. Only the top 15 most influential model parameters are shown. An analysis for all model parameters is given in figure S3. Vital rate functions can be found in Table 1. Parameters $\beta_0$ through $\beta_4$  refer to the model formulae in Tables 2 and S2-4. The $\beta_0$ always denote a (logit) intercept, $\beta_1$ a size-dependent parameter (slope over size) and $\beta_3$  a change in variance with size. $\beta_4$ always refers to a density-dependent parameter.}
\label{fig:chap4fig5}
\end{figure*}
\end{landscape}

\section{Discussion}

The mechanisms that regulate tree abundances in tropical forests are unknown \citep{Wright2002, Leigh2004, Siepielski2010, Vellend2010}. We found that populations of a widespread and "hyper-dominant" species (\textit{sensu}  \citealt{Steege2013}), the palm \textit{Attalea butyracea}, are strongly regulated by negative density dependence in seedling establishment, and are limited by crucial bottlenecks at juvenile life stages at Barro Colorado Island (BCI), Panama (Fig \ref{fig:chap4fig3}-\ref{fig:chap4fig5}). Earlier studies documented density dependence at several life stages \citep{Wright1983, Visser2011a, Jansen2014}, Here, we extend these earlier studies and show that equilibrium densities approximate densities found in old-growth forest on BCI, with significant stabilization around this density and projected invasion growth rates that will, given current conditions, lead to rapid population recovery after collapse. NDD was thus indeed strong enough to regulate the population.   Strong negative density-dependence in seedling establishment was common among tree species at BCI \citep[e.g.][]{Harms2000a}.  In recent global review, NDD was shown to have strong effects on seedlings \citep{Comita2014}, but weak or non-significant effects on seeds.  Previously, we showed that seed survival in \textit{Attalea} is density-independent because tri-trophic interactions among seeds, bruchid beetles and rodents counterbalanced density-dependent seed infestation by bruchids \citep{Visser2011a}.  Our results therefore appear to be typical for tropical trees.

The novel element of our results (table \ref{tab:chap4tab1} and Fig \ref{fig:chap4fig2}) is that NDD of seedling establishment is strong enough to regulate \textit{Attalea} populations. At first this may appear to be a paradox: why can NDD at the seedling stage lead to population regulation when demographic studies show that population growth in trees and other long-lived organisms is hardly sensitive to dynamics at these stages \citep{DeKroon2000, Visser2011}? The answer is simple:  small sensitivities can be countered by large effect sizes \citep[see e.g.][]{Zuidema2001}. Here, seedling establishment decreased exponentially with adult density, dropping an order of magnitude (Fig. \ref{fig:chap4fig2}a).  Nevertheless, the most influential factor limiting \textit{Attalea} population size was not seedling establishment, but the transition between juvenile stages.  

\subsection{Regulating versus limiting factors}
In theory, populations are subject to both limiting and regulating forces \citep[e.g.][]{Muller-Landau2016,Turchin1995}. Whereas a regulating force is a mechanism that results in NDD of population growth, a limiting force affects the abundance at which a population is regulated. Limiting forces may be density-dependent or -independent. The transitions between life-history stages are crucial population bottlenecks for \textit{Attalea}, being highly influential for local equilibrium abundances (Fig \ref{fig:chap4fig5}).  While seedling establishment regulates the population of \textit{Attalea}, the transition from seedling to rosette determines the abundance at which \textit{Attalea} is regulated, and our work suggests that it is far more influential.  

Very little is known about the growth requirements of \textit{Attalea} through ontogeny, though \citet{Araus1994} provide a key insight. \textit{Attalea} seedlings appear shade-tolerant, surviving in the shade for a period of 9 years ($\pm$ 25 cm tall under 1\% light; \citealt{Araus1994}). In gaps, however, \textit{Attalea} seedlings grow >4 m tall in the same period, and are apparently able to avoid dehydration and photoinhibition (under 70\% light). Hence, canopy gaps relax early bottlenecks in transitions between seedling, rosette and stemmed phases for \textit{Attalea}.  Thus, \textit{Attalea} is regulated by NDD of seedling establishment, but it is predominantly limited by light availability at the early life stages.
 
In tropical forest, gap creation frequently represents a major bottleneck to many species, and has a major impact on species relative abundance and the population structure of forests \citep{Brokaw1987, Alvarez-Buylla1994, Dalling1998, Farrior2016}.    Empirically, species abundance in at any site is likely regulated by negative density dependence, but can be limited more strongly by mechanisms independent of the local density of conspecifics - as our work quantitatively shows.

\subsection{Invasion growth rates}
The established criterion for stable coexistence is a species' ability to invade a community when it is rare \citep{Adler2010, Siepielski2010}. The invasion growth rate ($\lambda_i$) quantifies this, and is conceptually graphed as the intercept of the per capita population growth rate against focal species density \citep[e.g.][]{Adler2007}.  We quantified $\lambda_i$ of \textit{Attalea} from field data (e.g. Fig \ref{fig:chap4fig3}), but this required extrapolation (see horizontal axis in Fig \ref{fig:chap4fig3}). We note that, in a practical sense, the invasion growth rate is a theoretical concept that will always require extrapolation beyond the data \citep[as in e.g.][]{Adler2010} and can never be directly measured in the field.  

One problem is the precise definition of rarity. The literature defines $\lambda_i$ simply as the per capita population growth at "low abundance" \citep{Chesson2000, Siepielski2010}. This presents a practical problem, as we can expect the per-capita population growth to decline at low densities. For instance, when densities in the field are experimentally manipulated towards "low abundance", e.g. 1,2 or a handful of individuals, the per-capita  growth rate may decrease simply due to random mortality, or demographic stochasticity \citep{Lande1998}. This problem is well known from research on rare and endangered species. In conservation, the invasibility criterion is often replaced by the more practical probability of quasi-extinction, which is considered a better criterion for population viability, and more amenable to direct measurement \citep{Holmes2007}.

The expectation that demographic stochasticity reduces population growth when species are rare highlights a theoretical problem with the invasion growth rate: it may not be sufficient for coexistence. Effectively all species will be subject to demographic stochasticity at low abundance \citep[e.g.][]{Lande1998}, complicating the measurement of $\lambda_i$.  Species subject to Allee effects \citep[e.g.][]{Stephens1999} will by definition have negative per-capita population growth rates when sufficiently rare. Allee effects have been documented for a wide range of animal and plant species \citep{Courchamp1999}, and are known to complicate and confound invasion analyses \citep{Taylor2005}. Thus, our study highlights some concerns, both practical and theoretical, over the use of invasibility as the leading criterion for coexistence. 

\subsection{Clumps and groves}
We found that \textit{Attalea} was regulated to an equilibrium density of 0.23-1.3 stems ha$^{-1}$, which matches \textit{Attalea} densities in the old-growth forest of the Barro Colorado 50-ha plot (0.64 stems ha$^{-1}$).  Yet, we observed densities as high as 24 stems ha$^{-1}$. How did these high densities originate in the first place?  One plausible explanation is historical. The local people that once used this area influenced forests in numerous ways, often creating oligarchic forests dominated by useful trees \citep{Levis2012}.  Mono-dominant forest stands of \textit{Attalea speciosa} in Brazil have been linked to historic indigenous settlements \citep{Balee1993}. In Panama, \textit{Attalea butyracea} is associated with pre-Columbian and colonial settlements. The fronds are used for roofing, the fruits are used for oil (\textit{butyracea} from the latin "\textit{butyrum}" or butter), and the seeds were once used as food (R. Cooke pers. comm.). Thus, \textit{Attalea}, which is also potentially fire resistant \citep{Souza2004}, is often the only tree left standing in modern pastures in Panama. 

When in the 1880s, swidden agriculture along the Chagres river was abandoned for the construction of the Panama Canal, \textit{Attalea} was likely one of the few large-seeded forest species capable of rapidly colonizing abandoned fields and thereby attained high population density.  It is plausible that these dense stands remain today, as \textit{Attalea} population dynamics are typically slow. For instance, population growth was estimated to be 0.985/2 years within our most dense plots ($\approx$ 24 stems ha$^{-1}$). At this rate, the population would have a half-life of 90-100 years. It is thus conceivable that our densest plots are remnant groves formed more than 130 years ago \citep{Enders1935, Foster1982}. In contrast, densities in the 50-ha plot of 0.64 stems ha$^{-1}$ occur in old-growth forest which has been relatively undisturbed for approximately 1500 years \citep{Piperno1990}. We therefore conclude that the high \textit{Attalea} densities that we observed are consistent with our models. \textit{Attalea} attained high densities in a much younger forest, in conditions quite different from those we measured in recent years, and these relatively young stands have yet to reach equilibrium.  

\subsection{Habitat effects and positive density-dependence} 
We found evidence for both negative and positive density dependence, which raises the possibility that the gradient of adult density reflects favorable yet unknown habitat characteristics. We consider this unlikely, or at least unimportant, for three reasons. First, \textit{A. butyracea} is a habitat generalist, widespread across myriad habitats in Central and South America \citep{Steege2013}.  In island-wide studies of tree species distributions with respect to habitat variation on BCI, \textit{Attalea} abundance was unrelated to topography and soil properties, and was associated with young forest, where all our plots are situated \citep{Svenning2004,Svenning2006, Garzon-Lopez2014}. Finally, negative effects of conspecific abundance far outweigh positive effects in the IPMs (Fig. \ref{fig:chap4fig3}), indicating that any positive habitat effects are of secondary importance.

\subsection{Conclusions}
We found that negative density dependence in \textit{Attalea} was strongest at the early establishment and survival of seedlings, in line with previous studies \citep{Bagchi2010, Comita2014}. The observed negative density dependence was strong enough to regulate \textit{Attalea} populations well before \textit{Attalea} dominates forest stands, while allowing recovery from very low densities (Fig \ref{fig:chap4fig3} \& \ref{fig:chap4fig4}), thus meeting conditions for population regulation and species coexistence (Chesson 2000).  However, we also found crucial bottlenecks in the life cycle that heavily influenced equilibrium density (Fig \ref{fig:chap4fig5}). We conclude that the abundance of the common palm \textit{Attalea butyracea} is predominantly limited by light availability at the early life stages, and strongly regulated at this abundance by negative density-dependent seedling establishment. Our work suggests that density-dependent regulation is of secondary importance in determining relative abundances. We expect that \textit{Attalea} can attain far higher densities when and where the species is released from a key limiting bottleneck, which - for \textit{Attalea} - is achieved through environmental disturbance.  We conclude that patterns of diversity cannot be understood by investigating negative density-dependence alone.
\end{fullwidth}


\vspace*{20cm}

\begin{landscape}
\begin{figure}
\vspace*{-.6cm}\hspace*{4.4cm}\fbox{\includegraphics[width=19cm,height=22cm]{../figures/illustrations/chapter5.png}}
\hspace*{5cm}\begin{minipage}{18cm} 
\textit{ \footnotesize Seed mass and leaf size vary by 5-6 orders of magnitude
among species in neotropical forests. Wright \textit{et. al.} (2007) \nocite{Wright2007}}
\end{minipage}
\end{figure}
\end{landscape}
	
%------------------------------------------------
% Chapter 5
%--------------------------------------------------

\chapter{Functional traits as predictors of vital rates across the life cycle of tropical trees}
\label{ch5} 
\marginnote[-2cm]{Marco D. Visser, Marjolein Bruijning, S. Joseph Wright, Helene C. Muller-Landau,
Eelke Jongejans, Liza S. Comita and Hans de Kroon 
\textbf{Functional Ecology (2016) 30: 168-180}. Supplementary material can be found online: http://tinyurl.com/zghs7t8}

\section{Abstract} 
\begin{fullwidth}
\begin{enumerate}
\item The  'functional traits' of species have been heralded as promising predictors for species' demographic rates and life history. Multiple studies have linked plant species' demographic rates to commonly measured traits. However, predictive power is usually low - raising questions about the practical usefulness of traits - and analyses have been limited to size-independent univariate approaches restricted to a particular life stage.
\item Here we directly evaluated the predictive power of multiple traits simultaneously across the entire life cycle of 136 tropical tree species from central Panama. Using a model-averaging approach, we related wood density, seed mass, leaf mass per area and adult stature (maximum diameter) to onset of reproduction, seed production, seedling establishment, and growth and survival at seedling, sapling and adult stages.
\item Three of the four traits analysed here (wood density, seed mass and adult stature) typically explained 20-60\% of interspecific variation at a given vital rate and life stage. There were strong shifts in the importance of different traits throughout the life cycle of trees, with seed mass and adult stature being most important early in life, and wood density becoming most important after establishment. Every trait had opposing effects on different vital rates or at different life stages; for example, seed mass was associated with higher seedling establishment and lower initial survival, wood density with higher survival and lower growth, and adult stature with decreased juvenile but increased adult growth and survival. 
\item Forest dynamics are driven by the combined effects of all demographic processes across the full life cycle. Application of a multitrait and full-life cycle approach revealed the full role of key traits, and illuminated how trait effects on demography change through the life cycle. The effects of traits on one life stage or vital rate were sometimes offset by opposing effects at another stage, revealing the danger of drawing broad conclusions about functional trait - demography relationships from analysis of a single life stage or vital rate. Robust ecological and evolutionary conclusions about the roles of functional traits rely on an understanding of the relationships of traits to vital rates across all life stages.
\end{enumerate}


\section{Introduction}
Functional biology has raised the possibility that morphological and physiological traits, henceforth functional
traits, might be strongly related to interspecific variation in vital rates and serve as proxies for life history variation \citep{McIntyre1999, Westoby2002,  Westoby2006}. However, the predictive power of functional traits is often very low, raising questions about how 'functional' the selected traits really are \citep{Paine2015}. For
example, coefficients of determination ($R^2$) average just 0.08 for predictions of growth and mortality rates of
tropical trees \citep{Poorter2008, Wright2010, Iida2014a,Iida2014}. Such low predictive power strongly limits
the potential of traits to serve as proxies for life history variation or inform global vegetation models \citep{Cox2013, Friedlingstein2014}.

Previous studies in tropical forest trees have provided an incomplete picture of the role of traits in tree demography. Limitations include restriction of analyses to particular life stages, ignoring size dependence \citep{Poorter2008, Wright2010}, and/or consideration of only one trait at a time \citep{Iida2014,Iida2014a}. First, most previous studies consider a single vital rate and/or life stage, even though traits will generally have different roles and different predictive power for different demographic rates. For instance, seed mass is strongly negatively correlated with seed production \citep{Muller-Landau2008}, strongly positively correlated with
seedling establishment rates \citep{Moles2006}, and has weaker relationships with growth and survival later in
life \citep{Wright2010}.  Secondly, most previous studies ignore the size dependency of demographic rates, focusing instead on mean survival and growth over broad size classes (\citealt{Poorter2008, Wright2010}, but see \citealt{Iida2014a}). This overlooks important interactions between traits and size, as taller trees experience strikingly different resource and competitive conditions \citep{Poorter2005, Herault2011, Falster2016}. Thirdly, many studies consider only a single trait, thus inherently limiting the total explanatory power of traits \citep{Muller-Landau2008, Iida2014a,Iida2014}. Many traits will influence plant function simultaneously and each trait can be involved in multiple trade-offs with contradictory effects on vital rates \citep{Marks2006}. To improve understanding of how traits influence plant demography, we consider size dependence, evaluate trait-demography relationships across the entire life cycle, and consider multiple traits in a model-averaging framework.

We evaluate to what degree four key traits can explain variation in demography among tropical tree species. The
four key traits are adult stature, wood density, seed mass and specific leaf area, which provide largely independent
information about plant strategies (Table \ref{tab:chap5tab1}; \citealt{Westoby2002, Wright2007}). In contrast to previous work, our analyses are comprehensive including not only growth and mortality of trees \citep{Poorter2008,  Wright2010, Rueger2012, Iida2014}, but also reproductive schedules; seed production; and seedling establishment, growth and mortality. We use a model-averaging approach \citep{Burnham2002, Grueber2011} to simultaneously weigh the effects of all four traits on each vital rate for up to 136 tree and shrub species from Barro Colorado Island, Panama. We aim to quantify (i) which traits explain variation at different life stages; (ii) the predictive power of all traits combined to explain variation in all components of the life cycle; and (iii) the relative effect sizes of each trait at each life stage while mapping out contradictory effects across life stages.


\begin{landscape}
\begin{table}
\begin{center}
\footnotesize
\vspace*{4cm}
\hspace*{5cm}
\begin{tabular}{l l l p{5cm} p{4cm}}
\hline 
Trait & Mean (Sd) & Range & Associations & Sources \\ 
\hline 
Wood density (WD: $g/cm^3$) & 

$0.59 (0.14)$ & $(0.29,0.91)$ &

mechanical strength, vulnerability to hydraulic failure, defense against decay, growth-survival trade off & \citep{Chave2009,Larjavaara2010,Anten2010} \\ 

Seed mass (SM: $g$) & 

$0.06 (8.14)$ * & $(4.9 \cdot 10^{-5},22.87)$ & 

seed production, seed dispersal, seedling tolerance to stress, seedling competitive ability & \citep{Moles2004,Westoby2002,Muller-Landau2010} \\ 

Leaf mass per area (LMA: $g/m^2$) & 

$231.84 (314.89)$ & $(9.4,1891.35)$ & 

Leaf construction cost, photosynthetic capacity, respiration rates, leaf lifespan, leaf herbivory
 & \citep{Osnas2013,Wright2004,Westoby2002} \\ 

Adult stature ($D_{max}: mm$) & 

$398.97 (334.35)$ & $(12,1775.67)$ & 

Life-history variation & \citep{Kohyama1993,Westoby2002} \\ 
\hline 
\end{tabular} 
\label{tab:chap5tab1}
\hspace*{5cm}\begin{minipage}{19cm} Table \ref{tab:chap5tab1}: Details on the functional trait data, including units, mean and standard deviation, range, common ecological accociations and references.\\
 * geometric mean. \end{minipage} 
\end{center}
\end{table}
\end{landscape}


\section{Methods}
\subsection{Study site}
Our data are from the moist tropical forest of the 50-ha Forest Dynamics Plot (FDP) on Barro Colorado Island (BCI; (9$^{\circ}$9'N, 79$^{\circ}$51'W), Panama. Annual rainfall averages 2650 mm (since 1929), with a dry season between January and April, and temperature averages 27 $^{\circ}$C (see Leigh 1999, for details). 

\subsection{Vital rates}
We used five data sets to quantify vital rates for the entire life
cycle.

\begin{enumerate}
\item \textit{Trees}. In the FDP, all free-standing woody stems $\geq$1 cm diameter breast height (dbh, measured at 1.3 m) were censused in 1981-1982, 1985 and every 5 years thereafter. In each census, diameters of every stem are measured, and all new individuals are tagged, mapped, and identified to species. These censuses provide information on growth and survival for individuals $\geq$1 cm dbh (hereafter 'trees'). We analysed data from the 1990 to 2010 censuses, excluding earlier censuses because of small but important differences in measurement methods \citep{Condit1999, Rueger2009}.
\item \textit{Seed rain}. Seed rain has been recorded in 200 0.5-m$^2$ seed traps since January 1987 \citep{Wright2005a}. Traps are located in a stratified random manner along trails within the FDP. All reproductive parts (seeds, flowers, fruits and capsules) are identified to species and counted weekly (presence is recorded for flowers). We used seed data from 1993 to 2012, as these years correspond to records of newly recruiting seedlings (data set 3, below).

\item \textit{Small seedlings}. All seedlings and small saplings <1 cm dbh (with no limits on height) were censused annually in 600 1-m$^2$ seedling plots from 1994 through 2012. These plots are located 2 m from three sides of each of the 200 seed traps \citep{Wright2005a}.

\item \textit{Large seedlings}. Free-standing woody plants $\geq$20 cm in height and <1 cm dbh were censused in 20 000 1-m$^2$ seedling plots each year from 2001 through 2013, with the exceptions of 2005, 2007 and 2010 \citep{Comita2007, Comita2009}. These plots are located in a 5-m grid across the FDP. In each census,
the status (i.e. alive/dead) of previously tagged seedlings is checked, all individuals are measured for height (except in 2002, when only new recruits were measured), and new individuals are tagged and identified to species. In analyses of growth and survival all census intervals which include missing years were dropped (i.e. no intervals of >1 year are included for data set 4).

\item \textit{Reproductive status}. We assessed the reproductive status of 13 358 individual trees to quantify size-dependent probabilities of reproduction. For each species, a size-stratified sample of trees was randomly selected and visited during species-specific reproductive seasons. Reproductive status (fertile or sterile) was evaluated from the ground using binoculars. For eight dioecious species, we evaluated sex expression of all individuals within the FDP. Data were collected between January 1995 and January 1996 for 31 species \citep{Wright2005}, between
2005 and 2007 for 51 wind dispersed species, and between April 2011 and September 2014 for 81 species.
\end{enumerate}

\subsection{Trait data}
Trait data include seed mass (SM), leaf mass per area (LMA), adult size (D$_{max}$) and wood density (WD; Table \ref{tab:chap5tab1}; \citealt{Wright2010}). SM refers to endosperm and embryo dry mass determined after dissecting diaspores to isolate the endosperm and embryo. LMA was determined for shade leaves collected from the upper
canopy of the six smallest individuals of each species in the FDP. We could not use sun-exposed leaves as a basis of comparison because most FDP species are treelets that complete their entire life cycle in the shaded understorey \citep{King2005a}. D$_{max}$ is the mean dbh of the six largest individuals in the FDP (2005 census) and an additional 150 ha of mapped tree plots located within 30 km and mostly within 10 km of BCI. D$_{max}$ is well correlated with maximum tree height (r = 0.95 on a log-log scale). Species-specific WD was estimated from tree cores collected within 15 km of BCI, and was calculated as oven-dried (60 $\circ$C) mass divided by fresh volume (technically wood specific gravity). Further details can be found in \citet{Wright2010}. The four traits are largely independent of one another, with coefficients of determination (R$^2$ values) of 0.00068, 0.0056, 0.017, 0.052, 0.12 and 0.13 for LMA-D$_{max}$, SM-WD, WD-D$_{max}$, SM-D$_{max}$, LMA-SM and LMA-WD relationships, respectively \citep{Wright2010}.

We normalized trait values to enable model averaging, and facilitate comparison of effect sizes among traits with very different levels of interspecific variation \citep{Grueber2011}, using all 136 species evaluated here. Species-level trait values were normalized, with SM and D$_{max}$ first log-transformed, by subtracting mean trait values and then dividing by the standard deviation of the trait values (Table \ref{tab:chap5tab1}).

\subsection{Study species}
For each life stage and vital rate, we analysed all species with trait data and sufficient demographic data to ensure reasonable precision of species-specific vital rate estimates. Table \ref{tab:chap5tab2} gives exact selection criteria and the number of species in each analysis. Table S1 (Supporting information) gives the identities of the species in each analysis. Figure S1 shows the distribution of trait values across all species within each analysis.


\subsection{Fitting trait-based models for vital rates}
We evaluated relationships between size-dependent vital rates and traits, including trait-size interactions, using generalized linear mixed models (GLMMs), with species and individual included as random effects. The most complex full model had the following form:

\begin{equation}
y \sim \beta_0 + \beta_{1}s+\sum\limits_{i=1}^4 (\beta_{i+1}T_i + \beta_{i+5}T_is) + \epsilon_{sp} + \epsilon_{ind} + \epsilon_{residual}
\label{eq:chap5eq1}
\end{equation}

where \emph{y} is growth, (logit) survival or (logit) reproductive fraction; and \emph{s} is size in mm height, dbh or mm$^2$ basal area for analyses of seedlings, reproductive size or tree survival, and growth, respectively. Trait effects including their interactions with size are given by the expression in parentheses, where $T_i$ represents trait $i$ (corresponding to SM, WD, LMA or D$_{max}$). The random effects of species and individuals are denoted by $\epsilon_{sp}$ and $\epsilon_{ind}$ , respectively, and $\epsilon_{residual}$ is the residual error. For each size-dependent vital rate, we fit 82 possible models including eqn. \ref{eq:chap5eq1} and all subsets involving different combinations of the trait and trait by size effects (Table S2). Two vital rates, seed production and seedling establishment, were measured and analysed at species level, and we related these to traits directly using generalized linear models (GLMs) without size effects. Here too, we evaluated a suite of models including all subsets of SM, LMA, D$_{max}$and WD (16 models per vital rate; Tables S3 and S4). Details of model fitting for each vital rate follow.


\subsection{Reproduction}
The size-dependent probability of reproduction was evaluated with a logistic GLMM (eqn. \ref{eq:chap5eq1} with binomial error) using data set 5. 


\begin{landscape}
\begin{table}
\begin{center}
\footnotesize
\vspace*{4cm}
\hspace*{4.5cm}
\begin{tabular}{p{5cm}p{8cm}p{2.8cm}p{2cm}}
  \hline
Analysis & Selection criteria & Number of species & Years \\ 
  \hline
Reproduction (mm d.b.h.) & Reproductive status assessed for $>$ 20 trees. Species with too wide confidence intervals were excluded after visual inspection of fit. & 60 (8891) & 1995 - 2014 \\ 
  Seed production & Species had at least 30 seeds captured in traps, and must have been included in the reproductive analysis in order to estimate reproductive basal area & 38 (NA) & 1987 - 2010 \\ 
  Seedling establishment & 30 or more seedling recruits observed between 1995 and 2011, with $>$ 30 seeds observed for the fruiting years corresponding to 1995-2011 seedling recruitment (taking account of species-specific germination delays). & 69 (NA) & 1994 - 2011 \\ 
  Seedling survival (mm height) & $>$ 100 individuals in dataset & 80 (93082) & 2001 - 2013 \\ 
  Seedling growth (mm height) & $>$ 100 individuals in dataset & 80 (75990) & 2001 - 2013 \\ 
  Tree survival (mm d.b.h.) & $>$ 100 individuals in dataset & 117 (267469) & 1990 - 2010 \\ 
  Tree growth (mm$^2$ basal area) & $>$ 100 individuals in dataset & 117 (214373) & 1990 - 2010 \\ 
   \hline
\end{tabular}
\label{tab:chap5tab2}
\hspace*{4.5cm}\begin{minipage}{19.5cm} Table \ref{tab:chap5tab2}: Species selection criteria and sample sizes for each analysis. Each vital rate under the column "Analysis" is defined in the text, with corresponding units for size given in parenthesis when the analysis was done on individual data. Under "Number of species", the total number of species in each analysis is given with the total number of individuals between parentheses. This values is "NA" when the corresponding analysis concerns species-level data. The column "Years" gives the time span of data used in each analysis. A total of 136 unique species were included across all analyses. \end{minipage} 
\end{center}
\end{table}
\end{landscape}

\subsection{Seed production}
Species-specific seed production ($f_{seeds}$, seeds per year per m$^2$ of reproductive basal area) was quantified as the mean flux of seeds arriving (seeds per year per m$^2$ of trap area) divided by mean reproductive basal area density (m$^2$ of reproductive basal area per m$^2$ of plot area). We used seed trap and tree census data from 1993 through 2012. Reproductive basal area was calculated from the tree census data in combination with the fitted logistic models for size-dependent probability of reproduction. The logistic models predicted each individual's reproductive probability as a function of its size. We then weighted each individual's basal area by its reproductive probability to calculate total reproductive basal area. Total reproductive basal area was interpolated between FDP censuses to calculate annual values of $f_{seeds}$, which were then averaged over years to obtain a single mean value for each species. These simple estimates of seed production were qualitatively similar to more sophisticated
estimates obtained using inverse modelling (Text S1, Fig. S2). We chose to use the simple estimates because they were available for more species. Estimates of f$_{seeds}$ were then related to traits using linear regression.

\subsection{Seedling establishment}
Species-specific mean seed to seedling establishment probabilities were calculated as the mean flux of newly recruiting seedlings per year per m$^2$ in seedling plots in years 1995 to 2012 (data set 3) divided by the mean flux of seeds arriving per year per m$^2$ in seed traps for the corresponding fruiting years after accounting for germination delays \citep[][data set 2]{Wright2005a}. Seedling establishment rates were related to traits using GLMs (i.e. logit transform). 

\subsection{Growth}
Growth was modelled as height growth for seedlings (mm per year) and basal area growth for trees (mm$^2$ per year) using LMMs (eqn \ref{eq:chap5eq1}). We used basal area growth because general additive models (GAMs) showed that basal area growth was generally linearly related to size (mm$^2$ basal area). Growth rates were calculated as the difference in sizes divided by the time in years between censuses (data sets 1, 3 and 4). For data set 1, we excluded individuals marked as 'resprout', 'buttressed', 'leaning' and 'broken above 1.3 m' in each census, as well as those with growth rates more than four 4 standard deviations from the mean. These are likely measurement errors \citep{Rueger2011}.

\subsection{Survival}
The size-dependent (mm height and mm dbh for seedlings and trees, respectively) probability of survival was evaluated with a logistic GLMM (eqn 1 with binomial error) using data sets 1, 3 and 4.

\vspace*{.5cm}\hspace*{.4cm} We used model averaging to calculate average parameters \citep{Burnham2002}. All models were assigned a weight based on their AIC score and fitted parameters were averaged over the full set of models using these weights to obtain a final average model. The final average model provides a basis to compare effect sizes. Model averaging is superior to selecting the best model because models with similar fits are not ignored \citep{Burnham2002, Whittingham2006, Bolker2009}. For this reason, model averaging provides a more robust basis for inference and prediction, reducing bias in estimation of effect sizes, especially in cases where multiple variables influence the response variable \citep{Grueber2011}. This contrasts with stepwise multiple regression, which is seen as poorly suited to disentangle contributions of multiple traits to vital rates \citep{Whittingham2006}. We averaged parameters over all models having AIC weights >0
using the 'zero method' in which parameters are assigned the value zero where absent from models. This is a conservative approach to model averaging (i.e. leading to lower effect sizes) and is recommended when comparing effect sizes among variables \citep{Burnham2002, Grueber2011}. Confidence intervals for each weighted parameter were estimated following \citet{Buckland1997}.

We evaluated model fits to the full data sets, including variation among individuals, using marginal and conditional R$^2$ values developed for mixed-effects models \citep{Nakagawa2013}. These R$^2$ values provide information on how well the trait-based hierarchical models (the GLMMs) explain individual-level variation in vital rates over all species in the community. To evaluate linearity, we plotted model residuals against size for each model (Fig. S3). Residuals deviated from linearity for seedlings taller than 2.5 m and for trees with dbh >50 cm in the growth and survival analyses. To ensure linearity, we therefore excluded trees with dbh >50 cm and seedlings taller than 2.5 m
(corresponding to 0.41\% and 0.47\% of the data, respectively). Nonlinearity was not detected in our reproduction analysis. All analyses were performed in R 3.1.1 (R-core 2014), making use of the LME 4 package for mixed-effects models \citep{Bates2014}. An example R-code is provided for model averaging (Appendix S2).


\subsection{The power of traits to explain interspecific variation} 
We performed a second set of analyses to estimate the contributions of traits to explaining interspecific variation in demographic rates at particular sizes, and thereby to enable more direct comparisons of our results with earlier studies based on species-level estimates of vital rates \citep{Poorter2008, Wright2010, Iida2012}. We first calculated trait-based predictions for each species and each vital rate based on the fitted average models. We then compared these predictions with observed mean vital rates and calculated associated R$^2$ values. For seed production and seedling establishment, these comparisons used observed species-specific mean rates. For size-dependent vital rates (growth, survival, reproduction), we estimated species-specific moving averages using generalized additive models (GAMs; example code in Appendix S2), which make no prior assumptions on the functional shape of the relationship (Figs S4-S8). We then estimated size-specific R$^2$ values in three steps: (i) species-specific mean rates were given from GAM predictions for each vital rate at size points ranging from 0 to 1.5 m height for seedlings and from 1 to 25 cm dbh for trees (at these size ranges, analyses always included 15 or more species); (ii) demographic rates for these sizes were predicted using only fixed effects from each averaged trait models; and (iii) we calculated R$^2$ values for correlations between trait-based and GAM estimated mean rates at each size. To evaluate the predictive power of individual single traits, we repeated the calculation of R$^2$ values for single-trait average models for every trait, where single-trait models were based on averaging over models including only size, the single focal trait and/or the trait-size interaction as predictors. Single-trait models represent the best-case scenario, in terms of R$^2$, when using a single trait.

\subsection{Separate analysis at every size} 
Out of an abundance of caution, we performed a final analysis to guard against the possibility that underlying assumptions of linear relationships with size or the random effects structure in the mixed-effects model may unduly influence results. The (G)LMMs allow only linear relationships between vital rates and size, and might misstate the influence of predictors if there are underlying nonlinearities with size. Additionally, to improve computational
feasibility, we included only random species intercepts, but ignored random slopes with size, which potentially may impact effect sizes \citep{Schielzeth2009}. To address these concerns, we evaluated relationships between traits and species-specific vital rates separately for each size. We fit GLMs (sets of 16 models, as shown in Tables S3 and S4) to species mean rates (estimated from the GAMs) for seedlings between 0 and 1.5 m tall and trees between 1 and 25 cm dbh. We then compared effect sizes and R$^2$ values with the mixed-effect models (eqn \ref{eq:chap5eq1}). This analysis allows for varying intercepts, slopes and functional shapes between size and vital rates for each species. However, the analysis is also far less parsimonious and weights all species equally regardless of sample size and hence is likely to overestimate the strength of trait-vital rate relationships. When results are qualitatively similar this indicates no major problems with our assumptions. In this case, our initial analyses (eqn \ref{eq:chap5eq1}) are less biased for inference and yield the greatest predictive accuracy \citep{Gelman2006}.

\section{Results}
Our analyses included 136 different species, with 38 to 117 species for each vital rate (Table \ref{tab:chap5tab2}). The full range of trait values observed among BCI trees was well represented for each vital rate (Fig. S1). The species included a wide range of growth forms (shrubs to understorey and canopy trees), seed dispersal mechanisms (ballistic, wind, mammals and/or birds), and relative abundances [from 13.5\% (\textit{Faramea occidentalis}) to 0.29\% (\textit{Hampea appendiculata}) of all stems in the FDP]. WD, SM, LMA and D$_{max }$ varied by 0.5, 5.7, 2.3 and 2.2 orders of magnitude, respectively (Table \ref{tab:chap5tab1}).

\subsection{Seed production}
The average model for seed production was based on nine models with non zero weights (weights above 0.001;
Table S3), included all four traits and explained 65\% of interspecific variation (Table \ref{tab:chap5tab3}). Seed production was negatively related to SM (slope -1.13; Table \ref{tab:chap5tab3}, Fig \ref{fig:chap5fig1}a) with
all other traits having approximately 7-50 times smaller effect sizes (Table \ref{tab:chap5tab3}, Fig \ref{fig:chap5fig1}a).

\subsection{Seedling establishment}
The average model for seedling establishment was based on four models with nonzero weights (weights above
0.001; Table S4), included all four traits and explained 66\% of interspecific variation (Table \ref{tab:chap5tab3}). Seedling establishment increased with seed mass (slope 1.32; Table \ref{tab:chap5tab3}, Fig \ref{fig:chap5fig1}b) and decreased with D$_{max}$ (slope -1.06; Table \ref{tab:chap5tab3}, Fig \ref{fig:chap5fig1}b). LMA and WD hardly influenced seedling establishment rates (Table \ref{tab:chap5tab3}, Fig \ref{fig:chap5fig1}b).
\subsection{Seedling growth rates}
The average model for seedling growth rates was based on nine models with nonzero weights (weights above 0.001;
Table S5). It included all four size-trait interactions and explained 10\% of the individual-level variation, with the fixed effects of size, traits and their interactions explaining only 1\% of the individual growth variation (Table \ref{tab:chap5tab3}). Seedling growth rates decreased with SM and WD, with smaller effects for larger seedlings (Table \ref{tab:chap5tab3}, Fig \ref{fig:chap5fig2}a,b). LMA and D$_{max}$ hardly influenced seedling growth rates (Fig \ref{fig:chap5fig2}c,d).



\begin{landscape}
\begin{table}
\begin{center}
\footnotesize
\vspace*{1cm}
\hspace*{4.5cm}
\begin{tabular}{p{1.1cm}p{2.1cm}p{1.8cm}p{2cm}p{2cm}p{2cm}p{2cm}p{2cm}}
  \hline
 & Seed production & Seedling establishment & Seedling growth & Tree growth & Seedling survival & Tree survival & Reproduction \\ 
  \hline
Intercept & \textbf{-2.95 (0.0781)} & \textbf{-3.21 (0.0816)} & \textbf{42 (2.07)} & \textbf{108 (8.7)} & \textbf{1.31 (0.0582)} & \textbf{2.19 (0.0663)} & \textbf{-3.71 (0.0939)} \\ 
  SM & \textbf{-1.13 (0.0618)} & \textbf{1.32 (0.0866)} & \textbf{-11.1 (2.31)} & \textbf{-22.5 (9.5)} & \textbf{-0.22 (0.065)} & \textbf{0.387 (0.0722)} & \textbf{-0.226 (0.0724)} \\ 
  WD & 0.0152 (0.0312) & 0.0137 (0.0349) & \textbf{-13 (2.2)} & \textbf{-70.4 (9.79)} & \textbf{0.451 (0.0607)} & \textbf{0.536 (0.0748)} & 0.123 (0.0693) \\ 
  LMA & -0.166 (0.0844) & -0.00413 (0.035) & -0.0605 (2.32) & -2.19 (9.41) & \textbf{0.204 (0.0665)} & \textbf{0.249 (0.0724)} & \textbf{-0.281 (0.0875)} \\ 
  D$_{max}$ & -0.0247 (0.0448) & \textbf{-1.06 (0.0826)} & 0.434 (1.96) & \textbf{36.6 (9.97)} & \textbf{-0.348 (0.0578)} & \textbf{-0.297 (0.0758)} & \textbf{-0.386 (0.11)} \\ 
  size &  &  & \textbf{-0.0124 (0.000477)} & \textbf{0.00934 (7.63e-05)} & \textbf{0.00116 (2.34e-05)} & \textbf{-0.00282 (0.000116)} & \textbf{0.0174 (0.000224)} \\ 
  size:SM &  &  & \textbf{0.00479 (0.000332)} & \textbf{-0.00031 (3.14e-05)} & \textbf{0.000458 (2.1e-05)} & \textbf{-0.000976 (9.73e-05)} & \textbf{0.000675 (5.99e-05)} \\ 
  size:WD &  &  & \textbf{0.00496 (0.000596)} & \textbf{0.000858 (4.83e-05)} & 8.65e-06 (1.52e-05) & \textbf{-0.00121 (0.000107)} & -0.000109 (7.53e-05) \\ 
  size:LMA &  &  & 0.00112 (0.000772) & \textbf{-0.000946 (5.03e-05)} & \textbf{-0.000529 (3.04e-05)} & \textbf{-0.000642 (0.000104)} & -0.000143 (9.62e-05) \\ 
  size:D$_{max}$ &  &  & -1.98e-05 (2.29e-05) & \textbf{0.00255 (9.66e-05)} & \textbf{0.000272 (2.25e-05)} & \textbf{0.00581 (0.000161)} & \textbf{-0.00877 (0.000197)} \\ 
   \hline
$R^{2}_{size}$ &  &  & 0.0015 & 0.1676 & 0.0321 & 1e-04 & 0.1775 \\ 
  $R^{2}_{fixed}$ &  &  & 0.0084 & 0.2591 & 0.1581 & 0.0654 & 0.2469 \\ 
  $R^{2}_{species}$ &  &  & 0.0776 & 0.4015 & 0.2507 & 0.187 & 0.5063 \\ 
  $R^{2}_{individual}$ &  &  & 0.0343 & 0.5172 & 0.1581 & 0.0654 & 0.3451 \\ 
  $R^{2}_{full}$ & 0.54 & 0.57 & 0.1035 & 0.6595 & 0.2507 & 0.187 & 0.6045 \\ 
   \hline
\end{tabular}
\label{tab:chap5tab3}
\hspace*{4.5cm}\begin{minipage}{18cm} Table \ref{tab:chap5tab3}: Coefficients (and standard errors) from the full averaged model for each of the evaluated vital rates (columns). Bold face indicates significance ($\alpha<0.05$). The meassure of size was height (mm) for seedlings, dbh (mm) for survival and reproduction, and basal area for growth (mm$^2$). Complete lists of evaluated models with AIC values and AIC weights are given in tables S4-10. With the exception of seed production and seedling establishment, the $R^{2}$ values in the table are specific to mixed effect models \citep{Nakagawa2013}, and reflect the fit of the model including fixed effects of size only ("$R^{2}_{size}$"), fixed effects of size and traits ("$R^{2}_{fixed}$"), including fixed and species random effects ("$R^{2}_{species}$"), including fixed and individual random effects ("$R^{2}_{individual}$"), and the full mixed model ("$R^{2}_{full}$"). Trait variables were normalized prior to model fits. \end{minipage} 
\end{center}
\end{table}
\end{landscape}


\begin{figure*}
\hspace*{1cm}\includegraphics[width=15cm,height=16cm]{../figures/Chap5Fig1.pdf}
\caption[Fitted effects of deviations in seed mass (SM), wood density (WD), leaf mass per area (LMA) and maximum stature ($D_{max}$)][35cm]{.}
\label{fig:chap5fig1}
\hspace*{1cm} \begin{minipage}{12cm}
\small Figure \ref{fig:chap5fig1} Fitted effects of deviations in seed mass (SM), wood density (WD), leaf mass per area (LMA) and maximum stature ($D_{max}$) from their mean values on seed production (a), and the rate of seedling establishment (b) when other traits are held at their mean values. Predictions are plotted against standardized trait values (standard deviations from the mean). Observed trait value means and standard deviations are given in Table 1. Panels a and b show that traits can have opposing effects not only between life stages (SM in a and b) but also between different traits within a single life stage ($D_{max}$ and SM in Panel b).
\end{minipage} 
\end{figure*}


\begin{landscape}
\begin{figure*}
\hspace*{3cm}\includegraphics[width=17.5cm,height=19cm]{../figures/Chap5Fig2.pdf}
\caption[Fitted effects of each trait on size-dependent vital rates.][-14cm]{Fitted effects of each trait (columns) on size-dependent vital rates (rows) in the averaged models. The black lines present the vital rate-size relationships with all traits set to their mean values. The red plus signs and blue minus lines present the same relationships with one trait set to its mean plus or minus one standard deviation, respectively, and the three remaining traits set to their mean values. The trait whose value varies among the blue, black and red lines is named at the top of each column. Observed trait value means and standard deviations are given in Table 1. Grey text at the right outermost column gives the corresponding measure of size (in mm) for each row of panels. The figure shows that the effect sizes corresponding to different traits differ greatly throughout the life cycle of trees among species, as shown by the + and - lines from the top to the bottom of the graph.}
\label{fig:chap5fig2}
\end{figure*}
\end{landscape}



\subsection{Seedling survival rates}
The full model, including all size-trait interactions, contained 100\% of the weight for seedling survival (Table S6). This model explained 25\% of the individual-level variation in seedling survival, with the fixed effects of size, traits and their interactions explaining 16\% of the individual variation (Table \ref{tab:chap5tab3}). Seedling survival increased with WD and decreased with D$_{max}$, and these effects diminished with seedling size (Fig \ref{fig:chap5fig2}f,h). The direction of relationships between seedling survival and SM and LMA changed with
seedling size (Fig \ref{fig:chap5fig2}e,g). For small seedlings, survival was greater for species with smaller seeds and larger LMA. For larger seedlings, survival was greater for species with larger seeds and smaller LMA.

\subsection{Tree growth rates} 
The full model, including all size-trait interactions, also contained 100\% of the weight for tree growth rates
(Table S7). This model explained 66\% of the individual-level variation in tree growth, with the fixed effects of size, traits and their interactions explaining 26\% of the individual variation (Table \ref{tab:chap5tab3}). Tree growth rates decreased with WD and increased with D$_{max}$ (Table \ref{tab:chap5tab3}, Fig. \ref{fig:chap5fig2}j,l). The effect of WD decreased with tree size (Fig. \ref{fig:chap5fig2}j), and the effect of D$_{max}$ increased with tree size (Table \ref{tab:chap5tab3}, Fig. \ref{fig:chap5fig2}l). SM and LMA had much smaller effects on tree growth rates (Fig. \ref{fig:chap5fig2}i,k). 



\subsection{Tree survival rates}
The full model, including all size-trait interactions, contained 99.7\% of the weight for tree survival rates
(Table S8). The average model explained 19\% of the individual-level variation in seedling survival, with the fixed
effects of size, traits and their interactions explaining 7\% of the individual variation (Table \ref{tab:chap5tab3}). Tree survival was strongly influenced by D$_{max}$, and was generally larger for species with larger D$_{max}$ (Fig. \ref{fig:chap5fig2}p). Tree survival was also greater for species with larger SM, WD and LMA over nearly the full range of tree sizes, with the exception of the very largest individuals (Fig. \ref{fig:chap5fig2}m-o).

\subsection{Reproduction}
Nine models with nonzero weights contributed to the average model for reproduction (Table S9), which included all
size-trait interactions and explained 60\% of the individual-level variation, with the fixed effects of size, traits and their interactions explaining 25\% of the variation (Table \ref{tab:chap5tab3} and Table S9). D$_{max}$ had the largest effect on reproductive status, with larger-statured species becoming reproductive at larger sizes than smaller-statured species (Fig. \ref{fig:chap5fig2}t). The threshold size at which 50\% of individuals are reproductive is well predicted by the following simple equation: $R_{50} = \frac{1}{2} D_{max}$ (R$^2$ = 0.81; Fig. S9).

\subsection{The power of traits to explain interspecific variation}
Fig. \ref{fig:chap5fig3} summarizes the proportion of interspecific variation in vital rates explained by D$_{max}$, LMA, SM and WD throughout tree life cycles on BCI. These proportions are consistently higher than the proportions explained by traits and size in the GLMMs (Table \ref{tab:chap5tab3}) because the latter includes additional variation among individuals. The $R^2$ values for each trait separately are presented in Tables S10-12.

\subsection{Separate analysis at every size}
Traits had qualitatively similar influences on vital rates (Fig. S10) and explained similar proportions of interspecific variation (Fig. S11) in analyses that fit separate models for every size and in our main analyses.

\begin{landscape}
\begin{figure*}
\hspace*{3.5cm} \includegraphics[width=21cm,height=21cm]{../figures/Chap5Fig3.pdf}
\caption[The proportion of interspecific variation in various demographic rates explained by the four functional traits][35cm]{.}
\label{fig:chap5fig3}
\hspace*{4cm} \begin{minipage}{20cm}
\footnotesize Figure \ref{fig:chap5fig3} 
The proportion of interspecific variation in various demographic rates explained by the four functional traits throughout the life cycle, as measured by $R^2$ values. The rows show results for (top to bottom) size-dependent growth, size-dependent survival, and vital rates associated with reproduction. The $R^2$  value at the upper edge of the stacked colours represents the proportion of the total variation among species explained by the fixed effect terms in the full averaged model (i.e. including traits $D_{max}$, LMA, SM and WD), where the observed
values for each species are based on species-specific GAMs. The relative importance of different traits is indicated by the relative height of each colour band as a proportion of the total, with height scaled to the $R^2$  values for averaged models including only one trait (i.e. including only $D_{max}$, LMA, SM or WD; Tables S10-12). The number of species included in the analyses varies with size; for reference, the numbers of species included at various sizes are shown in solid circles above each graph. Variation in vital rates explained among species shows that seed mass is initially influential but diminishes in importance with wood density becoming more important.
\end{minipage}
\end{figure*}
\end{landscape}



\section{Discussion}
We systematically quantified trait-demography relationships across the entire life cycle of multiple co-occurring
species of tropical trees for the first time while incorporating individual-level, size-dependent variation in growth, survival and reproduction (Table \ref{tab:chap5tab3}, Fig. \ref{fig:chap5fig2}). Full models,
including random effects for individuals and species, explained 10-25\% of the overall variation for seedling
growth, seedling survival and tree survival (Table \ref{tab:chap5tab3}). Factors missing from our models clearly affect these three vital rates. Likely candidates include abiotic and biotic environmental variation associated with soils, local competitive effects and plant pests. The full models performed much better for tree growth rates and reproductive status, explaining 66\% and 60\% of overall variation, respectively. The fixed effects of size, wood density (WD), seed mass (SM), leaf mass per area (LMA), adult size (D$_{max}$) and interactions between size and traits explained 1-26\% of the variation observed over all individuals of all species for the five size-dependent vital rates (Table \ref{tab:chap5tab3}). Our trait-based average models explained between 4\% and 65\% of
interspecific variation in mean size-specific demographic rates, depending on the size and demographic rate, with
more variation explained for small than for large size classes (Fig. \ref{fig:chap5fig3}). In comparison with previous studies of species-level trait-demography relationships among tropical trees \citep{Poorter2008,  Wright2010, Iida2012, Iida2014}, our analyses provide clear improvements in predictive power and new insights into how effects vary with size.

\subsection{Wood density}
Higher WD is associated with higher resistance to hydraulic failure and to decay and with higher structural strength
for a given diameter, but at the cost of slower diameter growth rates \citep[reviewed by][]{Chave2009}. Previous
studies concur that wood density (WD) is the single trait best able to predict growth and survival among tropical
tree species, with coefficients of determination (R$^2$ ) averaging 0.093 ($\pm$0.077 SD) and 0.076 (($\pm$0.079 SD) for relationships with growth and survival, respectively (using the maximum R$^2$ values reported in \citealt{Poorter2008}; \citealt{Wright2010}; \citealt{Iida2014,Iida2014a}). Our size-specific coefficients of determination, which were always greater for wood density than for other traits for analyses of growth and survival (Table \ref{tab:chap5tab3}), are consistent with this conclusion. Wood density had progressively less predictive power for the survival and growth of larger trees, which is consistent with previous comparisons of broad sizes classes \citep{Poorter2008, Wright2010}. The benefits and costs associated with variation in WD affect basal area growth and survival directly, but have negligible effects on reproduction, seed production and seedling establishment (Fig. \ref{fig:chap5fig3}).

In interpreting the relationship of WD to growth and survival at constant diameter or constant height in this and
other studies, it is important to keep in mind that these relationships are dependent upon the size (and growth) currency used as the basis for comparison. The same 'size' in diameter or height is associated with larger biomass in
higher WD species, and the same biomass growth translates to less diameter, basal area, and height growth in higher
WD species \citep{Larjavaara2010}. Relationships of growth with WD may weaken or even disappear when growth is expressed on the basis of biomass instead of diameter \citep{Rueger2012}. Thus, the strong relationships
of growth with WD might in part be seen as an artefact of our choice of currency, much as mass-normalized leaf traits
show stronger interrelationships than area-normalized leaf traits \citep{Osnas2013}. Similarly, the use of height and
diameter as measures of size might introduce bias towards positive relationships between survival and WD, because
the same 'size' in diameter or height is associated with larger biomass in higher WD species, and survival increases
with biomass on BCI \citep{Muller-Landau2006}. Future analyses should evaluate how much variation in growth
and survival is explained by WD when these currency effects are eliminated.

\subsection{Seed mass}
Previous studies and our analyses (Fig. \ref{fig:chap5fig3}) concur that seed mass (SM) explains minimal variation in growth and survival among tropical trees (R$^2$ averages 0.002 $\pm$ 0.0022 SD and 0.101 $\pm$ 0.041 SD, respectively; Fig. \ref{fig:chap5fig3}; Poorter \emph{et al.} 2008; Wright \emph{et al.} 2010). This is unsurprising because seed reserves and direct effects of seed size are exhausted well before the large minimum sizes ($\geq$1 cm dbh) used to delimit trees. Those relationships with SM that remain at these large sizes reflect indirect effects, whose unknown causation must involve unrecognized correlations among traits and life histories.

SM has much stronger effects at the earliest stages of regeneration. SM was strongly negatively related to seed
production and strongly positively related to seedling establishment on BCI (Fig. \ref{fig:chap5fig1}). This is consistent with the well documented trade-off between seed quantity and per-seed investment \citep{Henery2001, Westoby2002, Muller-Landau2010}. SM also influenced seedling performance. Small-seeded species consistently grew faster than large-seeded species, although this effect diminished with seedling size (Fig. \ref{fig:chap5fig2}a). Small-seeded species also had lower survival rates than large-seeded species, but only at larger seedling
sizes. Among the smallest seedlings, small-seeded species actually had higher survival rates than large-seeded species (Fig. \ref{fig:chap5fig2}e). Environmental variation associated with germination sites likely confounds all of these relationships \citep{Lichstein2010,Muller-Landau2010}. Small-seeded species tend to establish in the least stressful locations, and those locations improve subsequent performance. Large-seeded species are able to establish in more competitive environments, which limit subsequent performance. On a population level, this causes smaller-seeded species to have lower establishment rates, because appropriate resource-rich sites are infrequent, and larger growth and survival rates than larger-seeded species that establish widely in less favourable environments. Clearly, observed vital rates are influenced not only by species traits, but by the habitats in which individuals are found (i.e. environmental filtering; \citealt{Lasky2013}). We discuss this issue in more detail below.


\subsection{Leaf mass per area}
Previous studies and our results concur that LMA explains minimal variation in growth and survival among tropical
trees (R$^2$ averages 0.08 $\pm$ 0.025 SD and 0.11 $\pm$ 0.071 SD, respectively; Fig. \ref{fig:chap5fig3}, this study; Poorter \emph{et al.} 2008; Wright \emph{et al.} 2010; \citealt{Iida2014a}). LMA is thought to be a minor factor affecting carbon gain in larger trees where crown architecture determines light interception \citep{Sterck2001}. Surprisingly, LMA was also vanishingly unimportant for seedling establishment, growth and survival (Figs 1b, 2c,g and 3). LMA had non-negligible correlations only with reproductive status, with species with lower LMA tending to reproduce at smaller sizes (Figs \ref{fig:chap5fig2} and \ref{fig:chap5fig3}). This relationship between LMA and reproductive status provides another example, as with seed mass, of an indirect effect, whose unknown causation must involve unrecognized correlations among traits and life histories.

There are at least two possible reasons for the negligible relationships of LMA with seedling growth and survival
(Figs \ref{fig:chap5fig2}c,g and \ref{fig:chap5fig3}). First, our LMA values are for saplings ($\geq$1 cm dbh), and seedling LMA values may differ. Previous studies show that leaf traits at juvenile and adult stages are generally strongly correlated \citep{Iida2014}; however, as LMA measurements for recently established seedlings remain rare, we cannot discount ontogenetic changes in LMA between seedlings and saplings \citep{Spasojevic2014}. LMA values determined for seedlings might yet yield stronger relationships between LMA and seedling performance. A second possible cause of weak relationships between LMA and growth and survival applies to both seedlings and trees. Costs and benefits associated with LMA variation might balance, yielding similar growth and survival rates on a population level. Low LMA species tend to have low construction costs and large leaf turnover rates, whereas high LMA species tend to have larger construction costs and lower leaf turnover rates \citep{Wright2004}. This may result in similar net carbon gain over time, minimizing potential relationships between LMA and demographic rates.

\subsection{Adult stature}
Previous studies and our results concur that larger-statured species have larger growth and survival rates among tropical trees (Fig. \ref{fig:chap5fig2}l,p). R$^2$ values average 0.11 $\pm$ 0.039 SD and 0.2 $\pm$ 0.066 SD, respectively (using the maximum of reported R$^2$ values; Fig. \ref{fig:chap5fig3}; Poorter \emph{\emph{et al.}} 2008; Wright \emph{et al.} 2010; Iida \emph{et al.} 2014a,b). This led to the conclusion that smaller-statured species have lower survival and growth rates as they may have less access to light (Poorter \emph{et al.} 2008; Wright \emph{et al.} 2010; Iida \emph{et al.} 2014a). We see opposing effects on seedlings, however, with taller species at an inherent disadvantage in early life \citep{King2005a}. Seedling establishment and survival decreased with increasing adult stature (Figs \ref{fig:chap5fig1}b and \ref{fig:chap5fig2}h). These opposing relationships between maximum size and performance between life stages are a condition for coexistence in the 'forest architecture hypothesis'. Large-statured species will out-compete smaller species, if not handicapped during establishment \citep{Kohyama1993}. The handicap observed among large-statured species during establishment might be related to a trade-off between allocation to traits and architectures that enable survival in the forest understorey vs. rapid vertical growth towards the canopy (for a more detailed discussion see \citealt{Kohyama2003}; \citealt{Poorter2005}).


D$_{max}$ was the single most influential trait explaining interspecific variation in reproductive size thresholds (cf. Fig. \ref{fig:chap5fig2}q-s vs. t). Species that grow larger only begin to reproduce at larger sizes. The threshold size for reproduction also increased for taller species in a Malaysian forest (Thomas 1996; \citealt{Davies1999}). For BCI trees, the simple equation $R_{50} = \frac{1}{2}D_{max}$ explains 81\% of interspecific variation in reproductive size thresholds (Fig. S9). That is, at the time individuals attain half of their species' maximum observed diameter, they have a 50\% probability of being reproductive. This is consistent with expectations from game
theory models \citep[reviewed by][]{Falster2003}, which predict that to maximize reproductive output individuals
should first invest heavily in growth, and then only after reaching an optimal size start to invest in reproduction.
Extending this, we would expect large-statured species to become reproductive only when they attain a position in the
forest canopy \citep{Thomas1996a, Zuidema2002}, while smaller understorey species will likely reproduce when they
obtain optimal crown depth \citep{Kohyama2003} or foliage cover. However, whether this optimal size on average
corresponds to the $\frac{1}{2}$ D$_{max}$ size threshold reported here remains to be tested.

\subsection{Traits as quantitative predictors of tree demography}
\citet{Westoby1998} suggested that specific leaf area (the inverse of LMA), seed mass and adult stature are three
readily measurable traits that represent important dimensions of variation in plant ecology. Our results suggest
that, at least for tropical forests, a more promising combination would be adult stature, seed mass and wood density. In our study system, adult stature, seed mass and wood density (but not LMA) each explained substantial
interspecific variation in particular vital rates or particular life stages (Fig. \ref{fig:chap5fig3}). Nevertheless, a large proportion of interspecific and individual variation remained unexplained (Table \ref{tab:chap5tab3} and Fig. \ref{fig:chap5fig3}). Why is the explained variation not higher, and what are the implications for the functional
trait research agenda?

Variance partitioning suggests that a considerable fraction of the unexplained interindividual variation is due to species effects not captured by the functional traits included in this study. Variance partitioning quantifies the unexplained variation and, thus, the potential for additional factors to explain variation at each grouping level. Mixed-effects models allow variance partitioning \citep{Bolker2009}, and we calculated conditional R$^2$ values to quantify variation explained at the individual and species levels \citep{Nakagawa2013}. The addition of species random effects increased conditional R$^2$ values on average by a factor of 3.2 (compare R$^2_{fixed}$ with R$^2_{species}$ in Table \ref{tab:chap5tab3}). This demonstrates that substantial unexplained interspecific variation remains, variation that might potentially be explained by additional traits. However, species-environment associations might also contribute to unexplained interspecific variation \citep{Messier2010, Lasky2013}.

Variation in plant performance among individuals depends strongly on local environment as well as on species traits and their interaction \citep{Uriarte2016}. Local environmental variation includes both abiotic factors such
as soil nutrients \citep{Condit2013} and water availability \citep{Comita2009a}, and biotic factors such as
local competitive neighbourhoods \citep{Uriarte2004}. Such environmental variation is not explicitly included in
our models, and thus contributes to variation among individuals captured here by individual-level random effects.
Environmental variation can also confound interspecific comparisons of vital rates \citep{Lichstein2010, McMahon2011,  Baraloto2012}. For example, in closed canopy tropical forests, small-seeded species only establish successfully in relatively high light microsites and higher light levels then contribute to higher initial growth and survival rates (Fig. \ref{fig:chap5fig2}a,e). Inclusion of more information on key environmental covariates for each individual would make it possible to control for any systematic differences in environments among species, and thereby better estimate the true effects of traits on performance \citep{Paine2011, Lasky2014}. Other sources of variation will remain stochastic and unpredictable. These include negative height growth due to stem breakage for seedlings (e.g. from falling branches), which happens regularly on BCI \citep{Paciorek2000}.


Our full-life cycle approach shows that individual traits can have opposing effects on different vital rates. This
raises the possibility that effects at one life stage or vital rate may be offset by opposite effects at another life stage or vital rate. Robust ecological and evolutionary conclusions, based on findings at single life stages or vital rates, will therefore depend on how effect sizes translate to net effects over the full life cycle. Trait-based models that map full-life cycle demographic patterns across trait axes (Figs \ref{fig:chap5fig1} and \ref{fig:chap5fig2}) may help resolve full-life cycle effects. For instance, when an empirical study finds an effect of increased seed production - which comes at the cost of reduced seed size - trait-based models can be used to calculate whether a net positive effect on seedling recruitment can be expected. A trait-based framework may even be
used in a population modelling context \citep[\textit{sensu}][]{Visser2011, Merow2014}, to calculate expected net effects when information on the whole life cycle is lacking. In this context, a full-life cycle trait-based approach may add value to ecological research by enabling robust assessments of relationships between traits and population fitness. 

The holy grail of the functional traits research agenda is the identification of easily measured traits that are good
predictors of life history and demographic performance, and the parameterization of associated models for inferring
life history and demography from these traits. With increasing amounts of vital rate variation across species explained, researchers start daring to ask the exciting question of whether trait-based vital rate models can be used to interpolate vital rates for species for which they have trait information but lack demographic data. Such interpolations would increase the number of species that can be included in Earth system models and community-wide studies. How certain do we have to be about these interpolations for the resulting multispecies analyses to be trustworthy? Our analyses still show a large proportion of unexplained interspecific variation, implying that trait-based models should be used with caution. Nevertheless, we have shown that a single individual measurement (size) and three species-level traits (D$_{max}$, SM and WD) explained on average 41\% of interspecific variation in vital rates (mean R$^2$ ranged between 0.11 and 0.66; Fig. \ref{fig:chap5fig3}), despite unquantified environmental effects on vital rates. This is quite remarkable and represents a substantial improvement over earlier studies. Functional biology may yet improve understanding of tropical forest dynamics.

\end{fullwidth}

\vspace*{20cm}

\begin{landscape}
\begin{figure}
\vspace*{-.6cm}\hspace*{4.4cm}\fbox{\includegraphics[width=19cm,height=22cm]{../figures/illustrations/chapter6.png}}
\hspace*{5cm}\begin{minipage}{18cm}
 \textit{ \footnotesize "There is much difficulty in understanding why hermaphroditic plants should ever have been rendered dioecious" - Charles Darwin (1877)}
\end{minipage}
\end{figure}
\end{landscape}
	
\chapter{Surviving in a cosexual world: a cost-benefit analysis of dioecy in tropical trees}
\label{ch6} 
\marginnote[-2.2cm]{Marjolein Bruijning, Marco D. Visser, Helene C. Muller-Landau, S. Joseph Wright,
Liza S. Comita, Stephen P. Hubbell, Hans de Kroon, Eelke Jongejans.\\\noindent \textbf{The American Naturalist, in press}. Supplementary material can be found online: http://tinyurl.com/zghs7t8}

\section{Abstract} 
\begin{fullwidth}
Dioecy has a demographic disadvantage compared to hermaphroditism: only about half of reproductive adults produce seeds. Dioecious species must therefore have fitness advantages to compensate for this cost through increased survival, growth and/or reproduction. We used a full life-cycle approach to quantify the demographic costs and benefits associated with dioecy, while controlling for demographic differences between dioecious and hermaphroditic species related to other functional traits. The advantage of this novel approach is that we can focus on the effect of breeding system across a diverse tree community. We built a composite integral projection model for hermaphroditic and dioecious tree populations from Barro Colorado Island (Panama) using long-term demographic and newly collected reproductive data. Integration of all costs and benefits showed that compensation was realized through increased seed production, resulting in no net costs of dioecy. Compensation was also facilitated by the low contribution of reproduction to population growth. Estimated positive effects of dioecy on tree growth and survival were small and insignificant for population growth rates. Our model revealed that, for long-lived organisms, the cost of having males is smaller than generally expected. Hence, little compensation is required for dioecious species to maintain population growth rates similar to hermaphroditic species.

\section{Introduction}

Although separate sexes are very common in mobile animals, dioecy is quite rare in plants. Only 6-10\% of angiosperms are dioecious \citep{Renner1995}. This rarity may reflect demographic costs associated with dioecy \citep{Bawa1980}: when a large proportion of the population are males, population growth is reduced - all else being equal - compared to when all individuals produce seeds. Having fewer seed-producing individuals is thus expected to constitute a substantial fitness disadvantage \citep{Queenborough2007, Vamosi2007, Vamosi2008}. Nonetheless, dioecy evolved independently at least 100 times \citep{Charlesworth2002, Barrett2010} and is represented among almost half of angiosperm families \citep{Renner1995}. It follows that in order to coexist with hermaphroditic species, dioecious species must have compensatory fitness advantages \citep{Bawa1980, Heilbuth2001}. Although increased genetic variation and decreased inbreeding depression may partly explain the benefits, it is difficult to explain dioecy simply as a mechanism favoring outbreeding given that most hermaphroditic species also have efficient outbreeding mechanisms \citep{Bawa1974, Renner1995, Freeman1997}. Identification of the benefits of dioecy and quantification of how dioecious species compensate for the costs is therefore an intriguing and longstanding challenge in ecology \citep[e.g.][]{Darwin1877, Opler1978, Armstrong1989, Renner1995, Freeman1997, Vamosi2008, Queenborough2009}.

Different compensating mechanisms have been proposed. Dioecious species may be able to invest more in reproduction, growth and/or survival because they produce only staminate or pistillate flowers \citep{Bawa1980, Queenborough2007, Queenborough2009, Vamosi2008}. Reproductive investment focused on one flower type may enable increased seed production, an earlier age of reproduction, more frequent reproduction, and/or production of better quality seeds leading to higher seedling establishment or survival \citep{Heilbuth2001, Queenborough2009}. However, the two studies that have compared these vital rates between dioecious and non-dioecious species have found little evidence that females in dioecious species benefit from the absence of costs associated with the production of staminate flowers \citep{Vamosi2008, Queenborough2009}. 

A challenge in making such comparisons between dioecious and non-dioecious species is that species invariably differ in other traits as well - traits that may also influence interspecific variation in vital rates. For example, seed production is closely related to seed mass \citep{Moles2004, Muller-Landau2008} and growth and mortality rates vary with wood density and adult stature \citep{Muller-Landau2004, Kraft2010}. Functional traits like seed mass, wood density and maximum size vary widely among tropical tree species and explain significant variation in vital rates \citep{Poorter2008, Wright2010, Visser2016}. Proper estimation of demographic effects of breeding systems in particular thus requires controlling for the effects of interspecific variation in these other traits. 
A complete picture of the benefits and costs of dioecy also requires integrating effects across the entire life cycle, something no previous study has done. A life cycle analysis is critical for two reasons. First, different life stages contribute differently to population growth. For this reason, the size of effects on individual vital rates provides limited insight into effects on population growth \citep{DeKroon1986, Ehrlen2003}. Second, breeding system affects multiple fitness components; benefits in one life stage may be offset by costs in other life stages \citep{Visser2016}. Simple comparisons of single vital rates between breeding systems, without considering correlations and tradeoffs, cannot demonstrate differences in population growth rates. Hence, we do not know how dioecious plant species survive in a co-sexual world.

Here, we perform an analysis that encompasses the entire life cycle to evaluate the fitness consequences of dioecy versus hermaphroditism for trees from the tropical moist forests of Barro Colorado Island (BCI), Panama. We take advantage of the relatively high occurrence of dioecy in tropical forests (21\% of species on BCI, \citealt{Croat1978}; 16-28\% elsewhere, \citealt{Opler1978, Zapata1978, Bawa1980}) and the demographic data available for many BCI tree species \citep{Hubbell1983, Hubbell1992, Wright2005a, Muller-Landau2008, Comita2010} and add individual data on fruiting or flowering probabilities for 23 hermaphroditic and 17 dioecious species and on sex for eight dioecious species. We model vital rates as a function of individual size and four species traits: breeding system, seed mass, wood density and adult stature \citep{Poorter2008, Wright2010, Visser2016}. We use the resulting vital rate functions to construct a 'composite' integral projection model (IPM) as a function of trait values (including breeding system), rather than separate IPMs for each species. Our composite approach quantifies the effects of breeding system across a tree community, while controlling for interspecific demographic variation related to other functional traits. We evaluate six hypotheses concerning possible benefits associated with dioecy, all premised on reallocation of resources consumed by male and female function in hermaphrodites \citep{Heilbuth2001,  Barot2004, Vamosi2008, Queenborough2009}. Dioecious females might allocate more resources to seed production than do hermaphrodites leading to (1) increased seed production, and/or higher quality seeds characterized by (2) greater seedling establishment, (3) higher seedling survival and/or (4) higher seedling growth rates. Dioecious males and females might both allocate more resources to their own maintenance than do hermaphrodites leading to greater (5) adult survival and/or (6) adult growth rates. We use vital rate functions to evaluate each of these six hypotheses individually. We then use the composite integral projection model to evaluate how each vital rate difference individually and all vital rate differences simultaneously affect intrinsic population growth rates.

\section{Methods}

\subsection{Model framework}
Species vary not only in breeding system (the focus of this study) but also in other traits that influence population dynamics, and variation due to other traits may dilute or mask the demographic effects of breeding system. For instance, plant breeding system is hypothesized to affect crop size \citep{Queenborough2009} but this effect will be modest compared to the much larger effect of seed mass, a major determinant of crop size \citep{Moles2006,Muller-Landau2008}. When comparing a random dioecious species with a random hermaphroditic species, the effect of the difference in breeding system cannot be distinguished from the effects of differences in other traits. We therefore used a community approach with as many species as possible, while including important species traits in the analyses to isolate the role of breeding system. Moreover, plant traits have opposing effects across the life-cycle and robust ecological inference therefore requires integration of the net effect through population models \citep{Visser2016}. To do so, we build a composite integral projection model to integrate the effects of breeding system across the life-cycle while simultaneously controlling for the effects of three other key traits. This enables a clear focus on one particular aspect of life history strategies: breeding system.

We begin by explaining the structure of the composite integral projection model (IPM) and vital rate functions. The vital rate functions incorporate three functional traits (seed mass, wood density or adult stature; see e.g. Visser \textit{et al.} 2016) and breeding system. Second, we explain our statistical approach for fitting the vital rates through model averaging, and provide an overview of the study site and data used. Finally, we explain the cost-benefit analysis of dioecy.  Here, we project the intrinsic growth rate, a measure for population fitness \citep{Charlesworth1980, Caswell2001, Metcalf2007}, of a hermaphrodite species at fixed trait values. We then systematically exchange hermaphroditic vital rate functions for dioecious vital rate functions. The resulting differences in intrinsic growth rate measure the population-level cost or benefit of dioecy while controlling for the effects of other traits. We then repeated this for different trait values to explore the possibility that dioecy might be favored over some subset of seed mass, wood density or adult stature values.

\subsection{Composite integral projection model}
Integral Projection Models (IPMs) describe population dynamics in discrete time, with continuous functions that relate vital rates (survival, growth, reproduction) to continuous state variables \citep{Easterling2000, Ellner2006}, here size and traits. We constructed a two-stage composite IPM across a wide range of species in which all vital rates were functions of both individual size and species-specific trait values \textbf{z}. Here, \textbf{z} is an array that contains trait values for seed mass (SM), wood density (WD), adult stature (D$_{max}$), breeding system (B; DIO or HERMA) and female proportion (set at 1 for hermaphrodites where all individuals produce seeds). We chose for a single-sex model rather than separately modeling males and females of dioecious species, as no data exist on mating and pollination rates for the community. These processes could play a different role in dioecious compared to hermaphroditic species. However, by directly looking at the seed production of females, the net outcome of these processes can be examined, without the need for explicitly modeling these unobserved rates.
	The two-stage composite IPM consisted of a seedling and tree stage, with height (mm) and DBH (mm) as continuous state variables, respectively. Seedling height ranged up to a species-specific C$_{seedling}$, and tree DBH ranged between 10 mm and a species-specific C$_{tree}$. C$_{seedling}$ is defined as the estimated height at which stems reach 10 mm DBH, and C$_{tree}$ is defined as the maximum DBH (D$_{max}$). The time step was annual (year t to year t+1).
The two-stage IPM consisted of four kernels that describe all size-dependent transitions within and between the two stages (Fig. \ref{fig:chap6fig1}). All four kernels included multiple underlying vital rate functions, as explained below (Eqs. \ref{eq:chap6eq3}-\ref{eq:chap6eq6}). The IPM can be written using two equations which are the addition of two kernels that describe transitions between sizes and stages. The first describes how the number of seedlings of various sizes in year t+1 is a function of contributions by seedlings (through survival) and trees (through reproduction) of the previous year:

\begin{equation}
S_{t+1}(u, \textbf{z}) = \int_{0.01}^{C_{seedling}} P_{seedling}(u,q,\textbf{z})S_t(q,\textbf{z})dq +
\int_{10}^{C_{tree}} F(u,x,\textbf{z})T_t(x,z) dx
\label{eq:chap6eq1} 
\end{equation}

Here, P$_{seedling}$(u,q, \textbf{z}) describes the survival and growth of seedlings with size q to size u, S$_t$(q, \textbf{z}) describes the size distribution of seedlings in year t, and F(u,x, \textbf{z}) describes the arrival of new seedlings with size u based on the size distribution of trees in year t (T$_t$(x, \textbf{z})) (Eqs. \ref{eq:chap6eq3}-\ref{eq:chap6eq4}).

The second equation describes the contributions to the number of trees of various sizes at time t+1:	

\begin{equation}
T_{t+1}(u, \textbf{z}) = \int_{10}^{C_{tree}} P_{tree}(y,x,\textbf{z})T_t(x,\textbf{z})dx +
\int_{0.01}^{C_{seedling}} P_{new tree}(y,q,\textbf{z})S_t(q,z) dq
\label{eq:chap6eq2} 
\end{equation}

Here P$_{tree}$(y,x, \textbf{z}) describes the growth and survival of trees with size x to size y, and P$_{new_tree}$(y,q, \textbf{z}) the transition from seedlings to trees (Eqs. \ref{eq:chap6eq5}-\ref{eq:chap6eq6}).
	To enable computation of the asymptotic population growth rate ($\lambda$), we discretized the composite IPM into a matrix of 800$\times$800 size classes (200 seedling height and 600 tree DBH classes). $\lambda$ is then calculated as the dominant eigenvalue of the matrix. Our size class resolution resulted in robust estimates of population growth rates ($\lambda$) with any increase in dimensions having negligible effect on $\lambda$ (< 0.00001). The phenomenon of 'eviction' - when individuals near the size boundaries are predicted to grow outside the range (Williams et al. 2012) -  was avoided by adding the probabilities of growing smaller or larger than the boundaries to the outer size classes. Size class widths in height and diameter are species-specific, depending on C$_{seedling}$ and C$_{tree}$, respectively, and are calculated as C$_{seedling}$/200 for seedlings and C$_{tree}$/600 for trees. At average trait values, this results in widths of 12 mm height and 0.25 mm DBH.


\begin{figure*}
\includegraphics[width=10cm,height=10cm]{../figures/Chap6Fig1.png}
\caption[The proportion of interspecific variation in various demographic rates explained by the four functional traits][-8cm]{Overview of the composite integral projection model (IPM), which includes four kernels (F = reproduction, P = survival and growth) that together represent all transitions of and reproductive contributions by individuals of size q (seedling height in mm) and size $\times$ (tree diameter in mm) at time t to the number of individuals of size u and y at time t+1. All kernels are functions of trait values z (including seed mass, wood density, maximum DBH, breeding system and - if dioecious - the proportion females).}\label{fig:chap6fig1}
\end{figure*}


\subsection{Underlying vital rate functions}
	The four kernels that form the composite IPM (Eqs. \ref{eq:chap6eq1} and \ref{eq:chap6eq2}) are constructed with multiple trait-dependent vital rate functions. The reproduction kernel gives the expected production of seedlings of sizes u at time t+1 by a tree of size x at time t:

\begin{equation}
F (u,x,\textbf{z})=p_{female} (\textbf{z}) p_{repr}(x,\textbf{z}) f_{seeds} (x,\textbf{z}) p_{establishment} (\textbf{z}) f_{dist}(u,\textbf{z})
\label{eq:chap6eq3} 
\end{equation}

where p$_{female}$ is the probability of being female (i.e., proportion females), p$_{repr}$ is the probability of being reproductive, f$_{seeds}$ is the expected seed production of an individual reproductive female tree, p$_{establishment}$ is the probability of seedling establishment and f$_{dist}$ is the offspring size distribution, giving the probability of a new recruit having initial height u. 
	The seedling kernel represents survival and growth of seedlings with height q at time t that remain seedlings with height u at time t+1: 
	
\begin{equation}
P_{seedling}(u,q,\textbf{z})=p_{sdl survival}(q,\textbf{z}) p_{sdl growth}(u,q,\textbf{z})
\label{eq:chap6eq4} 
\end{equation}

Please note that the P$_{seedling}$ kernel only includes new sizes up to the height C$_{seedling}$. Surviving seedlings that grow larger than C$_{seedling}$ are represented in the P$_{new tree}$ kernel, which represents transitions of seedlings with height q at time t to trees with diameter y at time t+1. This P$_{new tree}$ kernel is a function of seedling survival and growth and of the function for converting tree DBH to seedling heights (u=h(y)). This conversion is needed to properly integrate over the DBH classes, and thus translate the output of the seedling growth function from height to DBH:

\begin{equation}
P_{new tree}  (y,x,\textbf{z})=p_{sdl survival} (x,\textbf{z}) p_{sdl growth} (h(y),q,\textbf{z}) (dh/dy)
\label{eq:chap6eq5} 
\end{equation}

Finally, the tree kernel represents transitions of trees with DBH x at time t to DBH y at time t+1, and is a function of tree survival (P$_{tree}$ survival) and tree growth (P$_{tree}$ growth):

\begin{equation}
P_{tree} (y,x,\textbf{z})=p_{tree survival}(x,\textbf{z})p_{tree growth}(y,x,\textbf{z})
\label{eq:chap6eq6} 
\end{equation}

\subsection{Functional traits}
The composite IPM is a function of three key functional traits \citep{Visser2016} which included seed mass, wood density, and adult stature in addition to breeding system and a female proportion. Seed mass (SM; g) is the dry mass of the embryo and endosperm only. Wood density (WD; g/cm$^3$) is oven-dried mass divided by fresh volume (technically wood specific gravity). We averaged values for mass dried at 60$^\circ$C and at 100$^\circ$C. Maximum diameter (D$_{max}$; mm), henceforth referred to as maximum size, is a proxy for adult stature which correlates well with measured maximum heights (r=0.93, on a log-log scale). Preliminary analyses also included specific leaf area, but it explained minimal variation in vital rates and was therefore dropped \citep{Visser2016}. \citet{Wright2010} describe detailed methods and provide values for SM, WD, maximum height and specific leaf area. To facilitate comparison among effect sizes, we normalized SM, WD and D$_{max}$ (by subtracting the mean and dividing by one standard deviation) using all species in the databases (see below), with SM and D$_{max}$ log-transformed prior to normalization. Breeding system (B) is hermaphroditic (HERMA) or dioecious (DIO), and is obtained from \citet{Croat1978}. We excluded monoecious and polygamous species, which comprised 8\% and 4\% of tree species at Barro Colorado Island, respectively.  

We used data for 123 species. For each vital rate we included varying subsets of species depending on data requirements and availability at each life stage (Table \ref{tab:chap6tab1}, Appendix S1). This allowed us to maximize power by using all available data at each vital rate and life stage to test for the effect of breeding system while accounting for interspecific variation associated with traits. To ensure that estimated effects of breeding system were not biased by different trait or size distributions between breeding systems, we compared the distributions of wood density (WD), seed mass (SM), adult stature (D$_{max}$) and individual size between breeding systems for each vital rate (Appendix S2).

\subsection{Statistical fitting of vital rate functions}
For each vital rate function, we fit multiple models, one for each possible combination of traits and, where relevant, trait-size interactions, and then applied model averaging over these models (weighted by the Akaike Information Criterion, AIC) to calculate average parameters \citep{Burnham2002}. Model averaging is considered more robust for inference and prediction than simply using the single best model, because models with a similar fit to the best model are not ignored \citep{Burnham2002, Whittingham2006, Bolker2009}. On the condition that model assumptions are met (e.g. no multicollinearity or nonlinearities; \citealt{Cade2015}), model averaging provides a more robust basis for inference and prediction, reducing bias in estimation of effect sizes, especially in cases where multiple variables influence the response variable \citep{Grueber2011}. Our normalized traits were only weakly correlated (r$^2_{SM,WD}$ = 0.026; r$^2_{SM,D_{max}}$ = 0.021; r$^2_{WD,D_{max}}$ = 0.0019) and the statistical approach was also vigorously tested by \citet{Visser2016} who, using the same vital rates and trait data, found no issues with non-linearity, heteroscedasticity, nature of the random effect structure and lack of random slopes - issues which all can potentially impact effect sizes and robustness of conclusions. Parameters were averaged over all models, using the 'zero method'. This means that when a variable is not in a model, the parameter is assigned the value zero. This is a conservative approach, leading to lower effect sizes \citep{Burnham2002}. Standard errors were calculated for all parameters of the weighted model, following \citep{Buckland1997}. Analyses were performed with the lme4 package in R \citep{Bates2014, RDCT2016}.




%% LONG TABLE
\begin{landscape}
\advance\vsize4cm
\textheight=\vsize
\csname @colht\endcsname=\vsize
\footnotesize
\setlength\LTleft{-.3cm}
\setlength\LTright{-2cm}
\begin{longtable}{@{}p{6cm}p{6cm}p{6cm}p{1cm}}
\hline
Species selection criteria & Methodology & Fitted functions with estimated parameter values & Eq. \\ 
\hline 

\hline

\endfirsthead
\hline

Species selection criteria & Methodology & Fitted functions with estimated parameter values & Eq. \\ 

\hline
\endhead
\hline \multicolumn{4}{r}{\emph{Continued on next page}}
\endfoot
\endlastfoot

%%%%%%%%%%%%%%%%%%%%%%%%%%%%%%%%%%%%%%%%%%%%%%%%%%%%%%%%%%%%%%%%%%%%%%%%%%%%%%%%%%%%%%%%%%%%%%%%%%%%%%%%%%%%%%%%%%%%%%%%%%%%%%%%%%%%%%%%%%%%
%% NOW START THE LONG TABLE!

\multicolumn{4}{c}{\underline{\emph{Seed production probability (as a function of tree diameter)}}} \\
\vspace{-\baselineskip}\begin{itemize}[nosep]
\item Reproductive status assessed for >20 trees.
\item	Visual inspection confirmed a correlation between size and probability of reproduction (excludes 20 species with too wide confidence intervals, which were mostly species with very few observed reproductive individuals).
\item	$N_{DIO}$ = 17; $N_{HERMA}$ = 23.
\end{itemize}  & Mixed effects logistic regression models including species as random effects ($\sigma^2_{species}$ = 1.432). Details in Appendix S3.4. & 
\vspace{-\baselineskip}\begin{itemize}[nosep]
\item $p_{repr}(x,z) = \frac{1}{(1+e^{-v (x,z)}} z_{prop females}$ where 
\item $v(x,z)=-2.705-3.411 \cdot 10^{-2} \cdot z_{sm}+1.628e^{-2}  \cdot z_{wd} -7.56 \cdot 10^{-1} \cdot z_{D_{max}}-6.586 \cdot \cdot10^{-2} \cdot z_b+(2.461 \cdot 10^{-2}+1.697 \cdot 10^{-5} \cdot z_sm+7.231 \cdot 10^{-6} \cdot z_{wd} -1.182 \cdot 10^{-2} \cdot z_Dmax+ 1.185 \cdot 10^{-4} \cdot z_{b} \cdot) x$ 
\end{itemize}

& 1 \& 2 \\
\multicolumn{4}{c}{\underline{\emph{Seed production (per unit basal area of reproductive female or hermaphroditic tree)}}} \\
\vspace{-\baselineskip}\begin{itemize}[nosep]
\item Reproductive status assessed for >30 trees.
\item Visual inspection confirmed a correlation between size and probability of reproduction, excluding species with too wide confidence intervals.
\item >50 seeds were counted during 2008-2012.
\item Seeds were counted in >20 traps during 2008-2012. 
\item $N_{DIO}$ = 6; $N_{HERMA}$ = 12.
\end{itemize}  &
Inverse modeling to estimate species-specific seed production parameter $\beta$ (seeds cm$^{-2}$ basal area) \citep{Muller-Landau2008}.  
Details in Appendix S3.1. Generalized linear models relating $\beta$ to traits across species.
 & 
 \vspace{-\baselineskip}\begin{itemize}[nosep]
\item $f_{seeds} (x \rvert \beta)=e^{\beta} \cdot \frac{x}{2}^2 \cdot \pi $ where 
\item $\beta (z)=-1.363 +7.50 \cdot 10^{-2} \cdot z_{wd}+0.670  \cdot z_b -1.470 \cdot z_{sm}-0.649 \cdot z_{Dmax}$
\end{itemize}
& 3 \& 4 \\

\multicolumn{4}{c}{\underline{\emph{Seedling establishment (seedlings per seed)}}} \\
\vspace{-\baselineskip}\begin{itemize}[nosep]
\item >30 seedling recruits were observed in 1995-2011 (Wright et al. 2005b; Puerta-Pi\~nero et al. 2013).
\item >30 seeds were observed for the fruiting years corresponding to 1995-2011 seedling recruitment (taking account of germination delay).
\item	N$_{DIO}$ = 21; N$_{HERMA}$ = 44.
\end{itemize}  &
Species-specific seedling establishment rates Details in Appendix S3.2.
Generalized linear models relating seedling establishment to traits across species.
& 
$
p_{establishment} (z)=logit (-3.254+1.120 \cdot z_sm-1.316 \cdot 10^{-1} \cdot z_b +1.689 \cdot 10^{-1} \cdot z_{wd} -1.588 \cdot z_{Dmax})
$
& 5 \\

\multicolumn{4}{c}{\underline{\emph{Height distribution of recruits}}} \\
\vspace{-\baselineskip}\begin{itemize}[nosep]
\item >40 seedling recruits were observed during 1995-2011 \citep{Puerta-Pinero2013, Wright2005a}.
\item $N_{DIO}$ = 22; $N_{HERMA}$ = 47.
\end{itemize}  &
Maximum likelihood estimates of species-specific recruit height distributions (Weibull). 
Generalized linear models to relate estimated species-specific parameters to traits across species.
Details in Appendix S3.3. \vspace{1cm}
 & 
 \vspace{-\baselineskip}\begin{itemize}[nosep]

\item $f_{dist} (u \rvert k,\lambda)  \sim weibull(k,\lambda)$
where
\item $k(z)=2.122+0.517 \cdot z_{sm} -3.425 \cdot 10^{-2} \cdot z_{wd} +2.141 \cdot 10^{-2} \cdot z_{Dmax} +8.858 \cdot 10^{-2} \cdot z_b$
and 
\item $\lambda (z)= 109.3 + 20.72 \cdot z_{sm} -0.158 \cdot z_{wd} +8.344 \cdot z_{Dmax} -0.318 \cdot z_b$
\end{itemize}
& 6,7 \& 8 \\
\multicolumn{4}{c}{\underline{\emph{Seedling survival}}} \\
\vspace{-\baselineskip}\begin{itemize}[nosep]
\item	>100 individuals in seedling dataset (Comita et al. 2007).
\item	N$_{DIO}$ = 20; N$_{HERMA}$ = 42.
\end{itemize}  &
Mixed effects logistic regression models including species and individual as random effects ($\sigma^2_{species}$ = 0.3788; $\sigma^2_{tag}$ = 2.497 $\cdot$ 10$^{-11}$). Details in Appendix S3.5.
 & 
 \vspace{-\baselineskip}\begin{itemize}[nosep]
\item $p_{sdl survival} (q,z)=1/(1+exp⁡(-w (q,z)) )$
where
\item $w (q,z)= 1.505 -9.044  \cdot 10^{-2} \cdot z_{sm} + 0.4518 \cdot z_{wd} -0.4811  \cdot z_{Dmax} -3.279 \cdot 10^{-3} \cdot z_b+ (1.380 \cdot 10^{-3} -4.308 \cdot 10^{-4} \cdot z_sm+4.154  \cdot 10^{-5}  \cdot z_wd+4.427 \cdot 10^{-4} \cdot z_{Dmax}-2.348  \cdot 10^{-5} \cdot z_b ) \cdot q $
\end{itemize}
& 9 \& 10 \\

\multicolumn{4}{c}{\underline{\emph{Seedling growth}}} \\
\vspace{-\baselineskip}\begin{itemize}[nosep]
\item	>100 individuals in seedling dataset (Comita et al. 2007).
\item	N$_{DIO}$ = 20; N$_{HERMA}$ = 42.
\end{itemize}  &
Mixed effects logistic regression models including species and individual as random effects ($\sigma^2_{species}$ = 1.415 $\cdot$ 10$^3$; $\sigma^2_{tag}$ = 4.545 $\cdot$ 10$^3$). Species-specific variation (normally distributed) in growth was estimated and related to traits across species using General Linear Models. Details in Appendix S3.5.
& 
\vspace{-\baselineskip}\begin{itemize}[nosep]
\item $p_{sdl growth} (u,q,z)= N(u \rvert \gamma_{sdl growth} [q,z]+q, \sigma_{seedling} [z])$
where
\item $\gamma_{sdl growth} (q,z)= 50.02 - 15.36 \cdot z_{sm}-16.83 \cdot z_{wd} -1.719 \cdot z_b -4.593 \cdot z_{Dmax} + (-1.150 \cdot 10^{-2} +9.574  \cdot 10^{-3} \cdot z_{sm} +8.782  \cdot 10^{-7} \cdot z_{wd} -4.644 \cdot 10^{-6} \cdot z_{Dmax} -5.146  \cdot 10^{-4} \cdot z_b ) \cdot q$
\item $\epsilon(z) \sim N(\theta,\sigma_{seedling [z]}$
\item $\sigma_{seedling} (z)= 166.0 - 64.84 \cdot z_{sm} -22.51 \cdot z_{wd} +10.14  \cdot z_{Dmax} +9.925 \cdot z_b$ 
\end{itemize}

&
11, 12, 13 \& 14 \\

\multicolumn{4}{c}{\underline{\emph{Relation between DBH and seedling height (for transitions from seedlings to trees)}}} \\
\vspace{-\baselineskip}\begin{itemize}[nosep]
\item	>100 individuals in seedling dataset \citep{Comita2007}.
\item	Visual inspection confirmed a correlation between height and DBH.
\item	N$_{DIO}$ = 19; N$_{HERMA}$ = 41.
\end{itemize}  &
Mixed effects models including species and individual as random effects ($\sigma^2_{species}$ = 1.545; $\sigma^2_{tag}$ = 1.270).
Details in Appendix S3.6.
& 
$
g(q,z)=-4.233-1.1580 \cdot z_{sm} +0.2117 \cdot z_{wd} +0.5344 \cdot z_{Dmax} +2.065 \cdot z_b +(5.882 \cdot 10^{-3} +4.941 \cdot 10^{-4} \cdot z_{sm} -7.185  \cdot 10^{-4} \cdot z_{wd} -3.101  \cdot 10^{-6} \cdot z_{Dmax} -6.175  \cdot 10^{-4}  \cdot z_b ) \cdot q$ - Here, the upper bound on seedling size (i.e., the height at which a seedling becomes tree, DBH=10), $C_{seedling}$, is defined as: g(q,z)=10
& 15 \\


\multicolumn{4}{c}{\underline{\emph{Tree survival}}} \\
\vspace{-\baselineskip}\begin{itemize}[nosep]
\item	>100 individuals in tree dataset.
\item	N$_{DIO}$ = 34; N$_{HERMA}$ = 71.
\end{itemize}  &
Mixed effects logistic regression models including species and individual as random effects ($\sigma^2_{species}$ = 0.4471; $\sigma^2_{tag}$ = 2.467 $\cdot$ 10$^{-35}$). Details in Appendix S3.7. \vspace*{1.8cm}
 & 
 \vspace{-\baselineskip}\begin{itemize}[nosep]
\item $p_{tree survival} (x,z)=[ \frac{1}{1+e^{-p(x,z)}}]^{1/5}$
where
\item $p(x,z)=2.067+0.4420 \cdot z_{sm} +0.4925 \cdot z_{wd} -0.2146 \cdot z_{Dmax} -5.958 \cdot 10^{-2} \cdot z_b+(-4.830 \cdot 10^{-3} -1.179 \cdot 10^{-3}  \cdot z_{sm} -9.437 \cdot 10^{-4} \cdot z_{wd} +5.302 \cdot 10^{-3}  \cdot z_{Dmax} +1.160 \cdot 10^{-3} \cdot z_b ) \cdot x$ 
\end{itemize} 
& 16 \& 17 \\

\multicolumn{4}{c}{\underline{\emph{Tree growth}}} \\
\vspace{-\baselineskip}\begin{itemize}[nosep]
\item	>100 individuals in tree dataset.
\item	Excluding stems marked as 'resprout', 'buttressed', 'dead', 'leaning' and 'broken above 1.3 m'.
\item	Excluding growth measurements >4 standard deviations from mean growth.
\item	N$_{DIO}$ = 34; N$_{HERMA}$ = 71.
\end{itemize}  &
Mixed effects logistic regression models including species and individual as random effects ($\sigma^2_{species}$ = 0.1659; $\sigma^2_{tag}$ = 0.2646). Species-specific variation (normally distributed) in growth was estimated and related to traits across species using General Linear Models. Details in Appendix S3.7.
& 
\vspace{-\baselineskip}\begin{itemize}[nosep]
\item $p_{tree growth} (u,q,z)= N(u \rvert \gamma_{tree growth} [q,z]+q, \sigma_{tree} [z])$
where
\item $\gamma_{tree growth} (q,z)= 0.6893-0.153 \cdot z_{sm} -0.3125 \cdot z_{wd} -1.096  \cdot z_{Dmax} +2.852 \cdot 10^{-2} \cdot z_b +(5.683 \cdot 10^{-2}+6.862 \cdot 10^{-3} \cdot z_{sm} +2.216 \cdot 10^{-5} \cdot z_{wd} +4.018 \cdot 10^{-1} \cdot z_{Dmax} +7.971 \cdot 10^{-3} \cdot z_b ) \cdot log⁡(x)$
\item $\epsilon(z) \sim N(\theta,\sigma_{tree [z]}$
\item $\sigma_{tree} (z)= 1.140-0.115 \cdot z_{sm}-0.239 \cdot z_wd+0.3534  \cdot z_{Dmax} +4.356 \cdot 10^{-2} \cdot z_b$ 
\end{itemize}

&
18, 19, 20 \& 21 \\

%%%%%%%%%%%%%%%%%%%%%%%%%%%%%%%%%%%%%%%%%%%%%%%%%%%%%%%%%%%%%%%%%%%%%%%%%%%%%%%%%%%%%%%%%%%%%%%%%%%%%%%%%%%%%%%%%%%%%%%%%%%%%%%%%%%%%%%%%%%%
%% END OF LONG TABLE

\bottomrule
\hspace*{0cm}\begin{minipage}{20.5cm}
\vspace{0.2cm}
Table \ref{tab:chap6tab1}: Overview of the species selection criteria, fitting methodology and estimated parameters for the vital rate models. \textbf{z} is the array containing trait values, which includes the vectors  $z_{sm}$ = normalized (log-transformed) seed mass, $z_{wd}$ = normalized wood density, $z_{D_{max}}$ = normalized (log-transformed) maximum diameter, $z_{b}$ = breeding system (hermaphrodite [0] or dioecious [1]) and $z_{prop females}$= proportion females ranging between 0 and 1 for dioecious species and equal to 1 for hermaphrodite species. q and u are seedling height (in mm) at time t and time t+1; x and y are tree DBH (in mm) at time t and time t+1, respectively. $N_{DIO}$ and $N_{HERMA}$ are the number of dioecious and hermaphrodite species included for each vital rate analysis. $\sigma^2_{species}$ and $\sigma^2_{tag}$ are estimated random effects for species and individual, respectively. Estimated parameters shown in the fitted functions are weighted averages based on AIC values. Coefficients whose values are significantly different from zero (p<0.05) are highlighted in bold.
\end{minipage} 
\label{tab:chap6tab1}\\

\end{longtable}
\end{landscape}



The following three trait-dependent functions involved parameters calculated at the population-level for each species: seed production (f$_{seeds}$), seedling establishment (p$_{establishment}$) and the shape and scale parameters of initial seedling size distributions (f$_{dist}$) (detailed methods in Appendix S3.1-3.3; estimates for f$_{seeds}$ and p$_{establishment}$ in Appendix S4 and S5 respectively). For each of these species-level parameters, we used generalized linear models to fit all 16 possible models involving additive combinations of traits, with the most complex model being 

\begin{equation}
species value \sim B + SM + WD + D_{max}
\label{eq:chap6eq7}
\end{equation}

The six remaining trait-dependent functions involve parameters calculated at the individual-level using all species simultaneously: reproductive status (p$_{repr}$), seedling survival (p$_{sdl survival}$), seedling growth (p$_{sdl growth}$), tree survival (P$_{tree survival}$), tree growth (P$_{tree growth}$) and the relation between seedling DBH and height (g). For these functions, we included effects of individual size (seedling height or tree DBH), species traits, and interactions between size and traits (details in Appendix S3.4-3.7). We log-transformed initial sizes in tree growth models. We used linear mixed effects models with species and individual as random effects to fit models for all 82 possible combinations of traits and trait-size interactions, with the most complex model being 

\begin{equation}
individual value \sim size*B+size*SM+size*WD+size*D_{max}
\label{eq:chap6eq8}
\end{equation}

We additionally tested how well our model-averaged trait-based models captured interspecific variation by comparing trait-based model results with species-specific fits for each species individually, and calculating R$^2$ values across species (Appendix S6). We also tested whether the trait-based model-averaging approach was capable of capturing the well-known growth-survival trade-off \citep[][Appendix S7]{Wright2010}.

In order to test whether sex affected tree growth and whether sex ratios changed with size, we also fitted species-specific vital rate models for eight dioecious species for which we collected new data on individual reproduction status (see below). Tree growth was fitted as a linear model as a function of log-transformed DBH, with sex as a factor influencing both slope and intercept. We tested for size-dependency in sex ratios using logistic regressions of sex as a function of DBH. Finally, we fitted the probability of reproduction as a function of tree size (DBH) using logistic regression for each focal dioecious species.

\subsection{Study site and species}
Demographic data were collected in moist tropical forest in the 50-ha Forest Dynamics Plot (FDP) on Barro Colorado Island (BCI). BCI is a 1562-ha island in central Panama. Annual rainfall averages 2600 mm, and there is a pronounced dry season between January and April \citep{Leigh1999}. More information on BCI and associated datasets can be found in \citet{Croat1978, Condit1999, Leigh1999}. 

\subsection{Tree, seed, and seedling censuses}
All free-standing woody stems ('trees' hereafter) larger than 1 cm diameter breast height (DBH, measured at 1.3 m height) are measured, tagged, mapped and identified to species (in 1980-1982, 1985 - 2010 at 5 year intervals) \citep{Condit1998}. We excluded data from the first two censuses in analyses of tree growth and survival due to small but important differences in measurement methods \citep{Condit1998}. 

Fruit and seeds were identified to species in weekly censuses of 200 0.5-m$^2$ seed traps starting in January 1987 (details in \citealt{Wright2005a}). Fifty additional seed traps were established in newly formed canopy gaps between 2002 and 2004 \citep{Puerta-Pinero2013}. We use seed data from all 250 traps and inverse modeling (details in Appendix S3.1) to estimate seed production for 2008 through 2012, which corresponds to our reproductive tree censuses (see Methods: Reproductive status censuses).

Seedlings and new recruits were tagged and identified to species in annual censuses of 600 1-m$^2$ plots starting in 1994 (details \citealt{Wright2005a}). Our recruit size distributions and seed-to-seedling establishment probabilities are based on seedling recruits from 1995 through 2011 and seed production for corresponding fruiting years (after accounting for species-specific germination delays). 

All free-standing woody plants $\geq$ 20 cm tall and < 1 cm DBH were tagged, measured for height and identified to species in annual censuses of 20,000 1-m$^2$ seedling plots starting in 2001 (details in \citealt{Comita2007}). We estimated seedling survival and growth and the allometric relationship between seedling height and DBH using the 2001, 2002, 2003, and 2004 censuses. In 2002, the status (alive or dead) of previously tagged seedlings was recorded, but height was only measured for newly recruited seedlings.

\subsection{Reproductive status censuses}
We combined three censuses of the reproductive status of individual trees. Between January 1995 and January 1996, we censused all individuals of 15 species in the FDP and subsampled 16 species \citep{Wright2005}. Between April 2011 and January 2013, we subsampled 73 additional species. Between March and June 2012, we censused all individuals of eight dioecious species (\textit{Alchornea costaricensis, Triplaris cumingiana, Virola sebifera, Pouteria reticulata, Protium tenuifolium, Cecropia insignis, Cecropia obtusofolia} and \textit{Simarouba amara}). For species that were subsampled, trees were initially selected randomly from among those larger than estimated reproductive DBH thresholds and visited while flowering or fruiting. If trees of the estimated threshold size were reproductive, the minimum DBH threshold was lowered further. Reproductive status was evaluated from the ground, using binoculars where necessary, and was scored on a five-point scale (details in \citealt{Wright2005}). For dioecious species, sex expression was determined by viewing flowers in the crown using binoculars or examining abscised flowers on the ground. Flowering trees for which sex could not be determined were revisited during the fruiting season.

\subsection{Evaluating the population-level effects of dioecy}
We used the composite IPM to quantify the effects of dioecy on intrinsic population growth rates, r (ln[$\lambda$]).  We calculated r for a hermaphroditic species with average trait values (normalized SM, WD and D$_{max}$ set to 0), and compared it to r for a dioecious species with the same trait values assuming p$_{female}$ = 0.5 (incorporating dioecy effects on all vital rates, Table \ref{tab:chap6tab1}).  The difference between these two intrinsic growth rates ($\Delta$r) measures the net demographic effect of dioecy at average trait values (for similar life cycle analyses see \citealt{Metcalf2007} and \citealt{Visser2011}).  For both breeding systems and for average trait values, we calculated the proportional sensitivity (elasticity) of these population growth rates to small perturbations in the stage- and size-dependent vital rate functions using standard methods \citep{DeKroon1986, Easterling2000}. We also calculated per capita population growth rates for every trait combination observed among the BCI tree species included in our analyses, and compared population growth rates between dioecious species and hermaphrodites.
   
We further used the composite IPM to separate the contributions of dioecy effects associated with different vital rates.  To isolate the costs of having male individuals that do not produce seeds from any other differences in vital rates associated with dioecy, we calculated r for a species with p$_{female}$ = 0.5 and all other vital rates equal to mean values observed for hermaphrodites. The change in intrinsic growth rate ($\Delta$r) relative to a species with p$_{female}$ = 1 measures the demographic cost of simply halving the proportion of seed-producing individuals.  To further explore how the demographic cost of males depends on sex ratio, we calculated $\Delta$r with p$_{female}$ equal to the observed sex ratios of the eight focal dioecious species (Table \ref{tab:chap6tab2}). Note that here we assume that a change in sex ratio does not change pollination dynamics, which may be frequency dependent.
We then used the breeding system coefficients shown in Table \ref{tab:chap6tab1} to change the value of one vital rate at a time, keeping all other rates at the hermaphroditic level,  to the value observed for dioecious species with average trait values and recalculated $\Delta$r relative to a hermaphroditic species with the same trait values. The difference between these two intrinsic growth rates quantifies the demographic impact (cost or benefit) of dioecy on this vital rate alone at average trait values. 

We evaluated the robustness of these cost-benefit analyses given uncertainty in breeding system coefficients (intercept and, where relevant, slope). For each vital rate separately, we varied predictions up to one standard deviation away from their mean, constructed a new IPM, and recalculated $\Delta$r. To quantify uncertainty in the net effect, we sampled breeding system effects in all vital rates simultaneously. Coefficients were sampled from normal distributions with means and standard deviations estimated from model fits. We performed 100 resamples to calculate standard deviations for $\Delta$r.

To evaluate the extent to which costs and benefits of dioecy vary with trait values, we repeated the cost-benefit analysis for trait values up to 1 standard deviation away from the mean for each trait separately.  Finally, we calculated the sensitivity of r for changes in SD, SM and D$_{max}$ (details in Appendix S8).



\begin{landscape}
\begin{table}
\begin{center}
\footnotesize
\vspace*{4cm}
\hspace*{4.5cm}
\begin{tabular}{l l l l l l l l}
\hline 
Species & Minimum DBH (mm)	& \# Females	& \# Males &	\# Sterile trees	 & Proportion female &	Reproduction intercept	& Reproduction slope  \\ 
\hline 
\emph{Alchornea costaricensis} &	67	& 47 & 36 &	59 & 0.57 (0.47-0.67) & -3.12 $\pm$ 0.619 & 0.013 $\pm$ 0.0025 \\
\emph{Cecropia insignis} & 	11 &	102	& 118 & 97 & 0.46 (0.40-0.53) & -2.009 $\pm$ 0.368	& 0.021 $\pm$ 0.0030 \\
\emph{Cecropia obtusifolia} &	10 & 47 & 64 & 25	& 0.42 (0.34-0.52) & 0.852 $\pm$ 0.404 & 0.008 $\pm$ 0.0035 \\
\emph{Simarouba amara}	 & 100	& 31 & 32 & 71 & 0.49 (0.37-0.61) & -5.963 $\pm$ 1.023	& 0.019 $\pm$ 0.0035 \\
\emph{Pouteria reticulata} &	100	& 38 &	56	& 107 &	0.40 (0.31-0.51) &	-3.918 $\pm$ 0.566 &	0.012 $\pm$ 0.0019 \\
\emph{Protium tenuifolium} &	67  &	97 &	121 &	348 &	0.44 (0.38-0.51) &	-3.601 $\pm$ 0.290 &	0.019 $\pm$ 0.0017 \\
\emph{Triplaris cumingiana} &	60	& 46 &	34  &	51  &	0.58 (0.47-0.68) &	-2.274 $\pm$ 0.573 &	0.017 $\pm$ 0.0036 \\
\emph{Virola sebifera}	& 67 &	58 &	116 & 244 & 0.33 (0.27-0.41) & -4.271 $\pm$ 0.449	& 0.019 $\pm$ 0.0021 \\
\hline 
\end{tabular} 
\label{tab:chap6tab2}
\hspace*{4.6cm}\begin{minipage}{20cm} 
\vspace{0.1cm}
Table \ref{tab:chap6tab2}: Numbers of observed males and females of eight dioecious species for which sex information was collected. All individuals in the 50-ha plot larger than listed species-specific minimum DBH thresholds were visited. The proportion female is given with binomial confidence intervals. Logistic regressions were used to estimate the relation between DBH and species-specific reproductive probability (combining males and females). Estimated coefficients ($\pm$ 1 S.E.) are given: larger trees are significantly more likely to be reproductive in all species.
 \end{minipage} 
\end{center}
\end{table}
\end{landscape}


\section{Results}
For each vital rate, the average model is given in Table \ref{tab:chap6tab1}, and the five best performing models (based on AIC), weighted coefficients, standard errors and random effects variances are given in Appendix S9. Note that model averaging was done over all included models, and that parameter values which are given below are weighted parameters over all models.

\subsection{Sex ratios and reproductive probability}
The proportion of female individuals averaged 0.46 among the eight dioecious species with sex expression data (Table \ref{tab:chap6tab2}). \textit{Triplaris cumingiana} had the strongest female bias (0.57) and \textit{Virola sebifera} the strongest male bias (0.33). We found no consistent pattern of size dependency of sex ratios, with three species showing a negative relationship between DBH and proportion females, and one species a positive relationship (Appendix S10). Reproductive probability increased significantly with DBH for all eight species (Table \ref{tab:chap6tab2}).
The trait-based probability of reproduction (p$_{repr}$(x, \textbf{z})) was unaffected by breeding system (Fig. \ref{fig:chap6fig2}a). The best model did not include breeding system, and the total weight of models including breeding system was 0.32. The average model predicted a difference of just 1 mm in the DBH at which reproductive probability reached 50\% for hermaphroditic and dioecious species, and the narrow confidence intervals for relevant parameters included zero (see Fig. \ref{fig:chap6fig2}a, Eq. 8 in Table \ref{tab:chap6tab1}, Table S9.2). Observed species-specific values were well-predicted by the average model (Fig. S6.1, R$^2$=0.49 using only fixed effects).

\subsection{Seed production and dispersal}
Seed production (f$_{seeds}$(x, \textbf{z})) was significantly larger for dioecious than hermaphroditic species, supporting the first hypothesis for compensatory benefits of dioecy (Figs. \ref{fig:chap6fig2}b and \ref{fig:chap6fig3}a). The summed weight of all models including breeding system was 0.75, and dioecious species had higher seed production in the averaged model (effect DIO: 0.67 $\pm$ 0.26 s.e., Eq. 10 in Table \ref{tab:chap6tab1}, Table S9.4). The estimated effect in the average model implies that female individuals of dioecious species produce 95\% more seeds per unit reproductive basal area than hermaphrodites when controlling for seed mass and other species traits (calculated as $e^{0.67}$; see eqs. 9 and 10 in Table \ref{tab:chap6tab1}). The average model included a strong negative effect of seed mass (-1.47 $\pm$ 0.12 s.e.), a weak negative effect of D$_{max}$ (-0.649 $\pm$ 0.20 s.e.) and a weaker (insignificant) positive effect of WD (0.075$\pm$ 0.075 s.e.) on seed production. Species-level variation was well explained by the averaged model (R$^2$=0.77, Fig. \ref{fig:chap6fig3}a). There were no consistent differences in seed dispersal distances between hermaphrodite and dioecious species (Appendix S5). 



\begin{landscape}
\begin{figure*}
\hspace*{4cm} \includegraphics[width=20cm,height=16.5cm]{../figures/Chap6Fig2.pdf}
\caption[The proportion of interspecific variation in various demographic rates explained by the four functional traits][35cm]{.}
\label{fig:chap6fig2}
\hspace*{4cm} \begin{minipage}{20cm}
\footnotesize Figure \ref{fig:chap6fig2} 
Comparisons of estimated size-dependent vital rates between hypothetical hermaphroditic (dotted line) and dioecious (solid line) tree species having identical community average values of seed mass (SM), wood density (WD) and maximum size (D$_{max}$), with uncertainty in the dioecy effect (grey shading shows $\pm$1SD), as calculated from the trait-dependent models fitted to combined data for many tropical tree species (Table 1). In a,b,e,f sizes range between DBH=10 mm and DBH=158 mm, the maximum DBH for an average species (D$_{max}$ set to 0). C$_{seedling}$ equals 2420 mm height for an average species, and is the seedling height at which a seedling is predicted to have a DBH=10 mm, and enter the tree stage.
\end{minipage}
\end{figure*}
\end{landscape}

\begin{landscape}
\begin{figure*}
\hspace*{4cm} \includegraphics[width=20cm,height=17cm]{../figures/Chap6Fig3.pdf}
\caption[The proportion of interspecific variation in various demographic rates explained by the four functional traits][35cm]{.}
\label{fig:chap6fig3}
\hspace*{4cm} \begin{minipage}{20cm}
\footnotesize Figure \ref{fig:chap6fig3} 
Species-specific estimated seed production (a) and seedling establishment probability (b) versus seed mass for dioecious (filled circles, solid lines) and hermaphroditic (open circles, dashed lines) tree species. Plotted fecundities are standardized to those estimated for a fully reproductive tree with a DBH of 200 mm. Lines represent the fitted average models including all traits for dioecious (solid line) and hermaphroditic (dashed line) species for average WD and D$_{max}$. 
\end{minipage}
\end{figure*}
\end{landscape}

\subsection{Seedling establishment, growth and survival}
For seedling establishment (p$_{establishment}$( \textbf{z})), the total weight of models including breeding system was 0.38, and the full averaged model explained species-level variation relatively well (R$^2$ =0.62, Fig. \ref{fig:chap6fig3}b).  The estimated seed-to-seedling transition probability was 3.73\% for hermaphroditic species and 3.27\% for dioecious species at average trait values (Eq. 11 in Table \ref{tab:chap6tab1}, Table S9.10), thereby predicting 12\% lower seedling establishment in dioecious than in hermaphroditic species (calculated as 1-3.27/3.73). However, confidence interval on the dioecy effect included zero (effect DIO: -0.13 $\pm$ 0.12 s.e.).
For seedling growth (p$_{sdl growth}$(u,q, \textbf{z})), the total weight of models including breeding system was 0.92. Both uncertainty (Fig. \ref{fig:chap6fig2}d) and variation in species and individual random effects were large (Tables \ref{tab:chap6tab1} and S9.12). The average model (using fixed effects only) predicted greater seedling growth rates for hermaphroditic than dioecious species over all seedling sizes (Fig. \ref{fig:chap6fig2}d), although the wide confidence intervals on the dioecy effect include zero (effect DIO: -1.72 $\pm$ 8.26 s.e.). Only a small proportion of observed interspecific variation was explained by the average model (Fig. S6.3, R$^2$=0.08 using only fixed effects).  

For seedling survival (p$_{sdl survival}$(q, \textbf{z})), the total weight of models including breeding system was 0.38. The average model (using fixed effects only) predicted very similar seedling survival for hermaphroditic and dioecious species, and the narrow confidence intervals on the dioecy effect included zero (see Fig. \ref{fig:chap6fig2}c, Table S9.14).  The averaged model predicted observed species-level values reasonably well (Fig. S6.2, R$^2$=0.29 using only fixed effects).

\subsection{Tree growth and survival}
For tree growth (P$_{tree growth}$(y,x, \textbf{z})), the total weight of models including breeding system was 0.50. The average model (using fixed effects only) predicted observed species-level values well (Fig. S6.5, R$^2$=0.53). The average model predicted greater tree growth rates for dioecious than hermaphroditic species (effect DIO: 0.0285$\pm$0.030 s.e.) with the absolute growth advantage associated with dioecy increasing with size (effect DIO*log(size): 0.00797 $\pm$ 0.00497 s.e.), but confidence intervals include zero for both the additive effect of breeding system, and the size*breeding system interaction (Fig. \ref{fig:chap6fig2}f). The average model predicts annual DBH growth rates at 100 mm DBH of 0.95 and 1.02 mm per year for hermaphroditic and dioecious species, respectively (Eq. 25 in Table \ref{tab:chap6tab1}, Table S9.20).

For tree survival (P$_{tree survival}$(x, \textbf{z})), the total weight of models including breeding system was 1.00. The average model (using fixed effects only) predicted observed species-level values reasonably well (Fig. S6.4, R$^2$=0.29). The average model predicted a decreased intercept for dioecious species (effect DIO: -0.0596 $\pm$ 0.142 s.e.) and an increased slope (effect DIO*log(size): 0.00116 $\pm$ 0.000140 s.e.); note that the confidence interval of the effect of dioecy on the intercept effect easily overlaps zero, while the narrow confidence interval on the slope does not come close to overlapping zero  (Fig. \ref{fig:chap6fig2}e). This indicates that for species with average D$_{max}$, trees do not reach the sizes at which dioecy provides a demographic benefit (Fig. \ref{fig:chap6fig2}e). For average trait values, the model predicted annual survival probabilities at 100 mm DBH of 0.963 and 0.965 for hermaphroditic and dioecious species, respectively (Eq. 22 and 23 in Table \ref{tab:chap6tab1}, Table S9.22). However, differences in estimated survival become more pronounced at larger sizes. When D$_{max}$ is set at 1, predicted yearly survival probability of a 400 mm DBH tree is 0.976 for a hermaphroditic species, while 0.984 for a dioecious species.

When we included the sex of individuals as a factor in species-specific growth models, we found no consistent patterns across species and individually significant effects in only two species: \textit{Alchornea costaricensis} had a decreased slope in males, while \textit{Pouteria reticulata} had an increased slope in males (Appendix S11).

\subsection{Population-level effects of dioecy}
For average trait values, intrinsic growth rate r (ln$[\lambda]$) based on the composite IPM was 0.0176 per year for hermaphrodites, compared to 0.0179 per year for dioecious species with 50\% males. Population growth rates $\lambda$ were 1.0179 and 1.0180, respectively, suggesting annual increases of 1.78\% for hermaphrodite and 1.80\% for dioecious species. Elasticity values were largest for tree stages P$_{tree}$(y,x, \textbf{z}) (HERMA: 0.69; DIO: 0.66), intermediate for seedling stages P$_{seedling}$(u,q, \textbf{z}) and P$_{new tree}$(y,q, \textbf{z}) (HERMA: 0.31; DIO: 0.33) and very small for reproduction F(u,x, \textbf{z}) (HERMA: 0.01; DIO: 0.009). When constructing IPMs using combinations of trait values (WD, SM and D$_{max}$, B ) observed for real species occurring on BCI, 95\% of the simulated intrinsic growth rates of hermaphrodites ranged between -0.029 and 0.032 and of dioecious species between -0.0082 and 0.022, indicating fairly stable populations overall.

\begin{landscape}
\begin{figure*}
\hspace*{3cm} \includegraphics[width=23cm,height=23cm]{../figures/Chap6Fig4.pdf}
\caption[The estimated individual vital rate and total net effects of dioecy on population intrinsic growth rate ][35cm]{.}
\label{fig:chap6fig4}
\hspace*{3.4cm} \begin{minipage}{22.3cm}
\footnotesize Figure \ref{fig:chap6fig4} 
The estimated individual vital rate and total net effects of dioecy on population intrinsic growth rate (r = ln($\lambda$)) in tropical trees on BCI, as calculated from the composite integrated projection model (IPM, Table \ref{tab:chap6tab1}). The baseline point of comparison ($\Delta$r=0) is the estimated intrinsic growth rate of a hermaphroditic species.  In both panels, estimated consequences for intrinsic growth of dioecy effects on a single vital rate (grey dots) are calculated as the difference from the baseline of the r of a hypothetical species having one vital rate replaced by a dioecious vital rate (both with average trait values), while estimated net effects (grey dot, bottom row) are calculated as the difference when all vital rates are those of a dioecious species with average trait values.  The cost of males naturally varies with the sex ratio; values shown (A, top) are for the species-specific sex ratios observed here (ranging between 0.33 and 0.57; Table \ref{tab:chap6tab2}).   The error bars differ between panels. In panel A, uncertainty in dioecy coefficients is propagated to uncertainty in effects at the population level (black bars show effects under $\pm$1SD in the dioecy coefficients).  In panel B, the trait-dependence of effects of dioecy are estimated for each vital rate are shown by recalculating the baseline and $\Delta$r using trait values plus (or minus) one standard deviation away from their mean (blue = D$_{max}$, red = SM, green = WD).
\end{minipage}
\end{figure*}
\end{landscape}

When the effects of dioecy on individual vital rates were evaluated at average trait values for SM, WD and D$_{max}$, we found that the largest cost of dioecy was the presence of non-seed-producing individuals (males) in the population (Fig. \ref{fig:chap6fig4}a, grey circles indicate effects for average trait values). All else equal, having 50\% males decreased the intrinsic growth rate r by 0.00652, translating to a shift from a 1.78\% annual increase to a 1.11\% annual increase. Smaller costs were found when adding the dioecious-specific decrease in seedling establishment and seedling growth (decreasing r by 0.00125 and 0.00037, respectively). The largest benefit associated with dioecy was increased seed production in dioecious females compared to hermaphrodites, resulting in an increase of 0.00699 in r. The second largest benefit associated with dioecy concerned tree growth, resulting in an increase of 0.00151 in r. Total estimated benefits compensate for total estimated costs in the model parameterized for dioecious species with a sex ratio of 50\% (total $\Delta$r=0.0000321, which represents a shift from +1.78\% to +1.80\% annual population growth). Sex-ratio had the predictable effect of inflating costs with increasing male bias; for observed sex ratios, $\Delta$r ranged between -0.00526 (57\% females) and -0.010 (33\% females; grey dots in upper row in Fig. \ref{fig:chap6fig4}a).

When integrating uncertainty in breeding system effects in the IPMs, we found that many effects on the population growth rate included zero within 1 standard deviation (Fig. \ref{fig:chap6fig4}a). Indeed, the sole effects that were significantly different from zero were the presence of males and the enhanced seed production in females. When incorporating uncertainty in dioecious parameters across all vital rates, the net effect on a population level included zero within 1 standard deviation (bottom row in Fig. \ref{fig:chap6fig4}a).

The estimated effects of dioecy on population growth rates differed systematically with functional traits in many cases (Fig. \ref{fig:chap6fig4}b, Appendix S8). The overall effects of dioecy varied most strongly with maximum adult stature (D$_{max}$). Larger stature was associated with lower costs for having males, lower benefits in seed production, higher benefits in tree survival, lower benefits in seedling transition, and more positive overall net effects. Individual demographic costs and benefits also varied to a lesser degree with seed mass (SM) and wood density (WD), but net effects were little affected.  Higher values of SM and WD were associated with lower costs for males and lower benefits in seed production.


\section{Discussion}
Our study provides the most complete quantification to date of the various demographic costs and benefits of dioecy relative to hermaphroditism, based on 20 years of demographic data on the BCI tree community. Our evaluation with a composite IPM that accounted for the effects of three other key life history traits suggested that, on average, the demographic benefits offset the costs when integrated over all life stages, as one might expect based on the continued persistence of dioecious species. We show that there are both benefits and costs at different life stages, with strikingly different elasticities. Benefits or costs at a single stage were not indicative of total population-level effects, illustrating the importance of full life cycle analyses. In our population-level analyses, the main cost associated with dioecy was the reduced number of seed-producing trees, and the largest benefit was the greater seed production of dioecious females. Our best estimates of dioecy effects suggest that dioecious species benefit from increased seed production (supporting hypothesis 1) and, to a lesser extent, tree growth and survival (supporting hypotheses 5 and 6), but do not benefit from increased seedling establishment, seedling growth or seed survival (contrary to hypotheses 2-4).  Interestingly, costs and benefits were estimated to vary considerably depending on other species traits, with large-statured species in particular estimated to have more positive net effects of dioecy (Fig. \ref{fig:chap6fig4}b).  

Despite the large datasets brought to bear here, there remained considerable uncertainty in all estimated effects of dioecy (Figs. \ref{fig:chap6fig2} and \ref{fig:chap6fig4}a). We interpret this as reflecting not only the inherent noisiness of demographic data, but also the fundamentally small size of the effects of dioecy on population growth rates in comparison with other factors.  Both on a population and vital rate level, the demographic effects of breeding system were relatively small compared to the effects of SM, WD and D$_{max}$ (Fig. S8.1 and coefficients in Table \ref{tab:chap6tab1}). Moreover, a substantial proportion of vital rate variance was captured by the random effects for species (Appendix S9), indicating an important role for other unmeasured traits and trait-environment interactions (see also Visser \textit{et al.} 2016). This is consistent with the idea that factors influencing reproduction should have a relatively small impact on long-lived organisms (Visser \textit{et al.} 2011), an issue discussed further below.

\subsection{Variation in sex ratios}
For six of eight dioecious species, we observed more reproductive males than females, which is in accordance with the general finding of male-biased populations in dioecious tropical forest trees \citep{Opler1978, Queenborough2007}. However, this male bias was statistically significant in only one species, \textit{Virola sebifera }(116 males; 58 females), which belongs to a genus and family (Myristicaceae) previously shown to contain male-biased species \citep{Queenborough2007}. \textit{Triplaris cumingiana} was found to be female-biased (although not significantly), in accordance with studies on a congener, \textit{Triplaris americana} \citep{Bawa1977, Melampy1977}. We found no consistent patterns in the size-dependency of sex ratios across these eight species (Appendix S10), in agreement with Queenborough \textit{et al.} (2007).

\subsection{Seed production}
Dioecious species are hypothesized to produce more seeds per female compared to hermaphroditic species because they do not invest in male flowers \citep{Heilbuth2001, Barot2004, Vamosi2008}. Consistent with this hypotheses, females in gynodioecious species, which have both females and hermaphrodites \citep{Bawa1981}, produce up to 70\% more seeds than do hermaphrodites \citep{Ashman1994, Ashman1999, Asikainen2003, Spigler2011}.  However, evaluation of this hypothesis for purely dioecious species is more difficult. Ours is the first study to compare seed production of dioecious and hermaphroditic species while correcting for the well-known trade-off between seed number and seed mass \citep{Moles2004, Muller-Landau2008}. 

Our results suggest that dioecious species indeed benefit from higher seed production. Dioecious females produced on average 95\% more seeds than did hermaphrodites, after controlling for seed size. On a population level, this increase is almost enough to fully compensate the loss of seeds due to the presence of males. Confidence intervals on this increased seed production were quite large, with 95\% confidence intervals encompassing increases of 16\% to 228\%. Despite this uncertainty, increased seed production remained a confirmed benefit at the population level after integrating all effects and uncertainty (Fig \ref{fig:chap6fig4}a). Overall, our seed production estimates fall in the same range as previous estimates for these species, estimates which incorporated data for more years, but did not benefit from information on which trees were female \citep{Muller-Landau2008}.

\subsection{Seedling phase}
Even a doubling of the number of seeds produced per female might be insufficient to compensate for the full costs of dioecy due to increased local competition or increased attack by density-dependent natural enemies, such as pathogens and insect herbivores \citep{Heilbuth2001, Barot2004}. Increased local competition and natural enemy attack is predicted among seedlings of dioecious species because they are expected to be more aggregated in space around females, and aggregation leads to decreased seedling performance \citep{Harms2000a, Comita2014, Lebrija-Trejos2014}. Dioecious species, through increased seed production per mother tree, are therefore expected to have a 'seed-shadow handicap' that limits the benefit of increased local seed production \citep{Heilbuth2001, Barot2004}. It has been hypothesized that dioecious species might thus allocate for increased seed dispersal \citep{Heilbuth2001}, but we found no consistent differences in dispersal distances between dioecious and hermaphroditic species (Appendix S5).  Consistent with the seed shadow handicap hypothesis, our average model estimates of dioecy effects on seedling establishment, growth, and survival are negative, but these effects are highly uncertain and not significantly different from zero (Table \ref{tab:chap6tab2}, Figs. 2d and 3b). The low elasticity for these vital rates meant that estimated population level costs were, in any case, small (Fig. \ref{fig:chap6fig4}a). Note that elasticity values may be sensitive to the sex ratio \citep{Haridas2014}; something we could not evaluate with a single-sex model.

\subsection{Tree phase}
Dioecious species had greater growth and survival in the tree phase, and the advantage over hermaphroditic species increased with individual size (Figs. \ref{fig:chap6fig2}e and \ref{fig:chap6fig2}f). Greater growth and survival in dioecious species is expected if resource allocation to reproduction is reduced because individuals produce only one type of reproductive organ \citep{Bawa1980}. Therefore, an increase in growth and survival should only occur after trees start reproducing. This suggests that any breeding system effects in terms of increased tree growth and survival are only visible at large tree sizes, which is supported by our results, as reflected by the increased slope (Figs. \ref{fig:chap6fig2}e and \ref{fig:chap6fig2}f). As a result of this benefit becoming more pronounced at larger sizes, the cost-benefit analysis found a strong survival benefit for dioecy for tree species with whose maximum adult size was considerably above the average (of 158 mm DBH), despite the lack of such a benefit for species of average or below average size (Fig. \ref{fig:chap6fig4}b).

The estimated effects of dioecy on tree growth and survival were highly uncertain (Figs. \ref{fig:chap6fig2}e and \ref{fig:chap6fig2}f). One possible cause of this uncertainty could be sex-specific effects of individuals. It seems unlikely that dioecious females could increase seed production and simultaneously increase tree growth and/or survival. One hypothesis to explain these findings is that the increased seed production may be a female effect, while the increased growth and survival could be largely a male effect. However, we found no consistent sex differences in growth (Appendix S11). This is in agreement with \citet{Queenborough2007}, who found no significant sex effects on annual diameter growth in twelve tropical tree species in Ecuador. Tests of the hypothesis remain inconclusive, as we could not account for potential spatial segregation of the sexes, which has been found in some previous studies \citep{Cox1981, Bierzychudek1988, Queenborough2007, Forero-Montana2010, Ortiz-Pulido2010}. When males occur more frequently in less suitable habitats, sex-specific survival and growth benefits may be obscured. 

\subsection{Balance of costs and benefits}
Our population models project that the intrinsic population growth rate of an average hermaphroditic species is 0.0176 per year compared to 0.0178 for a dioecious species with the same trait values and a 50\% sex ratio. This suggests that the combined demographic costs are compensated by fitness advantages in other aspects of the life history of dioecious species, although there is considerable uncertainty in estimates of individual vital rate and total effects (Figs. \ref{fig:chap6fig2}, \ref{fig:chap6fig3} and \ref{fig:chap6fig4}). After integrating breeding system effects across the full life-cycle, increased seed production was the most important and only significant compensating factor (Fig. \ref{fig:chap6fig4}a).

The presence of males was the strongest cost of dioecy. For average trait values, our population-level cost-benefit analysis shows that the presence of males, when the sex ratio was 50\%, reduced the intrinsic growth rate r by 0.00652. It is generally assumed that dioecy as a breeding system comes with demographic costs, and intuitively this makes sense. Our results not only show that the increased seed production compensates for these costs, but also, that the inherent costs are low to begin with. This is because for long-lived organisms such as trees, we expect low $\lambda$-elasticities for factors involving reproduction \citep{DeKroon2000}, as was also shown for reproductive strategies such as mast fruiting in tropical trees \citep{Visser2011}.
 
At global and regional scales, the proportion of species that are dioecious within plant communities varies greatly, but tends to be higher in tropical than in temperate forests \citep{Vamosi2010}. Multi-site studies that similarly quantify the costs and benefits of breeding systems at the population level would help explain this wide variation in the success of the dioecious breeding system. In conclusion, our results support the hypotheses that the advantages of dioecy include increased seed production by females, and to a lesser extent increased adult growth and adult survival. Together the benefits compensate for the costs of dioecy, which are found to be smaller than generally assumed because of the low elasticity of population growth rates to seed production in long-lived tree species. 

\end{fullwidth} 

\begin{landscape}
\begin{figure*}
\vspace*{-.6cm}\hspace*{4.4cm}\fbox{\includegraphics[width=19cm,height=22cm]{../figures/illustrations/chapter7.png}}

\hspace*{5cm}\begin{minipage}{18cm}
 \textit{\footnotesize ".. Only surviving bombers are observed, .. downed planes are unobservable and not part of the sample .." - M. Mangel and F. Samaniego (1984) on Abraham \vspace*{-0.4cm} Wald's "A method of estimating plane vulnerability based on damage of survivors" (1943) \nocite{Mangel1984}.}

\end{minipage}
\end{figure*}
\end{landscape}

\let\cleardoublepage\clearpage

\chapter{Parasite-host interactions in tropical trees: lianas differentially impact population growth rates of tropical tree species}
\label{ch7} 
\marginnote[-2.2cm]{Marco D. Visser, Stefan A. Schnitzer,  Helene C. Muller-Landau, Eelke Jongejans, Hans de Kroon, Liza S. Comita, Stephen P. Hubbell and S. Joseph Wright. \textbf{Journal of Ecology, in review.} \\\noindent Supplementary material can be found online: http://tinyurl.com/zghs7t8}


\section{Abstract} 
\begin{fullwidth}
Lianas are structural parasites of trees, which reduce individual host tree growth, survival, and fecundity. Lianas are also increasing across the Neotropics, and this may have profound consequences for tree species composition if lianas reduce and differentially affect host tree species population growth rates. Here, we used integral projection models to integrate effects across the life cycle and quantify the effect of lianas on population growth rates for 33 tree species in Panama. Liana infestation decreased tree growth, survival and reproduction, with strong effects on survival in fast-growing species and on reproduction in large-statured species. On average, populations of trees with >50\% of their canopy infested by lianas declined 1.9\% annually (or 33\% per decade). The reduction in population growth rates was greatest among fast-growing species, which showed higher sensitivity to liana infestation ($\sim$12.5\% annual decreases). Our results demonstrate that lianas have strong differential effects on tree species populations, and that increases in liana abundance are likely to alter tree species composition in Neotropical forests. 

\section{Introduction}

Lianas are an important component of tropical forests, where they frequently occur in the crowns of more than 50\% of all canopy trees \citep{Putz1984a, Heijden2009, Ingwell2010}.  Lianas reduce reproduction, growth and survival of individual host trees \citep{Clark1990, Phillips2005, Wright2005, Wright2015, Pena-Claros2008, Ingwell2010, Schnitzer2010}. Liana abundance is negatively correlated with forest biomass across sites \citep{Duran2013}, and experimental removal of lianas causes dramatic increases in tree biomass accumulation \citep{Schnitzer2014, Heijden2015}. 	Lianas are also increasing in abundance in Neotropical forests. Nine studies have reported changes in liana abundance and all report increases over time, ranging from 0.21\% to 4.1\% annually (mean 1.98\%; \citealt{Schnitzer2014}, \citealt{Wright2015}). Of these, five studies also report changes in tree density, with tree densities either declining or increasing at a slower rate than lianas (reviewed in Schnitzer 2015, Wright \textit{et al.}, 2015). The percentage of tree crowns infested by lianas increased from 32\% in 1968 to 75\% in 2007 at our study site on Barro Colorado Island (BCI), Panama \citep{Ingwell2010}. This raises the question of whether increasing liana abundance might alter tree communities. If the detrimental effects of lianas differ among tree species, increasing lianas will favor some tree species over others altering tree community composition and causing additional indirect effects on ecosystem functioning \citep{Heijden2008, Schnitzer2011}. 

It is plausible that tree species differ in their response to liana infestation. In closed canopy forests, lianas prosper in tree-fall gaps \citep{Schnitzer2000, Ledo2014}, and gap-dependent tree species may encounter lianas more frequently. However, gap-dependent and fast growing tree species are often disproportionately liana free \citep{Clark1990}. Yet, does this reflect a lower tolerance - where infested individuals die rapidly leaving disproportionately more liana-free trees standing - or an improved ability to avoid liana infestation \citep{Putz1984a, Heijden2008, Schnitzer2010}? Most lianas grow towards full sunlight \citep{Putz1984a}, and large-statured tree species, with full sun exposure in the canopy, may be colonized preferentially by lianas \citep{Phillips2005}. Alternatively, large-statured individuals may tolerate lianas better (less negative effect) once infested because of greater insolation \citep{Wright2015}. Finally, some studies report that shade-tolerant, slow-growing species show stronger negative responses to lianas \citep{Schnitzer2010}, while others find that lianas have similar negative effects on tree performance regardless of species identity \citep{Martinez-Izquierdo2016, Schnitzer2005}. However, a comprehensive and large-scale assessment of tree-liana interaction strength across all host vital rates and ontogenetic stages is lacking. 

Changes in liana abundance would influence tree species composition only if lianas have different effects on the population growth rates of different host tree species. Studies of effects at a single life-stage and vital rate are inadequate because differences at one life stage and vital rate are unlikely to parallel total net effects on fitness for two reasons \citep{Caswell1983, Ehrlen2003, Metcalf2007}. First, effects of lianas at one life stage or vital rate may be offset by opposing effects at another \citep{Visser2016}. Second, different vital rates in life stages make highly unequal contributions to overall population growth in trees, and the relative importance of particular vital rates and stages varies among species \citep{DeKroon2000, DeKroon1986}. Similar absolute effects on vital rates among species with diverse life histories may translate into dissimilar effects on population growth rates. For example, an equal absolute decrease in growth rate may be more detrimental for fast-growing species than for slow-growing species. Accurate estimation of the consequences of liana infestation for tree communities thus requires integrating the effects of lianas over the entire life cycle of their host trees for many tree species. No study to date has evaluated how liana infestation affects tree species population growth rates, much less quantified variation among tree species in these effects.

In this study, we evaluated the effect of lianas on population dynamics of tropical tree species on BCI through analyses of large, long-term datasets spanning 28 years. We tested whether lianas have the capacity to alter tree community composition by quantifying effects of liana infestation on all key aspects of a tree's life cycle and by comparing the differential effects of lianas on the population growth rates of 33 tree species. We quantified the tolerance/sensitivity of each host species to liana infestation as the difference between liana free population growth rates and heavily infested population growth rates. Finally, we evaluated relationships between tolerance of liana infestation and adult stature and position on the growth-survival trade-off, two important axes of life history variation among tropical tree species \citep{Kohyama2003, Wright2010}.  We show that liana infestation has strong negative effects on tree species vital rates, and that liana infestation severely decreases population growth rates particularly for fast-growing tree species. 

\section{Methods}

\subsection{Study site}
We conducted the study on BCI, a 1562-ha island covered with lowland moist tropical forest located within Gatun Lake in central Panama (9$\circ$9'N, 79$\circ$51'W). Annual rainfall averages 2632 mm (1929-2014), and there is a distinct dry season between January and April that averages only 285 mm of rain (Leigh 1999). Detailed descriptions of BCI can be found in Croat (1978) and Leigh (1999).  

\subsection{Datasets}
To include all tree life history stages and two key functional traits, we used seven datasets from BCI: 
\begin{enumerate}
\item Tree growth and survival from censuses of the 50 ha plot and five 4-ha plots on BCI. In the 50-ha forest dynamics plot (FDP), all free-standing woody plants with diameter at 1.3 m height (DBH) $\geq$ 1 cm have been mapped, measured for diameter and identified to species in 1980-82, 1985, 1990, 1995, 2000, 2005 and 2010 \citep{Hubbell1983}. The same methods were used to census all trees $\geq$ 20 cm DBH in the five 4-ha plots in 2003-04 and 2014. 
\item Tree reproductive status and proportion of the canopy covered by lianas (liana load) in the 50-ha plot. We assessed 2907 trees of 30 species in 1996 and 2007 \citep{Wright2005, Ingwell2010}. We assessed an additional 18,157 trees of 129 species between March 2010 and October 2014. For each focal species, we selected a size-stratified sample of individuals, which we assessed during the species' reproductive season. For each species, we balanced samples among three size classes of equivalent dbh width, with the upper and lower limits being the largest individual and the largest size at which no individuals reproduce (as assessed by field observations), respectively. We quantified tree liana load and reproductive status from the ground, using binoculars for the taller crowns. We scored both liana load and reproductive status on a five-point scale with zero indicating a liana free or reproductively sterile tree and scores 1 - 4 indicating trees with 1-25\%, 26-50\%, 51-75\% and 76-100\% of the crown bearing lianas or reproductive structures (flowers and/or fruits). For reproductive status, we located the approximate center of a projection onto the ground for each crown, used this central point to divide each crown into four quarter-sections, and scored each quarter separately. For liana load, we recorded a single score for the entire crown. For dioecious species, we also visually determined sex expression (male/female) from the ground or from abscised flowers collected beneath each individual. \citet{Ingwell2010} and \citet{Wright2015} have previously shown that this method for measuring liana infestation results in precise and repeatable reproductive scores. 
\item Liana load for all trees (3312 total) in the five 4-ha plots between 2005 and 2006, using the methods described for dataset 2. 
\item Seed production, quantified weekly since January 1987 using 200 0.5-m$^2$ seed traps located within the FDP \citet{Wright1999}. 
\item Seedling establishment, growth, and survival for individuals < 1 cm DBH monitored annually since 1994 in 600 1-m$^2$ plots, with three plots paired with each seed trap \citep{Wright2005a}. 
\item Growth, mortality, and recruitment of seedlings $\geq$ 20 cm tall and < 1 cm DBH monitored annually or biannually since 2001 in 20,000 1-m$^2$ plots located in a uniform design throughout the FDP \citep{Comita2007}). 
\item Previously published species trait data. As a measure of adult stature we used H$_{max}$, defined as the mean height of the six largest individuals in the 50-ha plot in 2007/8 \citep{Wright2010}. As a proxy for a species' position along the growth-survival tradeoff (hereafter slow-fast axis), we used the first principal component factor score of a principal components analysis including the mortality and mean relative growth rate \citep[obtained from][]{Condit2006} for trees > 10 cm DBH, with higher values indicating higher growth and lower survival.  A species' position on the slow-fast axis is an excellent estimate of shade-tolerance with greater values indicating increasing light requirements \citep{Wright2010}. H$_{max}$ and factor scores on the slow-fast axis were independent of each other (r = 0.04, n.s.).
\end{enumerate}

\subsection{Study Species }
We used two different, but overlapping, sets of tree species in our analyses. First, we evaluated community-level mean effects of liana load on tree reproduction, growth and survival in hierarchical analyses (one per vital rate). We included tree species with at least 15 individuals for the relevant vital rate to ensure reasonable precision for species-specific estimates (rarer species comprised 1-5\% of the data). The 15-individual criterion resulted in 92, 62 and 62 tree species for analyses of liana effects on reproduction, growth and survival, respectively (Table S1 in Supporting Information). Each species had at least one infested individual, 97\% had 5 or more, and 90\% had 10 or more infested individuals. 

Second, we conducted species-specific analyses to quantify how lianas affect population growth rates by constructing Integral Projection Models (IPMs; see below). We included only those species with sufficient data for liana infestation (datasets 2 \& 3) in addition to all other life-history stages and vital rates. Vital rates included seed production (dataset 4), seedling recruitment (dataset 5), seedling and tree growth and survival (datasets 1, 5 and 6), and reproduction (datasets 2 and 3). There were sufficient data for all life stages and vital rates for 33 species (Tables S2-4). 

\subsection{Liana effects on tree growth, survival and reproduction}
We quantified how lianas affected tree growth, survival and reproduction (individuals > 1 cm DBH; datasets 1-3) by fitting generalized linear mixed models. Growth, survival and reproduction were first expressed as functions of individual size (DBH in mm), after which the liana infestation level (hereafter \textbf{L}) was included as a factor. We evaluated alternative models that included main effects of \textbf{L} and interactions between size and \textbf{L}. All models included random slopes and intercepts for species and, if individuals were measured multiple times in repeated censuses, additional random intercepts for individual. The exact model formulations are presented in tables S5-8, but in all cases the most complex model had the following form: 

\begin{equation}
Y_{si} \sim (\beta_0 + \epsilon_s  + \epsilon_i )+(\beta_1+ \mu_s)S_i + (\beta_2 +\gamma_s)L_i + \beta_3 L_i S_i    
\label{eq:chap7eq1}
\end{equation}

where $Y_{si}$ represents the annual growth rate, survival state, or reproductive status of individual i from species s, $\beta_0$ the intercept, $\epsilon_s$ a species random effect, $\epsilon_i$ an individual random effect (where appropriate), $\beta_1$ the main effect of individual tree size $S_i$ , $\mu_s$ the random species effect of size, \textbf{$\beta_2$} a 4 $\times$ 1 vector for the main effects of different levels of the factor $\beta_2$ (1-4), $\gamma_s$ a 4 $\times$ 1 vector for random effects of factor \textbf{L} on species s, and $\beta_3$ a 4 $\times$ 1 vector of the interactions between size and factor \textbf{L}. \textbf{L} is a vector of binary variables denoting the liana infestation score of individual i with zero and ones - a vector of four zeros represents a liana free individual while vectors with three zeros and a one in position 1, 2, 3 or 4 represents an individuals with the corresponding level of infestation. All other models were simpler subsets of eq \ref{eq:chap7eq1}. Models were fit using Laplace approximation of the true likelihood with the lme4 R package \citep{Bates2014}. All models share the same random effect structure, and model selection was based on AIC. We considered a model to be significantly better than a simpler model only when the $\Delta$AIC > 10 (excluding equivalent but more complex models; \citealt{Bolker2009}). More details on the estimated vital rates are given below. We calculated R$^2$ values for mixed models in two ways: marginal values, R$^2_m$, including only fixed effects, and conditional values R$^2_c$, including both fixed and random effects, using the method of \citet{Nakagawa2013}. We plotted model residuals against size for each model to evaluate the appropriateness of the linearity assumptions, and found no evidence of non-linearity. 

To test whether the effect of lianas on vital rates are related to tree life-history strategies, we analyzed correlations between species-specific liana-infestation coefficients (every element of $\beta$2 + $\gamma$s) and the two axes of life-history variation (slow-fast and H$_{max}$). Significance levels were Bonferroni-corrected ($\alpha$ levels given in results).

\subsection{Liana effects on species population growth rates}
	We estimated population growth rates ($\lambda$) as a function of liana infestation (\textbf{L}) for 33 species by constructing integral projection models (IPMs; \citealt{Easterling2000}; \citealt{Ellner2006}). In IPMs, vital rates are continuous functions of size, spanning the entire life-cycle, which together create projection kernels that quantify the transitions of existing individuals and the birth of new individuals in (discrete) time. When the distribution (number of individuals) in a population at time t is described by W(x,t) as a function of size x, then the distribution of individuals sized y at time t+1 is given by:  

\begin{equation}
W(y,t+1)=\int_{A}^{Z} P(y,x)W(x,t)dx						
\label{eq:chap7eq2}
\end{equation}

where the limits [A, Z] represent the minimum and maximum sizes of individuals, respectively, and P(y,x) is a projection kernel that describes the transition of individuals of size x at time t that survive and grow to size y at time t+1, and the production of new individuals sized y at time t+1 by individuals sized x at time t. The projection kernel P(y,x) is built from two functions. 

Function G relates growth and survival from size x to size y in one time step, and function F quantifies the production of new recruits of size y by individuals of size x. The projection kernel P can then be expressed as:

\begin{equation}
P(y,x)=G(x,y,\textbf{L})+F(x,y,\textbf{L})
\label{eq:chap7eq3}
\end{equation}
 
where G(x,y,\textbf{L}) is a transition kernel, constructed using fitted models describing the liana dependent growth and survival of each species, and F(x,y,\textbf{L}) is a reproduction kernel that describes the distribution of new individuals of size y produced by individuals of size x and liana load \textbf{L}. 

We combined data on tree and seedling demography in two-stage IPMs, where size is measured by diameter (d) and height (h) at the tree and seedling stages, respectively. The two stage IPMs in turn consisted of four kernel functions that described all size-dependent transitions between seedlings and trees. These included (1) a function describing reproduction P$_{d\rightarrow h}$ (y$_h$,x$_d$, \textbf{L}), which incorporates liana effects, (2) a function describing the survival and growth of seedlings (< 1 cm DBH) P$_h$(y$_h$, x$_h$), (3) a function quantifying the transition between seedlings and trees P$_{h\rightarrow d}$(y$_d$, x$_h$) and (4) a function describing survival and growth of trees P$_d$(y$_d$, x$_d$, \textbf{L}), which also incorporates liana effects. Liana infestation among seedlings was ignored, as smaller seedlings never have lianas and larger seedlings have them only extremely rarely. Table \ref{tab:chap7tab1} describes the formulation of each of the four kernels. The four functions are then combined in a mega-matrix (\textbf{M}) which describes the transition within and between seedlings and trees: 

\begin{equation}
\textbf{M}=
  \begin{bmatrix}
P_h (y,x) & P_{d\rightarrow h} (y,x,\textbf{L}) \\ 
P_{h\rightarrow d} (y,x) & P_d (y,x,\textbf{L})
  \end{bmatrix}
\label{eq:chap7eq4}
\end{equation}

The dimensions of M were set at 500$\times$500 (corresponding to 100 seedling height and 400 tree DBH classes) as further increases in matrix dimensions had a negligible influence on $\lambda$ (changes < 0.0001\%). 
Using the IPMs, we calculated per capita population growth rates for each species at each level of liana infestation. We use $\lambda_0$, $\lambda_1$, ... $\lambda_4$ to denote the intrinsic rate of increase for the five infestation categories (free, up to 25\%, 26-50\%, 51-75\% and 76-100\% infested), respectively. The intrinsic growth rate is the natural log of the per capita population growth rate, where negative and positive values indicate decreasing and increasing populations, respectively. We then tested whether liana infestation caused significant changes in $\lambda$ compared to liana free rates by applying paired t-tests to contrast each liana infested population growth rate ($\lambda_1$, ... $\lambda_4$) with $\lambda_0$. To determine which demographic rates contributed most to the negative effect of lianas on tree population growth rates, we iteratively replaced model coefficients with zero values, and recalculated $\lambda$ values.  Finally, we conducted a correlation analysis of interspecific variation to evaluate whether changes in $\lambda$ under heavy infestation ($\lambda_4$- $\lambda_0$) were related to either adult stature (H$_{max}$) or position on the slow-fast axis. Significance levels were Bonferroni-corrected ($\alpha$-levels given in the results). 

We parameterized IPMs by combining species-specific models fit to all datasets (see general approach above and Table \ref{tab:chap7tab1}), where growth, survival, probably of reproduction and fraction of tree crowns bearing fruits were related to L using GLMMs for trees > 1 cm DBH (eq. \ref{eq:chap7eq1}).  Details on the procedure for each vital rate are given below. 



\begin{landscape}
\begin{table}
\begin{center}
\small
\vspace*{1cm}
\hspace*{4cm}\begin{tabular}{p{2.5cm}p{7.5cm}p{8.5cm}}
\multicolumn{3}{c}{Construction of the mega-matrix \textbf{M}} \\
\hline 
Kernel & Formulation & Description  \\
\\
\hline 
\\
$P_{h}(y_h,x_h) $ & $ = \begin{cases} 
S_h(x_h)N(y_h,\mu=G_h(x_h),\sigma=\sigma_h) & y_h \leq h_{d=10} \\
0 & y_h > h_{d=10} \end{cases}$ & 
$S_h()$ and $G_h()$ are height (h) dependent annual survival and height (h) growth (mm/year) functions for seedlings fit to data, $N()$ denotes the normal distribution, $\sigma_h$ is the standard deviation of individual height growth and $h_{d=10}$ is the height at which seedling attain a dbh of 10 mm (Figure S11). 
\\ 
$P_{h \rightarrow d}(y_d,x_h)$ & $= \begin{cases}  
0 & y_h \leq h_{d=10} \\
S_h(x_h)N(A(y_d),\mu=G_h(x_h),\sigma=\sigma_h) & y_h > h_{d=10}
 \end{cases}
$ & $A()$ is an allometric function that translates dbh to height (Figure S11).
\\ 
$P_{d}(y_d,x_d,L)$ & $= S_d(x_d)N(y_d,\mu=G_d(x_d,L),\sigma=\sigma_d) $ & 
$S_d()$ and $G_d()$ are diameter (d) dependent survival and growth functions for trees, fit to data. L is the crown infestation index, $\sigma_d$ is the standard deviation of individual diameter growth.
\\ 
$P_{d \rightarrow h}(y_h,x_d,L)$ & $ = R(x_d,L)C(x_d,L)F(x_d)S_d(x_d) \phi I(y_h)$ &  
In order, functions are the diameter (d) dependent functions denoting 
the fraction of individuals that are reproductive $R()$, the crown fraction bearing fruits $C()$, and seed production respectively $F()$. The seed to seedling transition rate is given by $\phi$ and $I()$ denotes the distribution of initial heights for seedlings (Figures S4-5).   
\\ 
\hline 
\end{tabular}
\label{tab:chap7tab1}
\hspace*{4cm} \begin{minipage}{20cm}
\footnotesize Table \ref{tab:chap7tab1} Equations of all transition kernels in the megamatrix \textbf{M}, including the definition of the various vital-rate functions and parameters. The four transition kernels ($P_{h}$, $P_{h \rightarrow d}$, $P_{d}$ and $P_{d \rightarrow h}$) together form a integral projection model with two continuous states: seedling height (h) and diameter at breast height (d). All functions described in this table were fit to data, with details given in the main text and Supporting Information (S1 Text).
\end{minipage}
\end{center}
\end{table}
\end{landscape}


\subsection{Details on vital rate estimation}
We parameterized IPMs by combining species-specific models fit to all datasets (see general approach above and Table \ref{tab:chap7tab1}), where growth, survival and reproduction were related to \textbf{L} using GLMMs for trees > 1 cm DBH (eq. \ref{eq:chap7eq1}). Briefly, the procedure for each vital rate follows.
  
\textit{Probability of reproduction}: We estimated the size-dependent probability of reproduction (R) using a GLMM, where tree reproductive status (sterile or reproductive; datasets 2 and 3) was related (logistically) to tree diameter and level of liana infestation (eq. \ref{eq:chap7eq1}). Reproductive status was dichotomized for the analyses using only sterile trees (score 0) versus fertile trees with reproductive structures (score 1-4).

\textit{Fraction of crown bearing fruits}: We used a linear mixed effect model to examine the relationship between field assessments of the fraction of the crown bearing reproductive structures (hereafter "crown reproductive fraction"; C) and tree diameter and the level of liana infestation (eq. \ref{eq:chap7eq1}; datasets 2 and 3).
 
\textit{Seed production}: Mean species-specific seed production per m$^2$ of reproductive basal area (F) was quantified by dividing seed production by reproductive basal area. Reproductive basal area was calculated for each census year as the sum of individual basal areas (dataset 1) weighted by size-dependent reproductive probability and crown reproductive fraction (datasets 2 and 3) and interpolated linearly between census years to obtain yearly values. Yearly seed production equaled plot area multiplied by the density of seeds arriving in the seed traps (dataset 4). Yearly estimates of seed production were divided by yearly estimates of reproductive basal area and averaged to obtain F. 

\textit{Seedling establishment}: We calculated species-specific mean yearly seedling establishment probability ($\phi$) as the density (per area) of newly recruiting seedlings across all seedling plots divided by the density of seeds arriving at all seed traps (datasets 4 and 5). 

\textit{Initial height of establishing seedlings}: The initial height distributions of establishing tree seedlings (I; dataset 5) were fit to exponential, log-normal and Weibull probability density functions. The Weibull provided the best fit for 31 out of 33 species. We therefore fit Weibull distributions for every species.

\textit{Seedling to tree transition}: We estimated the height at which tree seedlings entered the FDP census (h$_{d=10}$; 10 mm DBH) with a linear model that related height with DBH for each species using data on height and DBH from seedling data (dataset 6).

\textit{Growth}: We modeled growth (G) as height growth for seedlings (mm/year) and basal area growth for trees > 1 cm DBH (mm$^2$/year) using linear mixed effect models (eq. 1). We calculated growth rates as the difference in size divided by the time in years between censuses (datasets 1, 5 and 6). Due to measurement error \citep{Rueger2011}, a tiny fraction of the growth rates were unrealistically low or high (dataset 1), and likely erroneous, we therefore excluded any values greater than four standard deviations from the overall mean (0.4\% of all measurements). 

\textit{Survival}: We estimated the size-dependent yearly probability of survival (S) for both seedlings and larger individuals (> 1 cm DBH) using logistic mixed effects models (datasets 1, 5 and 6). Tree survival rates were related to \textbf{L} as shown in eq. \ref{eq:chap7eq1}. 
 
\textit{Minimum and maximum sizes}: Minimum sizes (A) in the IPM were set to 0 mm height for seedlings and 10 mm DBH for larger trees. Maximum sizes (Z) were the previously estimated species-specific height at which seedlings have 10 mm DBH for seedlings and 110\% (\emph{sensu} \citealt{Zuidema2010}) of the mean DBH of the 6 largest individuals of each species in the 50-ha plot (dataset 1). 


\section{Results}

\subsection{Mean liana effects on tree vital rates}
Tree growth, survival, reproductive probability, and crown reproductive fraction were all strongly negatively affected by lianas (Fig. \ref{fig:chap7fig1}). The proportion of reproductive trees decreased with increasing liana infestation (Fig. \ref{fig:chap7fig1}A), with the best model including size and main effects of L but no L-size interaction ($\Delta$AIC=791.19; table S5; R$^2_m$ =0.26, R$^2_c$=0.68). The fraction of the tree crown bearing reproductive structures increased strongly with tree size, with intercepts decreasing with increasing liana infestation (Fig. \ref{fig:chap7fig1}B). Again, the best model included size and main effects of L but no L-size interaction ($\Delta$AIC=25.16; table S6; R$^2_m$ =0.19 R$^2_c$=0.69). 

Tree growth was strongly influenced by lianas (Fig. \ref{fig:chap7fig1}C), with the best model including size, main effects of liana load (L) and interactions between L and size ($\Delta$AIC = 57.0; table S7; R$^2_m$ =0.03, R$^2_c$=0.43). Tree growth decreased with increasing L, with the lowest growth rates when trees were heavily infested (L = 4; 75-100\% crown cover). Larger trees were less affected when lianas infested less than 50\% (L=1 \& 2) of their crowns (see positive slopes of the yellow and orange lines in Fig. \ref{fig:chap7fig1}C).  
Tree survival declined noticeably only with L values > 50\% (Fig. \ref{fig:chap7fig1}D), with the best model including size and main effects of L, but no L-size interaction ($\Delta$AIC = 471.376; table S8; R$^2_m$ =0.08, R$^2_c$=0.35). Here again, an infestation level of >75\% resulted in the most severe decrease in tree survival. Table S9 presents coefficients and standard errors for each model.

\begin{figure*}
\hspace*{.5cm}\includegraphics[width=12.5cm,height=12.5cm]{../figures/Chap7Fig1.png}
\caption[The effects of liana infestation on expected demographic rates as a function of tree size][35cm]{.}
\label{fig:chap7fig1}
\hspace*{0.7cm} \begin{minipage}{12cm}
\footnotesize Figure \ref{fig:chap7fig1} 
The effects of liana infestation on expected demographic rates as a function of tree size, as predicted by fitted mixed-effect models.  Estimated effects on the probability of reproduction (A) are based on analyses of $14519$ individuals, effects on the proportion of the crown bearing fruit on $4814$  individuals, effects on tree growth (C) on $17770$  individuals, and effects on tree survival (D) on $6547$ individual trees.
\end{minipage}
\end{figure*}

\begin{figure*}
\hspace*{.25cm}\includegraphics[width=12.5cm,height=16cm]{../figures/Chap7Fig2.png}
\caption[The effects of liana infestation on expected demographic rates as a function of tree size][35cm]{.}
\label{fig:chap7fig2}
\hspace*{0.7cm} \begin{minipage}{12cm}
\footnotesize Figure \ref{fig:chap7fig2} 
The impact of heavy liana infestation on host vital rates related to maximum host size (panels A, C, E and G) and to host position on the growth-survival tradeoff axis (B, D, F and H). Vital rates include reproduction (panels A and B), fecundity (C and D), growth (E and F) and survival (G and H). The species-specific impact of lianas is estimated as the species-specific liana load coefficients ($\beta_2 + \gamma_{s}$) associated with heavy liana infestation, where $\beta_2$ represents the fixed effect and $\gamma_{s}$ the species-specific random effect (see eq. 1). Black lines indicate a statistically significant relationship (Bonferoni corrected significance level set at 0.00625) and r denotes the Pearson correlation coefficient. Larger values on the slow-fast axis correspond to faster growth and lower survival.
\end{minipage}
\end{figure*}


\subsection{Variation in liana effects on tree vital rates}
We found that tree species differed strongly in how severe infestation (75\% to 100\% of crown infested) affected vital rates, as measured by species-specific liana-infestation coefficients ($\gamma$s + $\beta$2). Large adult stature (H$_{max}$) was associated with increasingly negative effects of severe infestation on crown reproductive fraction (r=-0.36, p=0.0007), probability of reproduction (r=-0.32, p=0.0031) and growth (r=-0.54, p=0.00001), with a non-significant negative trend for survival (r=-0.11, p=0.39) (Fig. \ref{fig:chap7fig2}A-C). Fast-growing species, with high scores on the slow-fast axis, showed much stronger declines in survival than slower growing species (r=-0.56, p<0.00001; Fig. \ref{fig:chap7fig2}H). Observed trends for declines in growth, reproduction, and fecundity with respect to the slow-fast axis were not statistically significant (Bonferroni corrected $\alpha$ = 0.00625). 

\begin{figure*}
\hspace*{1.5cm}\includegraphics[width=14cm,height=14cm]{../figures/Chap7Fig3.png}
\caption[The effects of liana infestation on expected demographic rates as a function of tree size][35cm]{.}
\label{fig:chap7fig3}
\hspace*{0.7cm} \begin{minipage}{12cm}
\footnotesize Figure \ref{fig:chap7fig3} 
Projected mean annual population growth rates for A) populations with different levels of liana infestation and B) when iteratively removing the effects of lianas on vital rates. In A, 95\% confidence intervals are given by the grey error bars). Asterisks indicate a statistically significant difference in growth rate from the liana free population at the $p<0.001$ level (Bonferroni corrected $\alpha$ = 0.0125). In B, the effects of severe infestation ($\geq 75$\% crown infested) on population growth (black bar in A) are decomposed into the contributions from different vital rates. Letters correspond to survival (S), growth (G), crown reproductive fraction (F) and probability of reproduction (R). Letters code including S, G, F, and R correspond to the vital rates included in the analysis, with the effects of infestation set to zero for the vital rates absent as indicated by the letter codes. 
\end{minipage}
\end{figure*}

\subsection{Liana effects on tree species population growth rates}
The mixed model vital rate predictions used to calculate integral projection models performed well for all 33 species, reproducing species average rates with an R$^2$ that ranged from 0.71-0.97 (figures S1 - S11). Population growth rates ($\lambda$), calculated from IPMs and estimated for liana-free populations, varied between -0.047 and 0.043 per year. Population growth rates declined with increasing severity of liana infestation and were significantly lower for heavy liana infestation (L=3 and 4) compared to liana free populations (L=0, Fig. \ref{fig:chap7fig3}, paired t-test, p<0.0001, N=33; Bonferroni corrected $\alpha$ = 0.0125).  The iteratively replacement of model coefficients with zero values, for each vital rate, revealed that species-specific effects of lianas on tree survival (shown in Fig. \ref{fig:chap7fig2}H) were largely responsible for the reduction in population growth rates (Fig \ref{fig:chap7fig3}B).   

The negative effect of lianas on population growth rates was strongly related to life history variation, but not to adult stature (Fig. \ref{fig:chap7fig4}). Liana-free population growth rates, $\lambda_0$ , tended to increase with species positions on the slow-fast axis, but this correlation was not significant after Bonferroni correction (Fig. \ref{fig:chap7fig4}D, r=0.38, $\alpha$ = 0.0083). Population growth rates when heavily infested decreased significantly with species position on the slow-fast axis (Fig. \ref{fig:chap7fig4}E, r=-0.616, p=0.002,). The sensitivity of population growth rate to liana infestation ($\lambda_4$ - $\lambda_0$) also decreased significantly with species position on the slow-fast axis (Fig. \ref{fig:chap7fig4}F, r=-0.783, p<0.00001). 


\begin{figure*}
\hspace*{0cm}\includegraphics[width=14cm,height=18cm]{../figures/Chap7Fig4.png}
\caption[The effects of liana infestation on expected demographic rates as a function of tree size][35cm]{.}
\label{fig:chap7fig4}
\vspace*{0.3cm}

\hspace*{0.2cm} \begin{minipage}{14cm}
\vspace{.1cm}
\footnotesize Figure \ref{fig:chap7fig4} 
 Projected population growth rates of host trees (intrinsic rate of increase, $\lambda$) related to maximum host size (panels A, B and C) and to host position on the growth-survival tradeoff axis (D, E and F). Projected population growth rates are for liana-free hosts ($\lambda_{L=0}$, panels A and D) and heavily infested hosts ($\lambda_{L=4}$, B and E) and their difference ($\lambda_{L=4}-\lambda_{L=0}$, C and F). Species with higher growth and lower survival rates tended to be more sensitive to liana infestation. Solid lines indicate a statistically significant relationship (Bonferoni corrected significance level set to $0.0083$) and r is the Pearson correlation coefficient. Larger values on the slow-fast axis correspond to faster growth and lower survival.
\end{minipage}
\end{figure*}

\section{Discussion}
This is the first study to systematically quantify the net effects of liana infestation on population growth rates for multiple tree species. Previous studies evaluated the effects of lianas on one or more vital rates, but have not explored possible effects on population growth \citep{Schnitzer2005, Ingwell2010, Alvarez-Cansino2015, Wright2015,  Martinez-Izquierdo2016}. We found that lianas strongly influenced individual vital rates (Figs. \ref{fig:chap7fig1} and \ref{fig:chap7fig2}), and when integrating all lower-level effects on vital rates, heavy liana infestation severely decreased population growth rates particularly for fast-growing tree species (Figs. \ref{fig:chap7fig3} and \ref{fig:chap7fig4}).  This strong differential effect of lianas demonstrates that lianas have the potential to alter tree species composition and may also play a greater role in tree species coexistence than previously appreciated. 
Our finding that liana infestation is more harmful to fast-growing tree species (Figs. \ref{fig:chap7fig4}E and \ref{fig:chap7fig4}F) appears to be at odds with the general expectations in the literature \citep{Putz1984a, Clark1990, Schnitzer2000, Heijden2008, Heijden2009, Schnitzer2010}. Previous studies proposed that shade-tolerant, slow-growing tree species should be disproportionately negatively affected by liana increases \citep{Schnitzer2002} because shade-tolerant species have higher levels of liana infestation than do light-demanding species. Higher levels of infestation have been interpreted as evidence that shade-tolerant, slow-growing species are more vulnerable to becoming infested (\citealt{Heijden2008}, \citealt{Clark1990}; but see \citealt{Ingwell2010}). Our results suggest an alternative interpretation. Once infested, slow-growing, shade-tolerant species are better able to tolerate lianas than are fast-growing, light-demanding species (Fig. \ref{fig:chap7fig4}F). Survivor bias affects static levels of liana infestation, with greater survival once infested (Fig. \ref{fig:chap7fig2}H) contributing to the observation that levels of liana infestation are greater in slow-growing species than in fast-growing species. Static levels of liana infestation rates integrate vulnerability to and subsequent tolerance of liana infestation.  

\subsection{Potential mechanisms behind liana tolerance and intolerance}
When lianas infest tree crowns they form a layer of leaves on top of their host crown \citep{Avalos1999}. Liana leaves are placed close to host leaves and intercept a large proportion of the light that hits the canopy \citep{Stevens1987, Avalos1999}. Fast-growing, light-demanding and gap specialist trees tend to have low leaf area indices \citep[leaf area per ground area or LAI;][]{Kitajima2005} and shallow crowns \citep{Kohyama2003}, presumably because their leaves are constructed for full sun, having high light compensation points that would suffer heavily from self-shading \citep{Kitajima1994}. Because liana leaves displace host tree leaves on a 1-to-1 mass basis \citep{Kira1971}, the displacement of the one and only layer of leaves of a gap-specialist tree species (e.g., \textit{Cecropia} sp.) could be fatal \citep{Kitajima2005}. As a result, the only fast-growing or gap-specialist trees that survive are those without lianas \citep[as observed by][]{Clark1990}.
 
By contrast, later-successional and more shade-tolerant trees species tend to have deep crowns \citep{Kohyama2003}, with up to eight layers of leaves and low light compensation points for individual leaves \citep{Kitajima2005}. In theory, an additional layer of liana leaves at the top of the host canopy should only displace the lowest, most heavily shaded layer of host leaves. Thus, while liana infestation would reduce light at the top of the host tree, shade-tolerant trees should survive because many leaves are adapted for lower light conditions in their deep canopies. This novel hypothesis for how trees across a range of shade-tolerance vary in response to liana infestation may explain the muted response and apparent "tolerance" of lianas for population growth of shade-tolerant tree species and the contrasting sensitivity for light-demanding tree species (Figs. \ref{fig:chap7fig4}E and \ref{fig:chap7fig4}F). 

Belowground competition between lianas and trees for water and nutrients may also play an important role \citep{Dillenburg1993, Schnitzer2005}. Both lianas and fast-growing tree species require high concentrations of nitrogen, phosphorus and potassium per unit leaf area to maintain photosynthesis in high light \citep{Asner2012}. Tree species with high potential growth rates also tend to require more water, and lianas are strong competitors for water \citep{Cai2009, Alvarez-Cansino2015}. For these reasons, slow-growing tree species with conservative nutrient and water use should be better suited to tolerate belowground competition from lianas than fast-growing tree species. 

Lianas also exert strong mechanical stress and torque on tree limbs \citep{Putz1984a}. Species with high potential growth rates tend to have low density wood \citep{Wright2010}. Wood rupture strength is proportional to the product of wood density and basal area \citep{Larjavaara2010}. Hence, we would expect fast-growing species to be at greater risk of stem breakage or limb loss, factors that increase tree mortality \citep{Paciorek2000}. Competition for light, water and nutrients and resistance to breakage might all favor tolerance among slow-growing tree species and sensitivity among fast-growing species once infested with lianas.  

\subsection{Predicting the effects of liana increases on tree communities }
To predict how the increasing importance of lianas will affect tree species composition, we need more information than just the ability of different tree species to tolerate lianas once infested (Figs. \ref{fig:chap7fig4}E and \ref{fig:chap7fig4}F). We also need to know whether tree species differ in their ability to avoid initial colonization and infestation by lianas, to reduce liana load once infested or to shed lianas all together. Previous work has equated a species' ability to avoid lianas with the proportion of its individuals infested with lianas \citep{Putz1984a, Clark1990, Heijden2008}. The traits often cited as predictors of reduced liana infestation include rapid growth, monopodial stems, few branches, and large leaves, all of which characterize many gap-dependent tree species \citep{Putz1984,Putz1984a} We observed that gap-dependent species die rapidly when infested by lianas (Fig. \ref{fig:chap7fig2}H). When only liana-free individuals of gap-dependent species survive, static infestation rates will create the impression that fast growing species are unaffected by lianas \citep[e.g.][]{Clark1990}. This phenomenon is known as survivorship bias \citep[e.g.][]{Zens2003}. A crucial focus for future research should be obtaining unbiased estimates of how species differ in their ability to evade liana infestation and shed lianas when infested. One crucial question is whether interspecific variation in vulnerability to liana infestation (Figs. \ref{fig:chap7fig2} and \ref{fig:chap7fig4}) may be balanced at least in part by variation in ability to avoid infestation, as species that are more vulnerable would experience stronger selection to avoid infestation. This question could be addressed by studies that quantify species differences in liana infestation rates while controlling for both survivorship bias caused by different rates of liana induced mortality and spatial location relative to areas of high liana density \citep[e.g. in gaps;][]{Ledo2014}.

It remains difficult to predict what the net outcome of increased liana infestation will be for the tree community. One prediction is that the increasing abundance of lianas increases tree mortality rates (Figs. \ref{fig:chap7fig1}D and \ref{fig:chap7fig2}H) and therefore the rate of gap creation. At BCI, 80\% of gap-making trees carry lianas \citep{Ingwell2010}, and these trees frequently pull down neighboring trees connected by shared lianas  \citep[on average 2.6 trees > 10 cm][]{Putz1984a}. Greater liana abundance may therefore result in increased gap dynamics, including increased gap sizes. More gaps, in turn, might favor both lianas and fast-growing tree species  \citep{Schnitzer2000, Schnitzer2010, Ledo2014}. However, trees growing near liana-infested trees are far more likely to become infested with lianas  \citep{Heijden2008}, hence individuals of slow-growing tree species that survive well when infested with lianas may facilitate the infestation of fast-growing neighbors. Thus, it is unclear whether increasing liana abundance will favor slow-growing species that tolerate liana infestation (Figs. \ref{fig:chap7fig4}E and \ref{fig:chap7fig4}F) or fast-growing species that benefit from intensified gap creation.
	
\subsection{Conclusions}
The world is changing, and tropical forests are no exception  \citep{Malhi2014}. To date, multiple studies have found that lianas are increasing in abundance in the Neotropics  \citep[reviewed in][]{Schnitzer2011, Wright2015} and that lianas reduce tree growth, reproduction and survival  \citep{Heijden2008, Ingwell2010, Wright2015} throughout ontogeny (Fig. \ref{fig:chap7fig1}) with different effects on different species (Fig. \ref{fig:chap7fig2}). Given the increasing importance of lianas in the Neotropics, we expect concurrent increases in tree mortality and decreases in biomass accumulation - as increased liana biomass hardly compensates for lost tree biomass  \citep{Schnitzer2014, Heijden2015}. Our results reveal that lianas have the potential to strongly influence multiple tree demographic rates (Fig. \ref{fig:chap7fig3}), and that these effects differ predictably among tropical tree species, showing strong relationships with life-history strategy. Thus, the ongoing increase in lianas may directionally change Neotropical tree community composition, favoring some species over others. Altered tree community composition could, in turn, have serious repercussions for ecosystem functioning and services such as carbon storage.

\end{fullwidth}


\vspace*{20cm}

\begin{landscape}
\begin{figure}
\vspace*{-.6cm}\hspace*{4.4cm}\fbox{\includegraphics[width=19cm,height=22cm]{../figures/illustrations/chapter8.png}}
\hspace*{5cm}\begin{minipage}{18cm}

\textit{ \footnotesize "The same lianas as creep on the ground, reach the tops of the trees, and pass from one to another at the height of more than one hundred feet" - Alexander von Humboldt  (1814) "Personal narrative of travels to the equinoctial regions of the new continent during the years 1799-1804".}

\end{minipage}
\end{figure}
\end{landscape}
	


\chapter{Host-parasite interactions in tropical trees: explaining variation among tree species in liana prevalence}
\label{ch8} 
\marginnote[-2cm]{Marco D. Visser, Helene C. Muller-Landau, Stefan A. Schnitzer,  Hans de Kroon, Eelke Jongejans and S. Joseph Wright. Supplementary material can be found online: http://tinyurl.com/zghs7t8 \\\noindent. }

\section{Abstract} 

\begin{fullwidth}
Lianas reduce tropical tree growth, survival and reproduction, which greatly alters forest structure, carbon stores and species diversity. Yet surprisingly, little is known about the quantitative processes that govern liana abundance. Here we apply a disease ecology approach to explain the proportion of infested trees (termed liana prevalence) among 15 tree species on Barro Colorado Island, Panama. We apply a Markov chain model to estimate liana prevalence as the integrated effect of species-specific rates of host colonization (analog to disease transmission), liana loss (host recovery), infested tree mortality (parasite lethality) and base host mortality. We continue to quantify the relative importance of each rate in explaining species-specific parameters. Our models predicted the observed rate of liana prevalence well, explaining 58\% of the variation across species. We found that host recovery and parasite lethality were the most important processes controlling liana prevalence at our site. Thus, liana prevalence may be controlled through traits that are inherently associated with host species, namely the ability of hosts to shed and/or tolerate lianas.


\section{Introduction}

Lianas are a highly diverse group of woody plants \citep{Putz1984a}, that are both globally widespread and abundant \citep{Dewalt2015}.  Lianas are woody climbing vines that use trees as trellises to grow into the forest canopy, benefiting from the structural investments made by trees \citep{Schnitzer2002}. Lianas typically deploy their foliage above that of their host tree, thus deriving light resources at their host expense \citep{Putz1984a, Avalos1999} while simultaneously competing intensely with for resources below ground \citep{Schnitzer2005, Toledo-Aceves2014}. Consequently, liana infestation has strong negative effects on tree growth, survival and reproduction \citep{Schnitzer2002, Ingwell2010, Wright2015} which in turn depresses per capita population growth rates of tree species (chaper \ref{ch7}),  impacts forest structure \citep{Schnitzer2000, Schnitzer2014}, alters carbon stores \citep{Schnitzer2014, Heijden2015}, and the diversity of other plants and animals \citep[reviewed in][]{Schnitzer2002}. Yet, surprisingly, despite the documented ecological importance of lianas we know little about the factors that influence liana abundance within forests \citep{Muller-Landau2016}.

The abundance of lianas within a given forested area depends for a large part on the proportion of trees infested by lianas (hereafter termed "liana prevalence"). Tree species, in turn, vary considerably in the proportion of individuals infested with lianas \citep{Clark1990}. Classical liana literature assumes that differences in liana prevalence among tree species reflect differences in colonization and loss rates \citep{Heijden2008}. Liana colonization and loss rates (here after recovery) are in turn are attributed to either tree defenses or habitat effects due to variation in liana abundance where the trees are found.  Various hypothesized defenses are large leaves, flexible trunks, fast monopodial growth or ant symbionts \citep{Putz1980,Putz1984,Putz1984a, Hegarty1989}. Habitat effects, on the other hand, include higher liana prevalence in edge and gap habitats among tree species that are characteristic of these habitats \citep{Putz1984a}. However there is very limited evidence for any of these with no studies linking infestation and recovery rates to defenses, habitats or infested proportions. Rather, the literature relates only liana prevalence to habitat type and hypothesized defenses \citep{Putz1984,Putz1984a, Clark1990}. Disease ecology offers an additional explanation, the proportion of infested hosts for any (micro)parasite depends on transmission (colonization), recovery, the effects of the disease agent on their host (lethality) and host base mortality \citep{Anderson1982}. Parasite prevalence is the result of the integrated effect of all these rates, and in turn on the mechanisms that determine these rates \citep[e.g.][]{Roberts2009}. 

We may expect, therefore, that liana prevalence not only depends on the lianas' ability to colonize or their hosts ability to recover, but also on their host species dynamics, and on their effects on their host \citep[lethality;][]{Anderson1982,  Holt2007}. Correlations between species traits and liana prevalence may reflect anyone of these rates. In fact, liana prevalence among tree species may depend for a large part on the demography of the host trees \citep{Muller-Landau2016}. First, tree species with shorter lifespans have less time to accumulate lianas and hence should likely have a lower proportion of infested individuals. Second, species that experience higher mortality specifically when infested should display lower proportion infested, because the infested individuals exit the population faster.  It is well known that both baseline mortality varies extensively among tree species \citep[e.g.][]{Condit2006}, as will mortality of infested individuals vary among species (chapter \ref{ch7}). Yet, the idea that host demography may shape observed interspecific variation in liana infestation has received almost no attention in the literature (chapter \ref{ch7}).  It is therefore an open question in ecology of how much the variation in liana infestation among tree species relates to variation in colonization vs. recovery vs. tree demography.

Here we show that the proportion of trees infested by lianas can be estimated by applying models from disease ecology. We estimate rates of liana-free mortality, liana-infested mortality, liana colonization, and liana loss rate for 21 tropical tree species on Barro Colorado Island, Panama from field data. We then use a simple host-parasite model to predict the proportion of individuals infested for each species, and assess how well these models predict interspecific variation in the liana prevalence -  defined as the proportion of infested individuals \citep[as in parasitology e.g.][]{Roberts2009}. Finally, we test the hypothesis that liana prevalence is predominantly caused by host demography; host tree mortality and liana induced lethality, rather than colonization and recovery. To do so we investigate the relative contributions of liana colonization rates, liana recovery rates, and tree demography to explain interspecific variation in liana infestation.

\section{Methods}

\subsection{Study site}
Liana infestation data are from the 50-ha Forest Dynamics Plot (FDP) and three 4 ha plots on Barro Colorado Island (BCI; 9$^{\circ}$9'N, 79$^{\circ}$51'W), Panama. BCI hosts a moist tropical forest: temperature averages 27$^\circ$C, and mean annual rainfall is 2650 mm (since 1929), with a dry season between January and April (see \citealt{Leigh1999} for details). 

\subsection{Tree and liana data }
We assessed the proportion of the canopy covered by lianas (liana load) for 1781 trees greater than 20 cm dbh in the 50 ha plot in 1996 and 2007 \citep{Wright2005, Ingwell2010} and all 1537 trees greater than 20 cm dbh in the three 4-ha plots located adjacent to the 50 ha plot in 2005 and 2015. For each tree, we quantified the liana load from the ground using binoculars. We scored liana load on a five-point scale with zero indicating a liana-free tree and scores 1 - 4 indicating trees with 1-25\%, 26-50\%, 51-75\% and 76-100\% of the crown bearing lianas. \citet{Ingwell2010}, \citet{Schnitzer2010}, \citet{Heijden2010} and \citet{Wright2015} show that this method results in repeatable liana load scores. In the analyses here, we combined classes 1-4 as liana-infested and compared it to no liana infestation.  

We classified each tree as liana-free (F) or liana-infested (I) in the initial census, and as liana-free, liana-infested, or dead (D) in the final census. For each species, we then constructed a matrix giving the number of trees observed in each combination of categories in the two censuses. This matrix, \textbf{N$_{0\rightarrow t}$}, has elements n$_{ij}$ denoting the number of individuals initially in state j at time 0, and in state i at time t (years), with states ordered as F, I, and D in the columns and rows. This transition matrix was the basis for our subsequent model fits. For each species, we also calculated the observed proportion of individuals infested in the initial census, as a basis for comparison against model predictions.  


\begin{figure*}
\hspace*{1cm}\includegraphics[width=14cm,height=16cm]{../figures/Chap8Fig1.png}
\caption[The effects of liana infestation on expected demographic rates as a function of tree size][35cm]{.}
\label{fig:chap8fig1}
\hspace*{0.7cm} \begin{minipage}{12cm}
\vspace{0.1cm}
\footnotesize Figure \ref{fig:chap8fig1} 
Conceptual diagrams of the Markov transition model used to predict the proportion of infested individuals (A) and yearly transitions between the liana free, infested and dead states between census periods. Yearly transitions of trees between liana free, infested and dead states (A) are regulated by species specific rates of $R$, $C$, $M$ and $L$ (recovery, colonization, background mortality and additional mortality when infested).  All transitions between states can take place each year (B), and a census conducted in year 0, and repeated in year 2, will miss key transitions in year 1 yielding biased estimates of yearly rates (figure S1).
\end{minipage}
\end{figure*}


\subsection{Estimating liana colonization rates, liana loss rates, and tree mortality rates}
We used the transition matrices to estimate probabilities per time step (defined below) of mortality in liana-free trees ($M$; hereafter mortality), additional mortality in liana-infested trees ($L$; lethality, constrained to be $\geq 0$), liana colonization of liana-free trees ($C$; colonization), and loss of lianas from liana-infested trees ($R$; recovery). These parameters define the transition probabilities per time step, for example, the probability of transitioning from liana-free to liana-infested is the product of survival probability of a liana-free individual and liana colonization, $C$(1-$M$) (Figure \ref{fig:chap8fig1} a).  The full transition matrix for changes in states in a time step t, A, is then defined as 

\begin{equation}
\textbf{A}=
  \begin{bmatrix}
 (1-C)(1-M)& R(1-(M+L)) & 0 \\ 
 C(1-M) & (1-R)(1-(M+L)) & 0\\ 
 M & M + L  & 1			
  \end{bmatrix}
\end{equation}

 (recall that the order of states in columns and rows is F, I, and D).  The estimated transition matrix for 2 time steps is then \textbf{A}$\times$\textbf{A}.  That is, the probability that an individual that is liana-free in time 0 is dead in time 2 is the sum of the probability it takes paths F0-F1-D2, F0-I1-D2 and F0-D1-D2 (Figure \ref{fig:chap8fig1} b).  More generally, the estimated transition matrix P(t) for a total of t time steps is the product of  P(t)=\textbf{A}$^t$. 
  
The choice of time step is important, because it determines the potential number of transitions that can be made in a given time.  The time interval between our censuses (10-11 years) is long enough for individual trees to potentially make multiple transitions. For example, a tree that was initially liana-free might be colonized by lianas and then die while infested. Or a tree that was initially liana-infested might lose its lianas only to be re-colonized. Failure to consider the possibility of multiple transitions would lead to underestimation of colonization and loss rates, overestimation of liana-free mortality, and underestimation of liana-infested mortality. We tested a variety of time steps, and found that parameter estimates converged as the duration of the time step decreased, with little change for time steps smaller than 1-2 years (Fig. S1). Thus, we chose to use annual time steps, with ten or eleven time steps between initial and final censuses depending on the census interval for the plot. 

We restrict our analyses to species that had at least 35 individuals in the combined datasets.  For each species, we obtained maximum likelihood estimates of all rates ($C$, $R$, $M$, $L$) by searching for the parameter combinations that maximized the multinomial likelihood of the observed transitions (N) given the transition probabilities expected (P(t)=\textbf{A}$^t$) under the parameter values. The parameter space was searched using generalized simulated annealing, due to its robustness in finding global maxima \citep{Xiang2013}. We estimated standard errors for each model parameter through numerical approximation of the second partial derivative matrix of the log-likelihood function at the maximum-likelihood estimate (or the Hessian). The inverse of the Hessian is an estimate of the variance-covariance matrix \citep[e.g.][]{Bolker2008}. When a given species had any parameter with very large confidence intervals (including 0 and 1), we deemed the estimate untrustworthy and dropped the species from the analysis. This left 21 species with confidence intervals excluding 0 and 1 for all parameters. Our data and the R-script used to fit the models are given in the supplemental material (Text S1 \& Table S2). 

\subsection{Predicting the proportion of trees infested with lianas}
Using the estimated species-specific parameter values for mortality, colonization, and loss rates, we predicted the proportion of trees expected to be infested under the Markov model defined by the matrix \textbf{A}. That is, we calculated the asymptotic stable state distribution, the dominant right eigenvector \citep{Caswell2001}, and then used the first two elements to calculate the asymptotic proportion of living trees expected to be infested if current dynamics continued indefinitely. This should approximate the observed proportions if the population is close to a stable state and the infestation dynamics of trees < 20 cm d.b.h. do not differ substantially from those > 20 cm d.b.h. The assumption that trees smaller and larger than 20 cm d.b.h. have similar infestation rates is necessary because we have no observations on recruits below the 20 cm size threshold used throughout. Model performance was evaluated by comparing observed (in the initial census) with predicted proportions of liana-infested individuals across species. We quantified performance using 1) the coefficient of determination (r$^2$), a measure of variance explained; 2) the root mean squared error (RMSE), a measure of the typical deviation between predicted and observed; and 3) the differences in the predicted and observed means and standard deviations, measures of model bias and model ability to capture interspecific variation, respectively.   

\subsection{The importance of different factors for interspecific variation in liana prevalence}
We investigated the relative importance of interspecific variation in colonization, recovery, liana-free vs invested mortality for explaining variation in the proportion of infested trees among species. To do this, we compared predictions under models where all different unique combinations of parameters were set either to species-specific or to species-averaged values. Species-averaged parameter values were set to the arithmetic means over species. We calculated the above metrics of model fit for all combinations, but are especially interested in the following cases:
\begin{enumerate}
\item	Full model, including species-specific rates of all parameters ($M_s$, $L_s$, $R_s$, $C_s$);
\item	Tree demography only model - species-specific mortality rates and species-averaged colonization and recovery ($M_s$, $L_s$, $R_s$, $\bar{C}$);  
\item	Colonization and recovery only model - species-specific colonization and recovery rates and species-averaged mortality rates ($\bar{M}$, $\bar{L}$, $R_s$, $C_s$);
\item	Species-specific values of one parameter and species-averaged values of the other three;
\item	Species-specific values of three parameters and species-averaged values of the final parameter.
\end{enumerate} 

\section{Results}

A total of 25 species had a sample size greater than 35, and out of these 21 had credible maximum likelihood estimates for $R_s$, $C_s$, $M_s$ and $L_s$ (Table S2). Mean annual rates ($\pm$ sd) of colonization, recovery, tree mortality and lethality were 2.14\% ($\pm$ 2.46\%), 1.93\% ($\pm$ 5.17\%) 1.4\% ($\pm$ 1.61\%) and 1.7\% ($\pm$ 2.2\%), respectively. The observed liana prevalence (proportion of infested individuals) among these 21 species ranged over 0.12 - 0.93 at the initial census.

Predicted stable proportions of infested trees from the full models for each species tended to underestimate the average proportion of infested (table \ref{tab:chap8tab1}). Compare, for instance, the difference between the distributions of the observed (solid lines) and predicted (dashed lines) in the inset plot (Fig. \ref{fig:chap8fig2}A) and the bias statistic (observed - predicted) in table \ref{tab:chap8tab1}. We, however, capture the interspecific variation well, explaining 58\% of the variation and predicting the full observed range. 



\begin{landscape}
\begin{table}
\begin{center}
% latex table generated in R 3.3.1 by xtable 1.8-2 package
% Fri Sep  2 16:17:37 2016
\small
\hspace*{4cm}\begin{tabular}{lllllrrrrlr}
  \hline
Scenario & mortality & lethality & recovery & colonization & $r^2$ & RMSE & Bias & $\Delta\sigma$ & range & N \\ 
  \hline
Full model ($ M_s,L_s,R_s,C_s $) & - & - & - & - & 0.58 & 0.16 & 0.08 & 0.01 & [ 0.13 - 0.92 ] & 4.00 \\ 
  Average mortality ($ \bar{M},L_s,R_s,C_s $) & X & - & - & - & 0.58 & 0.16 & 0.08 & 0.01 & [ 0.13 - 0.94 ] & 3.00 \\ 
  Average liana lethality ($ M_s,\bar{L},R_s,C_s $) & - & X & - & - & 0.47 & 0.18 & 0.09 & 0.03 & [ 0.13 - 0.92 ] & 3.00 \\ 
  Average recovery ($ M_s,L_s,\bar{R},C_s $) & - & - & X & - & 0.27 & 0.22 & 0.15 & 0.08 & [ 0.16 - 0.77 ] & 3.00 \\ 
  Average colonization ($ M_s,L_s,R_s,\bar{C} $) & - & - & - & X & 0.58 & 0.15 & 0.02 & 0.06 & [ 0.14 - 0.92 ] & 3.00 \\ 
  Recovery and colonization ($ \bar{M},\bar{L},R_s,C_s $) & - & - & X & X & 0.47 & 0.18 & 0.09 & 0.03 & [ 0.13 - 0.92 ] & 2.00 \\ 
  Recovery and mortality ($ M_s,\bar{L},R_s,\bar{C} $) & X & - & X & - & 0.54 & 0.17 & 0.03 & 0.09 & [ 0.15 - 0.87 ] & 2.00 \\ 
  Recovery and lethality ($ \bar{M},L_s,R_s,\bar{C} $) & - & X & X & - & 0.58 & 0.15 & 0.02 & 0.06 & [ 0.14 - 0.92 ] & 2.00 \\ 
  Colonization and mortality ($ M_s,\bar{L},\bar{R},C_s $) & X & - & - & X & 0.10 & 0.24 & 0.15 & 0.09 & [ 0.17 - 0.77 ] & 2.00 \\ 
  Colonization and lethality ($ \bar{M},L_s,\bar{R},C_s $) & - & X & - & X & 0.28 & 0.22 & 0.15 & 0.08 & [ 0.16 - 0.77 ] & 2.00 \\ 
  Mortality and lethality ($ M_s,L_s,\bar{R},\bar{C} $) & X & X & - & - & 0.41 & 0.19 & 0.11 & 0.18 & [ 0.33 - 0.56 ] & 2.00 \\ 
  Only mortality ($ M_s,\bar{L},\bar{R},\bar{C} $) & - & X & X & X & 0.00 & 0.25 & 0.12 & 0.25 & [ 0.49 - 0.5 ] & 1.00 \\ 
  Only liana lethality ($ \bar{M},L_s,\bar{R},\bar{C} $) & X & - & X & X & 0.41 & 0.19 & 0.11 & 0.18 & [ 0.33 - 0.56 ] & 1.00 \\ 
  Only recovery ($ \bar{M},\bar{L},R_s,\bar{C} $) & X & X & - & X & 0.54 & 0.17 & 0.03 & 0.09 & [ 0.15 - 0.87 ] & 1.00 \\ 
  Only colonization ($ \bar{M},\bar{L},\bar{R},C_s $) & X & X & X & - & 0.10 & 0.24 & 0.15 & 0.09 & [ 0.17 - 0.77 ] & 1.00 \\ 
   \hline
\end{tabular}
\label{tab:chap8tab1}
\hspace*{4cm} \begin{minipage}{20cm}
\footnotesize Table \ref{tab:chap8tab1} 
Summary statistics for evaluating model performances of 9 different scenarios. Statistics are based comparing observed (in the initial census) with predicted liana prevalence across species. Prevalence is defined as the proportion of liana-infested individuals within a species. The crosses (X) under the columns mortality, lethality, recovery and colonization imply that the respective rate had species-specific values. Other columns include the coefficient of determination ($r^2$), indicating how much variation in species infestation rates was explained; 2) the root mean squared error (RMSE), a measure of the typical deviation between predicted and observed; and 3) bias: the difference in the mean predicted and mean observed prevalence among species (observed mean prevalence was 0.584); 4) $\Delta\sigma$ indicates differences in the standard deviation of predicted and observed (0.28) infestation rates; and 5) the predicted range. The observed range was between 12\% and 92\% of the individuals within a species infested. N gives the number of free parameters per species. Predicted prevalence for all scenarios and species are given in table S5.
\end{minipage}
\end{center}
\end{table}
\end{landscape}


\begin{landscape}
\begin{figure*}
\hspace*{4cm}\includegraphics[width=20cm,height=18cm]{../figures/Chap8Fig2.pdf}
\caption[The effects of liana infestation on expected demographic rates as a function of tree size][35cm]{.}
\label{fig:chap8fig2}
\vspace*{0.3cm}
\hspace*{4cm}\begin{minipage}{20cm}
\footnotesize Figure \ref{fig:chap8fig2} 
Observed and predicted proportions of infested individuals, under six different scenarios. Black circles indicate observed and predicted proportions of infested individuals, with the 95\% confidence intervals of the observed proportion (at year 10/11) given by the grey vertical lines. The grey dashed lines indicate the 1:1 line. Each inset figure represents the distribution of either the observed (black), or predicted (dashed black).  The three scenarios include A) the full model, when including species-specific rates of $R$, $C$, $M$ and $L$ (recovery, colonization, background tree mortality and lethality or "mortality when infested"); B) Recovery and colonization fixed at the mean, but species-specific rates of mortality and mortality when infested; C) All tree species have equal rates of tree mortality ($\bar{M}$, $\bar{L}$) but species differ in rates of recovery and colonization.
\end{minipage}
\end{figure*}
\end{landscape}

\begin{figure*}
\hspace*{.1cm}\includegraphics[width=14.5cm,height=19cm]{../figures/Chap8Fig3.pdf}
\caption[Coefficients of determination for a set of scenarios][35cm]{.}
\label{fig:chap8fig3}
\vspace*{0.3cm}
\hspace*{.5cm}\begin{minipage}{14cm}
\footnotesize Figure \ref{fig:chap8fig3} 
Coefficients of determination for a set of scenarios given by the right hand text between predicted and observed proportion of infested individuals. The decrease from the full scenario is given by the grey bar, and the white text within each bar. 
\end{minipage}
\end{figure*}

The predicted stable proportion varied greatly under different scenarios (Table \ref{tab:chap8tab1}, Fig. \ref{fig:chap8fig2}B-C, Fig. \ref{fig:chap8fig3}). The combined effect of recovery and colonization, rather than the combined effect of tree demography (mortality and additional liana induced mortality; lethality), proved to be a more important combination in explaining the variation in liana prevalence among tree species (compare Fig \ref{fig:chap8fig2}B. and C). The single most influential rate was the rate at which tree lost their lianas (recovery rate $R$, Fig. \ref{fig:chap8fig3}, Fig. \ref{fig:chap8fig4}D). The second most influential parameter influencing host tree abundance was lethality ($L$), which was the liana-derived mortality rate (Fig. \ref{fig:chap8fig4}B). The rate of colonization was the third most influential (Table 1, Fig. \ref{fig:chap8fig2} C). This indicates that the result in Fig. \ref{fig:chap8fig2} B-C, showing that tree demography is of secondary importance, is driven by the importance of recovery. The mortality of un-infested individuals had the least influence of all (Fig. \ref{fig:chap8fig3}, Table \ref{tab:chap8tab1}). 

The observed proportion of infested individuals was negatively related with the rates of recovery and lethality (Fig. \ref{fig:chap8fig4}). It was correlated weakly positively with the rates colonization, and unrelated to tree mortality (Bonferroni corrected significance level was set to 0.05/4 = 0.0125). No significant correlations were found among parameters (Fig S2). 


\begin{figure*}
\hspace*{.3cm}\includegraphics[width=13cm,height=12cm]{../figures/Chap8Fig4.pdf}
\caption[The effects of liana infestation on expected demographic rates as a function of tree size][35cm]{.}
\label{fig:chap8fig4}
\hspace*{1cm}\begin{minipage}{12cm}
\vspace*{0.1cm}
\footnotesize Figure \ref{fig:chap8fig4} 
Observed proportion infested plotted against the estimated rates of natural mortality ($M$), and additional mortality when infested ($L$; lethality), recovery ($R$) and colonization ($C$). The size of each filled circle is proportional to the species sample size. Significant relations (p<0.05/4), as predicted through linear regression, are indicated by the solid lines, with confidence intervals given by the dashed lines. Removal of the rightmost outlier in panel B, reduced the $r^2$ value to 0.36 (p=0.023).
\end{minipage}
\end{figure*}


\section{Discussion}
Lianas have a series of negative effects on their tree hosts (see chapter \ref{ch7}), but the net effect of lianas on a tree population will depend on their prevalence, which varies greatly among tree species (Fig. \ref{fig:chap8fig2}; \citealt{Clark1990}). This study is the first to explain liana prevalence among tree species by integrating species-specific rates of colonization, recovery, mortality and lethality. There is a wide variation among tree species in the proportion of individuals infested by lianas (0.12 to 0.93), and this proportion can be predicted with a combination of two parameters. Just the rates of recovery ($R_s$), and parasite lethality ($L_s$) are enough to explain 58\% of the variation among tree species. The relative importance of colonization, recovery, mortality and lethality in explaining interspecific variation in the proportion infested was robust across three analyses. The rank order was first recovery, then lethality, colonization and last tree mortality for i) predictive power when only one variable was included, ii) the loss in r$^2$ when only variable was excluded (Table 1, Fig. \ref{fig:chap8fig2}-\ref{fig:chap8fig3}), and iii) the simple Pearson correlation with proportion infested (Fig. \ref{fig:chap8fig4}).  

The next step is to understand what our results imply about the controls and regulation of liana prevalence and what it is about tree or liana species, particular to their life history strategies, traits or habitats that leads them to have specific recovery rates or lethality rates (the most influential rates as found here). Below we discuss these questions in light of classic host-parasite coevolution (Box 1).  \vspace{0.3cm}

\shadowbox{
\begin{minipage}{13.5cm}

\textbf{Box 1 (liana fitness)}: In disease ecology parasite fitness  is often expressed as R$_0$, the basic reproductive rate, or the number of new infections generated over the course of an infection \citep{Anderson1982}; 

\begin{equation}
R_0= \frac{TS}{(M + L + R)}		
\label{eq:chap8eq2}			
\end{equation}

With parameters:
T is the rate of transmission, the number of new infections from an infected host over the course of an infection,
S is the susceptible host population size,
$M$ is the mortality rate of uninfested hosts, 
$L$ is the additional mortality due to infection (lethality), and
$R$ is the rate of recovery.

\hspace{0.4cm} Anderson and May (1982) formulated that parasite fitness, in a population of S susceptible hosts, depends on the rate of new infections from an infected host (T) and the expected duration of an infection ($M$ + $L$ + $R$)$^{-1}$.   It follows from equation \ref{eq:chap8eq2}, that lianas can increase fitness by either increasing transmission rates (producing more seeds or vegetative shoots), or prolonging infestation (e.g. through prudent exploitation to reduce lethality). In general, any increase in transmission will require greater exploitation of the host which in turn leads to increased lethality (and thus decreases in the length of infection e.g. Alizon et al. 2009). Hence parasites face a trade-off that should result in the evolution of optimal schedule - one that balances transmission and the host' s ability to tolerate infection \citep{Anderson1982, Frank1996}. Note that the colonization rate ($C$) used here, is not directly translatable to the transmission rate (T), but $C$ will be a function of the community average T and S.

\end{minipage}
}

 
\subsection{The dynamics of liana prevalence}
In classic epidemiology, parasites can increase fitness by either infecting more individuals per unit time or prolonging infestation but not both \citep[Box 1;][]{Alizon2009}. Hosts, in turn, can also influence parasite fitness through immune responses that increase the rates of recovery or reduce transmission. Such dynamics are likely also true in a liana host-tree system. First, we can expect that increased production of seeds or shoots will increase transmission rates for lianas, but simultaneously will require more vigorous exploitation of the host tree (see Box 1), increasing lethality through greater resource competition above and below ground \citep{Dillenburg1993, Avalos1999, Schnitzer2005, Alvarez-Cansino2015}. Assuming that the death of the host is inadvertently bad for the liana \citep[due to loss of canopy-access and increased liana mortality;][]{Phillips2005}, we can theoretically expect that lethality has a limiting impact on liana fitness. As lianas are clearly detrimental to tree fitness (chapter \ref{ch7}), trees in turn, should experience selection to evolve in a way that reduces parasite fitness.  Next to preventing infestation, an obvious pathway is the selection on traits that decrease the length of an infestation. Consequently, we may expect that the rate of recovery to be a key factor with which trees can influence liana fitness. Our analyses indeed showed that the loss of lianas (recovery), and the tolerances of host to infestation (lethality), are important predictors of liana prevalence and hence our results seem to be in line with these basic expectations from epidemiology. 

\subsection{Tree recovery} 
Liana prevalence among tree species appears to be especially sensitive to the rate at which trees recover from infestation (e.g. Fig. \ref{fig:chap8fig3}). Recovery rates varied considerably among tree species, and has often been related to tree species traits. Previous work related a variety of host tree architectural traits that may have evolved, at least in part, to help shed lianas. These include traits as flaky bark (bark shedding), the ability to drop branches (self-pruning), trunks/branch flexibility and long leaves may also improve the chances for shedding lianas \citep{Putz1984, Heijden2008}. Interpretation of these traits as related to tree recovery is however complicated, as previous studies compared static liana prevalence with tree architectural traits. As we show here, any static observation of liana prevalence results from the interplay of 4 rates, of which both recovery and lethality are key determinants of liana prevalence (see also Fig. \ref{fig:chap8fig4}). Trees species with low liana prevalence could indeed have traits that reflect rapid shedding of lianas, or we may only observe these low infestation rates as infested individuals die off rapidly so traits may reflect high intolerance of infestation. Future work that relates recovery and lethality to host species traits should be able to estimate relationships between traits and rates that are statistically unconfounded. 

\subsection{Lethality}
The rate at which lianas kill their hosts, was found to be the second most influential factor governing liana prevalence among tree species. As the death of the host is inadvertently bad for the liana, we may expect that lianas should evolve to limit lethality (\citealt{Frank1996}, \citealt{Alizon2009}; Box 1). However, this is complicated by the fact that liana species are capable of infesting multiple host species, varying widely in traits and life history. It is hard to imagine strong co-evolutionary dependence between any single liana and tree species \citep[see also][]{Garrido-Perez2010}. Co-infestation can also change the optimal strategy as prudent exploitation of one parasite species may result in greater relative success of the other, invoking a tragedy of the commons where the optimal strategy is far more heavy exploitation of the host \citep{Frank1996}. Nevertheless, we  may still expect parasite species to adapt to the most abundant genotype, host species, or functional group in multi-host species systems \citep[e.g.][]{Lively2000}. In the case of mature tropical forest shade-tolerant species are the most abundant group by far \citep[e.g.][]{Gilbert2006}. As a consequence, lianas may have evolved optimal lethality based on the aggregate traits of shade-tolerant species. If this is true, then trees should be progressively affected by lianas across a gradient from shade-tolerant towards light-demanding. In chapter \ref{ch7}, we show that lethality due to liana infestation is strongly related to shade tolerance, with shade-tolerant tree species being far more capable of tolerating liana infestation. However, it is an open question what explains the tolerance of lianas by shade-tolerant tree species. Did selective forces shape lianas into being prudent exploiters, are shade-tolerant species simply more tolerant of stress and low resources in general, or is it a combination of both?
  
Our analysis on these points is left wanting, as we lack more detailed information on the natural history of liana species. Additionally, our spatially implicit host-parasite model is evidently a simplification of the much more intricate nature of liana-tree interactions and these simplifications muddle interpretation of the results. Interactions may be confounded in space when certain tree and liana species are both only found in certain habitats. For instance, gap-dependent or light demanding lianas may be inclined to grow more vigorously, exploiting hosts more intensely despite the costs of greater lethality. Indeed, some lianas thrive despite the loss of a tree host, suppressing tree recruitment and regeneration in gaps for decades \citep{Schnitzer2000}.  It seems plausible that liana exploitative strategies (and thus induced lethality) may differ among liana species. Clearly, more detailed information is needed to definitively answer whether some liana species are more prudent exploiters of their trees host than others or whether lethality is largely explained by host traits as shade tolerance. 
 
\subsection{Colonization}
Lianas will infest trees either from the ground up or laterally growing from an infested neighbor \citep[e.g.][]{Heijden2008}. If the primary mode of infestation is from the ground up, we would expect tree traits as, long straight boles, smooth bark or number of low branches that provide trellis support to generate the observed rates \citep{Putz1984,Putz1984a}.  Yet, as multiple liana species infest multiple host species, host traits will interact with liana climbing mode. Any observed colonization rate will therefore be a function of the tree species traits in interaction with the various liana species that may infest it. This may make the rate of infestation, from the ground up, an occurrence that depends more on the aggregate traits of infesting liana species than on host species identity and traits. Moreover, lianas on BCI often infect multiple trees (on average 1.4 trees per liana; \citealt{Putz1984a}). Instances of lateral (crown to crown) infestation are thus frequent. Here, the rate of colonization should depend not only on host traits or liana climbing mode but also on the density of infested neighbors and the perimeter of the focal tree crown. Hence, also lateral colonization may thus be a haphazard occurrence depending for a large part on the local availability of infested neighbors \citep{Heijden2008}, or nearby disturbance \citep{Ledo2014}, and less on host species identity or traits. As the rate of colonization was found to be have little predictive power, this may indeed be the case.
 
\subsection{Natural tree mortality rates}
Tropical tree species vary continuously along an axis of low-mortality and slow growth towards fast-growth and high mortality \citep{Gilbert2006, Wright2010}. We initially expected that longer-lived hosts, being exposed for longer, would accumulate more lianas over time. Yet, the natural tree mortality rate was found to be the least influential parameter in explaining liana prevalence. Why don't we detect the hypothesized accumulation of lianas over time? There are two potential explanations. First, long-lived slow-growing species simply encounter lianas less frequently.  Lianas are far more abundant in disturbed areas \citep{Putz1984a, Ledo2014},  and hence any accumulation effect will be confounded with an encounter rate, leading mortality to be a weak predictor of liana prevalence. The second explanation is that any effect of tree longevity (the inverse of mortality) is masked by recovery rates. This is plausible as recovery rates are independent of tree mortality (Fig. S2) - and large enough to overturn any accumulation effect. Average recovery rates were larger than mortality rates (respectively 1.9 vs 1.4 per annum), and at least 3 times as variable (with a standard deviation of 5.17 vs 1.61 respectively). Given the larger magnitude, variability and importance of recovery rates (Fig. \ref{fig:chap8fig3}), it is not surprising that any accumulation effect is overturned. 

\subsection{Liana increases and expectations from host-parasite dynamics}
The incidence of new liana infestations can be expected to be a function of the number of infested host trees, which are sources of new infestations (e.g. \citealt{Muller-Landau2016}). Concurrently, lianas are increasing in the neotropics with mean liana prevalence rates reported to be higher than ever \citep{Wright2004a, Wright2015, Schnitzer2011}.  Should we therefore expect the incidence of new infestations and mean prevalence to increase further? Will liana abundances be controlled or will Neotropical forest slowly become climber dominated "liana forests" \citep{Perez-Salicrup2001}? Our model yields some understanding here, as do basic insights from epidemiology. 
 We may expect that the rate of incidence depends not only on the number of infested individuals, acting as sources for new infestations, but also on the number of susceptible host trees. When either is low, the number of new infestation will be depressed. Given that lianas have displayed long-term increase in prevalence \citep{Ingwell2010}, the density of susceptible hosts has inevitably declined. This predicts that new infestations will eventually be balanced by the rates of lethality and recovery \citep{Anderson1982}.  Here, we observed that the "asymptotic" stable-state proportion predicted the observed variation in liana prevalence well (r$^2$=0.58; Fig. \ref{fig:chap8fig2}). This potentially indicates long-term saturation in the system with prevalence rates appearing relatively stable because current rates projected toward infinitely predict them well. Yet, does this mean that lianas populations are controlled at Barro Colorado Island? Has the decades long increase in liana abundance (e.g. \citep{Wright2004a} reached a new equilibrium?  
 
Any such conclusion would be premature, as liana abundance will be influenced by more factors than prevalence alone. Liana population abundance will, additionally, be a function of the liana load (liana density) across tree crowns, and liana densities in treefall gaps. Therefore, although this study has improved our understanding of what controls liana prevalence, a quantitative understanding of the regulation and limitation of liana densities in tree crowns and gaps will be needed before we can make definitive conclusions concerning the population controls on liana abundance. 
 
\subsection{Conclusion}
Lianas are globally widespread and species-diverse plant group that are vital components of forest ecosystems, having profound impacts on animal diversity \citep{Yanoviak2015}, tree population dynamics (chapter 7), and ecosystem processes as carbon sequestration \citep{Schnitzer2014, Heijden2015}. Yet, we know little about the mechanisms that control liana populations at any given site \citep{Muller-Landau2016}. Here we evaluated potential explanations for the proportion of tree species infected. Liana prevalence was well predicted by asymptotic stable stage distributions (Fig. \ref{fig:chap8fig2}), with liana prevalence among species being most sensitive to the rates of recovery and lethality. Our work demonstrates that an epidemiological approach is insightful and provides a sound basis for further explorations in the factors that regulate liana population changes. We conclude that the theoretical aspects of liana population, community and evolutionary dynamics are severely underdeveloped but there exists fertile ground for improvement. 

\end{fullwidth}

\begin{landscape}
\begin{figure}
\vspace*{-.6cm}\hspace*{4.4cm}\fbox{\includegraphics[width=19cm,height=22cm]{../figures/illustrations/chapter9.png}}
\hspace*{5cm}\begin{minipage}{18cm} 
\textit{ \footnotesize "There is little need to worry about optimization, when computational problems can be solved overnight" Reviewer \# 2}
\end{minipage}
\end{figure}
\end{landscape}
	

\chapter{Speeding Up Ecological and Evolutionary Computations in R; Essentials of High Performance Computing for Biologists}
\label{ch9} 
\marginnote[-2cm]{Marco D. Visser, Sean M. McMahon, Cory Merow, Philip M. Dixon, Sydne Record, Eelke Jongejans  \\\noindent \textbf{PLOS Computational Biology (2015), 11, e1004140}. Supplementary material can be found online: http://tinyurl.com/zghs7t8}

\section{Abstract} 
\begin{fullwidth}
Computation has become a critical component of research in biology. A risk has emerged that computational and programming challenges may limit research scope, depth, and quality. We review various solutions to common computational efficiency problems in ecological and evolutionary research. Our review pulls together material that is currently scattered across many sources and emphasizes those techniques that are especially effective for typical ecological and environmental problems. We demonstrate how straightforward it can be to write efficient code and implement techniques such as profiling or parallel computing. We supply a newly developed R package (aprof) that helps to identify computational bottlenecks in R code and determine whether optimization can be effective. Our review is complemented by a practical set of examples and detailed Supporting Information material (S1-S3 Texts) that demonstrate large improvements in computational speed (ranging from 10.5 times to 14,000 times faster). By improving computational efficiency, biologists can feasibly solve more complex tasks, ask more ambitious questions, and include more sophisticated analyses in their research.

\subsection{Introduction}
Emerging fields such as ecoinformatics and computational ecology \citep{Petrovskii2012,Michener2012} bear witness to the fact that biology is becoming more quantitative and interdisciplinary. Such research often requires intensive computing, which may be limited by inefficient code that confines the size of a simulation model or restricts the scope of data analysis. It is therefore increasingly necessary for biologists to become versed in efficient programming \citep{Wilson2012}, as well as in mathematics and statistics \citep{Ellison2010}.

Computer scientists have developed many optimization methods (e.g., \citealt{Hager2010}), however, the efficient translation of mathematical models to computer code has received very little attention in biology \citep{Petrovskii2012}. Here we present an overview of techniques to improve computational efficiency in a wide variety of settings. Much of the information we present is currently scattered throughout various textbooks, articles, or online sources, and our goal here is to provide a convenient summary for biologists interested in improving the efficiency of their computational methods. In short, we 1) highlight the processes that slow down computation; 2) introduce techniques, which, via an R package, help to decide whether and where optimization is needed; 3) give a step-by-step guide to implementing various basic, but powerful techniques for optimization; and 4) demonstrate the speed gains that can be achieved. We supplement this with more background information and detailed examples in the Supporting Information (S1-S3 Texts). The widespread adoption in the biological sciences of the R programming language has motivated a focus on techniques that are directly applicable to R - although many principles
hold for other platforms. 

\section{Focus}

Many analyses in biology are computationally demanding. Examples include large matrix operations \citep{Zuidema2010}, optimizing likelihood functions with complex functional forms \citep{Putten2012}, many applications of bootstrapping or other randomization-based inference, network analysis \citep{Nagarajan2013}, and Markov Chain Monte Carlo fits of hierarchical Bayesian models (e.g., \citealt{Comita2010}). Here, we focus on common issues with large databases and stochastic simulation models, applying general approaches for optimizing code to two simple examples:

\begin{enumerate}
\item Bootstrapping mean values 10,000 times in a moderately large dataset of 750 million records. This example is highly suited for parallel computation and employs common data protocols: indexing and grouping, resampling and calculating means, and formatting and saving output (Fig. \ref{fig:chap9fig1}A-C).
\item A simple stochastic two-species Lotka-Volterra competition model, which utilizes basic mathematical operations, randomly sampling statistical distributions, and saving fairly large simulation results. Additionally, as change depends on the state of the population in a previous time step (a Markov process), a single run cannot be conducted in parallel (Fig. \ref{fig:chap9fig1}D-F).
\end{enumerate}

The optimization of these examples can be followed in detail in S1 Text. In all cases, we obtained speed-ups of 10.5 to 14,000 times with benefits that increase with the amount of computation (Fig. \ref{fig:chap9fig2}). Finally we show the relevance of these techniques when applied to two previously published problems concerning spatial models \citep{Merow2011} and the analysis of fitness landscapes \citep{Visser2011} (documented in S2 and S3 Texts).

\section{When (Not) to Optimize?}
One should consider optimization only after the code works correctly and produces trustworthy results \citep{Chambers2009}. Correct code should be the primary goal in any analysis. Before optimizing, it is important to recall a fact that is recognized by programmers: "Everyone knows that debugging is twice as hard as writing a program in the first place. So if you're as clever as you can be when you write it, how will you ever debug it?" \citep{Kernighan1978}. Optimized code may be faster but tends to lose robustness and generality, be more complex and less accessible, introduce new bugs, and have limited portability and maintainability. Loosely written code, in a high-level language, may be slow, but it will be faster to develop and easier to prototype. In concurrence, it is sensible to prioritize robust, general, and simple code above "fast code" -robust and general programs work in multiple situations (S1 Text: examples 2.14 and 2.15), are reusable, and hence save development time, while clear simple code saves time when revisiting old code (or when sharing among peers). Clearly, slower code will lead to lower total project time if it is more generally applicable, or when additional development and debugging time exceeds what is saved in run time. Therefore, before attempting to optimize code, one should first determine if it will be worthwhile.

\begin{landscape}
\begin{figure*}
\hspace*{3.5cm}\includegraphics[width=17cm,height=15cm]{../figures/Chap9Fig1.png}
\caption[Visualization of profiling output using the aprof package for R code][-14.5cm]{
Visualization of profiling output using the aprof package for R code, where the amount of time spent in each line of code is indicated by the blue bars. In A, B and C, a bootstrap algorithm is shown, and in D, E, and F, a stochastic Lotka-Volterra competition model is shown. The consecutive optimizations described in the text are indicated with the red lines in B, C, E, and F indicating the altered pieces of code. (A) An inefficiently coded bootstrap algorithm, with most time spent in lines 7-8. This algorithm shuffles the values of a large matrix (750,000 $\times$ 1000) stored in object "d", and then calculates columnwise the difference between the mean column values and the overall mean. (B) A slightly improved code where the overall mean calculation is stored in object "avg."  (C) A further improved version of the code where column means are calculated by a specialized and vectorized function (colMeans). (D) A slow running stochastic Lotka-Volterra model of species coexistence that runs a simulation over T years where species have

}\label{fig:chap9fig1}
\footnotesize
\vspace*{0.3cm} \hspace*{3.3cm}\begin{minipage}{21cm}
normally distributed intrinsic growth rates (r $\sim$ Norm(rm,rs)) and competition coefficients (a $\sim$ Norm(am,as)). (E) the Lotka-Volterra model is more efficient when the pre-allocation-and-fill method is applied. (F) Switching to a matrix to store results further decreases run time. A detailed description of each optimization step with profiling analysis is given online (S1 Text, sections 2 and 6).
\end{minipage}
\end{figure*}
\end{landscape}

\begin{figure*}
\hspace*{.9cm}\includegraphics[width=12cm,height=12cm]{../figures/Chap9Fig2.png}
\caption[Execution time in minutes, required to complete various computational problems][35cm]{.}
\label{fig:chap9fig2}
\vspace*{0.3cm}
\hspace*{1cm} \begin{minipage}{12cm}
\footnotesize Figure \ref{fig:chap9fig2} 
Execution time in minutes, required to complete various computational problems, using the optimization techniques discussed. Panels A
and B show the execution time as a function of problem size for 10,000 bootstrap resamples conducted on datasets varying in size (A) and time required to run a stochastic population model against the number of time steps (B). "Naive" R code, in which no optimizations are applied, uses most computing resources (solid lines in A and B). Optimized R code, with use of efficient functions and optimal data structures pre-allocated in memory (dashed lines in A and B), is faster. In both panels A and B, the largest speed-ups are obtained by using optimal R code (black lines). Subsequent use of parallelism causes further improvement (dot-dashed green line) in A. In panel B, using R's byte compiler improved execution time further above optimal R code (dotted lines in green) while the smallest execution times were achieved by refactoring code in C (red dot-dashed lines). Panels C and D give the computing time (in minutes) needed to conduct the calculations from (C) \citet{Merow2011} and (D) the calculations represented by Fig. 3 in \citet{Visser2011}. Bars in panel C represent the original unaltered code from \citet{Merow2011} (I), the unaltered code run in parallel (II), the revised R code where we replaced a single data.frame with a matrix (III) and the revised code run in parallel (IV). Bars in panel (D) represent the original unaltered code \citet{Visser2011} (I), original run in parallel (II), optimized R code (III), optimized R code using R's byte compiler (IV), optimized R code run in parallel (V), optimized R code using byte compiler run parallel (VI), code with key components refactored in C (VII), and parallel execution of refactored code (VIII). All parallel computations were run on 4 cores, and code is provided in S1 Text, section 3, S2 and S3 Texts.
\end{minipage}
\end{figure*}



\section{What to Optimize?}
Amdahl's law (Fig. \ref{fig:chap9fig3}) \citep{Amdahl1967} provides insight into the value of making a specific section of code more efficient: unless this code section uses a very large fraction of the overall execution time,
the reduction in run time for the whole program may be modest. For example, consider code that requires 120 minutes to run, but one section can be sped up by a factor of 2. If that section consumes 95\% of the original run time, optimization will improve total run time to 64 minutes. If that section consumes only 50\% of the original run time, total run time will only improve to 90 minutes (Fig. \ref{fig:chap9fig3}A). Amdahl's law also shows that increased effort in optimization has diminishing returns (Fig. \ref{fig:chap9fig3}B).

Empirical studies in computer science show that small sections of code often consume large amounts of the total run time \citep{Porter1990}. Identifying these code sections allows effective and targeted optimization. "Code profilers" are software engineering tools that measure the performance of different parts of a program as it executes \citep{Bryant2010}. When dealing with large data sets or large matrices, where memory storage is limiting, memory profilers (e.g., \textit{Rprofmem}) provide statistics to gauge memory efficiency. We illustrate the value of profiling in S1 Text, sections 1-3, using R's profiler (\textit{Rprof}) and a newly developed R package (\textit{aprof}: "Amdahl's profiler"). This package helps to rapidly (and visually) identify code bottlenecks and potential optimization gains (as illustrated in Fig. \ref{fig:chap9fig1}).


\begin{figure*}
\hspace*{.5cm} \includegraphics[width=11cm,height=14cm]{../figures/Chap9Fig3.png}
\caption[Projected improvements in total program run time using Amdahl's law.][-12.5cm]{
Projected improvements in total program run time using Amdahl's law. (A) Realized total speed up when a section of code, taking up a fraction $\alpha$ of the total run time, is improved by a factor I (i.e., the expected program speed-up when the focal section runs I times faster). We see that optimization is only effective when the focal section of code consumes a large fraction of the total run time ($\alpha$). (B) Total expected speed-up gain for different levels of $\alpha$ as a function of I (e.g., the number of parallel computations). Theoretical limits exist to the maximal improvement in speed, and this is crucially and asymptotically dependent on $\alpha$ - thus code optimization (and investment in computation hardware) are subject to the law of diminishing returns. All predictions here are subject to the scaling of the problem (S1 Text, section 2).}\label{fig:chap9fig3}
\end{figure*}


\section{How to Optimize?}
After bottlenecks have been identified, the precise nature of any optimization depends on the specific properties of the programming language. However, generally large gains can be achieved by avoiding common inefficiencies (See S1 Text, section 1.4 for more background information).

\begin{enumerate}

\item \textit{Nonessential operations}.
Eliminating unnecessary function calls, printing statements, plotting, or memory references can increase efficiency. Many functions in high-level languages (see below) like R have default options enabled that may incur unnecessary cost. When profiling identifies a specific function as a bottleneck, check its inputs. For example, using \emph{unlist()} on a list with named vectors can be sped up considerably with \emph{use.names = FALSE}, while loading large datasets with \emph{read.table()} or \emph{read.csv()} is expedited by setting the colClasses input. 

\item \textit{Memoization}.
Store the results of expensive function calls that are used repeatedly. For instance, transpose a matrix or calculate a mean once prior to entering a loop rather than repeatedly within a loop. Replacing the repeatedly recalculated mean(d) in line 10 of Fig. \ref{fig:chap9fig1}A, with an object "avg" to store the mean of d, results in a drastic improvement in efficiency with a speedup of $\approx$ 28 times (red lines in Fig. \ref{fig:chap9fig1}B).

\item \textit{Vectorized operations}.
Writing a loop to calculate elements of a vector or rows of a matrix is inefficient. In R, vectorized functions are faster because the actual loop has been pre-implemented in a lower-level, compiled language (in most cases C; \citealt{Schmidberger2009}). Replacing the operation of calculating the mean differences over columns in lines 8-10 of Fig. \ref{fig:chap9fig1}B with its vectorized and highly specialized equivalent "\emph{colMeans(d[index,])-avg}" (Fig. \ref{fig:chap9fig1}C, line 7), the overall execution speed is improved an additional 1.4 times. Note that very large vectors will be inefficient in R. In those cases chunk based iteration is an effective compromise (see section on large data below).

\item \textit{Growing data}.
"Growing data" refers to adding values incrementally to data frames, matrices, or vectors. When a new value is added and the object is lengthened, the new, longer, object must be written to free space in the memory. In the next iteration this process repeats itself, becoming ever more time-consuming. It is much faster to pre-allocate memory that is sufficiently large for the final object than to fill in new values as they are computed. Replacing line 16 from Fig. \ref{fig:chap9fig1}D with a pre-allocate-and-fill operation (lines 6, 7 and 17 in Fig. \ref{fig:chap9fig1}E) results in an $\approx$ 5 times speed-up. 

\item \textit{Dispatch overhead}.
Another potential speed-up strategy is to create custom functions to avoid overhead in base- or package-provided functions. The object-oriented philosophy of R encourages general purpose functions; these perform a large number of checks prior to doing the desired task. Custom written functions perform only the desired task, without these checks, and can lead to significant speed-ups. Another strategy would be to use lower-level functions (see S1 Text, section
1.4) instead of their default counterparts (e.g., \emph{lm.fit} vs \emph{lm}). Note that custom and lower-level functions should be used cautiously as they provide speed at the cost of requiring much stricter compliance to input rules (e.g., S1 Text: examples 2.14-2.15). For example, the Lotka-Volterra competition model code in Fig. \ref{fig:chap9fig1}F stores results in a \textit{data.frame}. In R, \textit{data.frames} are used for storing multiple types of data (e.g., integers, characters, factors etc.) however this functionality is not needed when only using numeric data. Switching to an efficient way of storing a single data type (a matrix) speeds up computation by a factor of $\approx$ 20 (compare Fig. \ref{fig:chap9fig1}E,F, Fig. \ref{fig:chap9fig2}B).
\end{enumerate}

After each optimization step confirm that new code versions produce identical results compared to previous slower versions. Some simple functions for formal results checking in R include \emph{identical()} and \emph{all.equal()}.


\subsection{Parallelization}
Parallel computing divides calculations into smaller problems and solves these simultaneously, using multiple computing elements (hereafter "workers"). In the biological sciences, many computationally intensive problems are "embarrassingly parallel" \citep{Grama2003}, where almost all calculations can be completed in parallel. Common examples are Monte-Carlo simulation and bootstrapping (S1 Text, section 3). Popular parallel computing systems include computations on single multi-cored machines or "distributed computing" on clusters of workstations connected
via a network. Our focus here is on modern multi-cored machines, where parallel computing has become relatively easy to implement, and which most people have access to - though we highlight where distributed computing will be particularly useful. Users should note before implementing a parallel algorithm that parallel code can be more challenging to debug. Accordingly, a handful of basic rules are worth reviewing (details in S1 Text: section 3):

\begin{enumerate}

\item There is a start-up cost to initializing a collection of jobs to run in parallel, so a collection
of small jobs may run faster sequentially (e.g., Fig. \ref{fig:chap9fig2}A), and more parallel processes do
not necessarily lead to faster program execution (\citealt{Hager2010}; i.e., parallel algorithms are also subject to Amdahl's law, see Fig. \ref{fig:chap9fig3}B). When finalizing a parallel run, results need to be copied back to the parent process and collated from each worker; this can be expensive, especially when results are large.

\item In most computing devices, random access memory (RAM) is shared among parallel processes \citep{Schmidberger2009}. Ensure that enough memory is available for each worker, so parallel workers do not have to wait for memory to become available. Because shared memory decreases geometrically with each added worker, such systems are unsuited for big data. Parallel computing on a cluster, where memory is distributed (i.e., increases proportionately with the number of threads), or an algorithm that partitions the data proportionally to each worker, will be more feasible.

\item Independence of random number sequences must be ensured for valid scientific results \citep[e.g.][]{LEcuyer2012, Grama2003}. Ensure that random numbers sequences are unique, reproducible, and will
not overlap (examples in S1 Text, section 3.4).

\item Avoid load imbalances, where one processor has more work than the others causing them to wait. Attempt to split jobs equally. This is especially challenging on a cluster where jobs should match the available resources on each host machine. 
\end{enumerate}

Starting with the optimized but serial R-bootstrap code (Fig. \ref{fig:chap9fig1}C) we created a parallel algorithm for use on a single machine (S1 Text, section 3), with which we achieved a speed-up by a
factor of 2.5 with 4 cores (Fig. \ref{fig:chap9fig2}A).

\subsection{Calling Low-Level Languages}
Parallel computing can reduce run time, but it essentially does not make code run any faster. In other cases parallel computing may not be possible (e.g., Fig. \ref{fig:chap9fig1}D-F). Substantial improvements in execution time can still be made by rewriting key sections of code in a "lower-level" or compiled language. Beginning R-programmers with limited familiarity with compiled languages are advised to pursue other "R-specific" routes of optimization first. These are more straightforward and lead to the greatest relative speed-ups (Fig. \ref{fig:chap9fig2}), while C is more complicated to develop and debug (requiring memory management and missing data (\emph{NA}) handling). 

In general, there are two types of programming languages: interpreted (R, MATLAB) and compiled (C, Fortran). In interpreted languages, like R, code is indirectly evaluated by an evaluation program (hereafter the R-interpreter; \citealt{Chambers2009}). In compiled languages, like C, code is first translated to machine language (i.e., machine-specific instructions) by a compiler program and then directly executed on the central processing unit (CPU). The differences in the type of programming language used can have large effects on execution speed \citep{Chambers2009}.

Compiled and interpreted languages exhibit a trade-off in run time versus programmer time, respectively. Interpreted languages have the benefits of being relatively easy to understand, debug, and alter. However, there is usually much higher CPU overhead as each line must be translated (i.e., "interpreted") every time it is executed. Compiled languages tend to be more challenging to code and debug, but are highly efficient when executed, as "translation
overhead" occurs just once, when the source code is compiled.

In the Lotka-Volterra code in Fig. \ref{fig:chap9fig1}F, we find no clear bottlenecks, with most time consumed by the repeated interpretation of mathematical operators (*, +, etc, S1 Text, section 6.15) and random number generation ("\emph{rnorm}"). We were able to remove such translation overhead by rewriting critical parts of the program in in C and calling the compiled code from R. With this we created a six-times-faster "vectorized" version of the model (Fig. \ref{fig:chap9fig2}B). In S1 Text (section 5) we give practical advice on extending R with C using the most common interfaces for extending R (through the .C and .Call interfaces; \citealt{Schmidberger2009}). In S3 Text, our applied example, we use Rcpp \citep{Eddelbuettel2011} and RcppArmadillo \citep{Eddelbuettel2013} to speed up a matrix multiplication by a factor of 400.

Many interpreted languages also provide special compilers for finished programs, which are simple to use. These represent a compromise between a true compiler and an interpreter. In the R \emph{compiler} package a byte-code compiler is used, which translates R code into more compact numeric codes. It does not produce machine-language code, but instruction sets designed for efficient execution by the interpreter. This may be a quick fix to speed up some code, but most functions are already distributed in byte-compiled form, so further speed gains using byte-compiling are modest. In our examples, we did find that using this compiler decreases execution time (Fig. \ref{fig:chap9fig2}B and S1-Text-section 6.5.1).

\subsection{Large Data}
R loads data into memory by default: datasets comparable in size to the amount of memory available will slow R to a crawl while datasets exceeding the memory space will fail altogether. In these cases researchers can either 1) use databases stored outside R, accessing these in R via languages like SQL (via, for example, \emph{RSQLite}) or 2) use more memory-efficient algorithms. The latter usually involves sequential algorithms, which restrict memory usage to one block of data at a time. Many statistics can be calculated sequentially (e.g., \citealt{Robbins1951}), but problems will take longer to solve as accessing data from a storage disk is slower than from memory. We provide a short example on how to do this for the bootstrap example in S1 Text (section 4), using the \textit{ff} package \citep{Adler2014}.

\subsection{Using More Efficient Algorithms}
A final method to speed up computations is to use a more efficient algorithm. These are mathematically equivalent, but computationally smaller, methods (i.e., they use fewer operations). Although this is highly problem specific, we nevertheless highlight this point, as it is worth scrutinizing the efficiency of the algorithm in use since substantial speed-up may be gained when alternatives exist \citep{Petrovskii2012}. For example, matching m values in a table of n elements requires on the order of m $\times$ n operations with a loop and on the order of m + n options when a hash table is constructed first \citep{Chambers2009}. Subsequent matches will be even faster if the hash table is stored, as in the \textit{fastmatch} package \citep{Urbanek2012}.  Additional examples include using the turning bands algorithm \citep{Finley2011, Mantoglou1982} instead of a Cholesky (variance-covariance) decomposition when simulating a large spatially correlated random field or using an algorithm like Broyden-Fletcher-Goldfarb-
Shanno in non-linear optimization, which requires fewer evaluations of the objective function because the Hessian matrix is built up from information about the first derivatives.

\section{Recommendations}
Optimizing code can provide efficiency gains of orders of magnitude, as our benchmark results show (e.g., Fig. \ref{fig:chap9fig2}). However, we do not recommend optimizing immediately. Realize that one will inevitably sacrifice clarity, generality, and robustness for speed. At the start of a project, the most productive approach (e.g., \citealt{Wilson2012}) is often to write code in the highest-level language possible ensuring the program runs correctly. High-level languages enable rapid decision-making and prototyping, and correct code enables checking of more optimized versions. When a performance boost is deemed worthwhile, for example, through profiling, only optimize those parts identified as bottlenecks to avoid sacrificing development time in favour of optimization
\citep{Chambers2009, Wilson2012}. The primary route for optimization should be efficient R code which, as we show in
Fig. \ref{fig:chap9fig2}, yields the largest gains for the least effort.

The fastest running code examples shown here are the instances where we called compiled code from R (Fig. \ref{fig:chap9fig2}, S1 Text: section 6). This technique is especially powerful when one can use the vast libraries of algorithms that already exist in C (and Fortran), which are often optimized and efficiently coded \citep{Chambers2009}. However, a programming language like C has a steeper learning curve and when learning C requires too much time, we encourage biologists to collaborate with computer scientists in their research or to include contracts for computational consultation in grant budgets \citep{Wilson2012}.

\section{Conclusion}
Learning how to program and efficiently use computational resources is not only convenient. Computing has become fundamental to the practice of science \citep[e.g.][]{Merali2010, Wilson2012, Petrovskii2012, Michener2012}. In biology, research is striving toward ever more accurate projections to inform public leaders on nature management or make predictions regarding how ecosystems respond to change \citep[e.g.][]{Isbell2011, Brook2000, Guisan2006}. More often than not, such accurate predictions will require high levels of detail as natural systems are variable and include intricate levels of biotic and abiotic interactions \citep[e.g.][]{Bohrer2009, Moran2011}. With these challenges ahead, the use of computationally intensive analyses in the biological sciences should not be constrained by programming practices. 

\end{fullwidth}

\vspace*{20cm}

\begin{landscape}
\begin{figure}
\vspace*{-.6cm}\hspace*{4.4cm}\fbox{\includegraphics[width=19cm,height=22cm]{../figures/illustrations/chapter10.png}}
\hspace*{5cm}\begin{minipage}{18cm} 
\textit{ \footnotesize "There is a regulating force at work here: remove a tree and a tree grows back." - \citet{Condit1992a}}

\end{minipage}
\end{figure}
\end{landscape}

\let\cleardoublepage\clearpage

\chapter{Increasing the scale of inquiry}
\label{ch10} 

\begin{fullwidth}
What maintains species diversity? Few questions have generated quite as much interest and remain a conundrum \citep{Leigh1999, Wright2002, Vellend2010, Turcotte2016}. In this dissertation, I have endeavored to scale up some hypothesized coexistence mechanisms across broader spatial, temporal, and organizational scales. A cross-scale focus was combined with two conventional investigative approaches: detailed examination into a model system and the search for general organizing principles using cross-species comparisons.  Work on the \textit{Attalea} model system revealed that fine scale evidence of the Janzen-Connell effect does not necessarily scale up spatially - higher trophic levels affect the interactions between hosts plants and natural enemies. Cross species comparisons at the population scale showed that species respond differentially to liana infestation, or that an effect at one stage can be counterbalanced its inverse at another. Overall, each chapter shows that increasing the scale of inquiry broadens the range of trade-offs, resources or feedbacks that can permit or preclude coexistence of species. This realization is daunting, as with the expansion of scale and organizational level, the number of potential interactions grows exponentially. What are the implications of this? Does a foray across scales make coexistence endlessly more challenging to study? Will continued work require perpetually more detailed models and sophisticated tool sets (see e.g. Chapter 9; \citealt{Bolker2013})? Is the coexistence debate never-ending, doomed to never produce general principles \citep{Lawton1999, Simberloff2004}? In this synthesizing chapter, my response is a resolute negative to all these concerns. Instead, I believe this dissertation shows that we have every reason to be optimistic. Results consistently show that quantitatively predictable patterns emerge across each scale of inquiry. In this final chapter, I discuss each chapter in the light of what we can learn from cross-scale comparisons (summarized in table 10), while answering some remaining quandaries. 

\section{Increasing the spatio-temporal scale}

Ecologists study ecological patterns and variability on spatial scales that range from millimetres \citep{Seymour2000} to entire continents \citep{Chisholm2014} and on temporal scales ranging from milliseconds 
\citep{Bostwick2003} to millions of years \citep{Allmon1993}. The work in this thesis spans much more modest scales, yet even here we unequivocally see that patterns as clear as day at one scale disintegrate to noise when viewed from another scale. For instance, in chapter 2 we see that fine-scale patterns do not necessarily translate to the larger scale. Earlier fine-scale studies on the \textit{Attalea}-bruchid-rodent interaction showed that the proportion of seeds escaping predation increased with distance from the nearest fruiting palm \citep{Wright1983, Wright2001a}. Yet, seed predation was density-independent at the population scale due to top-down control. The conditions put forth by Janzen (1970) and Connell (1971) are present at the local spatial scale, but do not translate to the population level (see discussion in chapter 2). 

We observe a similar phenomenon in chapter 4. Seedling recruitment varied an order of magnitude among seedling plots, with adult density explaining 51\% of the variation (Fig \ref{fig:chap2fig2}). Nevertheless, on the whole plot scale, \textit{Attalea} is primarily limited by transition towards the rosette stage (Fig \ref{fig:chap4fig5}), a transition that requires the increased light conditions caused by canopy disturbance \citep{Araus1994}. Thus, our population model predicts that past disturbance should be the most important factor explaining variation in \textit{Attalea} abundance at greater spatial scales. Disturbance history is indeed the predominant factor explaining \textit{Attalea} abundance on the scale of the entire island \citep{Garzon-Lopez2014}. Such observations issue a warning: the presence of a strong effect at the local spatial scale provides little guarantee that the effect is important or even translates towards the larger scales relevant to the population or community. 

We witness yet another disconnect in the translation of pattern across spatial scales in chapter 3. In dense stands the vast majority of seeds fail to disperse to zones out of the influence of adult palms. This should amplify negative density-dependent recruitment as it increases local seed densities \citep{Janzen1970,  Wright2002}. Surprisingly, however, local seed arrival - at the m$^2$ scale of a seedling plot - was hardly affected by declining dispersal. The contribution of dispersal was completely overwhelmed by the effects of increasing source density (adults). In fact, the effect of increasing adult density was 27 times larger than that of dispersal (32/1.3 - see Chapter 3; Fig \ref{fig:chap3fig4}). Another lesson can be drawn here. We learn that the warning issued above works both ways: patterns acting at larger spatial scales may not be meaningful at finer spatial scales.

Lianas negatively affect tree performance \citep{Ingwell2010} and shade-tolerant species have higher levels of liana infestation than do light-demanding species \citep[e.g.][]{Clark1990}. Understandably, previous studies proposed that shade-tolerant tree species should be disproportionately negatively affected by liana increases \citep{Schnitzer2002}. Chapter 7 and 8 take a view across a longer time frame, showing that static liana prevalence scores are inherently confounded. In particular, chapter 7 raises the possibility that a tree species with low liana prevalence may simply die so quickly when infested that only individuals that escape infestation are observed. This artificially creates the impression, through survivorship bias, that a species is an excellent evader of lianas. Shade-tolerant trees may in fact be less affected by lianas due to their greater tolerance, a conclusion at odds with the general expectations in the literature \citep{Putz1984a, Clark1990, Schnitzer2000, Heijden2008, Schnitzer2010}. Chapter 8 improved on this by parameterizing a simple model from disease ecology (a Susceptible-Infected model). This model teaches us that low liana prevalence among species (at the community level) can be caused by lower colonization, higher tree recovery, greater liana induced lethality or simply through higher tree mortality. Tree traits that correlate with static liana infestation rates are often associated with susceptibility to liana colonization \citep[e.g.][]{Heijden2008}, but in reality may reflect any one of these rates. The four examples above paint a clear picture of the problem posed by spatio-temporal scale in community ecology. We must keep in mind that our perceptions of the importance of different processes vary in a scale-dependent manner.

\section{Cross trophic levels: natural enemies, dispersers and parasites}

\subsection{The enemy of my specialized natural enemy is my friend}
Specialist natural enemies, such as insect seed predators \citep{Lewis2008}, are thought to congregate and cause disproportionately more damage where their preferred food source is more abundant \citep{Hammond1998}. However, as the ancient proverb suggests, "the enemy of my enemy is my friend": top-down control by the plant enemies' enemies can complicate the process. When looking across tropic levels, as shown in Chapter 2, things no longer appear as elegantly straightforward as in Janzen's (1970) original hypothesis. Negatively density-dependent attack by natural enemies is clearly not only limited to plant populations: specialized insect predators are subject to top-down control by their enemies. Yet, does this limit the role invertebrates play in diversity maintenance? Can we say that negative density dependence (NDD), as often observed (Harms et al. 2000), is unlikely to be caused by invertebrate enemies of plants? Making any broad conclusions based on a single palm species would be premature, but what have we learned since the publication of chapter 2 \citep{Visser2011a} about which enemies are the likely culprits behind NDD in plant survival and seedling establishment? \end{fullwidth} \vspace{-.5cm}

It is beyond doubt that top-down control of insects takes place in tropical forests.  Intense insectivore predation strongly limits damage inflicted by arthropods \citep{Kalka2008, Williams-Guillen2008}\sidenote[][-2cm]{In general insectivorous bats weigh on average 11.g and typically eat 39 - 73\% of their body weight each night (Kuntz \textit{et al.}, 1995). This can add up to substantial numbers. A single colony of cave dwelling bats in Mexico was estimated to eat approximately 4000 kg of insect mass each night (Bateman \& Vaughan, 1974). On Barro Colorado Island in Panama, just one of the 28 species of insectivorous bats was projected to eat 1.3 million large insects per year (Kalko, 1997). Clearly, even bats alone have a profound impact on insect populations.} and insectivores also forage optimally where prey is more abundant \citep{Muler2012}. Plant fungal pathogens, in contrast, are generally not subject to top-down control from predators in a food web, and tend to be primarily regulated by host availability and climate \citep{Agrios2005, Thompson2010}. It is for these reasons that in chapter 2 we conclude that pathogens may be a fundamentally more powerful force for inducing negative density dependence than insects. Since this chapter was published in 2011, evidence has accumulated that microbes, particularly fungal pathogens, are the primary drivers of observed negative density dependence, as summarized below. \nocite{Kunz1995} \nocite{Bateman1974} \nocite{Kalko1997}
\begin{fullwidth} 
\hspace{0.5cm} \citet{Comita2014} reviewed over 4 decades of work on natural enemies induced mortality in seeds and seedlings in the context of the Janzen-Connell effect. \citet{Comita2014} observed that the strength of Janzen-Connell effect was significantly stronger at wet sites. This is in line with the expectation that fungal pathogens are foremost controlled through environmental conditions. In general the abundance of such small- bodied, desiccation-intolerant natural enemies is expected to increase with precipitation \citep{Givnish1999}. Interestingly, \citet{Comita2014} found no difference in the strength of negative density dependence between temperate and tropical regions. This is striking as insect herbivory rates tend to be 1.7 - 2.5 times lower in temperate sites \citep{Coley1996}. However, the strongest evidence for fungal pathogens as the major drivers of early life negative density-dependence comes from Las Cuevas in tropical Belize. In an elegant experiment, \citet{Bagchi2014} applied insecticide and fungicide in 252 one-m$^2$ seedling plots. They found widespread NDD in seedling establishment rates across species, but only the application of fungicide significantly weakened the strength of NDD. 

Current evidence is therefore strongly in favor of fungal pathogens as being the most potent cause of density-dependent mortality in early life-stages. However, is there no role for insects in coexistence? The truth is likely to be more nuanced, as demonstrated in chapter 4: both density-dependent and independent factors come together to determine the relative abundance of any plant species \citep[see also][]{Turchin1995}. \citet{Bagchi2014}, for instance, also found that insecticide increased total seedling recruitment by 2.7 times independent of focal species density. Insecticide treatment also increased similarity among plots, swelling the abundance of some - mostly small seeded - species. Hence, generalist insect herbivores are a major cause of seedling density-independent mortality and affect community composition - as they tend to target some species more than others \citep[e.g.][]{Green2014}. Generalist insect herbivores may, therefore, certainly act as a limiting force which can have strong impacts on plant relative abundances. In conclusion, it is by no means established what the relative roles of insects and pathogens are in determining plant species abundances. This will require an integrated understanding of how natural enemies both regulate and limit plant populations in the context of the full-life cycle. 

It is still equally uncertain whether lack of top-down control is the primary mechanism allowing pathogens to cause disproportionate damages to seedlings. The existence of hyperparasitism, for instance, demonstrates that top-down control can certainly take place among plant pathogens \citep{Pal2006} though direct predation is uncommon \citep{Agrios2005}. Evidence presented above and in chapter 2-4, however, does point to one certainty: trophic interactions play a substantial role in plant population regulation. Host-parasite interactions are a core constituent in natural communities, and may be vital to understanding coexistence, as I argue next. 

\subsection{Host-parasites interactions}
In general, parasites may regulate host abundance whenever they increase host mortality in a density-dependent manner \citep[e.g.][]{Roberts1995}. Regulation may operate in the classical sense \citep[\textit{sensu}][]{Janzen1970} where each species is regulated by its own specialized micro-parasite. Interestingly, however, even generalist pathogens can produce observed density-dependent patterns among tree species (as in \citealt{Comita2010}; \citealt{Mangan2010}) when they display different levels of virulence (lethality) among species. In fact, such "shared parasitism" strongly affects interactions among host species and can even completely control dominance among competing species \citep{Holt1985, Holt2007}. Take, for instance, the dramatic invasion of 40 million hectare of Californian grasslands. Here, the invasion by European Mediterranean grasses was facilitated through novel micro-parasite infections \citep{Borer2007}. In another example, \citet{Power2004} show that \textit{Avena fatua} (wild oats) is highly susceptible to a generalist virus but, surprisingly, it is the abundance of competitors that is reduced. In this system, \textit{A. fatua} acts as a reservoir species, which greatly increases pathogen prevalence among several competing species. Hence, both theory and empirical evidence suggest that when species share parasites, host species that can better tolerate the parasite gain an advantage. Such parasite-mediated competition may even facilitate coexistence when competitively dominant species are more severely negatively impacted by parasites \citep{Packer2000}. Clearly, host-parasite interactions can have can have strong ramifications at the community level.

Theoretically, whenever parasites simultaneously share several host species, and differentially affect each host species, apparent competition may occur among host species. In tropical forests, multiple species of lianas are linked through competition for shared hosts. Additionally, lianas are essentially macro-parasites that do not invest in structural support, relying instead on tree hosts for access to canopy light while simultaneously usurping resources that would otherwise be available to their host. Liana infestation also has clear negative impacts on tree population dynamics, and as chapters 7 and 8 show, they affect some species more than others. Hence, the basic requirements for liana-mediated regulation of host abundance are present. Interestingly, the results parallel predator-mediated competition in which the host that can sustain more predators prevails \citep{Holt1994}. However, as explained in chapter 7, it remains unclear whether increasing liana abundance will favor slow-growing species that tolerate liana infestation or fast-growing species that benefit from intensified gap creation. Models will only get us this far. Experimental liana removal in combination with full-life cycle studies will be useful in investigating this further. 

Lianas are highly diverse group of organism that are considered key components of tropical forests \citep{Schnitzer2002}. \citet{Croat1978} identifies 175 species of liana on Barro Colorado Island alone. Lianas are also vital components of forest ecosystems, having profound impacts on tree growth and regeneration \citep[][Chapter 7]{Schnitzer2000, Ingwell2010}, animal diversity \citep{Montgomery1978, Yanoviak2015} and ecosystem level processes including carbon sequestration \citep{Heijden2015}. Yet almost nothing is known about the mechanisms that regulate liana abundance \citep{Muller-Landau2016}. Chapter 8 clearly demonstrates the complex interplay between host-tree and liana-parasite, wherein lianas dynamics shape their host's dynamics and are in turn shaped by their host's dynamics. If it is true that "shared parasitism" plays a large role in structuring tree communities, then understanding coexistence will depend on an understanding of the joint maintenance of parasite and host diversity. Ultimately, everything discussed above suggests a significant role for generalized micro and macro parasites in the structuring of tree communities and - as shown in chapters 7 and 8 - there exists fertile ground for future inquiry along these lines for both pathogens and lianas. 

\subsection{Competition for a scarce resource: dispersers}
Within-species competition for scarce resources can also cause negative density dependence, and, on larger spatial scales, seed dispersers are a finite resource for which trees must compete. Chapter 3 shows that intense competition for a finite population of animal dispersers leads to complete collapse of seed dispersal. However, as discussed earlier, the pattern of negatively \end{fullwidth} density-dependent dispersal across \textit{Attalea} populations had little influence on NDD of recruitment at the local scale (see Fig \ref{fig:chap3fig4}). Unfortunately, this analysis of seed arrival and recruitment was limited to the local scale, ignoring the effect of seed emigration outside each plot. How far organisms move influences how that organism perceives and responds to their environment (e.g. Levin 1992), and affects the scale at which population regulation takes place \citep{Turchin1995}\sidenote[][-2.5cm]{The scale of population regulation depends on the importance of movement relative to the birth-death process.}. Here, I investigate the implication of diminishing rates of seed emigration for coexistence at the scale of the community. \begin{fullwidth} 

\hspace{0.4cm} In general, the net contribution of intraspecific disperser competition to coexistence is complex to predict, as it depends on the interaction between seed dispersal and natural enemies \citep{Adler2005, Muller-Landau2007}. For example, poor dispersal causes local sites to become swamped with seeds, and when these seeds belong to a stronger competitor, they may displace other species \citep{Pacala1997}.  On the community scale this will lower diversity, unless countered by strong negative density-dependent offspring survival \citep{Bagchi2010}. Therefore, we may expect that stronger stabilization is needed when dispersal fails. On the other hand, negative density-dependent dispersal may also increase diversity. Poor dispersal will increase recruitment limitation as fewer seeds reach sites suitable for regeneration \citep{Nathan2000, Dalling2002}. Under this scenario, dispersal fails as a species becomes more abundant and emigration rates drop as does the expected number of recruitment sites reached (i.e. this increases recruitment limitation). A preliminary understanding of the community level effects of NDD dispersal would therefore require the integration of dispersal, competition for limited recruitment sites and interactions with natural enemies. I have set out to do this in box 1. 

\shadowbox{
\begin{minipage}{13.5cm}

\textbf{Box 1: Consequences of forest wide negative density dependent dispersal.}
Seed dispersal, recruitment limitation and the Janzen-Connell effect will interact, and can strongly affect the competitive dynamics of species. This can be demonstrated with a simple lattice model described by Pacala (1997), adapted to include negative density dependence of dispersal. In this model, a forest is viewed as a grid of patches, each occupied by a single adult tree. There are 1000 patches, and each patch represents a single tree crown. If each patch is 20 $\times$ 20 m, the scale of the model is 40 ha. In a single time step every tree dies with probability D, regardless of species identity. D was set to 30\%, so each time step represents approximately 30 years if annual mortality is 1\%. Tree deaths create empty patches that are colonized by the winner of a simple lottery. The probability of winning the recruitment lottery is proportional to the number of seeds the species disperse to the patch. Each tree, at each time step, disperses a proportion of E seeds (emigrants) which are spread out evenly over all other patches - and a proportion of 1-E seeds don't emigrate to outside the natal patch. E = 1 corresponds to perfect dispersal, all seed dispersal beyond the natal patch and E=$0$ corresponds to complete dispersal failure with all seed staying in the natal site. All species have equal fecundity. Natural enemy attack is simulated by reducing the number of seeds from species i in a patch (formerly) occupied by species i with proportion S. S =1 corresponds to no natural enemy attack. S = 0 represents complete local density dependence where all seeds die that end up in a patch with a conspecific adult. There is no seed bank or dormancy. Negative density dependent dispersal is introduced by making E a logistic function of community frequency as shown in Fig \ref{fig:chap10fig1}A. The shape the curve in panel A is based on the fitted dispersal kernel of \textit{Attalea} (from Chapter 3), which predicts the number of seeds dispersing beyond 20 meters (panel B). \vspace*{.01cm}

\hspace{.3cm} The model predicts that species richness is greatest when no offspring survives in conspecific patches (Fig \ref{fig:chap10fig1} C; S= $0$ when density-dependence is at its strongest). Poor dispersal and weak natural enemy attack (S>>$0$) always results in monodominance, as in neutral models without speciation and immigration from a meta-community. In essence the probability of eventually obtaining monodominance is proportional to species abundance as, in the model, recruitment sites are unlikely to be won by immigrants of rare species (with too few adults, species rapidly become seed-source limited and disappear). Efficient dispersal (E>0.5) allows more seeds of competing species to reach recruitment sites increasing species richness. Negative-density dependent dispersal always increases species diversity, as when a species becomes too common the number of seeds that reach recruitment sites drops. 
\end{minipage}
}


\shadowbox{
\begin{minipage}{13.5cm}

\textbf{Box 1 (continued): }
In combination with strong natural enemy attack, negative density-dependent dispersal maximizes species richness on the community scale. This takes place because more seeds land in natal patches, where deaths due to natural enemies is high, strongly limiting the dominance of common species (in the number of total seeds). This final interaction between NDD dispersal and natural enemies will increase population-level stabilization. The model is highly simplified, but highlights some pathways through which NDD dispersal may help to delay or prevent complete competitive exclusion and monodominance.  Simulations were run in R, with key computationally heavy components written in C++ (\textit{sensu} Chapter 9). All code required to run this model including model parametrization is provided on https://github.com/MarcoDVisser/thesis.
\end{minipage}
}



\begin{figure*}
\hspace*{.5cm}\includegraphics[width=12cm,height=14cm]{../figures/Chap10Fig1.png}
\caption[A community model of negative density-dependent dispersal][34cm]{.}
\label{fig:chap10fig1}
\footnotesize
\hspace*{1cm} \begin{minipage}{11cm}
\vspace*{0.1cm}
Figure \ref{fig:chap10fig1}: 
A) Empirical trend of the fraction of \textit{Attalea} seeds dispersing beyond 20 meters, plotted as a function of adult frequency. The dispersal rate was calculated from the previously fit models given in chapter 4, and frequency is calculated assuming there are approximately 100 adult trees per hectare.  
B) The fraction of seed dispersing beyond the natal patch, as a function of community frequency. Species dispersal efficiency (E), decreases with the fraction of patches it occupies under NDD dispersal in the model. 
C)  Model results after 250 000 time steps (or generations), starting with 50 randomly spaced species and no immigration from a meta-community. The graph shows species richness as a function of offspring mortality S in conspecific natal patches and dispersal efficiency (the emigration rate; E). The colored lines indicate model outcomes with (shades of blue-green) and without (shades red-yellow) negative frequency dependent respectively.  Shades correspond to values of E (varied between 1 and 0) as indicated by the color keys on the right of panel C.
\end{minipage}
\end{figure*}

\end{fullwidth} Box 1 summarizes results expanding on a simple lattice model described by Pacala (1997) with NDD dispersal. The model is a simple representation of a forest community using patches in a lattice. Here all species are competitively equal and compete for recruitment sites via local lotteries - the species with the most seeds at a recruitment site is most likely to win. Seeds disperse locally (under the parent) or globally based on a dispersal parameter, and survive under conspecifics based on a survival parameter which mimics the Janzen-Connell mechanism (Janzen 1970; Connell 1971). The model shows that for any given dispersal efficiency, diversity is maximized when NDD is strongest \citep{Bagchi2010}. Here, as no seeds from the previous occupant persist, recruitment patches cannot be won by a single species in two consecutive generations. However, when enemy attack is not complete (survival close to parents is greater than zero), more numerous species tend to have greater probabilities to recruit. This leads to displacement of others species at a rate dependent on dispersal efficiency. In the model, negative density-dependent dispersal increases diversity over the entire gradient of enemy attack strength simply because it causes fewer seeds from abundant species to reach recruitments sites. Here, when a species becomes more abundant, fewer of its seeds reach available sites. This tends to equalize the probability of recruitment among species in lottery recruitment\sidenote[][-10cm]{This model is of course a simplification. In reality as species become common the distance between a recruitment site and the nearest adult tree will also decrease on average. The model result is still valid, but will depend on how rapidly the dispersal distances decrease with each additional adult v.s. how rapidly the expected distance between two adults decreases with each adult added. This simple model is only a first pass exploration of the large scale implications of NDD dispersal and details like these must be kept in mind when interpreting the results.}.\begin{fullwidth}
 
\hspace{.5cm} The simple recruitment lottery model in box 1 shows that NDD dispersal may play a diversity-enhancing role at the community level by increasing recruitment limitation for common species - though the net effect depends on the strength of natural enemy attack. An obvious follow-up question is where do real communities lie with respect to the two parameters dictating dynamics in the model of box 1? First, evidence suggests that enemy attack is seldom complete (natal patch survival zero; S=$0$), as theoretically such strong negative density-dependent survival of offspring creates repelled distributions of offspring around parents \citep{Bagchi2010}. In reality, many tropical trees species show aggregated spatial distributions \citep{Condit1992, Condit2000}, and hence pathogen attack will be greater than zero (S>0). Second, dispersal efficiency is generally at the low end (small E), as the vast majority of seeds usually fall close to the parents \citep{Clark2005, Muller-Landau2008, Putten2012}. Finally, the vast majority of tree species are vertebrate dispersed (at least in the neotrpics; \citealt{Howe1982}), so intraspecific competition for seed dispersers may be a frequently occurring phenomenon. Hence, it is plausible that real communities are situated somewhere along the lower end of the blue shaded lines in Fig \ref{fig:chap10fig1}C in box 1. Negative density-dependence may plausibly play a role in countering the competitive edge of the most numerous species, by increasing their recruitment limitation.

\section{Integration across the life-cycle}

\subsection{From the individual to the population} 
Coexistence theory is based on population-level outcomes of interactions between species \citep{Chesson2000, Adler2007}. However, empirical studies often focus only on a small subset of the processes that affect populations \citep{Metcalf2007}, and assessing the ultimate impacts of such interactions requires integration across all relevant processes and life stages (e. g. chapters 4,6,7). A fundamental question brought up in this dissertation was whether the density-dependent responses measured for seeds and seedlings \citep{Harms2000a, Comita2010} are indicative of changes in the population growth rate. In chapter 4, integration of data across the full life-cycle showed that - at least for one tropical palm species - observed negative density dependence between the seed and seedling stage was sufficient to regulate the population. Yet, chapter 4 teaches a second lesson, that the strength of negative density-dependence may be mediated by the environment.

\textit{Attalea} seedlings, in old-growth, closed-canopy environments must persist in low light. Here, they obtain little resources other than those required for maintenance \citep{Araus1994}, and as a consequence they hardly grow (Fig  \ref{fig:chap4fig2}C). In such environments, they never develop rosettes and complex leaves (Fig \ref{fig:chap4fig2}E) making any transition towards the mature stage very unlikely. Low light levels will also make young \textit{Attalea} seedlings more vulnerable, as they lack the photosynthates needed to recover from pathogen attack. Consequently, at the population scale, births and deaths are matched at low abundance. Whenever disturbance creates a lighter environment, this limiting bottleneck is relaxed. Young \textit{Attalea} seedlings have more resources, simultaneously allowing fast growth, increased transition towards adult stages, and higher tolerance of pathogens. The result is that births and deaths are balanced at higher densities - at least temporarily - until adults start to shade offspring and the environment becomes limiting again, slowing dynamics and increasing vulnerability to pathogens. In \textit{Attalea} populations, negative density dependence will therefore arise due to competition for light resources with adults, and once seedlings are shaded due their increased vulnerability to pathogens.  Population regulation and limitation of \textit{Attalea} must be viewed within the context of the abiotic environment.  \textit{Attalea} populations are foremost light-limited and the observed level of negative density-dependence in chapter 4, can be viewed as a symptom of \textit{Attalea}'s inability to perform in shaded conditions\sidenote[][1cm]{This observation also leads to a pragmatic explanation for the observation that rare species suffer from more intense negative density dependence \citep{Comita2010, Mangan2010}. Rare species, compared to abundant species, are simply less optimally adapted to the abiotic environment - and as they struggle to survive in a suboptimal environment they are more vulnerable to pathogens.}. How general is this result? \end{fullwidth}  
 
One other life cycle study exists for the gap specialist \textit{Cecropia obtusifolia} \citep{Alvarez-Buylla1994}. \textit{Cecropia} populations were indeed regulated by negative density dependence in a single gap, but on larger spatial scales the population structure, relative abundance, and population growth rate depend principally on the frequency of gaps \citep[see e.g.][]{Alvarez-Buylla1994}. A fair expectation is that both \textit{Cecropia} and \textit{Attalea} are light-limited \citep{Foster1982a, Alvarez-Buylla1994, Araus1994}. More precisely, the light environment is the most important factor limiting their populations, and thus the best predictor of their abundance on larger spatial scales will be disturbance history. Can we, therefore, conclude that populations are foremost limited by the capacity of the environment to support them \citep[e.g.][]{White2004}? Empirical studies that integrate negative density-dependence across the entire life-cycle are exceedingly rare for tropical trees, and hence the generality of this conclusion is an open question. \begin{fullwidth}
 	
\hspace{.5cm} The very traits that allow species such as \textit{Attalea} to dominate in disturbed environments \citep{Araus1994, Souza2004}, and \textit{Cecropia} to dominate tree-fall gaps of certain sizes \citep{Brokaw1987}, limit their ability to survive and transition towards maturity in a predominately shaded environment \citep[e.g.][]{Sterck2011}. This trade-off \sidenote[][-4cm]{This is known as the growth-survival trade-off, which has been related to wood density (Wright \textit{et al.} 2010).} is the source of the bottleneck in \textit{Attalea}'s life-cycle. Shade-tolerant trees are limited because they grow slowly in the dark, and may experience additional key limiting bottlenecks along other axes. Broadly speaking, however, light cannot be a dominant limiting factor for most tropical tree species. For instance, the majority of species on Barro Colorado Island are small-statured species (Fig. \ref{fig:chap10fig2}). Such shade-tolerant, slow growing and early-maturing species, like \textit{Mouriri myrtilloides} or \textit{Rinorea sylvatica}, are unlikely to be limited by canopy disturbance. Yet, how are they limited instead? Do important bottlenecks exist somewhere in their life cycles? Chapter 5 exposes two more trade-offs (Figs. \ref{fig:chap5fig1} \& \ref{fig:chap5fig2}): 1) the benefits of early reproduction are balanced by a shorter life-expectancy while, in species that mature at large size, relatively few individuals reach maturity \citep{Kohyama1993, Metcalf2009}; 2) the establishment benefits of large seeds are balanced by concomitant decreases in fecundity while copious numbers of small seeds are limited by hefty bottlenecks in early recruitment \citep{Alvarez-Buylla1994, Muller-Landau2010}. Each tree species at BCI can be found at some point along these three independent trait axes (Fig \ref{fig:chap10fig2}). It may very well be that shade-tolerant trees are limited because they grow slowly in the dark, and additionally experience limiting bottlenecks along other axes. The extent and importance among tropical tree species of limiting bottlenecks in determining the species abundances, equalizing fitness, and whether or how such bottlenecks originate from these three common trade-offs remains to be tested.

Negative density dependence has been shown to be almost universal across many species and taxa \citep{Harms2000a, Brook2006, Comita2014}. Yet, the examples of \textit{Attalea} and \textit{Cecropia} discussed here show that the role negative density dependence plays in determining a species' relative abundance must be viewed in the context of the abiotic environment and species ecological strategy (its traits and associated trade-offs). This result parallels what was achieved after a longstanding and heated debate in population ecology \citep[see e.g.][]{Turchin1995}: the dynamical patterns of persisting populations are the result of both density-dependent and density-independent processes \citep{Hassell1986}. Patterns of species abundance and diversity cannot be understood by investigating negative density-dependence alone.
 
\subsection{Trade-offs and traits across the life cycle}
\end{fullwidth}
In evolution, trade-offs sanction investment in phenotypic traits. Being good at one thing hampers one's ability in another \citep{Stearns1992, Fabian2012}. This prevents the emergence of "Darwinian demons"\sidenote[][-1cm]{Darwinian demons are hypothetical organisms that theoretically can evolve when selection is completely unconstrained (grow rapidly, reproduce continuously and don't age; \citealt{Law1979}).}, as a species' competitive ability is always context-dependent - you cannot win under all circumstances. In community ecology, trade-offs are expected to promote the maintenance \begin{fullwidth} of diversity as they foster niches among species \citep{Chesson2000}: each species' competitive ability is dependent on the environment. A trade-off therefore forces species to specialize towards one set of conditions, i.e. a niche. Selection towards fast growth in the light precludes survival in the dark \citep{Gilbert2006, Wright2010, Sterck2011}. Yet, the very traits thought to represent key trade-offs have discouragingly low power to explain species performance (details in chapter 5). How important can functional traits and associated trade-offs be when they cannot explain meaningful amounts of variation in species vital rates, and thus by extension fitness? 

	The response of an individual tree, however, will reflect its traits in relation to its environment \citep[e.g.][]{Sterck2005}. Concurrently, as individual trees grow in size they experience large shifts in the biotic and abiotic conditions they experience \citep[e.g.][]{Poorter2005}. Additionally, as the environment experienced changes often strikingly from seed to forest giant, different traits may be important at different phases during each trees life. Without the correct integration of this changing environment across ontogeny, trait-mediated responses may be hard to detect. In chapter 5, we therefore assessed the predictive power of traits across ontogeny and hence across all environments experienced (by the average individual). Accounting for size and the interacting effect of three species-level traits allowed us to explain on average 41\% of interspecific variation in vital rates across the life cycle. Moreover, effects of traits on one life stage or vital rate were offset by opposing effects at another stage, which is the signature of life-history trade-offs. In chapter 6, we integrated the results from chapter 5 and show that all three traits have strong impacts on per capita population growth rates (Fig \ref{fig:chap6fig4}), demonstrating the importance of the associated trade-offs (chapter 1, box 1). To what extent these three axes equalize fitness remains to be demonstrated, but the evidence arising from the cost-benefit analysis, in chapter 6, does at least show that the costs of dioecy as a breeding strategy are counterbalanced by the benefits. No discernable fitness differences between hermaphroditic and dioecious species could be detected. Taken together, chapters 5 \& 6 show that species performance at the population level is influenced by multiple trade-offs simultaneously. This means that, among competing species, fitness is constrained - and potentially equalized - along multiple independent axes (see Fig. \ref{fig:chap10fig2}).
	 
Decreases in population growth rates due to liana infestation differed strongly among species, with the majority of variation explained by each species' position on the growth-survival trade-off (chapter 7, Fig \ref{fig:chap7fig4}).  This response was highly predictable (R$^2$= 0.61), and the predominant cause was the effect of liana infestation on tree survival with effects on reproduction or growth hardly contributing to the trend. Essentially, the act of scaling up narrowed things down, revealing which lower-level mechanisms were influential at the population scale.  In conclusion, species population level responses to environmental change depend on species ecological strategy and, critically, the importance of the vital rate and life-stage upon which change acts \citep{Zuidema2001}. The response can be predicted based on an understanding of the species ecological strategy associated with the most influential life-stage and/or vital rate. Collectively, chapters 5-7 provide evidence that species differ in their ecological strategies, that communities are therefore functionally divergent, and that species responses to change vary predictably at the population scale. 


\begin{figure*}
\hspace*{-.1cm} \includegraphics[width=15cm,height=15cm]{../figures/Chap10Fig2.png}
\caption[Positions along three important trait axes for 151 species][35.6cm]{.}
\label{fig:chap10fig2}
\footnotesize
\hspace*{.1cm} \begin{minipage}{14cm}
\vspace*{.4cm}
Figure \ref{fig:chap10fig2}: 
Scatterplot of 151 species, from the tree community on Barro Colorado Island, showing their positions along three important trait axes (as determined in chapter 5). The axes include wood specific gravity (g/cm$^3$), maximum height (m) and seed mass (g). Panel A, shows each species position in three dimensional space, colors and numbering correspond to a 3 dimensional clustering of species determined through k-mean clustering.  The plane indicates the projection shown in panel B, that of wood specific gravity against maximum height. The colored polygons show the projections of each convex hull, encompassing all species within a cluster, on the plane created by the axes of wood specific gravity and maximum height. Panels C \& D show the projection on the planes of wood specific gravity and maximum height against seed mass respectively. Each axis can be associated with a trade-off at a particular life stage: seed mass with seed production and establishment, wood density with development through ontogeny, and maximum height with the timing of reproduction. Projected convex hulls, that delineate species groupings, show large overlap and there are no significant correlations between the axes. This indicates independent ecological strategies and different evolutionary constraints at different life stages.
\end{minipage}
\end{figure*}


\subsection{The importance of cross-scale comparisons in community ecology}
\end{fullwidth}
I have been immensely fortunate to be able to work on multiple large-scale and long-term datasets that enabled this investigation across many levels of organization. The forest dynamics plot on Barro Colorado Island in Panama was established before I was born and many of the datasets were started before I was a teenager. Therefore, it is the foresight of my predecessors\sidenote[][-2cm]{Steve Hubbell and Rick Condit established the BCI 50 ha plot in 1980, two years before I was born. Joe Wright established his seed traps in 1987, when I was 5 years old, and his seedling plots when I was 12. Ages at which I was blissfully unaware of the issue of scale. Around the time I finally figured out I wanted to work in biology, Liza Comita established her 20 000 seedling plots. If I have seen across scales, it is by standing on the shoulders of these giants} that has granted me the chance to integrate across scales. Often my expectations turned out to be false during the scaling-up exercises described in each chapter. Stephen Pacala notes that "humans have remarkably poor quantitative intuition, and so require models to determine the quantitative consequences of even simple assumptions" \citep{Pacala1997}. This is most certainly the case for me, and the writing of this thesis has left me with the impression that the activity of using models to scale up phenomena is indispensable to ecological science. \begin{fullwidth}

\hspace{.4cm} In his seminal paper on pattern and scale in ecology, \citet{Levin1992} deduced three key points. First, he reasoned that theory and models play an essential role to demonstrate that specific lower-level patterns can at least give rise to the pattern at the scale of interest. Second, he argued that integration of highly detailed processes across scales forces abstraction, a step vital towards recognizing what fine detail is relevant and what is noise at the level of interest. Finally, Levin then predicted that patterns will tend to become increasingly predictable, with increasing abstraction towards higher scales.  This thesis bears witness to the validity of these statements. First, integration across scales showed that perceptions of importance are scale-dependent. Clear patterns at one scale crumble to noise at another scale (chapters 2-4) and ecological conundrums are not as large a mystery when all life stages are considered (chapter 6). Second, only a handful of processes are influential in predicting sensitivity to liana infestation and prevalence (chapters 7 \& 8). Finally, abstraction toward larger scales most certainly increased predictably: individual tree responses are highly unpredictable but predictability increased 2.8 fold at the species level\sidenote[][8.5cm]{R$^2$ values of individual variation averaged 14.6\% while species-level R$^2$ values averaged 41\%.}. (chapter 5), and population-level responses to lianas were substantially more predictable that any lower-level vital rate (chapter 7).  When all chapters are taken together (table \ref{tab:chap10tab1}) they demonstrate the profound influence scale has on how we interpret results and understand the links between processes operating at different levels. 

The work documented here, and elsewhere \citep[e.g.][]{DeKroon2016}, issues a warning: mechanisms observed at fine spatial, temporal and organizational scale may not translate well towards larger, theoretically more appropriate scales. Almost every chapter in this thesis shows that there are biases attributable to the investigator's choice of scale (table \ref{tab:chap10tab1}). Most ecological studies focus on scales that are amenable to experimental tests. Reviews in the literature show that the vast majority of ecological studies choose scale arbitrarily \citep{Wheatley2009}, focusing on fine spatial scales with replicates at the scale of a meter or less \citep{Kareiva1988, Gardner2001}, on single life stages such as seeds or seedlings \citep[e.g.][]{Comita2014}, and only a tiny fraction have a duration of 10 years or more \citep[e.g.][]{Salguero-Gomez2015}. As table \ref{tab:chap10tab1} demonstrates, it is not guaranteed that dynamics at \end{fullwidth} fine scales are influential at the larger scales relevant to our research questions. This does not imply that the vast amounts of work at finer scales are not major achievements\sidenote[][.5cm]{As argued by Simberlof 2004, understanding of lower level details are critical to a mechanistic understanding of a system.}. Rather, it simply implies that a singular focus on small spatial, temporal or single organizational scales may blind us. Confounding issues can arise when data, characterizing a single temporal, spatial or organizational scale, is used to draw conclusions about potential impacts on a different scale. This is a lesson that must always be borne in mind. 


\begin{landscape}
\begin{table}
\begin{center}

\hspace*{4cm}\begin{minipage}{20cm}
 \begin{center}
 Table \ref{tab:chap10tab1}
\end{center}  
\end{minipage}
\footnotesize
\hspace*{4cm}\begin{tabular}{p{1cm} p{2.5cm} p{5cm} p{5cm} p{5cm}}
\hline
\multicolumn{5}{c}{Scale} \\
\multicolumn{1}{c}{Phenomenon} & \multicolumn{1}{c}{Topic} & \multicolumn{1}{c}{Small*} & \multicolumn{1}{c}{Population} &  \multicolumn{1}{c}{Community**} \\
 \hline
 \parbox[t]{3mm}{\multirow{4}{*}{\rotatebox[origin=c]{90}{\textbf{Negative density dependence (NDD)}\hspace{1cm}}}} & 
 Seed predation	
 & Proportion of seeds that escape predation increases with distance from nearest seed-bearing adult. Conditions emphasized by Janzen (1970) present, which should lead to population-level density dependence of seed predation.  
 &  Proportion of seeds escaping predation is density-independent across populations spanning twenty-fold variation in adult density - contrary to the expectation generated by patterns at finer scales. Trophic interactions at the population level responsible for pattern (Fig \ref{fig:chap2fig1}). \vspace{0.2cm}  
& Disconnect between fine scale and population scale shows that studies on finer scales are inconclusive for regulation at the population level and higher.  \\
 & Dispersal 	
 & Dispersal 27x less influential predictor of seed arrival and local environment experienced by seedlings compared to adult density (Fig. \ref{fig:chap3fig4}). \vspace{0.3cm}
 & Strong NDD dispersal shown across populations spanning 40 ha of forest. 	
 & Potentially a strong increase in dispersal limitation with increasing abundance (box 1, Ch 10).\\
 
 & Recruitment 	
 & NDD highly influential in determining recruitment patterns in \textit{Attalea} (Fig. \ref{fig:chap4fig2}, R$^2$=0.5).	
 & Projected relative abundance most sensitive to density-independent factors (Fig. \ref{fig:chap4fig5}). Relative abundance should be most sensitive to disturbance history.\vspace{0.3cm}	
 & \textit{Attalea} abundance best predicted by disturbance history across the whole Island (Garzon-Lopez \textit{et al.} 2014). \\
 
& Stabilization
& Not relevant at this scale.
& NDD in recruitment was a strong determinant of the change in $\lambda$ with adult density (Fig. \ref{fig:chap4fig5}).	
& Theoretical requirement for coexistence (Chesson 2000). Extension towards multiple species required to determine generality.\\
\hline 

\parbox[t]{2mm}{\multirow{3}{*}{\rotatebox[origin=c]{90}{\textbf{Trade-offs and ecological strategies \hspace{0.1cm}}}}} 
& Functional traits.	
& Weak predictor of individual survival, growth or reproduction (mean R$^2$ $\approx$ 15\%; table \ref{tab:chap5tab3}). \vspace{0.3cm}
& Moderate to good predictor of species dynamics (mean R$^2$ $\approx$ 41\%, Fig. \ref{fig:chap5fig3}).\vspace{0.3cm}	
& Evidence of continuous variation in strategies along 3 axes (Fig. \ref{fig:chap10fig2}).  \\

& Equalization
& Not relevant at this scale. 
& Benefits of dioecy balanced by costs at the population average (Fig. \ref{fig:chap6fig4}). 
& Evidence for equalization. Requires extension to multiple trade-offs to determine generality. \vspace{0.3cm}  \\

& Liana prevalence  
& Static observations reveal higher liana prevalence among shade tolerant species. Hence, shade-tolerant species must suffer more from liana increases. 
& Dynamic observations show liana prevalence mostly determined by the rates of recovery and lethality (Fig. \ref{fig:chap8fig2}). Shade-tolerant species in fact more tolerant of liana infestation (Fig. \ref{fig:chap7fig4}). Causation the reverse of the static prediction.	
& Condition for parasite mediated competition. Shows that tree species will respond differentially to shifting lianas abundance, additional evidence for divergent ecological strategies. \vspace{0.4cm} \\

 
 \hline
 \end{tabular}
\label{tab:chap10tab1}
 \hspace*{4cm}\begin{minipage}{20cm}
 Table \ref{tab:chap10tab1}: 
* Small scale refers to replication across a fine spatial scale (i.e., m2, seedling plot), individual tree, single life-stage, single trophic level or a static observation. \\
** With a few exceptions, predictions at this scale are not the result of cross-scale integration and are just as suspect as the predictions from fine to population scale.  
\end{minipage}
\end{center}
\end{table}
\end{landscape}


\subsection{Future directions}
\begin{fullwidth}
Anthropogenic influences are changing the balance of forces maintaining species composition and diversity in plant communities - and thereby affecting carbon stores and fluxes, water runoff and transpiration, and other ecosystem services to humanity \citep{Chapin2000,Hooper2012}. Recent work forewarns that biodiversity may be altered by factors that increase the mortality of larger trees, such as increased frequency of extreme droughts \citep{Bennett2015} or drastic increases in liana abundance \citep{Phillips2002, Schnitzer2011}. How can we determine which of these scenarios present the greatest threat, when we know virtually nothing about the relative importance of different demographic processes that structure forest communities?

Two decades after Levin's influential paper on the role of scale in ecology \citep{Levin1992}, much of formal community theory is still focused on a single scale, assuming that local communities are closed and isolated \citep[see e.g.][]{Leibold2004}. However, exciting new developments are on the horizon. Community ecologists are starting to integrate scale in analyses and theory \citep[see e.g.][]{Chave2013}. New developments, such as scale transition theory \citep[e.g.][]{Chesson2012} , are providing frameworks to quantify scale-dependent diversity maintenance mechanisms - though applications are still scarce and usually limited in complexity (e.g. to simple one-dimensional systems as streams; \citealt{Holt2016}).  Theoretical developments along these lines provide the ability to assess the combined effects of multiple mechanism acting simultaneously across scales on patterns of diversity \citep{Benedetti-Cecchi2012}. Multiscale theory is likely to drive much research in the years to come. 

Theory can, however, only create a list of possible mechanism that drives population and community dynamics. The mathematical demonstration that a specific mechanism can give rise to patterns at the scale of interest, is no proof that the mechanism is actually responsible for generating this pattern \citep{Levin1992}. The dilemma is that experiments are needed to dissect this catalogue of theoretical mechanisms, but experiments at a scale that encompasses tropical forest communities are impractical, infeasible, and/or unethical. How then can we differentiate among the many theoretical explanations without experimentation?  This thesis demonstrates that traditional techniques\sidenote[][1.5cm]{Traditional approaches as, for instance, used in this thesis: model systems and cross-species comparisons.} can be complemented with approaches that assimilate data across scales, showing which local processes are influential at higher scales and so narrow down the long list of theoretical mechanisms. \end{fullwidth} 

One particular powerful tool towards this goal is sensitivity analysis. Demographic sensitivity analyses of full-life cycle models \citep{DeKroon1986} have a strong track-record in being useful \begin{fullwidth} - from conservation \citep{Caswell2000, Morris2003} and wild life management \citep{Wallace2013} to the courtroom \citep{Swartzman1996} - exactly because they quantify the relative contributions of the lower-level demographic processes that underlie population change. The extension of such an approach to connect individual-, population- and community-level dynamics is needed. To scale up individual dynamics towards the multispecies community, however, requires understanding of interaction networks among species and exactly this is a formidable challenge. We must understand not only how each species affects itself, but also how it affects others. For instance, \citet{Adler2010} quantified all interactions between just 4 competing grassland species, and generated a model with > 500 parameters. Conducting a similar study for the 300+ tree species on Barro Colorado Island would be impossible. Yet, work in this thesis shows that species responses may be predicted from traits, and recent progress made by \citet{Kunstler2016} shows that the same traits used here (chapters 5 \& 6) consistently influence competitive interactions.  Future work that brings together full life-cycle data (e.g. chapters 4 \& 7) and competitive interactions - while using functional traits (chapters 5 \& 6) to remain parsimonous and tractable - can provide understanding of what forces are essential in structuring communities across scales unamenable to experimentation.   

Species coexistence is a problem of multiple scales in space, time and organization. \citet{Lawton1999} stated that traditional community ecology is "intensely local in focus" and as a result will never understand complex systems. Today, ecologists have realized that communities should not only be studied locally but across a broad range of sites \citep{Chave2013, Borer2014}, and levels of biological and ecological organization \citep{Holt2007, Holt2016}. The translation and interrelation of phenomena across these scales is a critical step towards understanding and ultimately sustaining the diversity of tropical forests, the most diverse ecosystem on our biodiverse planet \end{fullwidth}

%--------------------------------------------------
\backmatter

%----------------------------------------------------------------------------------------
%	BIBLIOGRAPHY
%----------------------------------------------------------------------------------------
\begin{fullwidth} 
\footnotesize
\bibliography{MarcoDVisser_AllReferences} % Use the bibliography.bib file for the bibliography
%----------------------------------------------------------------------------------------
\end{fullwidth} 


%----------------------------------------------------------------------------------------
%	Abstract
%----------------------------------------------------------------------------------------

\chapter{Summary} % The asterisk leaves out this chapter from the table of contents

\begin{fullwidth}
Tropical forests are so rich in tree species that we don't even know how may tree species exist, and identifying the mechanisms structure this diversity has long challenged ecologists. The question is important, because only from an understanding of how diversity is generated and maintained, can we predict how tropical forests (and their associated services such as carbon storage) will be altered by global anthropogenic change. There has been ample interest in the subject, generating a vast body of work on the topic of diversity maintenance. In theory this body of work should help scientists predict how forest will change in diversity and composition  in relation to different threats. In practice, however, there is a disconnect because previous work has almost exclusively focused on fine spatial scales and limited levels of biological organization - such as a single trophic level and life stage.  Do processes relevant at a single spatial, temporal or organizational scale translate across scales? The overarching theme of this thesis is to scale up processes across spatial and organizational scales, evaluating whether and how processes at fine scales translate to higher levels of organization. In particular this dissertation takes a look at processes hypothesized to structure tropical forest communities across broad scales that encompass entire life-cycles, populations, communities and multiple trophic levels.  

Investigations focus on Barro Colorado Island in the Republic of Panama, one of the most intensely studied tropical forests. In general, the 8 chapters presented integrate several large-scale datasets and provide a look at ecological patterns, such as negative density dependence and trade-offs, over many hectares of forests and across multiple life stages and vital rates. Two general scientific approaches are used. The first is a model system of the palm \textit{Attalea butyracea}, studied on large spatial scale and for all life-stages, across ten 4-hectare plots spanning 20-fold variation in densities of adults (Chapters 2-4).  The second approach is a cross-species comparison, where we integrate 7 large, long-term datasets that included millions of observations spanning a three decades from seeds to forest giants (Chapters 5-8).   

Chapter 2 capitalizes on the existence a multi-year record of seed predation preserved in the soil bank. Excavation of the durable fruits from the palm \textit{Attalea butyracea} provides an opportunity to investigate tri-trophic interactions on an island-wide scale. Results show that the notion that species' performance is limited by host-specific natural enemies (so called Janzen-Connell effects) may be complicated by the enemies' enemies.  Predation of host-specific bruchid larvae by rodents negated the density-dependence of bruchid attack, causing seed survival to become density independent. Earlier work showed that Janzen-Connell effects are present at the local spatial scale, but chapter 2 reveals that this effect not necessarily translates to higher trophic and larger spatial scales relevant to the population. 

Chapter 3 tests whether intraspecific competition for dispersers results in negative density-dependent dispersal and whether this contributes to negative density dependence of recruitment. Frugivore visitation, seed removal and dispersal distance all declined with population density of \textit{A. butyracea}, demonstrating negative density dependence of seed dispersal and clear competition for dispersers.  Yet, unexpectedly, negative density dependence of dispersal across large spatial scales did not contribute substantially to changes in the quality of the seed distribution at finer spatial scales; seed densities and recruit neighborhood density patterns were dominated by effects due solely to increasing adult density. Hence, chapter 3 shows that strong patterns across large spatial scales may be not be influential at the smaller scales relevant to the dynamics of seeds and seedlings.

Chapter 4 integrates negative density dependence across the life cycle, using a population model. Results show that negative density dependence at the seedling stage is strong enough to regulate \textit{A. butyracea} populations well before \textit{A. butyracea} dominates forest stands. The observed negative density dependence allows recovery from very low densities, and hence the system meets the basic conditions for coexistence.  However, crucial bottlenecks in the life cycle exist that heavily influence projected equilibrium densities. \textit{A. butyracea} populations can attain far higher densities once the species is released from key limiting bottlenecks, which are density-independent, and may be achieved through environmental disturbance. Chapter 4 shows that patterns of species abundance and diversity are jointly influenced by density-dependent and -independent factors that limit and regulate the abundance of species. Moreover, the combined results of chapters 2-4 reveal that factors critical in determining patterns at a single spatial scale or life stage may be inconsequential at larger scales and levels of organization.

Chapter 5 provides a novel analysis of a very comprehensive set of relationships between traits and vital rates, by combining demographic data on seeds, seedlings and trees with data on four traits (seed mass, wood density, leaf mass area and adult stature). Results show strongly increased predictive power above previous studies, and a much more complete picture of how traits and vital rates interact across all life cycle components. Chapter 5 provides evidence that tropical tree species are simultaneously limited by multiple evolutionary constraints that act at different life stages.  Chapter 5 also shows that the effects of traits on one life stage or vital rate were sometimes offset by its opposing effects at another stage, revealing the danger of drawing broad conclusions from analyses of a single life stage or vital rate only.

Chapter 6 examines the fundamental ecological conundrum of how dioecious species can coexist with non-dioecious species when only approximately half of the individuals -the females- produce seeds. A novel composite Integral Projection Model is applied that, building on chapter 5, uses three functional traits (seed mass, wood density and adult stature) to control for variation in life-history among tree species. Results show that dioecious populations especially benefit from higher individual seed production, and when integrating all effects across the life cycle these benefits almost outweigh the demographic costs of having males. Nevertheless, the costs of having males were revealed to be far lower than expected due to the relatively low demographic importance of reproduction in long-lived organisms such as trees. This shows that hypothesized ecological conundrums may be less puzzling when all life stages are considered. 

Chapter 7 tested whether the net population level effects of lianas, structural parasites of trees, differ systematically among tree species. When integrating all lower-level effects on vital rates across the life cycle for 33 species, liana infestation severely decreased population growth rates, particularly for fast-growing light-demanding tree species. This finding is at odds with conventional wisdom which states that slow-growing shade-tolerant tree species should be disproportionately negatively affected by lianas because they tend to have greater proportions of infested trees.  Yet, chapter 7 suggests that shade-tolerant species have higher proportions of infested trees because they are less affected by infestation and simply better tolerate lianas. 

Chapter 8 applies a disease ecology model to explain the proportion of liana-infested trees (liana prevalence). This model shows that liana prevalence among species (at the community level) is caused by the interplay between colonization, tree recovery, liana induced lethality and tree mortality. Integration of these 4 rates narrowed down which rates were crucial in determining community level patterns as liana prevalence could be successfully predicted with only information on two rates: host recovery and parasite induced lethality. 

Chapter 9 reviews techniques from computer science that enabled the large-scale integration conducted in the previous chapters. It highlights methods that are indispensable in making cross-scale integration of large datasets practically feasible, and provides an accessible toolbox for any biologist in high-performance computing.

All chapters taken together demonstrate the profound influence scale has on how we interpret results and understand the links between processes operating at different levels (reviewed in chapter 10). We see that clear patterns at one scale may crumble to noise at another scale and only a handful of lower-level and fine-scale processes may be influential in predicting patterns at larger scales. This thesis therefore illustrates how perceptions of importance are scale-dependent and highlights how moving up in scale, forces abstraction: the identification of which details at the fine scale truly matter for the phenomenon at higher scales of organization. Ecologists are realizing that to understand a system, it is important to study it at all the appropriate scales, and to develop models that bridge levels of organization and spatiotemporal scales. Species coexistence is a process of multiple scales in space, time and organization. The translation and interrelation of phenomena across these scales is a critical step towards understanding, and sustaining, the diversity of tropical forests.

\chapter{Nederlandse samenvatting} % The asterisk leaves out this chapter from the table of contents
Tropische bossen kennen een onge\"evenaarde soortenrijkdom: er zijn zo veel soorten dat we niet eens weten hoeveel precies. Biologen worstelen al sinds de tijd van Charles Darwin met de vraag hoe zoveel soorten met elkaar kunnen samenleven. Dit is een belangrijke vraag aangezien we zonder deze kennis onmogelijk kunnen voorspellen hoe klimaatverandering deze soortenrijkdom, en geassocieerde ecosysteemdiensten zoals drinkwatervoorziening en koolstofopslag, zal veranderen. In de biologie is de laatste paar decennia ruime aandacht geweest voor dit probleem, hetgeen een grote verscheidenheid aan studies en kennis opleverde. In theorie zouden wetenschappers met deze kennis moeten kunnen voorspellen hoe bossen zullen reageren op verschillende bedreigingen. In de praktijk ontbreekt echter de integratie tussen studies, omdat wetenschappelijk onderzoek zich bijna uitsluitend gericht heeft op kleine ruimtelijke schaal, op een beperkte tijdschaal of op een enkel niveau van biologische organisatie - zoals een enkel trofisch niveau of levensfase. Echter, hoe vertalen al deze studies naar grotere ruimtelijke, temporele en organisatorische schaalniveaus? Zullen processen die belangrijk blijken op de kleine schaal, ook invloedrijk zijn op grotere schaalniveaus? Het overkoepelende thema van dit proefschrift is of (en zo ja, hoe) processen zich laten vertalen naar hogere organisatorische en grotere ruimtelijke en temporele schaalniveaus. In het bijzonder richt dit proefschrift zich op de processen waarvan verondersteld wordt dat zij  belangrijk zijn voor het structureren van tropische bosdiversiteit op kleine schaalniveaus, en integreert deze over gehele levencycli, boompopulaties, boom-gemeenschappen en meerdere trofische niveaus. 	

Het onderzoek richt zich op het bos van Barro Colorado Island (een eiland in het Panama-kanaal), een van de meest intensief onderzochte tropische bossen ter wereld. De gepresenteerde hoofdstukken integreren verschillende grote datasets, en werpen samen een blik op ecologische processen en patronen, zoals dichtheidsafhankelijkheid en "trade-offs", over vele hectares bos en over meerdere levensfases en trofische niveaus. In het algemeen worden twee wetenschappelijke benaderingen gebruikt. De eerste is het gebruik van een ecologisch modelsysteem, namelijk de tropische palm \textit{Attalea butyracea}, onderzocht op een grote ruimtelijke schaal en voor alle levensfases (Hoofdstukken 2-4). De tweede aanpak is een vergelijking van verschillende boomsoorten, waar zeven grote lange-termijn datasets worden ge\"integreerd, die miljoenen observaties bevatten over een tijdspan van 30 jaar, kijkend naar de dynamiek van zaden tot woudreuzen (Hoofdstukken 5-8). 

Hoofdstuk 2 maakt gebruik van een meerjarig archief van zaadpredatie, opgeslagen in de bodem van het bos. Het opgraven van de harde vruchten van de palm \textit{Attalea butyracea} geeft inzicht  in de ecologische interacties over drie trofische niveaus op de schaal van het gehele eiland. Resultaten laten zien dat het idee dat het succes van soorten wordt gelimiteerd door soort-specifieke natuurlijke vijanden (zogenaamde "Janzen-Connell effecten") verder wordt gecompliceerd door de vijanden van die vijanden. Knaagdieren prederen de larven van een gespecialiseerde bladkever (\textit{Bruchinae}), waardoor de keverpopulatie wordt gereguleerd. Als gevolg hiervan kan de kever geen dichtheidsafhankelijke zaadpredatie veroorzaken, en daarmee ook niet de palm \textit{A. butyracea} reguleren. Dit resulteert in dichtheidsonafhankelijke overleving van zaden, terwijl eerder onderzoek liet zien dat "Janzen-Connell effecten" aanwezig waren op de kleine ruimtelijke schaal. Hoofdstuk 2 laat zien dat processen op de kleine schaal niet noodzakelijkerwijs relevant zijn voor de populatiedynamiek op grote schaal, zodra men alle trofische lagen onder de loep neemt.

Hoofdstuk 3 toetst of competitie voor zaadverspreiders binnen een soort resulteert in negatieve dichtheidsafhankelijke zaadverspreiding en of dit bijdraagt aan negatieve dichtheidsafhankelijkheid van de vestiging van zaailingen. Bezoek van vrucht-etende dieren, verwijdering van vruchten en verspreidingsafstand daalden allemaal met toenemende populatiedichtheid van \textit{A. butyracea}. Dit toont duidelijke competitie voor zaadverspreiders aan. Dit dichtheidsafhankelijke patroon op grote schaal droeg echter niet bij aan veranderingen in de patronen van zaaddichtheden en lokale zaailingdichtheden. Patronen op deze fijne schaal werden gedomineerd door het ruimtelijk patroon van de volwassen palmbomen. Hoofdstuk 3 laat daarom zien dat sterke patronen op grote schalen niet noodzakelijkerwijs bijdragen aan patronen op de kleine schaal waarop de dynamiek van zaden en zaailingen bepaald wordt. 

Hoofdstuk 4 integreert dichtheidsafhankelijke patronen in groei, overleving en reproductie over alle levensstadia in een populatiemodel. Resultaten laten zien dat negatieve dichtheidsafhankelijke processen in de zaailingfase sterk genoeg zijn om de populatie van \textit{A. butyracea} te reguleren voordat deze soort het bos zou gaan domineren. Tegelijkertijd laten de resultaten zien dat de populatie groeit bij lage dichtheid, en zich dus kan herstellen. Dit systeem voldoet daarmee aan de theoretische basisvoorwaarden voor "coexistence", het duurzaam naast elkaar bestaan van soorten. Er zijn echter belangrijke knelpunten in de levenscyclus van A. butyracea die een sterke invloed op de geschatte equilibriumdichtheid hebben. \textit{A. butyracea} populaties kunnen veel hogere dichtheden bereiken wanneer bepaalde dichtheidsonafhankelijke knelpunten verdwijnen, bijvoorbeeld bij verstoring van de omgeving waarbij de lichtinval wordt vergroot. Hoofdstuk 4 laat daarmee zien dat patronen van soortenrijkdom en abundantie worden be\"invloed door zowel dichtheidsafhankelijke en -onafhankelijke processen. Bovendien laten de resultaten van hoofdstukken 2-4 samen zien dat factoren die cruciaal zijn voor het bepalen van patronen op een enkele ruimtelijke schaalniveau of levensstadium, onbelangrijk kunnen zijn op hogere schalen en niveaus van organisatie.

Hoofdstuk 5 bevat een nieuwe en zeer complete analyse van de relaties tussen uiterlijke kenmerken van planten (zogenaamde "functional traits", kenmerken die geacht worden belangrijk te zijn voor het functioneren van een individu) en demografische processen zoals groei, reproductie en overleving. Dit wordt gedaan door zaad-, zaailing- en boomdata te combineren met data over vier functionele kenmerken (zaadgewicht, houtdichtheid, blad gewicht per oppervlak en maximale eindhoogte van de boom). Resultaten laten een sterk toegenomen voorspellende kracht zien vergeleken met eerdere studies, en geven een veel completer beeld van de relaties tussen functionele kenmerken en demografische processen, en van hoe deze over de gehele levenscyclus van tropische bomen op elkaar inwerken. Hoofdstuk 5 levert bewijs dat tropische boomsoorten evolutionair beperkt zijn doordat meerdere demografische processen tegengestelde effecten hebben (zogenaamde "trade-offs") die tegelijkertijd inwerken op verschillende levesstadia. Hoofdstuk 5 laat ook zien dat de effecten van sommige uiterlijke kenmerken op \'e\'en levensstadium teniet kunnen worden gedaan door tegengestelde effecten op een ander stadium. Dit laat het gevaar zien van het trekken van vergaande conclusies gebaseerd op analyses van een enkel levensstadium.

Hoofdstuk 6 onderzoekt het fundamentele ecologische mysterie hoe tweehuizige soorten kunnen bestaan naast niet-tweehuizige soorten, aangezien slechts de helft van de tweehuizige individuen, alleen de vrouwelijke bomen, zaden produceert. Om deze paradox te onderzoeken ontwikkelden we een nieuw soort populatiemodel (Composite Integral Projection Model), dat rekening houdt met variatie in ecologische strategie\"en tussen soorten, door, bouwend op 
hoofdstuk 5, gebruik te maken van drie eerdergenoemde functionele kenmerken (zaadgewicht, houtdichtheid, en eindhoogte van de boom). Resultaten laten zien dat tweehuizige soorten vooral profiteren van hogere individuele zaadproductie, en na integratie over alle levensfases blijkt dat deze voordelen de kosten van tweehuizigheid nagenoeg opheffen. Daarnaast laat hoofdstuk 6 zien dat de kosten van de aanwezigheid van mannelijke bomen veel lager bleken te zijn dan verwacht, dankzij het relatief lage demografisch belang van reproductie voor lang-levende soorten, zoals bomen. Dit werk laat zien dat veronderstelde ecologische mysteries minder raadselachtig hoeven te zijn als alle levensfasen worden bekeken. 

Hoofdstuk 7 onderzoekt het netto effect van lianen op boompopulaties, en of deze varieert tussen soorten. Lianen zijn klimpanten, die parasiteren op de investering die bomen maken in hun constructie om zo zonder eigen investeringen een weg naar het kronendak te vinden. Lianen hebben sterk negatieve effecten op de groei, reproductie en overleving van bomen. Wanneer alle demografische effecten van lianen in een populatiemodel van 33 tropische boomsoorten worden samengenomen blijkt dat door de aanwezigheid van lianen de populatiegroeisnelheden van bomen sterk afnamen, en in het bijzonder die van snel-groeiende, licht-minnende boomsoorten. Deze bevinding druist tegen bestaande inzichten in die stellen dat juist de langzaam-groeiende en schaduw-minnende boomsoorten disproportioneel hard getroffen zullen worden door lianen omdat zij vaker met hogere percentages lianen bezet zijn. Hoofdstuk 7 suggereert echter dat schaduw-minnende boomsoorten met hogere percentages lianen bezet zijn omdat zij juist minder gevoelig zijn voor de negatieve effecten van lianen.

Hoofdstuk 8 past epidemiologische modellen toe om variatie in de bezettingspercentages van bomensoorten met lianen te verklaren. Het model laat zien dat de proportie van bomen met lianen afhangt van de wisselwerking tussen processen zoals kolonisatie, herstel, de mortaliteit van bomen zonder lianen en de extra sterfkans door de aanwezigheid van lianen. Door het integreren van deze vier processen konden de meest belangrijke processen worden ge\"identificeerd die cruciaal zijn in het bepalen van de variatie in bezettingspercentages tussen boomsoorten in de gemeenschap. De proportie van bomen met lianen kon worden voorspeld met alleen informatie over twee processen: het kwijtraken van lianen en de extra sterfkans van bomen door lianen. 

Hoofdstuk 9 geeft een overzicht van verschillende technieken uit de computerwetenschap die de grootschalige integratie van datasets, zoals beschreven in de vorige hoofdstukken, mogelijk maakt. Hoofdstuk 9 heeft als doel die "high-performance computing" bereikbaar te maken voor biologen.

Alle hoofdstukken samen demonstreren de verregaande invloed die schaalniveaus hebben op de uitoefening van biologie. Het laat zien hoe schaalniveau de interpretatie van onze resultaten en ons begrip van de koppelingen tussen processen op verschillende niveaus be\"invloedt (zoals besproken in hoofdstuk 10). We zien dat duidelijke patronen op \'e\'en schaalniveau verbrokkelen tot ruis wanneer op een andere schaalniveau gekeken wordt, en dat maar een handvol van de kleinschalige processen invloedrijk zijn op de patronen die gevormd worden op hogere schaalniveaus. Dit proefschrift laat zien hoe de beoordeling van het belang van of een bepaald proces schaalafhankelijk is, en illustreert hoe het opschalen van kleine naar grote schaal abstractie afdwingt: het identificeren welke kleinschalige details er werkelijk toe doen voor het fenomeen dat onderzocht wordt op hogere schaal. Ecologen realiseren zich steeds meer dat om een systeem goed te begrijpen, onderzoek moet plaatsvinden op de juiste organisatorische, ruimtelijke en temporele schaalniveaus. Het ontwikkelen van methoden die een brug kunnen slaan tussen verschillende ruimtelijke, biologische en organisatorische schaalniveaus, wordt daarom steeds belangrijker. De vraag hoe soorten kunnen samenleven is een vraag die afhangt van processen op meerdere schaalniveaus in ruimte, tijd en organisatie. De vertaling en koppeling van observaties tussen deze schaalniveaus is een cruciale stap die gemaakt moet worden, voordat we de diversiteit van tropische bossen kunnen hopen te begrijpen, laat staan behouden. 



\begin{figure}
\vspace*{-2cm}\hspace*{2cm}\fbox{\includegraphics[width=10cm,height=10cm]{../figures/acknowplot_names.pdf}}
\begin{minipage}{14cm}
\footnotesize
\vspace{0.2cm}
A gravitation plot of the holy trinity in the pursuit of advanced academic degrees. To survive and stay sane in graduate school requires a delicate balancing act of "Science", "Support" and "Procrastination". Without each in good measure, success, happiness and sanity are not ensured.  Different people play different roles throughout graduate school, with most playing a mix of all three. The positions of the names are based on each individual's relative contributions to Science, Support and Procrastination. Scores on Science and Support are measured by qualitative assessment by the author (using the "that seems about right" approach). Procrastination scores are based on an estimate of the volume of liquid consumption in various public houses and other venues in company of the author. Notice that positions are relative.  For instance, when a John Doe and a Jane Smith both contributed a score of 100 to Science, 30 to Support but Jane enjoyed a score of 30 (liter) more than John with the author on procrastination. This will place Jane closer to Procrastination but further from Science and Support than John - even though her role in Science and Support was equal to John's. Hence, the position of each name signifies the person's attraction towards each of the categories of "Science", "Support" and "Procrastination" based on their relative contribution. A balanced score results in placement between categories. Name colors indicate relative scores of Science, Support and Procrastination on the red, green and blue color channels.  Coordinates are calculated as 
$X_i =  W^{science}_i X^{science} + W^{support}_i  X^{support} + W^{procrastination}_i  X^{procrastination}$ and $Y_i =  W^{science}_i  Y^{science} + W^{support}_i Y^{support} + W^{procrastination}_i  Y^{procrastination}$. Where $X$ and $Y$ denote the Cartesian coordinates in the above figure for person i, or the central locations of science, support and procrastination while $W$ signifies the weight for person i in each category. Weights are the proportional score in each category relative to each individuals total score (sum over all three categories), and a spacing algorithm to improve legibility. Hence, clearly, positions do not reflect ranking, nor do they resemble total effort or input compared to others in the graph. Not all names could be included in the graph, due to crowding, legibility issues and the fact that after a while I had enough of typing names into a spreadsheet. \\

\end{minipage}

\end{figure}



\chapter{Acknowledgements}
After an intensive period of five years, today is the day: writing this note of thanks as the finishing touch of this dissertation. As is customary in a Dutch doctoral thesis, I would like to reflect on the people who have been indispensable not only towards the quality of science, but also to the maintenance of my sanity through their continued support and every now and then supplying some needed procrastination (see opening figure). This sections starts with acknowledging support in accordance with formal scientific hierarchy, but then quickly digresses towards a chaos from which no order or value can be assigned. 

I first met my promotor and daily supervisor, Hans de Kroon and Eelke Jongejans, at a workshop in Norway while I was an undergraduate almost 10 years ago.  Their encouragement spurred me on to publish and improve my work on mast fruiting of the Malaysian Dipterocarp \textit{Shorea leprosula}. I encountered Hans again during a fellowship on Barro Colorado Island, where he offered me the opportunity to write a grant for N.W.O which put us all on the track that eventually lead to this document. It is, therefore, thanks to his foresight and initiative that I am writing this final section. I owe Hans de Kroon my sincere gratitude for this opportunity, his continued support and enthusiasm throughout this thesis.  Hans has taught me that a mentor of students not only advices, but also promotes them, and allows them to make their own decisions. Hans, I thank for your patience and knowledge but especially the trust you put in me whilst allowing me the room to work in my own way or visit outside labs. This freedom has been pivotal in my development as a scientist. I also greatly enjoyed heading out to Coiba with Astrid, Joris and Marjolein. Eelke Jongejans, too, has played no small role in my development as a scientist. I thank you for the opportunities to speak at international conferences (as Miami) and inviting me to join decisive workshops (like Rostock) with you. You did your best to keep me on track, but also didn't stop deviations from the track. These may have delayed more relevant work, but also helped me develop into more of a well-rounded scientist.  I learned from you that it is important to be precise in general but also in the little stuff, to be constructive always in peer review, and that it is sometimes better to avoid conflict. Also thanks for your sand themed dinners and the many board games. Hans and Eelke, our collaborations have been both fun and fruitful and I hope that many more will follow in the future.

I met Joe Wright and Helene Muller-Landau during my first visit to Barro Colorado Island. Since then you have shown me unwavering support during the past 9 years, invited me into your home and family, and even rushed me to the emergency room when my fever was $\approx$ 105.  Thanks for your continued and decisive support when I searched for positions that would follow this PhD. I vividly remember the first meeting for my STRI short-term fellowship with my supervisor (now co-promotor), Helene Muller-Landau. She was discussing my research proposal, making excellent points faster than I could keep up, all while simultaneously working on an unrelated R-script and feeding her new born child. The experience was both astonishing and intimidating. Since then Helene has become my personal role model in what it means to be a scientist.  From her I learned to be my own toughest reviewer, to explore all other explanations, to minimize self-deception and to strive for honesty in all scientific communications. To not endlessly muddle in the literature, or dabble over established ideas but to stop and think for myself - typically aided by pen and paper for the scribbling of equations. From Joe Wright, who's knowledge of the forest is unsurpassed, I have learned much. I've learned to "acknowledge all contributions", to act with sincerity and strive for consistency both in thought and action in my work, and to write with brevity that doesn't overrule clarity or accuracy (I'm still a learner here). Yet, the most significant lesson is the value of balance. Be it balancing scientific ambitions with personal life or theory with natural history.  I learned that intimate knowledge of the organisms in the field, and detailed botanical experience are just as valuable as a keen mathematical sense and statistical know how. Without this balance, the detections of miscalculations and faults in theory would remain inconspicuous in clouds of model results (as was often pointed out by Joe in the graphs I produced). I will continue to strive to emulate this balance in all aspects of life and scientific work.  Joe and Helene, I thank you both for your continued support. You have both become personal friends, and I look forward to seeing Isaac and Eric grow up. \end{fullwidth}

Next, I would like to thank my colleagues from Radboud University for the general collegiality that each of them offered to me.  My attention span for administrative procedures is so small it makes any task seem excessively complicated and Kafkaesque. Hence, I may have contemplated suicide if not for the help of Jose Broekmans who made everything seem easy. I hope I fall into a favorable pigeonhole in your system\sidenote[1][-3.5cm]{Jose has a sign above her desk that classifies people into four classes: Bewust Bekwaam, Onbewust Bekwaam, Bewust Onbekwaam and Onbewust Onbekwaam (Consciously Competent, Unconsciously Competent, Consciously Incompetent and Unconsciously Incompent). I'm sure everyone who visits her office wonders where she places them}.  My appreciation also goes out to the faculty, staff, technicians, postdocs, fellow \begin{fullwidth} PhD's and students who have brightened many a dull lunchtime, or supplied me with ample laughs or interesting discussions at beer hour, movie nights, game night. A massive thank you, in no particular order, to Eric, Heidi, Liesje, Jose Adema, Peter Charpentier, Jeroen, Leon L., Leon v/d B., Philippine, Monique, Christiaan, Casper, Isabel, Onno, Nicky, Nick, Pieter, Michiel, Natan, Ernandes (when is that Brazilian BBQ planned?!), Marlous, Eva, Valerie, Anne, Nils, Dina, Ralf \& Ralph, Hannie, Annemiek, Xin, Arthur, Peter Cruijsen and Niels. Sarah Faye, my office buddy, thanks for the "gezelligheid" and I wish you the best in all your Arctic adventures.  To Qian Zhang I offer condolences as she got stuck with such a very messy office mate, but luckily he wasn't often there ;) 

I owe special thanks to those PhD students that shared my cubical, though I was away for large stretches of time, we were always able to support each other by deliberating over our problems and findings, but also by talking about things other than just our papers. The many dinners, shared vacations and outings were a blessing. Marloes Hendriks, you were a great support, and I certainly hope you are enjoying your new job. Let us run on the Veluwe again soon, no more then let's say a 10k? Janneke Ravenek's enthusiasm and adventures never failed to bring a smile to my face. Your new house is awesome and I hope we bump into each other on occasion.  Jan-Willem was always in for a good talk, while ensuring everyone did things safely. Thanks for your keen input in all our discussions. Never a dull moment with Maartje Groot, she keeps us entertained with her huge repertoire of anecdotes and (occasional dirty) stories. Don't worry I'll never buy pink for a girl again. Also, how brilliant is it that we will be defending on the same day? Laura Govers is a fantastic scientist, cook, colleague and true friend. Thanks for all the delicious meals, support in hard times, and great concerts. I am enthusiastic about our prospective collaborations, and am hoping she visits me in the states. 

 	I am very fortunate that I only supervised the brightest, most ambitious and hard working students. Thank you Annieke Borst, Anne ten Berge and Nadia Safar for your wonderful efforts. Annieke and Anne also travelled to Panama with me to an oppressively humid forest, crawling with insects, chiggers, ticks, sand flies, snakes, assassin beetles, immense hairy spiders and legions of other parasites. Annieke was fortunate enough to encounter a botfly maggot (\textit{Dermatobia  hominis}), which - after being transported by a mosquito - had hatched on her skin only to burrowed inside to consume her flesh - and I'm afraid Anne's experiences were quite similar! Yet, these were experiences from which they didn't flinch, or whine, no they carried on like true soldiers. You girls are hardcore tough and cool! I express my immense thanks to you both and hope you can look back on the experience with both pride and pleasure. Their excellent work resulted in first class scientific data, much of which is presented here. 
 	
 	
Whenever I visited the Smithsonian, my daily work there has been blessed with a friendly, inspiringly smart and cheerful group people from various nationalities. The work in this thesis has benefited hugely from the myriad of scientific discussions, official or not, from a great many people. I thank Martijn Slot, Michiel v. Breugel, Rachel Page, Kate \& Teague, Justin and Myra, Andy Jones, Scott Mangan, Sasha Wright, Hubert Herz, Christian Ziegler and Daisy Dent, Carol Puerta Pi\~nero, Lars Markesteijn, Omar Lopez, Janeene Touchton, Stuart Dennis, Stuart Davies, Jessica Stapley, Bettina Engelbrecht, Ioana Chiver, Meg Eckles, Meg Crowfoot, Ben Hirsch, Kisten Becklund, Alicia Ledo, Richard Condit and Dina Dechmann.  I hope to encounter each and every one of them many more times more in the wild. I also thank Allen Herre for the many BBQ's, Steve Hubbell for his advice and early copies of the census data, and Egbert Leigh for the many drinks and lively scientific discussions. To my old room mates Nelson, Christian Harris and Siewe Siewe Siewe thanks for sharing so much food and fun. David Brassfield's botanical expertise needs to be acknowledged as it laid the foundation for the now 7 years running reproductive census from which we still hope to gain much insight. I am obligated to Anayansi Cerezo who meticulously continued this work. The Royal Netherlands Academy Ecology fund generously funded another season of this census. Eric Oriel Vasquez was a great help with the endocarp excavations. In general I thank the Smithsonian Tropical Research Institute, with Oris Acevedo and Belkys Jimenez in particular, for the logistical and administrative support. Also many thanks goes out to all the great residents and employees on Barro Colorado Island, they make life bearable and island-fever fun. 

I would also like to thank those who fall into the categories great colleagues, friends or both. Ryan Chisholm, even though they don't build statues for critics I will always appreciate your critical eye and opinion. I wish you and Eli truly the best and am going to do my best to join you in Columbia for the wedding! Mateo and Rebecca I'm happy we will be in close contact at Princeton, I'm looking forward to many a lively interaction. Thanks also goes out to Bram van Putten for his enjoyable company, and for motivating me to critically evaluate and apply statistics, and for his innovations on anisotropic dispersal.  Stefan Schnitzer thanks for inviting me to Milwaukee, and into your home and family. I greatly enjoyed my short time in your lab. You have a fantastic bunch of students. Thanks Katie Barry and Sergio Estrada for the Polish dancing, the fish fry and those cheese curds. Liza Comita and Simon Queenborough, thanks for all your support and help with my YIBs proposal. I'm really sorry that it didn't work out but I plan to finish the proposed work with you nonetheless. Thanks for allowing me to visit your labs, and I also thank Jenalle, Meghna, Juan Carlos and Andrew for the all fun times in New Haven. Sean McMahon and Jessica Metcalf, your selfless hospitality and generosity is not forgotten! Thank you for sharing your home with me, and letting me crash on the couch when needed. Jessica, I hope to persuade to return to the wondrous world of plant demography one day. Sean, thanks for all the fun in Annapolis, Rostock, China, Nijmegen, Princeton and Baltimore.  Many more discussion and collaborations, both scientific and culinary are undoubtedly forthcoming.  Rob Salguero-Gomez thanks for organizing all those workshops, they not only resulted in fun inebriated times at Pleitegeier but also in wonderful papers (here credit must also go to Cory, Philip and Sydne). I greatly enjoyed Frank Streck's company in Montpellier, where we both have become fans of \textit{"Roadies of the D"}. Finally, I'd be remiss if I didn't acknowledge Patrick Jansen, for the first class advice, and for sharing his passion for great French wines and fine food. Patrick turns out to be a great dancer. Thanks keeping us on our toes when Frank and I wanted to go to bed. 

A good support system is paramount to staying sane in grad school. I'm obviously heavily biased, but I think I have one of the best. Jim and Nicole Henry are among the most agreeable, loving and helpful people in the world. Thank you so much for helping my sister in her times of distress, and I am very grateful that I got the chance to get to know you, and Samuel and Evalina of course. It so good to have family close by and I'll see you guys and Teal, Jude, Andy, Barb, Amy and Andrew soon enough. Coincidently, you are the people I think I could survive a zombie apocalypse with. Love goes out to my sister, Janneke Visser, you seriously cause me way too much stress sometimes, but I'm very proud of what you have become. I wish the best of luck and all the happiness in the world with Hamed in South Africa. To Sieger, I hope you find your way in life - we will always be ready to help. Beyond academia there are also my friends. I am always appreciative for Elmid Kooistra's dry wit, and thank you for keeping me on my toes during our sparring sessions. I'm sure you will kick ass in your upcoming fight. John de Vries, Roger Plekker, Sylvester Tanis, and Paul v/d Meer remain the best of buddies, though we live far apart they are still always ready to have a good time. Peter Vredeveld, during my PhD you got married and moved to many countries around the world. You have always been a supportive ear, and a loyal friend. Thanks for this.  Jeroen Bruijning, Catrien Mesman, Carla van Dongen and Miriam Bruijning, I thank you for welcoming me into your families, for the holidays, the good food, company, the gardening, and for all your support.  It would be highly neglectful if I didn't thank my parents in this section. Dad (and Ducky), thanks for your wise counsel and sympathetic ear, and for always helping me out whenever something needed to be build but I lacked technical skills to do so. Mom, thank you for always being my number one supporter, thank you for always seeming even more excited about my progress than I am. Life hasn't always been fair to us four (Janneke and Sieger included), but I hope to do justice to all the sacrifices you've made. 
 	
Looking back now, in the period that entailed the writing this thesis my life has gone through many changes (I almost died twice so I shouldn't forget to thanks my guardian angels!). Femke Maes, my buddy, my confidant, we brought out the best in each other, but also the worse. I am grateful for the times I shared with you, but also with Jack, Joke, Renske and Pablo. Thank you for all the support you gave me during this thesis, and as students from the Madelievenstraat, Speenkruidstraat, Bialowieza forest, Snoekstraat, BCI, de Udenstraat and finally the Apollostraat. You tell many a tall tale, and hold your own with the most fouled mouthed among us. Obviously, we still have quite a few beers to polish off before the end. Caspar Hallmann has provided a roof in times of hardship, made sure I never starved and, more importantly, also made sure I never went thirsty purchasing those drinks individuals of stature and sophistication drink. He has provided an experienced ear for my doubts, and an excellent help in all aspects of theory and analysis. One drawback though is that he makes his killer Greek kebab far too scarcely. 

And then there is one former student I haven't thanked, Marjolein Bruijning. Marjolein played a major role in this thesis and plays a major role in my life. Marjolein's kindness has offered me a different viewpoint on life. She has inspired me to improve myself on all fronts. Thank you for showing me Italy, the mountains, and for the adventures to come. Thank you for your love and support. You are an amazing scientist, way out of my league, and I hope you will keep putting up with me for a long time to come. \\

\vspace*{20cm} \begin{figure}
\vspace*{-.5cm}\hspace*{1.cm}\fbox{\includegraphics[width=12.5cm,height=12.5cm]{../figures/VisserExport.jpeg}}
\hspace*{.3cm}\begin{minipage}{14.2cm} \footnotesize Marco Dirk Visser was born in Springs South Africa, on the $5^{th}$ of October 1982. Growing up remotely and secluded amongst the farm houses and "mielie" fields of Sundra, he developed a deep love of the varied and beautiful animal inhabitants of Africa. In particular, he enjoyed catching "goggas" and "shongololos" or anything that slithers, wriggles, creeps or crawls to the dismay of his mother and sister. After emigrating to the Netherlands, this fascination for nature never died. Yet it wasn't until the first year of his bachelor, while volunteering to catch and follow shrews in the swamps of eastern Poland, that he got bitten by the scientific bug - \textit{Alea iacta est}. This fateful event led him to an internship at the Forest Research Institute of Malaysia and his first experience in a tropical forest. Though the work - which entailed conducting pre-felling inventories on trees in the lush swamp forests along the Pahang river just before those same trees were felled and shipped abroad - didn't really suit him. Luckily, he later arrived at Pasoh Forest Reserve where he became fascinated by general flowering events, where entire forests burst into flowering and fruiting after years of little more than vegetative growth. Then, while endeavouring to learn about the evolution of these colossal fruiting events,   he found that a single place would persistently pop up in the literature, a place called Barro Colorado Island. It wasn't until he happened to witness a guest lecture by Patrick Jansen that a chance to visit Barro Colorado Island materialized. It would be the focus of his master thesis. The mix of science and nature that this island had to offer was so intoxicating that he decided to become a frequent visitor from 2008 onwards. Under the mentorship of Helene Muller-Landau and Joe Wright, he first came back as a short term fellow, and then joined Hans de Kroon's lab in 2009 with ambition to write a grant that would fund a PhD, which in turn, would function as an excellent excuse to return often. He was kindly supported by Helene Muller-Landau as the first grant was rejected, and the second, and the third, until finally he started his PhD in 2011. The contents of this report are the fruits of that endeavour. In 2016 he will start a post-doctoral position in the United States, joining Steve Pacala's lab at Princeton University, where he hopes to continue to find excuses to witness the lavish variety of organisms on our beautiful blue planet.
\end{minipage}
\end{figure}

\chapter{Curriculum Vitae}

\begin{center}\rule{0.5\linewidth}{\linethickness}\end{center}

\Large \textbf{Publications (inc. submitted/ in
preparation)}
\vspace{1cm}
\footnotesize
\begin{description}

\item[2011]
\textbf{1.} \textbf{M. D. Visser}, E. Jongejans, M. van Breugel, P. A.
Zuidema, Y. Chen, A. R. Kassim, H. de Kroon. 2011. Strict mast fruiting
for a tropical dipterocarp tree: A demographic cost-benefit analysis of
delayed reproduction and seed predation. Journal of Ecology. 99,
1033-1044.

\textbf{2.} \textbf{M. D. Visser}, S. Joseph Wright, Helene C.
Muller-Landau, Gemma Rutten and Patrick A. Jansen. Tri-trophic
interactions affect density dependence of seed fate in a tropical forest
palm. 2011, Ecology Letters. 14, 1093-1100.
\item[2012]
\textbf{3.} B. van Putten, \textbf{M. D. Visser}, P. A. Jansen and H. C.
Muller-Landau. Distorted- distance models for directional dispersal: a
general framework and its application to a wind-dispersed tropical
forest trees. Methods in Ecology and Evolution. 2012.

\textbf{4.} B. T. Hirsch, \textbf{M. D. Visser}, R. Kays and P. A.
Jansen. Quantifying seed dispersal kernels from truncated seed-tracking
data. Methods in Ecology and Evolution. 2012
\item[2013]
\textbf{5.} \textbf{M. D. Visser}. aprof: Amdahl's profiler, directed
optimization made easy. R package version 0.1 - 0.3.1.
http://cran.r-project.org/web/packages/aprof/index.html. 2013.
\item[2014]
\textbf{6.} P. A. Jansen, \textbf{M. D. Visser}, S. J. Wright, G.
Rutten, H. C. Muller-Landau. Negative density-dependence of seed
dispersal and seedling recruitment in a Neotropical palm. Ecology
Letters 17: 1111--1120. 2014.
\item[2015]
\textbf{7.} \textbf{M. D. Visser}, S. M. McMahon, C. Merow, P. M. Dixon,
S. Record and E. Jongejans. Speeding Up Ecological and Evolutionary
Computations in R; Essentials of High Performance Computing for
Biologists. PLoS Comput Biol 11(3): e1004140.
doi:10.1371/journal.pcbi.1004140. 2015.
\item[2016]
\textbf{8.} \textbf{M. D. Visser}, M. Bruijning, S. J. Wright, H. C.
Muller-Landau, E. Jongejans, L. S. Comita and H. de Kroon. Functional
traits as predictors of vital rates across the life-cycle of tropical
trees. Functional Ecology.
\item[In press]
\textbf{9.} M. Bruijning, \textbf{M. D. Visser}, H. C. Muller-Landau, S.
J. Wright, L. S. Comita, S. P. Hubbell, H. de Kroon, E. Jongejans.
Surviving in a cosexual world: a cost-benefit analysis of dioecy in
tropical trees. In press at the American Naturalist.
\item[In revision]
\textbf{10.} E.J. Francis, H.C. Muller-Landau, S.J. Wright, \textbf{M.
D. Visser}, Y. Iida, A.R. Kassim, C. Fletcher, and S.P. Hubbell.
Re-evaluating the functional significance of wood density for
interspecific variation in growth and survival in tropical trees. Global
Ecology and Biogeography
\item[In review]
\textbf{11.} \textbf{M. D. Visser}, S. Joseph Wright, Helene C.
Muller-Landau, Eelke Jongejans, Liza S. Comita, Hans de Kroon and Stefan
Schnitzer. Parasite-host interactions in tropical trees: lianas differentially impact population growth rates of tropical tree species. In review at the Journal of Ecology.

\item[In prep]
\textbf{12.} \textbf{M. D. Visser}, S. Joseph Wright, Helene C.
Muller-Landau, Gemma Rutten and Patrick A. Jansen. Population-level density dependence in a tropical forest:  Regulation and limitation of a common palm. In preparation for Ecology Letters.

\textbf{13.} \textbf{M. D. Visser}, Helene C. Muller-Landau, Eelke
Jongejans, Liza S. Comita, Hans de Kroon and S. Joseph Wright.
Host-parasite interactions in tropical trees: explaining variation among tree species in liana prevalence. In prep
for Ecology.


\end{description}

\begin{center}\rule{0.5\linewidth}{\linethickness}\end{center}

~ ~ ~

\vspace{1cm}
\Large \textbf{About my research}
\vspace{1cm}
\footnotesize

\begin{description}

\item[2015]
\textbf{Salguero-G\'omez, R} (2015). Demography to infinity and beyond!
Journal of Ecology blog.
\url{https://jecologyblog.wordpress.com/2015/04/09/demography-to-infinity-and-beyond/}

\textbf{Wang, I} (2015). Recommendation F1000 prime.
\url{http://f1000.com/prime/725405210}
\item[2011]
\textbf{Sugden AM} (2011) Science Editors' choice. Ecology. The Enemy of
My Enemy is my? Science 334:569.

\textbf{Sugden AM} (2011) Science Editors' choice. Ecology. Why trees
skip a year. Science 333:386

\textbf{Rees M} (2011) Editor's Choice: Volume 99, Issue 4 (July).
Journal of Ecology.

\textbf{King, B} (2011), The enemy of my enemy is my friend. Smithsonian
Tropical Research Institute News 1:2

\textbf{Ecological Society of America} - young plant population
ecologist of the month (October 2011). Featured work: M. D. Visser et
al, 2011, Ecology Letters.

\textbf{Kouwen M} (2011) Mastjaar overtreft jaarlijkse zaadzetting.
Bionieuws 13:6.
\end{description}

\begin{center}\rule{0.5\linewidth}{\linethickness}\end{center}

\vspace{1cm}
\Large \textbf{Grants and awards}
\vspace{1cm}
\footnotesize

\begin{description}
\item[2016]
\begin{itemize}

\item
  \textbf{Grant}: Academy Ecology Fund. Royal Dutch Academy of Sciences
  (KNAW), Quantifying the effects of extreme years on tropical tree
  dynamics: capitalizing a rare El Ni\~no occurrence (6k).
\end{itemize}
\item[2011]
\begin{itemize}

\item
  \textbf{Grant}: NWO-ALW, What maintains the diversity of tropical tree
  species? Unravelling the importance of niche and neutrality with a
  life cycle approach. Co-wrote with Hans de Kroon, Helene
  Muller-Landau, Eelke Jongejans, S. J. Wright, P.A. Zuidema, P.A.
  Jansen and S. Tuljapurkar (230k).
\end{itemize}
\item[2009]
\begin{itemize}

\item
  \textbf{Award}: WUF-KLV thesis prize for the best thesis in the life
  sciences from Wageningen University awarded for my MSc thesis:
  Density-dependent dispersal and seed predation in a Neotropical palm.
\end{itemize}
\item[2008]
\begin{itemize}

\item
  \textbf{Grant}: Smithsonian Tropical Research Institute, short term
  fellowship awarded for the study: Quantifying density-dependent
  responses of seed predators in the Neotropical palm \textit{Attalea butyracea}.
  (\$ 5k).
\end{itemize}
\end{description}

\begin{center}\rule{0.5\linewidth}{\linethickness}\end{center}

\vspace{1cm}
\Large \textbf{International presentations}
\vspace{1cm}
\footnotesize
\begin{description}

\item[2015]
\textbf{Workshop} at the British Ecological Society Annual Meeting.
December 2015, Edinburgh. Speeding Up Ecological and Evolutionary
Computations in R; Essentials of High Performance Computing for
Biologists. Organizer.

\textbf{Workshop} at the Evolutionary Demography Society Annual Meeting.
October 2015, Lunteren. Speeding Up Ecological and Evolutionary
Computations in R; Essentials of High Performance Computing for
Biologists. Organizer.

\textbf{Speaker} at the at the Ecological Society of America Annual
Meeting 2015. August 2015, Baltimore. Differential effects of lianas on
population growth rates of tropical forest trees.

\textbf{Workshop} at the at the Ecological Society of America Annual
Meeting 2015. August 2015, Baltimore. Demography in a Continuous World:
New Advances in Integral Projection Models (IPMs). Co-organizer.

\textbf{Workshop} at the at the British Ecological Society Symposium
``Demography Beyond The Population''. March 2015, Sheffield. Speeding Up
Ecological and Evolutionary Computations in R; Essentials of High
Performance Computing for Biologists.

\textbf{Speaker} at the British Ecological Society Symposium
``Demography Beyond The Population''. March 2015, Sheffield.
Differential effects of lianas on population growth rates of tropical
forest trees.
\item[2014]
\textbf{Short Workshop} at the Yale School of Forestry \& Environmental
Studies. December 2014, New Haven. Speeding Up Ecological and
Evolutionary Computations in R; Essentials of High Performance Computing
for Biologists.
\item[2012:]
\textbf{Invited speaker} at the conference ``Everything disperses to
Miami'', December 14 - December 16, 2012, the University of Miami. The
fitness consequences of dispersal for a tropical palm; the role of
dispersers, natural enemies and negative density dependence.

\textbf{Invited speaker} at the Max Planck Intitute for Demographic
Research, workshop on Integral Projection Models, Rostock Germany. June
2012. A Blueprint for speeding-up calculations in R.

\textbf{Speaker} at the Netherlands Annual Ecology Meeting. February
2012. Quantifying dispersal kernels through inverse modeling.
\item[2010]
\textbf{Invited speaker} at the 5th International Symposium-Workshop on
Frugivores and Seed Dispersal. Montpellier, France. June 2010. Measuring
dispersal kernels through inverse modeling: density dependence of seed
dispersal in a Neotropical palm.

\textbf{Speaker} at Plant Population Biology: Crossing Borders.
Gfo-conference, Nijmegen, Netherlands. May 2010. Strict mast fruiting
for a tropical dipterocarp tree: a demographic cost-benefit analysis
\item[2009]
\textbf{Oral presentation} at the Smithsonian Tropical Research
Institute. Panama. December 2009. Density-dependent dispersal and seed
predation in a Neotropical palm.
\item[2008]
\textbf{Oral presentation} at the workshop on stochastic elasticity and
matrix modeling. Nijmegen, the Netherlands, June 2008. Strict masting in
the tropical tree species Shorea leprosula: demographic consequences and
evolutionary benefit of predator satiation.
\item[2007]
\textbf{Oral presentation} at the International workshop in Matrix
models of plant populations. Sogndal, Norway, June 2007. Demographic
consequences of strict masting for two tropical tree species Shorea
leprosula and Shorea parvifolia.
\end{description}

\begin{center}\rule{0.5\linewidth}{\linethickness}\end{center}

\vspace{1cm}
\Large \textbf{Reviewer for scientific journals}
\vspace{1cm}

\footnotesize

Biotropica, Canadian Journal of Forest Research, Ecology, Ecology and
Evolution, Ecology Letters, Journal of Biogeography, Journal of Ecology,
Methods in Ecology and Evolution, PLOS computational biology, The R
Journal.

\begin{center}\rule{0.5\linewidth}{\linethickness}\end{center}

%\printindex % Print the index at the very end of the document
\end{fullwidth}

\end{document}
